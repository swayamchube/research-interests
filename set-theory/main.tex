\documentclass[oneside,a4paper]{report}
\usepackage{../swayam}

\newcommand{\dom}{\operatorname{dom}}
\newcommand{\ran}{\operatorname{ran}}
\newcommand{\pred}{\operatorname{pred}}
\newcommand{\restrict}{\upharpoonright}
\newcommand{\type}{\operatorname{type}}
\newcommand{\ON}{\mathbf{ON}}
\newcommand{\V}{\mathbf{V}}
\newcommand{\WF}{\mathbf{WF}}
\newcommand{\rank}{\operatorname{rank}}
\newcommand{\trcl}{\operatorname{tr cl}}
\newcommand{\Cl}{\operatorname{cl}}
\newcommand{\cf}{\operatorname{cf}}
\newcommand{\val}{\operatorname{val}}
\newcommand{\Fn}{\operatorname{Fn}}

\title{Set Theory}
\author{Swayam Chube\\\textbf{Guide:} Prof. Bharat Adsul}
\date{\today}

\begin{document}
\maketitle

\begin{abstract}
    The main reference for this was \cite{kunen}. The SLP covered Chapters I through VII of the same. There are many proofs that have been omitted throughout this report. The reader is encouraged to read the source instead of these notes to gain a better understanding of the elements of Set Theory.
\end{abstract}

\tableofcontents

\chapter{The Zermelo-Fraenkel Axioms}
\section{Axioms of Set Theory}

We shall discuss Zermelo-Fraenkel Set Theory, which is a first order theory, with signature $\mathsf{ZF} = (\emptyset, \{\in\})$. That is, there are no function symbols and the only predicate is the ``belongs to'' relation.
\begin{enumerate}[label=\bfseries ZF\arabic*]
    \setcounter{enumi}{-1}
    \item (Nonempty Domain) There is at least one set.
    \begin{equation*}
        \exists x(x = x)
    \end{equation*}
    This axiom is redundunt since \textbf{ZF7} guarantees the existence of an infinite set and thus the domain of discourse must be nonempty.
    
    \item (Extensionality) Informally speaking, a set is determined uniquely by its elements.
    \begin{equation*}
        \forall x\forall y(\forall z(z\in x\Longleftrightarrow z\in y)\Longrightarrow x = y)
    \end{equation*}

    \item (Foundation/Regularity) This states that any nonempty set contains an element that is disjoint from it.
    \begin{equation*}
        \forall x\left[\exists y(y\in x)\Longrightarrow\exists y(y\in x\wedge\neg\exists z(z\in x\wedge z\in y))\right]
    \end{equation*}

    \item (Comprehension) Informally speaking, this axiom allows us to define sets in the set-builder notation. Let $\phi$ be a valid first order formula with free variables $w_1,\ldots,w_n,x,z$. Then 
    \begin{equation*}
        \forall z\forall w_1,\ldots,w_n\exists y\forall x\left(x\in y\Longleftrightarrow x\in z\wedge\phi\right)
    \end{equation*}
    Notice how this is the same as writing 
    \begin{equation*}
        y = \left\{x\in z\mid\phi\right\}
    \end{equation*}

    \item (Pairing) Informally, this states that given two sets $x$ and $y$, there is a set $z = \{x, y\}$.
    \begin{equation*}
        \forall x\forall y\exists z\forall w(w\in z\Longleftrightarrow(w = x\vee w = y))
    \end{equation*}

    \item (Union) This axiom allows us to take a union of a collection of sets.
    \begin{equation*}
        \forall\mathscr F\exists A\forall y(x\in y\wedge y\in\mathscr F\Longrightarrow x\in A)
    \end{equation*}

    \item (Replacement Scheme) Let $\phi$ be a valid formula without $Y$ as a free variable. Then, 
    \begin{equation*}
        \forall A(\forall x\in A~\exists!y~\phi(x,y)\Longrightarrow\exists Y~\forall x\in A~\exists y\in Y\phi(x,y))
    \end{equation*}
    Informally speaking, this allows us to replace the elements of a set to obtain a new set.

    \item (Infinity) There is an infinite inductive set.
    \begin{equation*}
        \exists x(\emptyset\in x\wedge\forall y\in x(S(y)\in x))
    \end{equation*}

    \item (Power Set) Every set has a set containing all its subsets. It is important to note that this need not be \textbf{the} power set.
    \begin{equation*}
        \forall x\exists y\forall z(z\subseteq x\Longrightarrow z\in y)
    \end{equation*}

    \item (Choice) Informally, given a collection of nonempty sets $X$, there is a choice function that chooses one element from each set in $X$.
    \begin{equation*}
        \forall X\left(\emptyset\notin X\implies\exists f: X\to\bigcup X,~\forall x\in X(f(x)\in x)\right).
    \end{equation*}
\end{enumerate}

We have been a bit sloppy in stating the axioms. Notice that our signature does not contain a predicate $\subseteq$ or the successor function $S$, neither do we know, a priori, of the existence of \textbf{the} empty set.

To define the formula $\subseteq(x,y)$, use 
\begin{equation*}
    \subseteq(x,y) := \forall z(z\in x\Longrightarrow z\in y)
\end{equation*}

As for the successor function, given any set $x$, using \textbf{ZF4}, there is a set $y = \{x\}$. Using \textbf{ZF5}, we may define $S(y) := x\cup y$. Finally, using \textbf{ZF0} and \textbf{ZF3}, we know of the existence of the empty set as 
\begin{equation*}
    \exists x(x = x\wedge\exists y\forall z(z\in x\Longleftrightarrow z\in y\wedge z\ne z))
\end{equation*}
Further, due to \textbf{ZF1}, the empty set is unique.

\section{Consequences of the Axioms}

\begin{theorem}
    There is no universal set. That is, 
    \begin{equation*}
        \neg\exists z\forall x(x\in z)
    \end{equation*}
\end{theorem}
\begin{proof}
If there were a universal set, then using \textbf{ZF3}, we may construct the set $y = \{x\in z\mid x\notin x\}$. Then, it is not hard to argue that 
\begin{equation*}
    y\in y\iff y\notin y,
\end{equation*}
a contradiction.
\end{proof}

\begin{definition}[Power Set]
    Let $x$ be a set. Due to \textbf{ZF8}, there is a set $z$ containing all the subsets of $x$. Using Comprehension, we may construct
    \begin{equation*}
        \mathscr P(x) := \{y\in z\mid y\subseteq x\}.
    \end{equation*}
    This is known as the \textbf{power set} of $x$.
\end{definition}

\begin{definition}
    Let $\mathscr F$ denote a set. Let $A$ be a set satisfying \textbf{ZF5}. Define 
    \begin{equation*}
        \bigcup\mathscr F := \{x\in A\mid\exists y\in\mathscr F(x\in y)\}
    \end{equation*}
    and 
    \begin{equation*}
        \bigcap\mathscr F := \{x\in A\mid\forall y\in\mathscr F(x\in y)\}.
    \end{equation*}
\end{definition}

\section{Relations, Functions and Well Ordering}

\begin{definition}[Ordered Pair]
    For sets $x,y$, define the ordered pair $\langle x,y\rangle$ by 
    \begin{equation*}
        \langle x,y\rangle := \{\{x\}, \{x,y\}\}.
    \end{equation*}
    The set on the right is constructed by using the pairing axiom twice.
\end{definition}

\begin{definition}[Cartesian Product]
    Let $A$ and $B$ be sets. Using Replacement, we may define, for each $y\in B$, 
    \begin{equation*}
        A\times\{y\} := \{z\mid\exists x\in A (z = \langle x,y\rangle)\}.
    \end{equation*}
    Again, by Replacement, define the set 
    \begin{equation*}
        \mathscr F := \{z\mid\exists y\in B(z = A\times\{y\})\}.
    \end{equation*}
    Finally, define 
    \begin{equation*}
        A\times B := \bigcup\mathscr F.
    \end{equation*}
\end{definition}

\begin{definition}[Relation, Function]
    Let $A$ be a set. A relation $R$ on $A$ is a subset of $A\times A$. Define the domain and range of a relation as 
    \begin{equation*}
        \dom(R) := \{x\in A\mid\exists y(\langle x, y\rangle\in R)\}\qquad\ran(R) := \{y\mid\exists x(\langle x, y\rangle\in R)\}.
    \end{equation*}
    We write $x R y$ to denote $\langle x,y\rangle\in R$.

    A relation $f$ is said to be a function if 
    \begin{equation*}
        \forall x\in\dom(f)\exists! y\in\ran(f)(\langle x,y\rangle\in f).
    \end{equation*}
    We use $f: A\to B$ to denote a function $f$ with $\dom(f) = A$ and $\ran(f)\subseteq B$.
\end{definition}

\begin{definition}[Total Ordering, Well Ordering]
    A \emph{total ordering} is a pair $\langle A, R\rangle$ where $A$ is a set and $R$ is a relation that is irreflexive, transitive and satisfies trichotomy. 

    We say $R$ \emph{well-orders} $A$ if $\langle A,R\rangle$ is a total ordering and every non empty subset of $A$ has an $R$-least element.
\end{definition}

We use $\pred(A,x,R)$ to denote the set $\{y\in A\mid y R x\}$.

\begin{lemma}
    Let $\langle A, R\rangle$ be a well-ordering. Then for all $x\in A$, $\langle A, R\rangle\not\cong\langle\pred(A,x,R), R\rangle$.
\end{lemma}
\begin{proof}
    Suppose $\langle A, R\rangle\cong\langle\pred(A,x,R), R\rangle$ and let $f: A\to \pred(A,x,R)$ be the order isomorphism. Let $x$ be the $R$-least element of the set 
    \begin{equation*}
        \{y\in A\mid f(y)\ne y\},
    \end{equation*}
    which obviously exists since the aforementioned set is nonempty. If $x R f(x)$, there is some $y\in A$ with $y R x$ and $f(y) = x\ne y$ a contradiction to the choice of $x$. On the other hand, if $f(x) R x$, then $f(f(x))\ne f(x)$ since $f$ is injective, a contradiction to the choice of $x$. This completes the proof.
\end{proof}

\begin{theorem}\thlabel{thm:well-orders-comparable}
    Let $\langle A, R\rangle$ and $\langle B, S\rangle$ be two well-orderings. Then exactly one of the following holds: 
    \begin{enumerate}[label=(\alph*)]
        \item $\langle A, R\rangle\cong\langle B, S\rangle$.
        \item $\exists y\in B\left(\langle A, R\rangle\cong\langle\pred(B,y,S), S\rangle\right)$.
        \item $\exists x\in A\left(\langle\pred(A,x,R), R\rangle\cong\langle B, S\rangle\right)$.
    \end{enumerate}
\end{theorem}
\begin{proof}
    Let 
    \begin{equation*}
        f := \{\langle v, w\rangle\mid v\in A,~w\in B,~\langle\pred(A,v,R), R\rangle\cong\langle\pred(B,w,S), S\rangle\}.
    \end{equation*}
    Due to the preceeding lemma, if $\langle v_1,w\rangle, \langle v_2,w\rangle\in f$, then $v_1 = v_2$. Similarly, if $\langle v,w_1\rangle, \langle v,w_2\rangle\in f$, then $w_1 = w_2$. Hence, $f$ is an injective function.

    It is not hard to argue that $f$ is an order isomorphism from an initial segment of $A$ to an initial segment of $B$. Both these segments may not be proper else we could find another isomorphism from an initial segment of $A$ to an initial segment of $B$ by extending one of the isomorphisms in $f$. This completes the proof.
\end{proof}

\chapter{Ordinal Numbers}
\section{Transitive Sets}

\begin{definition}
    A set $x$ is said to be \emph{transitive} if 
    \begin{equation*}
        \forall y\forall z(z\in y\wedge y\in x\implies z\in x).
    \end{equation*}
\end{definition}

\begin{proposition}
    A set $x$ is transitive if and only if 
    \begin{equation*}
        \forall y(y\in x\implies y\subseteq x).
    \end{equation*}
\end{proposition}
\begin{proof}
    Suppose $x$ is transitive and $y\in x$. Since for all $z\in y$, $z\in x$, we must have $y\subseteq x$. The converse is trivial.
\end{proof}

\begin{proposition}
    If $x$ is a transitive set, then so is $x\cup\{x\}$.
\end{proposition}
\begin{proof}
\end{proof}

\begin{proposition}
    If $x$ is a transitive set, then so is $\mathscr P(x)$.
\end{proposition}
\begin{proof}
\end{proof}

\begin{proposition}
    If $\mathscr F$ is a family of transitive sets, then so is $\bigcup\mathscr F$.
\end{proposition}
\begin{proof}
\end{proof}

\begin{proposition}\thlabel{prop:element-transitive}
    If $x$ is a transitive set, then so is every $z\in x$.
\end{proposition}
\begin{proof}
\end{proof}

\section{Ordinals}

\begin{definition}[Ordinal]
    A set $x$ is said to be an \emph{ordinal} if it is transitive and well ordered by $\in$. That is, the pair $\langle x,\in_x\rangle$ is a well ordering, where 
    \begin{equation*}
        \in_x := \{\langle v,w\rangle\in x\times x\mid v\in w\}.
    \end{equation*}
\end{definition}

\begin{theorem}[Properties of Ordinals]\thlabel{thm:properties-ordinals}\hfill
\begin{enumerate}[label=(\alph*)]
    \item If $x$ is an ordinal and $y\in x$, then $y$ is an ordinal and $y = \pred(x, y)$.
    \item If $x\cong y$ are ordinals, then $x = y$. 
    \item If $x,y$ are ordinals, then exactly one of the following is true: $x = y$, $x\in y$ or $y\in x$. 
    \item If $C$ is a nonempty set of ordinals, then $\exists x\in C~\forall y\in C(x\in y\vee x = y)$. That is, every nonempty set of ordinals has a minimum element.
\end{enumerate}
\end{theorem}
\begin{proof}
\begin{enumerate}[label=(\alph*)]
    \item Due to \thref{prop:element-transitive}, $y$ is a transitive and owing to it being the subset of a well ordered set, it is well ordered too, hence an ordinal.

    \item Let $f: x\to y$ be an isomorphism. Let 
    \begin{equation*}
        A := \{z\in x\mid f(z)\ne z\}.
    \end{equation*}
    Suppose $A$ is nonempty, then it has a least element, say $w\in x$. If $v\in w$, then $v = f(v)\in f(w)$ whence $w\subseteq f(w)$. On the other hand, if $v\in f(w)$, then there is some $u\in w$ such that $v = f(u) = u\in w$ and thus $f(w) = w$, a contradiction. 

    \item Follows from \thref{thm:well-orders-comparable}.

    \item First note that it suffices to find $x\in C$ with $x\cap C = \emptyset$ for if $y\in C$ is another ordinal with $x\ne y$, then $y\notin x$ lest $x\cap C\ne\emptyset$.

    Pick any $x\in C$. If $x\cap C = \emptyset$, then we are done. Else, let $x'\in x\cap C$ be the $\in$-least element. It is not hard to argue that $x'\cap C = \emptyset$ and we are done.\qedhere
\end{enumerate}
\end{proof}

\begin{lemma}
    If $A$ is a transitive set of ordinals, then $A$ is an ordinal.
\end{lemma}
\begin{proof}
    We must first show that the membership relation $\in_A$ is a linear order. This follows from \thref{thm:properties-ordinals} (c) and the fact that $A$ is a transitive set. Lastly, to see that $A$ is well ordered, simply invoke \thref{thm:properties-ordinals} (d).
\end{proof}

\begin{theorem}
    If $\langle A, R\rangle$ is a well ordering, then there is a \underline{unique} ordinal $C$ such that $\langle A, R\rangle\cong C$.
\end{theorem}
\begin{proof}
    Let 
    \begin{align*}
        B &:= \{a\in A\mid\exists x_a(x_a\text{ is an ordinal }\wedge\langle\pred(A,a,R), R\rangle\cong x_a)\},\\
        f &:= \{\langle b, x_b\rangle\mid b\in B\}.
    \end{align*}
    First, note that for all $b\in B$, $x_b$, since it exists must be unique and thus $f$ is a well defined function with $\dom(f) = B$.

    Let $C = \ran(f)$. We contend that $C$ is an ordinal. Let $y\in x\in C$ and $a\in B$ be such that $g:\pred(A,a,R)\to x$ is an isomorphism. Then, there is some $b\in\pred(A,a,R)$ with $g(b) = y$. It is not hard to see that the restriction $g:\pred(A,b,R)\to y$ is an isomorphism whence $y\in C$ and thus $C$ is an ordinal due to the preceeding lemma.

    The function $f: B\to C$ is obviously a surjection. We contend that it is an isomorphism. Indeed, let $a,b\in B$ with $a R b$ and $g: \pred(A,b,R)\to x_b$ be \emph{the} isomorphism. If $y = g(a)$, then the restriction $g:\pred(A,a,R)\to y$ is an isomorphism whence $f(a) = y\in x = f(b)$ and $f$ is an order isomorphism.

    Suppose $B\ne A$. Let $b\in A\backslash B$ be the $R$-least element. Then, $\pred(A,b,R)\subseteq B$. Now suppose $B\ne\pred(A,b,R)$, consequently, there is some $b'\in B\backslash\pred(A,b,R)$, then $bRb'$ and if there is an order isomorphism from $\pred(A,b',R)$ to some ordinal $x$, then there must be one from $\pred(A,b,R)$ as we have argued earlier, a contradiction. 
    
    Thus, either $B = A$ or $B = \pred(A,b,R)$ for some $b\in A$. In the latter case, the function $f$ is an order isomorphism between $\pred(A,b,R)$ and an ordinal $C$ whence $b\in B$, a contradiction. Thus $B = A$ and the proof is complete.
\end{proof}

\begin{definition}[Type of a Well Ordering]
    If $\langle A, R\rangle$ is a well ordering, then $\type(A,R)$ is the unique ordinal $C$ such that $\langle A, R\rangle\cong C$.
\end{definition}

Henceforth, we use Greek letters $\alpha,\beta,\gamma,\dots$ to vary over ordinals. That is, saying $\forall\alpha(\dots)$ is equivalent to saying $\forall x(x\text{ is an ordinal }\dots)$. Further, since the ordinals are well ordered, we write $\alpha < \beta$ to denote $\alpha\in\beta$ and similarly, $\alpha\le\beta$ means $\alpha\in\beta\vee\alpha = \beta$.

\begin{definition}
    Let $X$ be a set of ordinals. Define 
    \begin{equation*}
        \sup(X) := \bigcup X\quad\text{and}\quad\min(X) := \bigcap X.
    \end{equation*}
    Further, for an ordinal $\alpha$, let $S(\alpha)$ denote the set $\alpha\cup\{\alpha\}$.
\end{definition}

\begin{lemma}
\begin{enumerate}[label=(\alph*)]
    \item $\forall\alpha,\beta(\alpha\le\beta\iff\alpha\subseteq\beta)$.
    \item If $X$ is a set of ordinals, $\sup(X)$ is the least ordinal $\ge$ all elements of $X$ and if $X\ne\emptyset$, $\min(X)$ is the least ordinal in $X$.
\end{enumerate}
\end{lemma}
\begin{proof}
\begin{enumerate}[label=(\alph*)]
    \item The forward direction is obvious. Suppose $\alpha\subseteq\beta$. If $\alpha = \beta$, then we are done. If not, let $\gamma$ be the $<$-least element of $\beta\backslash\alpha$. We contend that $\gamma = \alpha$. Indeed, if $x\in\gamma$, then $x\notin\beta\backslash\alpha$ lest we contradict the minimality of $\gamma$ consequently, $x\in\alpha$ whence $\gamma\subseteq\alpha$. On the other hand, since $\alpha = \pred(\beta,\alpha)$, we have $\alpha\le\gamma$ and thus $\alpha\subseteq\gamma$. This shows that $\alpha = \gamma\in\beta$ and the conclusion follows.

    \item \qedhere
\end{enumerate}
\end{proof}

\begin{lemma}
    For an ordinal $\alpha$, $S(\alpha)$ is an ordinal, $\alpha < S(\alpha)$ and 
    \begin{equation*}
        \forall\beta(\beta < S(\alpha) \iff \beta\le\alpha).
    \end{equation*}
\end{lemma}

\begin{definition}[Successor, Limit Ordinal]
    An ordinal $\alpha$ is said to be a \emph{successor ordinal} if there is an ordinal $\beta$ such that $\alpha = S(\beta)$. On the other hand, $\alpha$ is said to be a \emph{limit ordinal} if $\alpha\ne\emptyset$ and $\alpha$ is not a successor ordinal.
\end{definition}

\section{Transfinite Induction and Recursion}

\subsection{Classes but informally}

Informally speaking, a class is any collection of the form 
\begin{equation*}
    \{x\mid \phi(x)\}
\end{equation*}
where $\phi(x)$ is a well defined first order formula. As we have seen earlier, the class 
\begin{equation*}
    \{x\mid x = x\}
\end{equation*}
is not a set. A \emph{proper class} is a class which is not a set. One uses boldface letters to denote classes. 

\begin{definition}
    Denote 
    \begin{equation*}
        \mathbf{V} := \{x\mid x = x\}\qquad\mathbf{ON} := \{x\mid x\text{ is an ordinal}\}.
    \end{equation*}
\end{definition}

To be completely formal, a class is simply a first order formula with one or more free variables. For example, the class of all ordinals can be thought of as the formula 
\begin{equation*}
    \ON(x) = x \text{ is an ordinal.}
\end{equation*}

We can extend this to define functions between classes $\mathbf A$ and $\mathbf B$. A function $\mathbf F:\mathbf A\to\mathbf B$ is given by a first order logic formula in two variables $\mathbf F(x,y)$ such that 
\begin{equation*}
    \forall x~\mathbf A(x)\implies\exists!y~\left(\mathbf B(y)\wedge\mathbf F(x,y)\right).
\end{equation*}

\begin{theorem}[Transfinite Induction on $\mathbf{ON}$]\thlabel{thm:transfinite-induction}
    If $\mathbf C\subseteq\ON$ and $\mathbf C\ne\emptyset$, then $\mathbf C$ has a least element.
\end{theorem}
\begin{proof}
    The proof is exactly like \thref{thm:properties-ordinals} (d).
\end{proof}

One must note that there is a significant difference between \thref{thm:properties-ordinals} (d) and \thref{thm:transfinite-induction}. The former is a single provable statement in \textsf{ZFC} while the latter is a theorem schema which represents an infinite collection of theorems. In particular, suppose the class $\mathbf C$ corresponded to a formula $\mathbf C(x,z_1,\dots,z_n)$, then \thref{thm:transfinite-induction} in this case says the following: 
\begin{align*}
    \forall z_1,\dots,z_n\big\{\left[\forall x(\mathbf C(x,z_1,\dots,z_n)\implies x\text{ is an ordinal})\wedge\exists x\mathbf C(x,z_1,\dots,z_n)\right]\\
    \implies\left[\exists x\left(\mathbf C(x,z_1,\dots,z_n)\wedge\forall y(\mathbf C(y,z_1,\dots,z_n)\implies y\ge x)\right)\right]\big\}.
\end{align*}

And \thref{thm:transfinite-induction} specifies one such formula for each well-formed sentence $\mathbf C$.

\begin{theorem}[Transfinite Recursion on $\ON$]\thlabel{thm:transfinite-recursion}
    If $\mathbf F: \V\to\V$, then there is a unique $\mathbf G:\ON\to\V$ such that 
    \begin{equation*}
        \forall\alpha\left(\mathbf G(\alpha) = \mathbf F(\mathbf G\restrict\alpha)\right).
    \end{equation*}
\end{theorem}
The formal restatement of the above in terms of first order logic is the following: 
\begin{equation*}
    \forall x\exists!y~\mathbf F(x,y)\implies\left[\forall\alpha\exists!y~\mathbf G(\alpha, y)\wedge\forall\alpha\exists x\exists y\left(\mathbf G(\alpha, y)\wedge\mathbf F(x,y)\wedge x = \mathbf G\restrict\alpha\right)\right]
\end{equation*}
where 
\begin{equation*}
    (x = \mathbf G\restrict\alpha) := \mathsf{function}(x)\wedge\dom(x) = \alpha\wedge\left(\forall\beta\in\dom(x)~\mathbf G(\beta,x(\beta))\right).
\end{equation*}
Similarly, one can encode the uniqueness condition.

\begin{proof}
    
\end{proof}

\section{Ordinal Arithmetic}
\subsection*{Addition}

\begin{definition}[Ordinal Addition]
    If $\alpha,\beta$ are ordinals, then define $\alpha + \beta = \type(\alpha\times\{0\}\cup\beta\times\{1\},R)$ where 
    \begin{align*}
        R = \left\{\langle\langle\xi,0\rangle,\langle\eta,0\rangle\rangle\mid \xi < \eta < \alpha\right\}\cup\left\{\langle\langle\xi,0\rangle,\langle\eta,1\rangle\rangle\mid\xi < \eta < \beta\right\}\cup\left[(\alpha\times\{0\})\times(\beta\times\{1\})\right].
    \end{align*}
\end{definition}

Informally speaking, we construct a new ordinal $\alpha + \beta$ by first ``placing'' $\alpha$ is a line and then placing $\beta$ after it linearly. This is best visualized when $\alpha$ and $\beta$ are finite ordinals.

To see that $R$ indeed gives $\alpha\times\{0\}\cup\beta\times\{1\}$ the structure of a well order, let $S$ be a nonempty subset. If $S\cap\alpha\times\{0\}$ is nonempty, then the minimal element of $S$ exists and is the minimal element of $S\cap\alpha\times\{0\}$. On the other hand, if $S\cap\alpha\times\{0\} = \emptyset$, the minimal element of $S$ exists and is the minimal element of $S\cap\beta\times\{1\}$. 

\begin{lemma}
    For ordinals $\alpha,\beta,\gamma$,
    \begin{enumerate}[label=(\alph*)]
        \item $\alpha + (\beta + \gamma) = (\alpha + \beta) + \gamma$.
        \item $\alpha + 0 = \alpha$.
        \item $\alpha + 1 = S(\alpha)$.
        \item $\alpha + S(\beta) = S(\alpha + \beta)$.
        \item If $\beta$ is a limit ordinal, then $\alpha + \beta = \sup\{\alpha + \xi\mid \xi < \beta\}$.
    \end{enumerate}
\end{lemma}
\begin{proof}
    We shall only prove (e) since the others are straightforward. First, note that $\alpha + \beta\ge\alpha + \xi$ for every $\xi < \beta$, which is easy to see by setting up an obvious order preserving injection. \todo{complete this argument}
\end{proof}

\begin{remark}
    One must note that ordinal addition is \textbf{not commutative}. Indeed, 
    \begin{equation*}
        1 + \omega = \sup\{1 + n\mid n < \omega\} = \omega
    \end{equation*}
    while 
    \begin{equation*}
        \omega + 1 = S(\omega)\ne\omega
    \end{equation*}
    where the last ``non-equality'' follows from the axiom of foundation. Thus, $1 + \omega\not\cong\omega + 1$.
\end{remark}

\subsection*{Multiplication}

\begin{definition}
    If $\alpha,\beta$ are ordinals, define $\alpha\cdot\beta = \type(\beta\times\alpha, R)$ where $R$ is the dictionary order, given by 
    \begin{equation*}
        R = \left\{\langle\langle\xi,\eta\rangle,\langle\xi',\eta'\rangle\rangle~\big\vert~\xi < \xi'\vee (\xi = \xi'\wedge\eta < \eta')\right\}.
    \end{equation*}
\end{definition}

We must first check that $R$ is indeed a well ordering. That it is a strict linear order is clear. Let $S\subseteq\beta\times\alpha$ be a nonempty subset. Let $S_1$ be the projection of $S$ onto $\beta$. This has a minimum element, say $\xi$. Consider now the set of all $\eta\in\alpha$ such that $\langle\xi,\eta\rangle\in S$. This is a nonempty subset of $\alpha$ and thus has a minimum element, say $\delta$. Then, $\langle\xi,\delta\rangle$ is a minimum element of $S$.

\begin{lemma}
    For ordinals $\alpha,\beta,\gamma$, 
    \begin{enumerate}[label=(\alph*)]
        \item $\alpha\cdot(\beta\cdot\gamma) = (\alpha\cdot\beta)\cdot\gamma$.
        \item $\alpha\cdot 0 = 0$.
        \item $\alpha\cdot 1 = \alpha$. 
        \item $\alpha\cdot S(\beta) = \alpha\cdot\beta + \alpha$. 
        \item If $\beta$ is a limit ordinal, then $\alpha\cdot\beta = \sup\{\alpha\cdot\xi\mid\xi < \beta\}$. 
        \item $\alpha\cdot(\beta + \gamma) = \alpha\cdot\beta + \alpha\cdot\gamma$.
    \end{enumerate}
\end{lemma}
\begin{proof}
    \todo{Proof of ordinal multiplication}
\end{proof}

\subsection*{Exponentiation}

\begin{definition}
    For ordinals $\alpha,\beta$, we define $\alpha^\beta$ by recursion on $\beta$ as 
    \begin{itemize}
        \item $\alpha^0 = 1$. 
        \item $\alpha^{\beta + 1} = \alpha^\beta\cdot\beta$. 
        \item If $\beta$ is a limit ordinal, $\alpha^\beta = \sup\{\alpha^\xi\mid\xi < \beta\}$. 
    \end{itemize}
\end{definition}

\begin{remark}
    Interestingly,
    \begin{equation*}
        2^\omega = \sup\{2^n\mid n < \omega\} = \omega.
    \end{equation*}
\end{remark}

\section{Equivalent forms of the Axiom of Choice}

\begin{theorem}[Well Ordering Theorem]\thlabel{thm:well-ordering}
    For every nonempty set $A$, there is a relation $R\subseteq A\times A$ such that $R$ well orders $A$.
\end{theorem}

\subsection*{AC \texorpdfstring{$\implies$}{} WO}

Let $A$ be a set. We shall explicitly construct a well ordering on $X$ using the Axiom of Choice. First, let $f:\calP(A)\backslash\{\emptyset\}\to A$ be a choice function and extend it to $f:\calP(A)\to A\coprod\{\emptyset\}$ by defining $f(\emptyset) = \emptyset$. We shall now use transfinite recursion to define a function $F$ on the ordinals as follows: 
\begin{align*}
    F(0) &:= f(A)\\ 
    F(\alpha) &:= f\left(\left\{x\in A\mid\forall\beta\in\alpha(F(\beta)\ne x)\right\}\right).
\end{align*}

First, note that if $F(\alpha) = F(\beta)\ne\emptyset$, then $\alpha = \beta$. Next, we contend that there must be an ordinal $\alpha$ with $F(\alpha) = \emptyset$. For if not, then we may apply the axiom of replacement and that of comprehension to obtain a set of all ordinals, a contradiction to the Burali-Forti paradox.

Let $\mathbf C$ denote the class of all ordinals $\alpha$ with $F(\alpha) = \emptyset$. Due to \thref{thm:transfinite-induction}, there is a minimal such ordinal, say $\alpha_0$, then 
\begin{equation*}
    f\left(\left\{x\in A\mid\forall\beta\in\alpha_0(F(\beta)\ne x)\right\}\right) = \emptyset\implies\left\{x\in A\mid\forall\beta\in\alpha_0(F(\beta)\ne x)\right\} = \emptyset.
\end{equation*}

Let $G: A\to\alpha_0$ denote the inverse function of $F$. Define the relation $R\subseteq A\times A$ by 
\begin{equation*}
    R := \{\langle x,y\rangle\mid G(x)\in G(y)\}.
\end{equation*}

That this is a well ordering is easy to see.

\subsection*{WO \texorpdfstring{$\implies$}{} AC}

This direction, on the other hand, is much easier. Let $X$ denote a collection of sets and let $Y = \bigcup X$. Let $R$ be a well ordering on $Y$. Define the function $f: X\to Y$ by $f(x) = \min(x)$, the $R$-least element, which can be chosen since $Y$ has been well ordered and $x\subseteq Y$.

\subsection*{AC \texorpdfstring{$\implies$}{} Zorn}

Let $X$ be a set and $P = (X,\leqq)$ be a poset on it such that every chain in $P$ has an upper bound. Let $f: \mathscr P(X)\backslash\{\emptyset\}\to X$ be a choice function. 

Suppose $P$ has no maximal element. Then, every chain in $P$ must have a strict upper bound. Let $\mathscr C$ be the set of all chains in $P$. Let $g: \mathscr C\to\mathscr P(X)$ map a chain in $P$ to the set of all \emph{strict} upper bounds. Consequently, $g(C)\ne\emptyset$ for every chain $C$ in $P$.

We shall define a class function $F:\ON\to X$ using transfinite recursion. Begin with $F(0) = F(X)$. Now, for any ordinal $\alpha\in\ON$, let $C_\alpha$ denote the chain $\{F(\beta)\mid \beta < \alpha\}$ and define 
\begin{equation*}
    F(\alpha) := f(g(C_\alpha)).
\end{equation*}

It is not hard to see that $F(\alpha) = F(\beta)$ if and only if $\alpha = \beta$ whence we may use Replacement to obtain a \emph{set} of all ordinals, which is absurd.

\subsection*{Zorn \texorpdfstring{$\implies$}{} AC}

Let $X$ be a collection of sets and $Y = \bigcup X$. Let $P$ be the poset of pairs $(S, f)$ where $S\subseteq X$ and $f: S\to Y$ is a function with $f(s)\in s$ for each $s\in S$. We say $(S, f)\leqq(S',f')$ if $S\subseteq S'$ and $f'\restrict_S = f$. 

Let $C = \{(S_\alpha, f_\alpha)\}$ be a chain in $P$. Define the function $f:S := \bigcup_{\alpha} S_\alpha\to Y$ by $f(x) := f_\alpha(x)$ if $x\in S_\alpha$. Then, $(S, f)$ is an upper bound for the chain $C$. Thus, due to Zorn's Lemma, $P$ contains a maximal element, say $(\wt S, F)$. We contend that $\wt S = X$. For if not, then there is $x\in X\backslash\wt S$ and the function $F$ can be extended to $\wt S\cup\{x\}$ by simply choosing an element of $x$ and assigning it to $x$ under $F$. This contradicts the maximality of $(\wt S, F)$ and hence, $F$ is the desired choice function.

\chapter{Cardinal Numbers}
\begin{definition}
    Sets $A$ and $B$ are said to be \emph{equinumerous} if there is a bijection $f: A\to B$. This is denoted by $A\approx B$. On the other hand, if there is an injection $f: A\to B$, it is denoted by $A\preceq B$. We write $A\prec B$ if $A\preceq B$ and $B\not\preceq A$.
\end{definition}

\begin{theorem}[Cantor-Schr\"oder-Bernstein]\thlabel{thm:cantor-schroder-bernstein}
    $A\preceq B\wedge B\preceq A\implies A\approx B$.
\end{theorem}

\begin{definition}
    For a set $A$, $|A|$ is the least $\alpha$ such that $\alpha\approx A$.
    $\alpha$ is a \emph{cardinal} if and only if $\alpha = |\alpha|$.
\end{definition}

From \thref{thm:well-ordering}, there is a well ordering $R$ on $A$ and thus an ordinal $\alpha$ with an order preserving bijection between $\langle A,R\rangle$ and $\alpha$, in particular, $A\approx\alpha$. Thus, $|A|$ is defined for every set. Further, note that $\alpha$ is a cardinal if and only if $\forall\beta < \alpha(\beta\not\approx\alpha)$ and for any ordinal $\alpha$, $|\alpha|\le\alpha$.

\begin{lemma}
    If $|\alpha|\le\beta\le\alpha$, then $|\beta| = |\alpha|$.
\end{lemma}
\begin{proof}
    Since $\beta\le\alpha$, we have $\beta\subseteq\alpha$ and thus $\beta\preceq\alpha$. On the other hand, $|\alpha|\subseteq\beta$. Composing this inclusion with the bijection $\alpha\approx|\alpha|$, we have $\alpha\preceq\beta$. We are done due to \thref{thm:cantor-schroder-bernstein}.
\end{proof}

\begin{lemma}
    If $n\in\omega$, then 
    \begin{enumerate}[label=(\alph*)]
        \item $n\not\approx n + 1$. 
        \item $\forall\alpha(\alpha\approx n\implies\alpha = n)$.
    \end{enumerate}
\end{lemma}
\begin{proof}
\begin{enumerate}[label=(\alph*)]
    \item Suppose not. Pick the smallest $n\in\omega$ such that $n\approx n + 1$. Note that $n\ne 0$. We have an injective function $f: n + 1\to n$. Composing appropriately, we may suppose that $f(n) = n - 1$ where $n\in n + 1$ and $n - 1\in n$. The restriction $f\restrict_n$ is an injective function from $n$ to $n - 1$ whence by \thref{thm:cantor-schroder-bernstein}, $n - 1\approx n$, a contradiction. 

    \item If $n < \alpha$, then $n + 1\le\alpha$ whence $n + 1\preceq\alpha$. On the other hand, $\alpha\approx n < n + 1$, consequently $\alpha\approx n + 1$, a contradiction to (a). 

    Now suppose $\alpha < n$. Then, $|n| = |\alpha|\le\alpha\le\alpha + 1\le n$, consequently, $|\alpha + 1| = |n|$. But since $\alpha + 1\approx n + 1$, we have $n + 1\approx n$, a contradiction to (a). Thus $\alpha = n$. \qedhere
\end{enumerate}
\end{proof}

\begin{corollary}
    $\omega$ is a cardinal and so is every ordinal $n < \omega$.
\end{corollary}

\begin{definition}
    $A$ is \emph{finite} if and only if $|A| < \omega$. $A$ is \emph{countable} if and only if $|A|\le\omega$. We use the shorthand \emph{infinite} to mean ``not finite'' and \emph{uncountable} to mean ``not countable''.
\end{definition}

\begin{definition}[Cardinal Arithmetic]
    For cardinals $\kappa$ and $\lambda$, define 
    \begin{equation*}
        \kappa\oplus\lambda := |\kappa\times\{0\}\cup\lambda\times\{1\}|,\quad\kappa\otimes\lambda := |\kappa\times\lambda|.
    \end{equation*}
\end{definition}

Unlike ordinal arithmetic, the operations $\oplus$ and $\otimes$ are commutative, which is obvious from the definition above. Furthermore, note that 
\begin{equation*}
    |\kappa + \lambda| = |\lambda + \kappa| = \kappa\oplus\lambda\quad\text{ and }\quad|\kappa\cdot\lambda| = |\lambda\cdot\kappa| = \kappa\otimes\lambda.
\end{equation*}

\begin{lemma}
    For $m,n\in\omega$, $n\oplus m = n + m < \omega$ and $n\otimes m = n\cdot m < \omega$.
\end{lemma}
\begin{proof}
    
\end{proof}

\begin{proposition}
    Every infinite cardinal is a limit ordinal.
\end{proposition}
\begin{proof}
    Suppose $\kappa = \alpha + 1$ is a cardinal. Then, $\alpha$ is not a finite ordinal, that is, $\omega < \alpha$ and thus there is an ordinal $\beta$ such that $\alpha = \omega + \beta$. Consequently, $1 + \alpha = 1 + \omega + \beta = \omega + \beta$ as we have seen previously that $1 + \omega = \omega$. Consequently, 
    \begin{equation*}
        |\kappa| = |\alpha + 1| = |1 + \alpha| = |\alpha|,
    \end{equation*}
    a contradiction to the fact that $\kappa$ is a cardinal.\todo{if $\alpha\le\beta$ there is an ordinal $\delta$ such that $\beta = \alpha + \delta$.}
\end{proof}

\begin{theorem}[Tarski]
    If $\kappa$ is an infinite cardinal, then $\kappa\otimes\kappa = \kappa$.
\end{theorem}
\begin{proof}
    We shall prove this statement by transfinite induction on $\kappa$. That this statement holds for $\kappa = \omega$ is well known. Suppose now that $\kappa > \omega$ and the statement holds for each cardinal $\lambda < \kappa$.

    Note that for an infinite ordinal $\alpha < \kappa$, we have $|\alpha| < \kappa$ and thus 
    \begin{equation*}
        |\alpha\times\alpha| = |\alpha|\otimes|\alpha| = |\alpha| < \kappa.
    \end{equation*}

    Let $\prec$ denote the strict lexicographic ordering on $\kappa\times\kappa$. Define the relation $\unlhd$ on $\kappa\times\kappa$ by $\langle\alpha,\beta\rangle\unlhd\langle\gamma,\delta\rangle$ if and only if 
    \begin{equation*}
        \max\{\alpha,\beta\} < \max\{\gamma,\delta\}\text{ or }\max\{\alpha,\beta\} = \max\{\gamma,\delta\}\text{ and }\langle\alpha,\beta\rangle\prec\langle\gamma,\delta\rangle.
    \end{equation*}
    That this relation is an ordering is immediate from the definition. We shall now show that this is a well ordering. Let $S\subseteq\kappa\times\kappa$ be nonempty. Using Replacement, construct the set $S'$ which consists of $\max\{\alpha,\beta\}$ for all $\langle\alpha,\beta\rangle\in S$. Since $S'\subseteq\kappa$, it contains a minimum element, say $\alpha_0$. Using Comprehension, construct the set $S''$ consisting of all pairs $\langle\alpha,\beta\rangle$ such that $\max\{\alpha,\beta\} = \alpha_0$. Now, $S''\subseteq\kappa\times\kappa$, and under the lexicographic order, it has a minimum element, which is also the minimum element of $S$ under the ordering $\unlhd$.

    Given any $\langle\alpha,\beta\rangle\in\kappa\times\kappa$, the set of all pairs preceeding it in $\langle\kappa\times\kappa,\unlhd\rangle$ is a subset of 
    \begin{equation*}
        (\max\{\alpha,\beta\} + 1)\times(\max\{\alpha,\beta\} + 1)
    \end{equation*}
    Since $\kappa$ is a limit ordinal, we have $\max\{\alpha,\beta\} + 1 < \kappa$ and due to the induction hypothesis, the cardinality of the above set is strictly smaller than $\kappa$ whence $|\kappa\times\kappa|\le\kappa$. There is an obvious injection from $\kappa$ into $\kappa\times\kappa$, forcing $|\kappa\times\kappa| = \kappa$ due to \thref{thm:cantor-schroder-bernstein}.
\end{proof}

\begin{corollary}
    Let $\kappa,\lambda$ be infinite cardinals. Then, 
    \begin{enumerate}[label=(\alph*)]
        \item $\kappa\oplus\lambda = \kappa\otimes\lambda = \max\{\kappa,\lambda\}$,
        \item $|\kappa^{<\omega}| = \kappa$.
    \end{enumerate}
\end{corollary}
\begin{proof}
    
\end{proof}

\begin{theorem}[Cantor]
    $\forall X\left(X\prec\mathscr P(X)\right)$.
\end{theorem}
\begin{proof}
    Suppose not, then $X\approx\mathscr P(X)$ for some $X$, which follows from \thref{thm:cantor-schroder-bernstein} and the fact that there is a canonical injection from $X$ to $\mathscr P(X)$. Let $f:X\to\mathscr P(X)\to X$ be a bijection. Using Comprehension, construct the set 
    \begin{equation*}
        S := \{x\in X\mid x\notin f(x)\}\subseteq X.
    \end{equation*}
    Let $s\in X$ be the unique element such that $f(s) = S$. Then, 
    \begin{equation*}
        s\in S\iff s\notin S,
    \end{equation*}
    a contradiction.
\end{proof}

\begin{theorem}
    $\forall\alpha\exists\kappa\left(\kappa > \alpha\text{ is a cardinal}\right)$ is true in \textsf{ZF}.
\end{theorem}
If we were to work in \textsf{ZFC} then we could just well order $\mathscr P(\alpha)$ and consider its cardinality.
\begin{proof}
    The statement is obvious for finite cardinals. Suppose now that $\alpha\ge\omega$. Let 
    \begin{align*}
        W := \{R\in\mathscr P(\alpha\times\alpha)\mid R \text{ well orders }\alpha\}
        S := \{\type(\langle\alpha, R\rangle)\mid R\in W\}.
    \end{align*}
    Let $\beta = \sup(S)$. We contend that $\beta$ is a cardinal and $\beta > \alpha$. First, note that if $\delta\in W$, then $S(\delta)\in W$, consequently, $\beta\notin W$. Further, $\beta\not\approx\alpha$ lest one could find a well ordering on $\alpha$ which is in order preserving bijection with $\beta$. Suppose $\beta$ were not a cardinal. Then, there is some $\gamma < \beta$ with $\gamma\approx\beta$. By definition, there is $\eta$ such that $\gamma\le\eta < \beta$ with $\eta\in W$, consequently, $\eta\approx\beta$ but $\alpha\approx\eta$, a contradiction. This completes the proof.
\end{proof}

\begin{definition}[Successor, Limit Cardinals]
    Let $\alpha$ be an ordinal. Denote by $\alpha^+$ the smallest \emph{cardinal} strictly greater than $\alpha$. A cardinal $\kappa$ is said to be a \emph{successor cardinal} if $\kappa = \alpha^+$ for some $\alpha$. On the other hand, if $\kappa > \omega$ and is not a successor cardinal, then $\kappa$ is said to be a \emph{limit cardinal}.
\end{definition}

\begin{definition}[Aleph Numbers]
    Define the numbers $\aleph_\alpha$ by transfinite recursion on $\alpha$.
    \begin{enumerate}[label=(\alph*)]
        \item $\aleph_0 := \omega$.
        \item $\aleph_{\alpha + 1} = (\aleph_\alpha)^+$.
        \item For a limit ordinal $\lambda$, define $\aleph_\lambda := \sup\{\aleph_\alpha\mid \alpha < \lambda\}$.
    \end{enumerate}
\end{definition}

\begin{theorem}
\begin{enumerate}[label=(\alph*)]
    \item Each $\aleph_\alpha$ is a cardinal.
    \item Every infinite cardinal is equal to $\aleph_\alpha$ for some $\alpha$.
    \item If $\alpha < \beta$, then $\aleph_\alpha < \aleph_\beta$. 
    \item $\aleph_\alpha$ is a limit cardinal ifa nd only if $\alpha$ is a limit ordinal. 
    \item $\aleph_\alpha$ is a successor cardinal if and only if $\alpha$ is a successor ordinal.
\end{enumerate}
\end{theorem}
\begin{proof}
    All of these follow immediately from the definition above.
\end{proof}

\begin{remark}
    One often writes $\omega_\alpha$ in place of $\aleph_\alpha$. We adopt both conventions and use them interchangeably.
\end{remark}

\begin{lemma}
    If there is a surjective function $f:X\to Y$, then $|Y|\le|X|$.
\end{lemma}
\begin{proof}
    Consider the set 
    \begin{equation*}
        S = \{f^{-1}(y)\mid y\in Y\},
    \end{equation*}
    which can be constructed using Replacement. Let $g: Y\to S$ be given by $g(y) = f^{-1}(y)$ and $F$ be a choice function on $S$. Then, the composition $F\circ g$ is an injective function from $Y$ to $X$, implying the desired conclusion.
\end{proof}

\begin{definition}[Cardinal Exponentiation]
    For sets $A$ and $B$, define 
    \begin{equation*}
        A^B := {}^BA := \{f\subseteq\mathscr P(B\times A)\mid f \text{ is a function}\}.
    \end{equation*}
    For cardinals $\kappa$ and $\lambda$, define $\kappa^\lambda := |{}^\lambda\kappa|$.
\end{definition}

\begin{theorem}
    Let $2\le\kappa\le\lambda$ and $\lambda$ an infinite cardinal. Then, $\kappa^\lambda = 2^\lambda$.
\end{theorem}
\begin{proof}
    Obviously, ${}^\lambda 2\approx\mathscr P(\lambda)$ which can be seen by looking at the characteristic function of each subset of $\lambda$. Then, we have 
    \begin{equation*}
        {}^\lambda k\preceq{}^\lambda\lambda\preceq\mathscr P(\lambda\times\lambda)\preceq\mathscr P(\lambda)\preceq{}^\lambda 2.
    \end{equation*}
    The conclusion follows from \thref{thm:cantor-schroder-bernstein}.
\end{proof}

\begin{theorem}
    Let $\mathscr B(\R)$ denote the Borel $\sigma$-algebra on $\R$ with the standard topology. Then, $|\mathscr B(\R)| = 2^{\aleph_0}$, the cardinality of the continuum.
\end{theorem}
\begin{proof}
    That $\mathscr B(\R)$ has cardinality at least that of the continuum is straightforward since it contains all singletons. Showing the reverse direction is a bit involved and requires transfinite recursion.

    First, note that $\R$ is second countable and thus has a countable abse for its topology, denote this by $S_0$. For an ordinal $\alpha < \omega_1$, let $S_{\alpha + 1}$ denote the collection of all unions of the form 
    \begin{equation*}
        \bigcup_{i} A_i\cup\bigcup_{j}(\R\backslash B_j)
    \end{equation*}
    where $A_i$ and $B_j$ are chosen from $S_\alpha$. Note that if $|S_\alpha|\le 2^{\aleph_0}$, then the number of these unions that can be formed is at most $(2^{\aleph_0})^{\aleph_0} = 2^{\aleph_0}$ since there is a surjection from the set of all functions $\aleph_0\to S_\alpha$ onto $S_{\alpha + 1}$. 

    On the other hand, if $\alpha$ is a limit ordinal, define 
    \begin{equation*}
        S_{\alpha} = \bigcup_{\lambda < \alpha} S_\lambda.
    \end{equation*}

    We contend that $S = \bigcup_{\alpha < \omega_1}S_\alpha$ is a $\sigma$-algebra. Obviously, $S$ contains $\emptyset$ and $\R$, and is closed under complementation. Let $\{A_n\}_{n = 1}^\infty$ be a sequence in $S$. For each positive integer $n$, let $\alpha(n)$ denote the minimal ordinal $\lambda$ such that $A_n\in S_{\lambda}$. Note that for each $n$, the cardinality $|\alpha(n)|\le\omega$. Hence, if $\beta = \sup_{n < \omega}\alpha(n)$, then $|\beta|\le\omega$, consequently, $\beta < \omega_1$ and $\{A_n\mid n < \omega\}\subseteq S_{\beta}$ implying that $\bigcup_{n = 1}^\infty A_n\in S_{\beta + 1}\subseteq S$. 

    As a result, $S$ contains $\mathscr B(\R)$ but the cardinality of $S$ is at most 
    \begin{equation*}
        |\omega_1|\otimes 2^{\aleph_0} = \aleph_1\otimes 2^{\aleph_0}\le 2^{\aleph_0}\otimes 2^{\aleph_0} = 2^{\aleph_0}.
    \end{equation*}
    This completes the proof.
\end{proof}

\begin{theorem}[Mazurkiewicz, 1914]
    There is a subset $A\subseteq\R^2$ which meets \emph{every} line in the plane at \emph{exactly} $2$ points.
\end{theorem}
\begin{proof}
    Let $\mathscr L$ denote the set of all possible lines in the plane. The cardinality of $\mathscr L$ is at least $2^{\aleph_0}$ and atmost $2^{\aleph_0}\otimes 2^{\aleph_0} = 2^{\aleph_0}$. Since this is in bijection with $2^{\aleph_0}$, it has an induced well ordering, which we denote by $\mathscr L = \{L_\alpha\mid\alpha < 2^{\aleph_0}\}$.

    We shall, using transfinite recursion construct a chain $X_\alpha$ of subsets of $\R^2$ for $\alpha < 2^{\aleph_0}$ such that $|X_\alpha| < 2^{\aleph_0}$ and $|X_\alpha\cap L_\beta|\le 2$ for each $\beta < 2^{\aleph_0}$. 

    Begin with $X_0 = \{x_0\}$ for any $x_0\in \R^2$. Suppose now that the sequence has been constructed for each $\beta < \alpha$ where $\alpha > 0$. Let $Y_\alpha := \bigcup_{\beta < \alpha} X_\beta$. Let $S_\alpha$ denote the set of all lines between two points in $Y_\alpha$. Note that the cardinality of $S_\alpha$ is strictly smaller than $2^{\aleph_0}$. 

    Let $\gamma$ be the smallest ordinal such that $|L_\gamma\cap Y_\alpha|\le 1$. If no such ordinal exists, then $Y_\alpha$ is the desired set. Suppose such a $\gamma$ does exist. Then, the set 
    \begin{equation*}
        L_\gamma\backslash\underbrace{\left(\bigcup_{L\in S_\alpha}L\cup\bigcup_{\beta < \gamma} L_\beta\cup Y_\alpha\right)}_{T}
    \end{equation*}
    which is non empty, since the intersection of $L_\gamma$ with $T$ has cardinality strictly smaller than $2^{\aleph_0}$. Let $x_\alpha$ be one such element in the above set and define $X_\alpha = Y_\alpha\cup\{x_\alpha\}$.

    It is not hard to see that $X_\alpha$ satisfies the desired properties and thus we may continue this procedure and obtain $\{X_\alpha\mid\alpha < 2^{\aleph_0}\}$. Let $X = \bigcup_{\alpha < 2{^\aleph_0}} X_\alpha$. This is the required set.
\end{proof}

\chapter{Well Founded Sets}
\textbf{\textcolor{red}{Throughout this chapter, we shall work in $\mathsf{ZF}^-$, which is $\mathsf{ZF}$ without the axiom of foundation.}}
\begin{definition}
    By transfinite recursion, deifne $R(\alpha)$ for each $\alpha\in\ON$ by 
    \begin{enumerate}[label=(\alph*)]
        \item $R(0) = \emptyset$, 
        \item $R(\alpha + 1) = \mathscr P(R(\alpha))$, 
        \item $R(\alpha) = \bigcup_{\lambda < \alpha} R(\lambda)$ when $\lambda$ is a limit ordinal.
    \end{enumerate}
    Finally, define the first order formula 
    \begin{equation*}
        \WF(x) := \exists\alpha(x\in R(\alpha)).
    \end{equation*}
    We denote by $\WF$ the class corresponding to the above formula.
\end{definition}

\begin{lemma}
    For each $\alpha$, 
    \begin{enumerate}
        \item $R(\alpha)$ is transitive. 
        \item $\forall\xi\le\alpha(R(\xi)\subseteq R(\alpha))$.
    \end{enumerate}
\end{lemma}
\begin{proof}
    We prove both statements by transfinite induction on $\alpha$. The base case with $\alpha = 0$ is trivial. Suppose $\alpha = \beta + 1$. Since $R(\beta)$ is transitive, so is its power set as we have seen earlier and obviously $R(\beta)\subseteq R(\alpha)$ since $R(\beta)\in R(\alpha)$. Finally, suppose $\alpha$ is a limit ordinal. Then, (b) is immediate and (a) follows from the fact taht the union of transitive sets is transitive.
\end{proof}

\begin{remark}
    As a consequence of the definition of $\WF$, for any $x\in\WF$, the least $\alpha$ for which $x\in R(\alpha)$ must be a \ul{successor ordinal}.
\end{remark}

\begin{definition}
    If $x\in\WF$, then $\rank(\alpha)$ is the \emph{least} $\beta$ such tdhat $x\in R(\beta + 1)$.
\end{definition}

\begin{lemma}
    For any $\alpha$, 
    \begin{equation*}
        R(\alpha) = \{x\in\WF\mid \rank(x) < \alpha\}.
    \end{equation*}
\end{lemma}
\begin{proof}
    Trivial.
\end{proof}

\begin{lemma}
    If $y\in\WF$, then 
    \begin{enumerate}[label=(\alph*)]
        \item $\forall x\in y(x\in\WF\wedge\rank(x) < \rank(y))$, and 
        \item $\rank(y) = \sup\{\rank(x) + 1\mid x\in y\}$.
    \end{enumerate}
\end{lemma}
\begin{proof}
    Let $\alpha = \rank(y)$. Then, $y\in R(\alpha + 1) = \mathscr P(R(\alpha))$ and thus $y\subseteq\mathscr R(\alpha)$, consequently, $x\in R(\alpha)$ and $\rank(x) < \alpha$.

    As for the second part, let $\alpha = \sup\{\rank(x) + 1\mid x\in y\}$. From (a), we know that $\alpha\le\rank(y)$. Further, each $x\in y$ has rank $< \alpha$ and thus $y\subseteq R(\alpha)$ whence $y\in R(\alpha + 1)$, consequently, $\rank(y)\le\alpha$.
\end{proof}

\begin{corollary}
    There is no $x\in\WF$ such that $x\in x$.
\end{corollary}
\begin{proof}
    If this were true, then $\rank(x) < \rank(x)$, a contradiction.
\end{proof}

\begin{lemma}
    \begin{enumerate}[label=(\alph*)]
        \item $\forall\alpha\in\ON(\alpha\in\WF\wedge\rank(\alpha) = \alpha)$.
        \item $\forall\alpha\in\ON(R(\alpha)\cap\ON = \alpha)$.
    \end{enumerate}
\end{lemma}
\begin{proof}
    We shall prove (a) using transfinite induction on $\alpha$. That (a) holds for $\alpha = 0$ is trivial. Now suppose (a) holds for each $\beta < \alpha$. Then, we have 
    \begin{equation*}
        \rank(\alpha) = \sup\{\rank(\beta) + 1\mid\beta < \alpha\} = \sup\{\beta\mid\beta < \alpha\} = \alpha
    \end{equation*}
    which proves (a). It is easy to see that (b) is immediate from (a).
\end{proof}

% \begin{lemma}
%     If $x\in\WF$, then 
% \end{lemma}

\begin{lemma}
    $\forall x(x\in\WF\iff x\subseteq\WF)$.
\end{lemma}
\begin{proof}
    The forward direction follows from the transitivity of $\WF$. As for the reverse direction, let $x\subseteq\WF$ and let 
    \begin{equation*}
        \alpha = \sup\{\rank(y) + 1\mid y\in x\}.
    \end{equation*}
    Then, $x\subseteq R(\alpha)$, consequently, $x\in R(\alpha + 1)$.
\end{proof}

\begin{lemma}
    \begin{enumerate}[label=(\alph*)]
        \item $\forall n\in\omega(|R(n)| < \omega)$.
        \item $|R(\omega)| = \omega$.
    \end{enumerate}
\end{lemma}
\begin{proof}
    (a) is immediate from induction on $n$. Obviously, $\omega\subseteq R(\omega)$. On the other hand, note that $R(\omega)$ is a countable union of countable sets and is thus countable.
\end{proof}

\section{Well Founded Relations}

\begin{definition}
    A relation $R$ is \emph{well-founded} on a set $A$ if 
    \begin{equation*}
        \forall X\subseteq A\left[X\ne\emptyset\implies\exists y\in X\left(\neg\exists z\in X\left(z R y\right)\right)\right].
    \end{equation*}
\end{definition}

For example, if $\langle A, R\rangle$ is a well-ordering, then $R$ is well-founded on $A$.

\begin{lemma}
    If $A\in\WF$, then $\in$ is well-founded on $A$.
\end{lemma}
\begin{proof}
    Let $X\subseteq A$ be nonempty and $\alpha = \min\{\rank(y)\mid y\in X\}$. Choose some $y\in X$ with $\rank(y) = \alpha$. Then $y$ is $\in$-minimal in $X$.
\end{proof}

\begin{lemma}
    If $A$ is transitive and $\in$ is well-founded on $A$, then $A\in\WF$.
\end{lemma}
\begin{proof}
    Suppose not. Then equivalently, $A\not\subseteq\WF$, whence $A\backslash\WF$ is nonempty. Let $y\in A\backslash\WF$ be the $\in$-least element of $A\backslash\WF$. If $z\in y$, then $z\in A$ due to the transitivity of $A$ but on the other hand, $z\notin A\backslash\WF$ lest one contradicts the minimality of $y$. Therefore, $z\in\WF$. Consequently, $y\subseteq\WF$ whence $y\in\WF$, a contradiction.
\end{proof}

\begin{definition}
    Let 
    \begin{equation*}
        \bigcup^0 A = A,
    \end{equation*}
    and for each $0 < n < \omega$, define, recursively, 
    \begin{equation*}
        \bigcup^{n + 1} A = \bigcup\left(\bigcup^n A\right).
    \end{equation*}
    Finally, set 
    \begin{equation*}
        \trcl(A) := \bigcup\left\{\bigcup^n A\mid n\in\omega\right\}.
    \end{equation*}
\end{definition}

\begin{lemma}
    Let $A$ be a set. Then, 
    \begin{enumerate}[label=(\alph*)]
        \item $A\subseteq\trcl(A)$.
        \item $\trcl(A)$ is transitive. 
        \item If $A\subseteq T$, and $T$ is transitive, then $\trcl(A)\subseteq T$. 
        \item If $A$ is transitive, then $\trcl(A) = A$. 
        \item $\trcl(A) = A\cup\bigcup\{\trcl(x)\mid x\in A\}$.
    \end{enumerate}
\end{lemma}
\begin{proof}
\begin{enumerate}[label=(\alph*)]
    \item Trivial. 
    \item If $x\in\trcl(A)$, then there is some $n$ such that $x\in\bigcup^n A$, therefore $x\subseteq\bigcup^{n + 1}A$, whence $x\subseteq\trcl(A)$. Thus $\trcl(A)$ is transitive.
    \item We shall show by induction on $n$ that $\bigcup^n A\subseteq T$. The base case is given to begin with. Suppose $\bigcup^n A\subseteq T$. Then, due to transitivity, $\bigcup^{n + 1}A = \bigcup\left(\bigcup^n A\right)\subseteq T$. The conclusion follows.
    \item Follows from (a) and (c) by taking $T = A$.
    \item First, note that if $x\in A$, then $x\in\trcl(A)$ and due to transitivity $x\subseteq\trcl(A)$. From (c), we have $\trcl(x)\subseteq\trcl(A)$. Let 
    \begin{equation*}
        T = A \cup\bigcup\{\trcl(x)\mid x\in A\}.
    \end{equation*}
    Then, it is easy to see that $T$ must be transitive and from what we concluded earlier, $T\subseteq\trcl(A)$ but since $A\subseteq T$, from (c), we must have $\trcl(A)\subseteq T$ whence $\trcl(A) = T$.\qedhere
\end{enumerate}
\end{proof}

\begin{theorem}
    For any set $A$, the following are equivalent: 
    \begin{enumerate}[label=(\alph*)]
        \item $A\in\WF$.
        \item $\trcl(A)\in\WF$.
        \item $\in$ is well-founded on $\trcl(A)$.
    \end{enumerate}
\end{theorem}
\begin{proof}
    
\end{proof}

\section{The Axiom of foundation}

Recall the axiom of foundation 
\begin{equation*}
    \forall x\left(x\ne\emptyset\implies\exists y\left(y\in x\wedge\neg\exists z\left(z\in x\wedge z\in y\right)\right)\right).
\end{equation*}
or equivalently, 
\begin{equation*}
    \forall x\left(x\ne\emptyset\implies\exists y\left(y\in x\wedge y\cap x = \emptyset\right)\right).
\end{equation*}

\begin{theorem}
    The following are equivalent:
    \begin{enumerate}[label=(\alph*)]
        \item the Axiom of Foundation. 
        \item $\forall A(\in\text{ is well-founded on }A)$
        \item $\V = \WF$.
    \end{enumerate}
\end{theorem}
\begin{proof}
    That (a) and (b) are equivalent is immediate from the definition of well-foundedness. Let $A\in\V$. Then, $\in$ is well founded on $A$ and thus on $\trcl(A)$, consequently, $A\in\WF$ and thus $\V = \WF$. The converse is trivial.
\end{proof}

\todo{Add picture of the universe}

\section{Induction and Recursion on Well-Founded Relations}

We extend the notion of well-foundedness to classes as follows. 
\begin{definition}
    $\mathbf R$ is well founded on $\mathbf A$ if and only if 
    \begin{equation*}
        \forall X\subseteq\mathbf A\left[X\ne\emptyset\implies\exists y\in X\left(\neg\exists z\in X(z\mathbf R y)\right)\right].
    \end{equation*}
\end{definition}

\begin{definition}
    $\mathbf R$ is \emph{set-like} on $\mathbf A$ if for all $x\in\mathbf A$, $\{y\in \mathbf A\mid y\mathbf R x\}$ is a set. If $\mathbf R$ is set-like on $\mathbf A$, then 
    \begin{enumerate}[label=(\alph*)]
        \item $\pred(\mathbf A, x,\mathbf R) = \{y\in\mathbf A\mid y\mathbf R x\}$. 
        \item $\pred^0(\mathbf A, x,\mathbf R) = \pred(\mathbf A, x, \mathbf R)$. 
        \item $\pred^{n + 1}(\mathbf A, x,\mathbf R) = \bigcup\{\pred(\mathbf A, y, \mathbf R)\mid y\in\pred^n(\mathbf A, x, \mathbf R)\}$. 
        \item $\operatorname{cl}(\mathbf A, x,\mathbf R) = \bigcup\{\pred^n(\mathbf A, x,\mathbf R)\mid n\in\omega\}$.
    \end{enumerate}
\end{definition}

\begin{lemma}
    If $\mathbf R$ is set-like on $A$ and $x\in\mathbf A$, then for each $y\in\Cl(\mathbf A, x,\mathbf R)$, $\pred(\mathbf A, y, \mathbf R)\subseteq\Cl(\mathbf A, x,\mathbf R)$.
\end{lemma}
\begin{proof}
    There is some nonnegative integer $n$ such that $y\in\pred^n(\mathbf A, y,\mathbf R)$. Then, $$\pred(\mathbf A, y, \mathbf R)\subseteq\pred^{n + 1}(\mathbf A, x,\mathbf R).$$ The conclusion follows.
\end{proof}

\begin{theorem}
    If $\mathbf R$ is well-founded and set-like on $\mathbf A$, then every non-empty subclass $\mathbf X$ of $\mathbf A$ has an $\mathbf R$-minimal element.
\end{theorem}
\begin{proof}
    Pick some $x\in\mathbf X$. If this $\mathbf R$-minimal, then we are done. If not, then consider $\mathbf X\cap\Cl(\mathbf A, x,\mathbf R)$ is a nonempty \emph{subset} of $A$, since $\Cl(\mathbf A, x, \mathbf R)$ is a set. This means that it has an $\mathbf R$-minimal element, say $y$. From the previous lemma, $y$ must be $\mathbf R$-minimal in $\mathbf X$.
\end{proof}

\begin{remark}
    Notice the similarity of the above with \thref{thm:transfinite-induction}. This in particular means that we can apply transfinite induction on well-founded set-like relations.
\end{remark}

\begin{theorem}[Well-Founded Transfinite Recursion]
    Assume $\mathbf R$ is well-founded and set-like on $\mathbf A$. If $\mathbf F: \mathbf A\times\mathbf V\to\mathbf V$, then there is a unique $\mathbf G:\mathbf A\to\mathbf V$ such that 
    \begin{equation*}
        \forall x\in\mathbf A\left[\mathbf G(x) = \mathbf F\left(x, \mathbf G\restrict\pred(\mathbf A, x,\mathbf R)\right)\right].
    \end{equation*}
\end{theorem}

\begin{definition}
    If $\mathbf R$ is well-founded and set-like on $\mathbf A$, define 
    \begin{equation*}
        \rank(x,\mathbf A,\mathbf R) = \sup\{\rank(y,\mathbf A,\mathbf R) + 1\mid y\mathbf R x\wedge y\in\mathbf A\}.
    \end{equation*}
\end{definition}

\begin{definition}\thlabel{def:mostowski-collapsing-function}
    Let $\mathbf R$ be well-founded and set-like on $\mathbf A$. Define the \emph{Mostowski collapsing function}, $\mathbf G$ of $\mathbf A, \mathbf R$ by 
    \begin{equation*}
        \mathbf G(x) = \{\mathbf G(y)\mid y\in\mathbf A\wedge y\mathbf R x\}.
    \end{equation*}
    The \emph{Mostowski collapse}, $\mathbf M$ of $\mathbf A,\mathbf R$ is defined to be the range of $\mathbf G$.
\end{definition}

\begin{definition}
    $\mathbf R$ is said to be \emph{extensional} on $\mathbf A$ if 
    \begin{equation*}
        \forall x,y\in\mathbf A\left(\forall z\in\mathbf A\left(z\mathbf R x\iff z\mathbf R y\right)\implies x = y\right).
    \end{equation*}
    Informally, this is equivalent to saying that the Axiom of Extensionality is true in $\mathbf A$ if $\in$ is interpreted as $\mathbf R$.
\end{definition}

\begin{theorem}[Mostowski Collapsing Theorem]\thlabel{thm:mostowski-collapsing}
    Suppose $\mathbf R$ is well-founded, set-like, and extensional on $\mathbf A$, then there is a transitive class $\mathbf M$ and a bijective map $\mathbf G: \mathbf A\to\mathbf R$ such that $\mathbf G$ is an isomorphism between $(\mathbf A,\mathbf R)$ and $(\mathbf M,\in)$. Furthermore, $\mathbf M$ and $\mathbf G$ are unique.
\end{theorem}

\chapter{Easy Consistency Proofs}
\section{Relativization}

\newcommand{\M}{\mathbf{M}}
\newcommand{\Con}{\operatorname{Con}}

\begin{definition}
    Let $\mathbf M$ be any class. For any formula $\phi$, define $\phi^{\M}$, the \emph{relativization} of $\phi$ to $\M$ by induction on $\phi$, by 
    \begin{enumerate}[label=(\alph*)]
        \item $(x = y)^{\M}$ is $x = y$. 
        \item $(x\in y)^{\M}$ is $x\in y$. 
        \item $(\phi\wedge\psi)^{\M}$ is $\phi^{\M}\wedge\psi^\M$.
        \item $(\neg\phi)^\M$ is $\neg\phi^\M$ 
        \item $(\exists\phi)^\M$ is $\exists x(x\in\M\wedge\phi^\M)$.
    \end{enumerate}
\end{definition}

\begin{lemma}
    Let $S$ and $T$ be two sets of sentences in the language of set theory under consideration, and suppose for some class $\M$, we can prove from $T$ that $\M\ne 0$ and $\M$ is a model for $S$. Then, $\Con(T)\implies\Con(S)$.
\end{lemma}
\begin{proof}
    Quite straightforward. Suppose $S$ were inconsistent. Then, there is a sentence $\chi$ such that we could prove $\chi\wedge\neg\chi$ from $S$. Then, we could begin a formal proof from $T$ and prove that $S$ is true in $\M$ and hence, $\chi^\M\wedge\neg\chi^\M$, a contradiction. Thus, $T$ is inconsistent.
\end{proof}

Recall the Axiom of Extensionality, 
\begin{equation*}
    \forall x,y\left(\forall z(z\in x\iff z\in y)\implies x = y\right).
\end{equation*}
Relativized to $\M$, it looks like 
\begin{equation*}
    \forall x,y\in\M\left(\forall z\in\M(z\in x\iff z\in y)\implies x = y\right).
\end{equation*}

\begin{lemma}
    If $\M$ is transitive, the Axiom of Extensionality is true in $M$.
\end{lemma}
\begin{proof}
    We have seen this in the previous chapter.
\end{proof}

\begin{lemma}
    If for each formula $\phi(x,z,w_1,\dots,w_n)$ with only the displayed variables free, 
    \begin{equation*}
        \forall z,w_1,\dots,w_n\in\M\left(\{x\in z\mid \phi^\M(x,z,w_1,\dots,w_n)\in\M\}\right),
    \end{equation*}
    then the Axiom of Comprehension is true in $\M$.
\end{lemma}
\begin{proof}
    Since the relativized instance of Comprehension is given by 
    \begin{equation*}
        \forall z,w_1,\dots,w_n\in\M\exists y\in\M\left(x\in y\iff x\in z\wedge\phi^\M(x,z,w_1,\dots,w_n)\right).
    \end{equation*}
    The conclusion is now obvious from the hypothesis.
\end{proof}
\begin{remark}
    In particular, if $\forall z\in\M(\mathscr P(z)\subseteq\M)$, then the Comprehension Axiom is true in $\M$. This shall turn out to be useful later on.
\end{remark}

\begin{theorem}
    Assume the consistency of $\mathsf{ZF}^-$. If $\M = \{0\}$, then Extensionality + Comprehension + $\forall y(y = 0)$ is consistent.
\end{theorem}
\begin{proof}
    Extensionality is true in $\M$ since it is transitive while Comprehension is true in $\M$ from the preceeding remark. 
\end{proof}

The above can be written as 
\begin{equation*}
    \Con(\mathsf{ZF}^-)\implies\Con\left(\mathsf{Extensionality} + \mathsf{Comprehension} + \forall y(y = 0)\right).
\end{equation*}

Let us now look at the Power Set Axiom. Recall that this is 
\begin{equation*}
    \forall x\exists y\forall z(z\subseteq x\implies z\in y).
\end{equation*}
We shall first see what $z\subseteq x$ becomes upon relativizing to $\M$. Note that $z\subseteq x$ is shorthand for $\forall w(w\in z\implies w\in x)$. This relativized to $\M$ is $\forall w\in\M(w\in z\implies w\in x)$ which is equivalent to writing $z\cap\M\subseteq x$. We may now write down the relativized Power Set Axiom as 
\begin{equation*}
    \forall x\in\M\exists y\in\M\forall z\in\M(z\cap\M\subseteq x\implies z\in y).
\end{equation*}

Suppose $\M$ is transitive. Then, $z\in\M$ implies $z\subseteq\M$ whence $z\cap\M = z$ nad hence, the Power Set Axiom holds in $\M$ if and only if 
\begin{equation*}
    \forall x\in\M\exists y\in\M\forall z\in\M(z\subseteq x\implies z\in y).
\end{equation*}

In particular, we have the following. 
\begin{lemma}
    If $\M$ is transitive, the Power Set Axiom holds in $\M$ if and only if 
    \begin{equation*}
        \forall x\in\M\exists y\in\M(\mathscr P(x)\cap\M\subseteq y).
    \end{equation*}
\end{lemma}

\begin{lemma}
    If 
    \begin{equation*}
        \forall x,y\in\M\exists z\in\M(x\in z\wedge y\in z),\quad\text{and}\quad\forall x\in\M\exists z\in\M(\bigcup x\subseteq z),
    \end{equation*}
    then the Pairing and Union Axioms are true in $\M$.
\end{lemma}
\begin{proof}
    Obvious.
\end{proof}

In particular, the Pairing and Union Axioms are true in $R(\omega)$ and $\WF$, which is trivial to see. 

Next, we shall show that Replacement is also true in $R(\omega)$ and $\WF$. First, a lemma.
\begin{lemma}
    Suppose for each formula $\phi(x,y,A,w_1,\dots,w_n)$ and each $A,w_1,\dots,w_n\in\M$, if 
    \begin{equation*}
        \forall x\in A\exists! y\in\M~\phi^{\M}(x,y,A,w_1,\dots,w_n),
    \end{equation*}
    then 
    \begin{equation*}
        \exists Y\in\M\left(\{y\mid\exists x\in A~\phi^{\M}(x,y, A, w_1,\dots,w_n)\}\subseteq Y\right).
    \end{equation*}
    Then the Replacement Scheme is true in $\M$.
\end{lemma}
\begin{proof}
    Again, obvious.
\end{proof}

Consider now the case $\M = R(\omega)$ or $\WF$ and let $Y$ be as defined in the lemma above. If $\M = \WF$, $Y\subseteq\WF$ and thus, $Y\in\WF$. On the other hand, if $\M = R(\omega)$, then $|Y|$ is finte and hence, $Y\subseteq R(n)$ for some $n$, whence $Y\in R(n + 1)$. This shows that replacement is true. 

We now move onto foundation. The relativization of foundation to $\M$ is 
\begin{equation*}
    \forall x\in\M\left(\exists y\in\M(y\in x)\implies\exists y\in\M(y\in x\wedge\ne\exists z\in\M(z\in x\wedge z\in y))\right).
\end{equation*}

Now, if $\M\subseteq\WF$ and $x\in\M$, pick $y\in\M\cap x$ with the smallest rank. It is routine to show that this satisfies the statement of the relativized axiom. This shows that Foundation is true in any $\M\subseteq\WF$. We now have the following. 

\begin{theorem}
    $\WF$ and $R(\omega)$ are models of $\mathsf{ZF} - \mathsf{Inf}$.
\end{theorem}

\section{Absoluteness}

\begin{definition}
    Let $\phi$ be a formula with $x_1,\dots,x_n$ free. 
    \begin{enumerate}[label=(\alph*)]
        \item If $\M\subseteq\mathbf N$, then $\phi$ is \emph{absolute} for $\mathbf M$, $\mathbf N$ if and only if 
        \begin{equation*}
            \forall x_1,\dots,x_n\in\mathbf M\left(\phi^\M(x_1,\dots,x_n)\iff\phi^{\mathbf N}(x_1,\dots,x_n)\right).
        \end{equation*}
        \item $\phi$ is said to be \emph{absolute} for $\M$ if it is absolute for $\mathbf M, \mathbf V$.
    \end{enumerate}
    Obviously, if $\phi$ is absolute for $\M$ and $\mathbf N$, and if $\mathbf M\subseteq\mathbf N$, then $\phi$ is absolute for $\M,\mathbf N$.
\end{definition}

\begin{remark}
    If $\M\subseteq\mathbf N$ and $\phi,\psi$ are both absolute for $\mathbf M, \mathbf N$, so are $\neg\phi$ and $\phi\wedge\psi$. This is trivial to prove. Further, note that $x = y$ and $x\in y$ are absolute for all $\mathbf M$, and any formula without quantifiers is built using the aforementioned atomic formulae and $\neg$ and $\wedge$. Therefore, a quantifier-free formula is absolute for any $\mathbf M$.
\end{remark}

\begin{lemma}
    If $\M\subseteq\mathbf N$ are both transitive and $\phi$ is absolute for $\M,\mathbf N$, then so is $\exists x\in y\phi$.
\end{lemma}
\begin{proof}
    Trivial.
\end{proof}

\begin{definition}
    The collection $\Delta_0$ of formulas are those built up inductively by: 
    \begin{enumerate}[label=(\alph*)]
        \item $x\in y$ and $x = y$ are $\Delta_0$.
        \item If $\phi,\psi$ are $\Delta_0$, so are $\neg\phi$ and $\phi\wedge\psi$. 
        \item If $\phi$ is $\Delta_0$, so is $\exists x(x\in y \wedge\phi)$.
    \end{enumerate}
\end{definition}

\begin{corollary}
    If $\M$ is transitive and $\phi$ is $\Delta_0$, then $\phi$ is absolute for $\M$.
\end{corollary}
\begin{proof}
    This follows from the previous lemma applied to $\M, \V$ and the above definition.
\end{proof}

\begin{lemma}
    Suppose $\M\subseteq\mathbf N$ are models for a set of sentences $S$, such that 
    \begin{equation*}
        S\vdash\forall x_1,\dots,x_n\left(\phi(x_1,\dots,x_n)\iff\psi(x_1,\dots,x_n)\right)
    \end{equation*}
    then $\phi$ is absolute for $\M,\mathbf N$ if and only if $\psi$ is.
\end{lemma}
\begin{proof}
    Follows immediately from the definition of absoluteness.
\end{proof}

Note that a function $F(x_1,\dots,x_n)$ can be defined through a formula in $\M$ if the following is true: 
\begin{equation*}
    \forall x_1,\dots,x_n\exists! y~\phi(x_1,\dots,x_n,y).
\end{equation*}

\begin{definition}
    If $\M\subseteq\mathbf N$ and $F(x_1,\dots,x_n)$ is a defined function, we say that $F$ is \emph{absolute} for $\M,\mathbf N$ if $\phi$ is.
\end{definition}

\begin{remark}
    There is now a tedious verification that almost all the operations that we use in set theory are absolute for a transitive model $\M$ that is a model for $\mathsf{ZF}^- - \mathsf P - \mathsf{Inf}$.
\end{remark}

\begin{lemma}
    Let $\M$ be a transitive model for $\mathsf{ZF}^- - \mathsf P - \mathsf{Inf}$. If $\omega\in\M$, then the Axiom of Infinity is true in $\M$. 
\end{lemma}
\begin{proof}
    Due to the remark above, $0$ and $S$ are absolute in $\M$. The Axiom of Infinity relativized to $\M$ now looks like 
    \begin{equation*}
        \exists x\in\M(0\in x\wedge\forall y\in x(S(y)\in x)),
    \end{equation*}
    which is seen to be true by taking $x = \omega$.
\end{proof}

The same argument yields that the Axiom of Infinity is false in $R(\omega)$, since any $x\in\WF$ containing $0$ and closed under $S$ has infinite rank. Thus, we have:

\begin{theorem}
    $R(\omega)$ is a model of $\mathsf{ZFC} - \mathsf{Inf} + (\neg\mathsf{Inf})$.
\end{theorem}
\begin{proof}
    It only remains to check that AC holds, for which we need to come up with a well-ordering for any $A\in R(\omega)$. But $A$ is finite and thus can be well ordered. Le $R\subseteq A\times A$ be a well-order. Then, $R$ must lie in $R(\omega)$ due to transitivity. The proof is complete due to the following lemma.
\end{proof}

\begin{lemma}
    Suppose $\M$ is a transitive model of $\mathsf{ZF}^- - \mathsf P - \mathsf{Inf}$. Let $A, R\in\M$ and suppose that $R$ well-orders $A$. Then $(R\text{ well-orders }A)^\M$.
\end{lemma}

\section{\texorpdfstring{$H(\kappa)$}{H(k)}}

\begin{definition}
    For an infinite cardinal $\kappa$, $H(\kappa) = \{x\mid|\trcl(x)| < \kappa\}$. Note that choice is not required for this definition, since we implicitly assume that $|y| < \kappa$ means that $y$ is well-orderable and $|y| < \kappa$.
\end{definition}

That each $H(\kappa)$ is a set and not a proper class follows from the following: 
\begin{theorem}
    For any infinite $\kappa$, $H(\kappa)\subseteq R(\kappa)$.
\end{theorem}

\chapter{Forcing}
\section{Some Infinitary Combinatorics}

\begin{definition}[Cofinal]
    If $f:\alpha\to\beta$, then $f$ maps $\alpha$ \emph{cofinally} if and only if $\ran(f)$ is unbounded in $\beta$. The \emph{cofinality} of $\beta$ is the least $\alpha$ that can be cofinally mapped into $\beta$.
\end{definition}

\begin{lemma}
    There is a cofinal map $f:\cf(\beta)\to\beta$ which is strictly increasing.
\end{lemma}
\begin{proof}
    Let $\alpha = \cf(\beta)$ and $f: \alpha\to\beta$ be a cofinal map. Define the map $g:\alpha\to\beta$ by 
    \begin{equation*}
        g(\eta) = max\{f(\eta), \sup\{f(\xi) + 1\colon \xi < \eta\}\}.
    \end{equation*}
    It is not hard to argue that $g:\alpha\to\beta$ is cofinal.
\end{proof}

\begin{lemma}
    If $\alpha$ is a limit ordinal and $f:\alpha\to\beta$ is a strictly increasing cofinal map, then $\cf(\alpha) = \cf(\beta)$.
\end{lemma}
\begin{proof}
    Upon composing $f$ with a strictly increasing cofinal map $\cf(\alpha)\to\alpha$, we see that $\cf(\beta)\le\cf(\alpha)$. Next, we would like to show that $\cf(\alpha)\le\cf(\beta)$. Let $g:\cf(\beta)\to\beta$ be a strictly increasing cofinal map. Define $h:\cf(\beta)\to\alpha$ by 
    \begin{equation*}
        h(\eta) = \inf_{\xi\in\alpha} f(\xi) > g(\eta).
    \end{equation*}
    It is not hard to see that $h$ is a cofinal map. Therefore, $\cf(\alpha)\le\cf(\beta)$ and this completes the proof.
\end{proof}

\begin{definition}
    $\beta$ is said to be \emph{regular} if $\beta$ is a limit ordinal and $\cf(\beta) = \beta$.
\end{definition}

\begin{lemma}
    If $\beta$ is regular, then $\beta$ is a cardinal.
\end{lemma}
\begin{proof}
    Cofinality is a cardinal.
\end{proof}

\begin{lemma}
    All successor cardinals are regular.
\end{lemma}
\begin{proof}
    Suppose there is a strictly increasing cofinal map $f:\kappa\to\kappa^+$. For each $\alpha\in\kappa$, note that $|f(\alpha)|\le\kappa$ and due to cofinality, $\kappa^+ = \bigcup_{\alpha\in\kappa} f(\alpha)$. This is a cardinality contradiction.
\end{proof}

\begin{lemma}[$\Delta$-system Lemma]
    Let $\mathscr A$ be an uncountable collection of finite sets. Then, there is an uncountable subcollection $\mathscr B$ of $\mathscr A$ and a set $r$ such that for all $x,y\in\mathscr B$ with $x\ne y$, $x\cap y = r$.
\end{lemma}
\begin{proof}
    There is a minimum cardinal $n < \omega$ such that there are uncountably many elements of $\mathscr A$ with cardinality $n$. Let $\mathscr A'$ denote the subcollection of these elements. We shall induct on the aforementioned unique $n$ for $\mathscr A$. 
    
    If $n = 1$, then there is nothing to prove since the elements of $\mathscr A'$ are disjoint. Now, suppose $n > 1$. If there is some $a$ that is in uncountably many elements of $\mathscr A'$, then consider $\mathscr A'' = \{x\backslash\{a\}\colon a\in x\in\mathscr A'\}$. We may now apply the induction hypothesis to find a subcollection $\mathscr B''$ of $\mathscr A''$ with the required property. Consider then the collection $\mathscr B = \{x\cup\{a\}\colon x\in\mathscr B''\}$.

    On the other hand, suppose there is no such $a$ that is in uncountably many elements of $\mathscr A'$. Begin with any $x_0\in\mathscr A'$. Let $\alpha < \omega_1$ be an ordinal and suppose the sequence $(x_i)_{i < \alpha}$ has been constructed. The union $x = \bigcup_{i < \alpha} x_i$ is a countable set (being the countable union of finite sets). Now, every element of $x$ is in countably many elements of $\mathscr A'$, consequently, there are uncountably many elements of $\mathscr A'$ that are disjoint from $x$. Define $x_\alpha$ to be any one of these and continue this process. 

    The sequence $(x_i)_{i < \omega_1}$ is now an uncountable collection of pairwise disjoint elements of $\mathscr A'$ and is our desired $\mathscr B$ with $r = \emptyset$.
\end{proof}

\begin{definition}
    Let $(\bbP, \le)$ be a partial order. Elements $p,q\in\bbP$ are said to be \emph{compatible} if there is an $r$ with $r\le p$ and $r\le q$. An \emph{antichain} is a collection of pairwise incompatible elements in $\bbP$.
\end{definition}

\section{Some Technicalities about Forcing}

Throughout this section, let $\bbP$ be a poset in $M$, a countable transitive model for $\mathsf{ZFC}$.


\begin{definition}[Forcing]
    Let $p\in\bbP$ and $\tau_1,\dots,\tau_n\in M^\bbP$. Let $\phi(x_1,\dots,x_n)$ be a formula with all free variables shown. Then, we write $p\Vdash\phi(\tau_1,\dots,\tau_n)$ if 
    \begin{equation*}
        \forall G\subseteq\bbP\left((G\text{ is generic}\wedge p\in G)\implies\phi^{M[G]}(\val(\tau_1, G),\dots,\val(\tau_n, G))\right).
    \end{equation*}
\end{definition}

\begin{definition}
    If $E\subseteq\bbP$ and $p\in\bbP$, then $E$ is said to be \emph{dense below} $p$ if $\forall q\le p\left(\exists r\in E(r\le q)\right)$.
\end{definition}

We now come to the ugliest definition of this article.

\begin{definition}
    Let $\phi(x_1,\dots,x_n)$ be a formula with all free variables shown, $p\in\bbP$ and $\tau_1,\dots,\tau_n\in V^\bbP$.
    \begin{itemize}
        \item $p\Vdash^*\tau_1 = \tau_2$ if and only if for all $\langle\pi_1,s_1\rangle\in\tau_1$,
        \begin{equation*}
            \{q\le p\colon q\le s_1\implies\exists\langle\pi_2,s_2\rangle\in\tau_2\left(q\le s_2\wedge q\Vdash^*\pi_1 = \pi_2\right)\}
        \end{equation*}
        is dense below $p$. And for all $\langle\pi_2,s_2\rangle\in\tau_2$, 
        \begin{equation*}
            \{q\le p\colon q\le s_2\implies\exists\langle\pi_1,s_1\rangle\in\tau_1\left(q\le s_1\wedge q\Vdash^*\pi_1 = \pi_2\right)\}
        \end{equation*}
        is dense below $p$.

        \item $p\Vdash^*\tau_1\in\tau_2$ if and only if 
        \begin{equation*}
            \{q\colon\exists\langle\pi,s\rangle\in\tau_2\left(q\le s\wedge q\Vdash^*\pi = \tau_1\right)\}
        \end{equation*}
        is dense below $p$.

        \item $p\Vdash^*\left(\phi(\tau_1,\dots,\tau_n)\wedge\psi(\tau_1,\dots,\tau_n)\right)$ if and only if 
        \begin{equation*}
            p\Vdash^*\phi(\tau_1,\dots,\tau_n)\text{ and }p\Vdash^*\psi(\tau_1,\dots,\tau_n).
        \end{equation*}

        \item $p\Vdash^*\neg\phi(\tau_1,\dots,\tau_n)$ if and only if there is no $q\le p$ such that $q\Vdash^*\phi(\tau_1,\dots,\tau_n)$.

        \item $p\Vdash^*\exists x\phi(x,\tau_1,\dots,\tau_n)$ if and only if 
        \begin{equation*}
            \{r\colon\exists\sigma\in V^\bbP\left(r\Vdash^*\phi(\sigma,\tau_1,\dots,\tau_n)\right)\}
        \end{equation*}
        is dense below $p$.
    \end{itemize}
\end{definition}

\begin{lemma}
    Let $phi(x_1,\dots,x_n)$ be a formula with all free variables shown and $\tau_1,\dots,\tau_n\in M^\bbP$. 
    \begin{enumerate}
        \item If $p\in G$ and $(p\Vdash^*\phi(\tau_1,\dots,\tau_n))^{M}$, then $\left(\phi(\val(\tau_1, G),\dots,\val(\tau_n, G))\right)^{M[G]}$.
        
        \item If $\phi(\val(\tau_1, G),\dots,\val(\tau_n, G))^{M[G]}$, then $\exists p\in G(p\Vdash^*\phi(\tau_1,\dots,\tau_n))^{M}$.
    \end{enumerate}
\end{lemma}
\begin{proof}
    Omitted due to length.
\end{proof}

\begin{theorem}
    Let $\phi(x_1,\dots,x_n)$ be a formula with all free variables shown and $\tau_1,\dots,\tau_n\in M^\bbP$. Then, 
    \begin{enumerate}
        \item for all $p\in\bbP$, 
        \begin{equation*}
            p\Vdash\phi(\tau_1,\dots,\tau_n)\iff\left(p\Vdash^*\phi(\tau_1,\dots,\tau_n)\right)^M.
        \end{equation*}

        \item for all $G$ that are $\bbP$-generic over $M$,
        \begin{equation*}
            \phi(\val(\tau_1, G),\dots,\val(\tau_n, G))^{M[G]}\iff\exists p\in G\left(p\Vdash\phi(\tau_1,\dots,\tau_n)\right).
        \end{equation*}
    \end{enumerate}
\end{theorem}

\section{Breaking $\mathsf{CH}$}

Throughout this section, let $M$ be a countable transitive model of $\mathsf{ZFC}$.

\begin{definition}
    Let $I, J\in M$. Define 
    \begin{equation*}
        \Fn(I, J) = \{p\colon p\text{ is a function }\wedge |p| < \omega\wedge\dom(p)\subseteq I\wedge\ran(p)\subseteq J\}.
    \end{equation*}
    Order $\Fn(I, J)$ by $p\le q$ if and only if $p\supseteq q$. Then, $\Fn(I, J)$ has a maximum element, $0$, the empty function.
\end{definition}

\begin{lemma}
    If $I, J\in M$ and $I$ is infinite, $J\ne\emptyset$, and $G$ is $\Fn(I, J)$-generic over $M$, then $\bigcup G$ is a surjective function $I\to J$.
\end{lemma}
\begin{proof}
    That $\bigcup G$ is a function is trivial. For $i\in I$, let 
    \begin{equation*}
        D_i := \{p\in\Fn(I, J)\colon i\in\dom(p)\}.
    \end{equation*}
    This is a dense set in $\Fn(I, J)$ and hence, intersects $G$. Thus, $i\in\dom\left(\bigcup G\right)$ and hence, $I = \dom\left(\bigcup G\right)$.

    Let $j\in J$ and consider the set 
    \begin{equation*}
        D_j := \{p\in\Fn(I, J)\colon j\in\ran(p)\}.
    \end{equation*}
    This is dense in $\Fn(I, J)$ and hence, intersects $G$. Consequently, $J = \ran\left(\bigcup G\right)$.
\end{proof}

\begin{lemma}
    If $\kappa\in M$ is a cardinal and $G$ is $\Fn(\kappa\times\omega, 2)$-generic over $M$, then $(2^\omega\ge|\kappa|)^{M[G]}$. 

    Note that we must use $|\kappa|$ instead of $\kappa$ since the extension may not preserve cardinals.
\end{lemma}
\begin{proof}
    Let $\bbP = \Fn(\kappa\times\omega, 2)$. Then, $f = \bigcup G$ is a function $\kappa\times\omega\to 2$. Define for any $\alpha,\beta\in\kappa$,
    \begin{equation*}
        D_{\alpha,\beta} = \{p\in\bbP\colon \exists n\in\omega\left(\langle\alpha, n\rangle\in\dom(p)\wedge\langle\beta, n\rangle\in\dom(p)\wedge p(\alpha, n)\ne\dom(\beta, n)\right)\}.
    \end{equation*}
    This is dense in $\bbP$ and hence, $G\cap D_{\alpha,\beta}$ is non empty. 

    Now, let $g_\alpha:\omega\to 2$ by $g_\alpha(n) = f(\alpha, n)$. Due to the above argument, $g_\alpha\ne g_\beta$ whenever $\alpha\ne\beta$. Note that all the $g_\alpha$'s are elements of $M[G]$. Consequently, there are at least $\kappa$ many distinct functions from $\omega\to 2$ in $M[G]$. The conclusion follows.
\end{proof}

\begin{lemma}
    If $I$ is any set and $J$ is countable, then $\Fn(I, J)$ has the countable chain condition.
\end{lemma}
\begin{proof}
    Let $\{p_\alpha\}$ be a collection in $\Fn(I, J)$. Let $a_\alpha = \dom(p_\alpha)$. Using the $\Delta$-system lemma, there is a subcollection $X\subseteq\omega_1$ such that $a_\alpha\cap a_\beta = r$ for every $\alpha\ne\beta$ in $X$. Note that $r$ must be a finite set. 

    Note that $J^r$ is countable and hence, there is an uncountable subcollection $Y\subseteq X$ such that for all $\alpha\in Y$, $p_\alpha\upharpoonright r$ is the same. In particular, this meanas that all the $p_\alpha$s for $\alpha\in Y$ are compatible. Thus, we cannot have an uncountable antichain in $\Fn(I, J)$.
\end{proof}


\begin{lemma}
    Let $\bbP\in M$, $(\bbP\text{ has c.c.c})^M$, and $A, B\in M$. Let $G$ be $\bbP$-generic over $M$ and let $f: A\to B$ be in $M[G]$. Then there is a map $F: A\to\mathscr P(B)$ with $F\in M$ such that 
    \begin{equation*}
        \forall a\in A(f(a)\in F(a))\quad\text{and}\quad\forall a\in A\left(|F(a)|\le\omega\right)^M.
    \end{equation*}
\end{lemma}
\begin{proof}
    Let $\tau$ be a $\bbP$-name in $M^{\bbP}$ such that $f = \tau_G$. Note that ``$\tau$ is a function from $\hat A\to\hat B$'' is a true statement in $M[G]$. Therefore, there is a $p\in G$ that forces the above statement. Define 
    \begin{equation*}
        F(a) := \{b\in B\colon\exists q\le p\left(q\Vdash\tau(\hat a) = \hat b\right)\}.
    \end{equation*}
    Since $\Vdash$ is definable, $F\in M$.

    Let $a\in A$ and $b = f(a)$. Then, there is $r\in G$ such that $r\Vdash \tau(\hat a) = \hat b$. Since $G$ is a filter, there is a $q\in G$ with $q\le r$ and $q\le p$. Consequently, $q\Vdash\tau(\hat a) = \hat b$, so $b\in F(a)$.

    Lastly, we must show that $\left(|F(a)|\le\omega\right)^M$.
\end{proof}

\begin{definition}
    If $\bbP\in M$, then $\bbP$ \emph{preserves cardinals} if whenever $G$ is $\bbP$-generic over $M$ 
    \begin{equation*}
        \forall\beta\in o(M)\left((\beta\text{ is a cardinal})^M\iff(\beta\text{ is a cardinal})^{M[G]}\right).
    \end{equation*}

    Note that $\omega$ is absolute and hence, we need only worry about preservation of cardinals $\beta > \omega$.
\end{definition}

\begin{corollary}
    If $\bbP\in M$ and $(\bbP\text{ has c.c.c})^M$, then $\bbP$ preserves cardinals.
\end{corollary}
\begin{proof}
\end{proof}

\begin{definition}
    If $\bbP\in M$, then $\bbP$ \emph{preserves cofinalities} if whenever $G$ is $\bbP$-generic over $M$ and $\gamma$ is a limit ordinal in $M$, 
    \begin{equation*}
        \cf(\gamma)^M = \cf(\gamma)^{M[G]}.
    \end{equation*}
\end{definition}

\begin{lemma}
    If $\bbP\in M$ and preserves cofinalities, then it preserves cardinals.
\end{lemma}
\begin{proof}
    If $\alpha\ge\omega$ is a regular cardinal of $M$, then 
    \begin{equation*}
        \cf(\alpha)^{M[G]} = \cf(\alpha)^M = \alpha,
    \end{equation*}
    whence $\alpha$ is a regular cardinal (in particular, a cardinal) of $M[G]$.

    Now, suppose $\beta$ is a limit cardinal in $M$. Then, every successor cardinal smaller than $\beta$ is a regular cardinal and hence, remains a regular cardinal in $M[G]$. Consequently, $\beta$ is also a limit cardinal in $M[G]$. This completes the proof. 
\end{proof}

\begin{lemma}
    Let $\bbP\in M$. Suppose whenever $G$ is $\bbP$-generic over $M$ and $\kappa$ is a regular uncountable cardinal of $M$, $(\kappa\text{ is regular})^{M[G]}$. Then $\bbP$ preserves cofinalities.
\end{lemma}
\begin{proof}
    Let $\gamma$ be a limit ordinal in $M$, and let $(\kappa = \cf(\gamma))^M$. Then, there is an $f:\kappa\to\gamma$ in $M$ that is cofinal and strictly increasing. If $(\kappa = \omega)^M$, then it is absolute and hence $(\kappa = \omega)^{M[G]}$. On the other hand, if $\kappa > \omega$, then $(\kappa\text{ is regular})^{M[G]}$ according to the given hypothesis. Since $f\in M[G]$, we have $(\kappa = \cf(\gamma))^{M[G]}$. This completes the proof.
\end{proof}

\begin{theorem}
    If $\bbP\in M$ and $(\bbP\text{ has c.c.c})^M$, then $\bbP$ preserves cofinalities.
\end{theorem}
\begin{proof}
    If not, then due to the previous lemma, there is a $\kappa\in M$ with $\kappa > \omega$, that is regular in $M$ but not $M[G]$. Hence, there is an $\alpha < \kappa$ and a map $f\in M[G]$ $f:\alpha\to\kappa$ that is cofinal. Hence, there is a map $F:\alpha\to\mathscr P(\kappa)$ such that for all $\xi < \alpha$, $f(\xi)\in F(\xi)$ and $|F(\xi)|\le\omega$ in $M$. 

    Define $S = \bigcup_{\xi < \alpha} F(\xi)$. Then, $S\in M$ and is unbounded subset of $\kappa$, further, has cardinality $|\alpha|$. This is a contradiction to $\kappa$ being regular in $M$.
\end{proof}

\subsection{\texorpdfstring{$\mathsf{ZFC} + \neg\mathsf{CH}$}{} is consistent}

Let $\bbP = \Fn(\kappa\times\omega, 2)$, then $\bbP$ has c.c.c in $M$ and thus preserves cardinals. As we have seen earlier, if $G$ is $\bbP$-generic, then $(2^\omega\ge\omega_2)^{M[G]}$ due to all the absoluteness results we have shown above.

\bibliographystyle{alpha}
\bibliography{../references.bib}
\end{document}