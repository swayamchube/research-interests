\section{Transitive Sets}

\begin{definition}
    A set $x$ is said to be \emph{transitive} if 
    \begin{equation*}
        \forall y\forall z(z\in y\wedge y\in x\implies z\in x).
    \end{equation*}
\end{definition}

\begin{proposition}
    A set $x$ is transitive if and only if 
    \begin{equation*}
        \forall y(y\in x\implies y\subseteq x).
    \end{equation*}
\end{proposition}
\begin{proof}
    Suppose $x$ is transitive and $y\in x$. Since for all $z\in y$, $z\in x$, we must have $y\subseteq x$. The converse is trivial.
\end{proof}

\begin{proposition}
    If $x$ is a transitive set, then so is $x\cup\{x\}$.
\end{proposition}
\begin{proof}
\end{proof}

\begin{proposition}
    If $x$ is a transitive set, then so is $\mathscr P(x)$.
\end{proposition}
\begin{proof}
\end{proof}

\begin{proposition}
    If $\mathscr F$ is a family of transitive sets, then so is $\bigcup\mathscr F$.
\end{proposition}
\begin{proof}
\end{proof}

\begin{proposition}\thlabel{prop:element-transitive}
    If $x$ is a transitive set, then so is every $z\in x$.
\end{proposition}
\begin{proof}
\end{proof}

\section{Ordinals}

\begin{definition}[Ordinal]
    A set $x$ is said to be an \emph{ordinal} if it is transitive and well ordered by $\in$. That is, the pair $\langle x,\in_x\rangle$ is a well ordering, where 
    \begin{equation*}
        \in_x := \{\langle v,w\rangle\in x\times x\mid v\in w\}.
    \end{equation*}
\end{definition}

\begin{theorem}[Properties of Ordinals]\thlabel{thm:properties-ordinals}\hfill
\begin{enumerate}[label=(\alph*)]
    \item If $x$ is an ordinal and $y\in x$, then $y$ is an ordinal and $y = \pred(x, y)$.
    \item If $x\cong y$ are ordinals, then $x = y$. 
    \item If $x,y$ are ordinals, then exactly one of the following is true: $x = y$, $x\in y$ or $y\in x$. 
    \item If $C$ is a nonempty set of ordinals, then $\exists x\in C~\forall y\in C(x\in y\vee x = y)$. That is, every nonempty set of ordinals has a minimum element.
\end{enumerate}
\end{theorem}
\begin{proof}
\begin{enumerate}[label=(\alph*)]
    \item Due to \thref{prop:element-transitive}, $y$ is a transitive and owing to it being the subset of a well ordered set, it is well ordered too, hence an ordinal.

    \item Let $f: x\to y$ be an isomorphism. Let 
    \begin{equation*}
        A := \{z\in x\mid f(z)\ne z\}.
    \end{equation*}
    Suppose $A$ is nonempty, then it has a least element, say $w\in x$. If $v\in w$, then $v = f(v)\in f(w)$ whence $w\subseteq f(w)$. On the other hand, if $v\in f(w)$, then there is some $u\in w$ such that $v = f(u) = u\in w$ and thus $f(w) = w$, a contradiction. 

    \item Follows from \thref{thm:well-orders-comparable}.

    \item First note that it suffices to find $x\in C$ with $x\cap C = \emptyset$ for if $y\in C$ is another ordinal with $x\ne y$, then $y\notin x$ lest $x\cap C\ne\emptyset$.

    Pick any $x\in C$. If $x\cap C = \emptyset$, then we are done. Else, let $x'\in x\cap C$ be the $\in$-least element. It is not hard to argue that $x'\cap C = \emptyset$ and we are done.\qedhere
\end{enumerate}
\end{proof}

\begin{lemma}
    If $A$ is a transitive set of ordinals, then $A$ is an ordinal.
\end{lemma}
\begin{proof}
    We must first show that the membership relation $\in_A$ is a linear order. This follows from \thref{thm:properties-ordinals} (c) and the fact that $A$ is a transitive set. Lastly, to see that $A$ is well ordered, simply invoke \thref{thm:properties-ordinals} (d).
\end{proof}

\begin{theorem}
    If $\langle A, R\rangle$ is a well ordering, then there is a \underline{unique} ordinal $C$ such that $\langle A, R\rangle\cong C$.
\end{theorem}
\begin{proof}
    Let 
    \begin{align*}
        B &:= \{a\in A\mid\exists x_a(x_a\text{ is an ordinal }\wedge\langle\pred(A,a,R), R\rangle\cong x_a)\},\\
        f &:= \{\langle b, x_b\rangle\mid b\in B\}.
    \end{align*}
    First, note that for all $b\in B$, $x_b$, since it exists must be unique and thus $f$ is a well defined function with $\dom(f) = B$.

    Let $C = \ran(f)$. We contend that $C$ is an ordinal. Let $y\in x\in C$ and $a\in B$ be such that $g:\pred(A,a,R)\to x$ is an isomorphism. Then, there is some $b\in\pred(A,a,R)$ with $g(b) = y$. It is not hard to see that the restriction $g:\pred(A,b,R)\to y$ is an isomorphism whence $y\in C$ and thus $C$ is an ordinal due to the preceeding lemma.

    The function $f: B\to C$ is obviously a surjection. We contend that it is an isomorphism. Indeed, let $a,b\in B$ with $a R b$ and $g: \pred(A,b,R)\to x_b$ be \emph{the} isomorphism. If $y = g(a)$, then the restriction $g:\pred(A,a,R)\to y$ is an isomorphism whence $f(a) = y\in x = f(b)$ and $f$ is an order isomorphism.

    Suppose $B\ne A$. Let $b\in A\backslash B$ be the $R$-least element. Then, $\pred(A,b,R)\subseteq B$. Now suppose $B\ne\pred(A,b,R)$, consequently, there is some $b'\in B\backslash\pred(A,b,R)$, then $bRb'$ and if there is an order isomorphism from $\pred(A,b',R)$ to some ordinal $x$, then there must be one from $\pred(A,b,R)$ as we have argued earlier, a contradiction. 
    
    Thus, either $B = A$ or $B = \pred(A,b,R)$ for some $b\in A$. In the latter case, the function $f$ is an order isomorphism between $\pred(A,b,R)$ and an ordinal $C$ whence $b\in B$, a contradiction. Thus $B = A$ and the proof is complete.
\end{proof}

\begin{definition}[Type of a Well Ordering]
    If $\langle A, R\rangle$ is a well ordering, then $\type(A,R)$ is the unique ordinal $C$ such that $\langle A, R\rangle\cong C$.
\end{definition}

Henceforth, we use Greek letters $\alpha,\beta,\gamma,\dots$ to vary over ordinals. That is, saying $\forall\alpha(\dots)$ is equivalent to saying $\forall x(x\text{ is an ordinal }\dots)$. Further, since the ordinals are well ordered, we write $\alpha < \beta$ to denote $\alpha\in\beta$ and similarly, $\alpha\le\beta$ means $\alpha\in\beta\vee\alpha = \beta$.

\begin{definition}
    Let $X$ be a set of ordinals. Define 
    \begin{equation*}
        \sup(X) := \bigcup X\quad\text{and}\quad\min(X) := \bigcap X.
    \end{equation*}
    Further, for an ordinal $\alpha$, let $S(\alpha)$ denote the set $\alpha\cup\{\alpha\}$.
\end{definition}

\begin{lemma}
\begin{enumerate}[label=(\alph*)]
    \item $\forall\alpha,\beta(\alpha\le\beta\iff\alpha\subseteq\beta)$.
    \item If $X$ is a set of ordinals, $\sup(X)$ is the least ordinal $\ge$ all elements of $X$ and if $X\ne\emptyset$, $\min(X)$ is the least ordinal in $X$.
\end{enumerate}
\end{lemma}
\begin{proof}
\begin{enumerate}[label=(\alph*)]
    \item The forward direction is obvious. Suppose $\alpha\subseteq\beta$. If $\alpha = \beta$, then we are done. If not, let $\gamma$ be the $<$-least element of $\beta\backslash\alpha$. We contend that $\gamma = \alpha$. Indeed, if $x\in\gamma$, then $x\notin\beta\backslash\alpha$ lest we contradict the minimality of $\gamma$ consequently, $x\in\alpha$ whence $\gamma\subseteq\alpha$. On the other hand, since $\alpha = \pred(\beta,\alpha)$, we have $\alpha\le\gamma$ and thus $\alpha\subseteq\gamma$. This shows that $\alpha = \gamma\in\beta$ and the conclusion follows.

    \item \qedhere
\end{enumerate}
\end{proof}

\begin{lemma}
    For an ordinal $\alpha$, $S(\alpha)$ is an ordinal, $\alpha < S(\alpha)$ and 
    \begin{equation*}
        \forall\beta(\beta < S(\alpha) \iff \beta\le\alpha).
    \end{equation*}
\end{lemma}

\begin{definition}[Successor, Limit Ordinal]
    An ordinal $\alpha$ is said to be a \emph{successor ordinal} if there is an ordinal $\beta$ such that $\alpha = S(\beta)$. On the other hand, $\alpha$ is said to be a \emph{limit ordinal} if $\alpha\ne\emptyset$ and $\alpha$ is not a successor ordinal.
\end{definition}

\section{Transfinite Induction and Recursion}

\subsection{Classes but informally}

Informally speaking, a class is any collection of the form 
\begin{equation*}
    \{x\mid \phi(x)\}
\end{equation*}
where $\phi(x)$ is a well defined first order formula. As we have seen earlier, the class 
\begin{equation*}
    \{x\mid x = x\}
\end{equation*}
is not a set. A \emph{proper class} is a class which is not a set. One uses boldface letters to denote classes. 

\begin{definition}
    Denote 
    \begin{equation*}
        \mathbf{V} := \{x\mid x = x\}\qquad\mathbf{ON} := \{x\mid x\text{ is an ordinal}\}.
    \end{equation*}
\end{definition}

To be completely formal, a class is simply a first order formula with one or more free variables. For example, the class of all ordinals can be thought of as the formula 
\begin{equation*}
    \ON(x) = x \text{ is an ordinal.}
\end{equation*}

We can extend this to define functions between classes $\mathbf A$ and $\mathbf B$. A function $\mathbf F:\mathbf A\to\mathbf B$ is given by a first order logic formula in two variables $\mathbf F(x,y)$ such that 
\begin{equation*}
    \forall x~\mathbf A(x)\implies\exists!y~\left(\mathbf B(y)\wedge\mathbf F(x,y)\right).
\end{equation*}

\begin{theorem}[Transfinite Induction on $\mathbf{ON}$]\thlabel{thm:transfinite-induction}
    If $\mathbf C\subseteq\ON$ and $\mathbf C\ne\emptyset$, then $\mathbf C$ has a least element.
\end{theorem}
\begin{proof}
    The proof is exactly like \thref{thm:properties-ordinals} (d).
\end{proof}

One must note that there is a significant difference between \thref{thm:properties-ordinals} (d) and \thref{thm:transfinite-induction}. The former is a single provable statement in \textsf{ZFC} while the latter is a theorem schema which represents an infinite collection of theorems. In particular, suppose the class $\mathbf C$ corresponded to a formula $\mathbf C(x,z_1,\dots,z_n)$, then \thref{thm:transfinite-induction} in this case says the following: 
\begin{align*}
    \forall z_1,\dots,z_n\big\{\left[\forall x(\mathbf C(x,z_1,\dots,z_n)\implies x\text{ is an ordinal})\wedge\exists x\mathbf C(x,z_1,\dots,z_n)\right]\\
    \implies\left[\exists x\left(\mathbf C(x,z_1,\dots,z_n)\wedge\forall y(\mathbf C(y,z_1,\dots,z_n)\implies y\ge x)\right)\right]\big\}.
\end{align*}

And \thref{thm:transfinite-induction} specifies one such formula for each well-formed sentence $\mathbf C$.

\begin{theorem}[Transfinite Recursion on $\ON$]\thlabel{thm:transfinite-recursion}
    If $\mathbf F: \V\to\V$, then there is a unique $\mathbf G:\ON\to\V$ such that 
    \begin{equation*}
        \forall\alpha\left(\mathbf G(\alpha) = \mathbf F(\mathbf G\restrict\alpha)\right).
    \end{equation*}
\end{theorem}
The formal restatement of the above in terms of first order logic is the following: 
\begin{equation*}
    \forall x\exists!y~\mathbf F(x,y)\implies\left[\forall\alpha\exists!y~\mathbf G(\alpha, y)\wedge\forall\alpha\exists x\exists y\left(\mathbf G(\alpha, y)\wedge\mathbf F(x,y)\wedge x = \mathbf G\restrict\alpha\right)\right]
\end{equation*}
where 
\begin{equation*}
    (x = \mathbf G\restrict\alpha) := \mathsf{function}(x)\wedge\dom(x) = \alpha\wedge\left(\forall\beta\in\dom(x)~\mathbf G(\beta,x(\beta))\right).
\end{equation*}
Similarly, one can encode the uniqueness condition.

\begin{proof}
    
\end{proof}

\section{Ordinal Arithmetic}
\subsection*{Addition}

\begin{definition}[Ordinal Addition]
    If $\alpha,\beta$ are ordinals, then define $\alpha + \beta = \type(\alpha\times\{0\}\cup\beta\times\{1\},R)$ where 
    \begin{align*}
        R = \left\{\langle\langle\xi,0\rangle,\langle\eta,0\rangle\rangle\mid \xi < \eta < \alpha\right\}\cup\left\{\langle\langle\xi,0\rangle,\langle\eta,1\rangle\rangle\mid\xi < \eta < \beta\right\}\cup\left[(\alpha\times\{0\})\times(\beta\times\{1\})\right].
    \end{align*}
\end{definition}

Informally speaking, we construct a new ordinal $\alpha + \beta$ by first ``placing'' $\alpha$ is a line and then placing $\beta$ after it linearly. This is best visualized when $\alpha$ and $\beta$ are finite ordinals.

To see that $R$ indeed gives $\alpha\times\{0\}\cup\beta\times\{1\}$ the structure of a well order, let $S$ be a nonempty subset. If $S\cap\alpha\times\{0\}$ is nonempty, then the minimal element of $S$ exists and is the minimal element of $S\cap\alpha\times\{0\}$. On the other hand, if $S\cap\alpha\times\{0\} = \emptyset$, the minimal element of $S$ exists and is the minimal element of $S\cap\beta\times\{1\}$. 

\begin{lemma}
    For ordinals $\alpha,\beta,\gamma$,
    \begin{enumerate}[label=(\alph*)]
        \item $\alpha + (\beta + \gamma) = (\alpha + \beta) + \gamma$.
        \item $\alpha + 0 = \alpha$.
        \item $\alpha + 1 = S(\alpha)$.
        \item $\alpha + S(\beta) = S(\alpha + \beta)$.
        \item If $\beta$ is a limit ordinal, then $\alpha + \beta = \sup\{\alpha + \xi\mid \xi < \beta\}$.
    \end{enumerate}
\end{lemma}
\begin{proof}
    We shall only prove (e) since the others are straightforward. First, note that $\alpha + \beta\ge\alpha + \xi$ for every $\xi < \beta$, which is easy to see by setting up an obvious order preserving injection. \todo{complete this argument}
\end{proof}

\begin{remark}
    One must note that ordinal addition is \textbf{not commutative}. Indeed, 
    \begin{equation*}
        1 + \omega = \sup\{1 + n\mid n < \omega\} = \omega
    \end{equation*}
    while 
    \begin{equation*}
        \omega + 1 = S(\omega)\ne\omega
    \end{equation*}
    where the last ``non-equality'' follows from the axiom of foundation. Thus, $1 + \omega\not\cong\omega + 1$.
\end{remark}

\subsection*{Multiplication}

\begin{definition}
    If $\alpha,\beta$ are ordinals, define $\alpha\cdot\beta = \type(\beta\times\alpha, R)$ where $R$ is the dictionary order, given by 
    \begin{equation*}
        R = \left\{\langle\langle\xi,\eta\rangle,\langle\xi',\eta'\rangle\rangle~\big\vert~\xi < \xi'\vee (\xi = \xi'\wedge\eta < \eta')\right\}.
    \end{equation*}
\end{definition}

We must first check that $R$ is indeed a well ordering. That it is a strict linear order is clear. Let $S\subseteq\beta\times\alpha$ be a nonempty subset. Let $S_1$ be the projection of $S$ onto $\beta$. This has a minimum element, say $\xi$. Consider now the set of all $\eta\in\alpha$ such that $\langle\xi,\eta\rangle\in S$. This is a nonempty subset of $\alpha$ and thus has a minimum element, say $\delta$. Then, $\langle\xi,\delta\rangle$ is a minimum element of $S$.

\begin{lemma}
    For ordinals $\alpha,\beta,\gamma$, 
    \begin{enumerate}[label=(\alph*)]
        \item $\alpha\cdot(\beta\cdot\gamma) = (\alpha\cdot\beta)\cdot\gamma$.
        \item $\alpha\cdot 0 = 0$.
        \item $\alpha\cdot 1 = \alpha$. 
        \item $\alpha\cdot S(\beta) = \alpha\cdot\beta + \alpha$. 
        \item If $\beta$ is a limit ordinal, then $\alpha\cdot\beta = \sup\{\alpha\cdot\xi\mid\xi < \beta\}$. 
        \item $\alpha\cdot(\beta + \gamma) = \alpha\cdot\beta + \alpha\cdot\gamma$.
    \end{enumerate}
\end{lemma}
\begin{proof}
    \todo{Proof of ordinal multiplication}
\end{proof}

\subsection*{Exponentiation}

\begin{definition}
    For ordinals $\alpha,\beta$, we define $\alpha^\beta$ by recursion on $\beta$ as 
    \begin{itemize}
        \item $\alpha^0 = 1$. 
        \item $\alpha^{\beta + 1} = \alpha^\beta\cdot\beta$. 
        \item If $\beta$ is a limit ordinal, $\alpha^\beta = \sup\{\alpha^\xi\mid\xi < \beta\}$. 
    \end{itemize}
\end{definition}

\begin{remark}
    Interestingly,
    \begin{equation*}
        2^\omega = \sup\{2^n\mid n < \omega\} = \omega.
    \end{equation*}
\end{remark}

\section{Equivalent forms of the Axiom of Choice}

\begin{theorem}[Well Ordering Theorem]\thlabel{thm:well-ordering}
    For every nonempty set $A$, there is a relation $R\subseteq A\times A$ such that $R$ well orders $A$.
\end{theorem}

\subsection*{AC \texorpdfstring{$\implies$}{} WO}

Let $A$ be a set. We shall explicitly construct a well ordering on $X$ using the Axiom of Choice. First, let $f:\calP(A)\backslash\{\emptyset\}\to A$ be a choice function and extend it to $f:\calP(A)\to A\coprod\{\emptyset\}$ by defining $f(\emptyset) = \emptyset$. We shall now use transfinite recursion to define a function $F$ on the ordinals as follows: 
\begin{align*}
    F(0) &:= f(A)\\ 
    F(\alpha) &:= f\left(\left\{x\in A\mid\forall\beta\in\alpha(F(\beta)\ne x)\right\}\right).
\end{align*}

First, note that if $F(\alpha) = F(\beta)\ne\emptyset$, then $\alpha = \beta$. Next, we contend that there must be an ordinal $\alpha$ with $F(\alpha) = \emptyset$. For if not, then we may apply the axiom of replacement and that of comprehension to obtain a set of all ordinals, a contradiction to the Burali-Forti paradox.

Let $\mathbf C$ denote the class of all ordinals $\alpha$ with $F(\alpha) = \emptyset$. Due to \thref{thm:transfinite-induction}, there is a minimal such ordinal, say $\alpha_0$, then 
\begin{equation*}
    f\left(\left\{x\in A\mid\forall\beta\in\alpha_0(F(\beta)\ne x)\right\}\right) = \emptyset\implies\left\{x\in A\mid\forall\beta\in\alpha_0(F(\beta)\ne x)\right\} = \emptyset.
\end{equation*}

Let $G: A\to\alpha_0$ denote the inverse function of $F$. Define the relation $R\subseteq A\times A$ by 
\begin{equation*}
    R := \{\langle x,y\rangle\mid G(x)\in G(y)\}.
\end{equation*}

That this is a well ordering is easy to see.

\subsection*{WO \texorpdfstring{$\implies$}{} AC}

This direction, on the other hand, is much easier. Let $X$ denote a collection of sets and let $Y = \bigcup X$. Let $R$ be a well ordering on $Y$. Define the function $f: X\to Y$ by $f(x) = \min(x)$, the $R$-least element, which can be chosen since $Y$ has been well ordered and $x\subseteq Y$.

\subsection*{AC \texorpdfstring{$\implies$}{} Zorn}

Let $X$ be a set and $P = (X,\leqq)$ be a poset on it such that every chain in $P$ has an upper bound. Let $f: \mathscr P(X)\backslash\{\emptyset\}\to X$ be a choice function. 

Suppose $P$ has no maximal element. Then, every chain in $P$ must have a strict upper bound. Let $\mathscr C$ be the set of all chains in $P$. Let $g: \mathscr C\to\mathscr P(X)$ map a chain in $P$ to the set of all \emph{strict} upper bounds. Consequently, $g(C)\ne\emptyset$ for every chain $C$ in $P$.

We shall define a class function $F:\ON\to X$ using transfinite recursion. Begin with $F(0) = F(X)$. Now, for any ordinal $\alpha\in\ON$, let $C_\alpha$ denote the chain $\{F(\beta)\mid \beta < \alpha\}$ and define 
\begin{equation*}
    F(\alpha) := f(g(C_\alpha)).
\end{equation*}

It is not hard to see that $F(\alpha) = F(\beta)$ if and only if $\alpha = \beta$ whence we may use Replacement to obtain a \emph{set} of all ordinals, which is absurd.

\subsection*{Zorn \texorpdfstring{$\implies$}{} AC}

Let $X$ be a collection of sets and $Y = \bigcup X$. Let $P$ be the poset of pairs $(S, f)$ where $S\subseteq X$ and $f: S\to Y$ is a function with $f(s)\in s$ for each $s\in S$. We say $(S, f)\leqq(S',f')$ if $S\subseteq S'$ and $f'\restrict_S = f$. 

Let $C = \{(S_\alpha, f_\alpha)\}$ be a chain in $P$. Define the function $f:S := \bigcup_{\alpha} S_\alpha\to Y$ by $f(x) := f_\alpha(x)$ if $x\in S_\alpha$. Then, $(S, f)$ is an upper bound for the chain $C$. Thus, due to Zorn's Lemma, $P$ contains a maximal element, say $(\wt S, F)$. We contend that $\wt S = X$. For if not, then there is $x\in X\backslash\wt S$ and the function $F$ can be extended to $\wt S\cup\{x\}$ by simply choosing an element of $x$ and assigning it to $x$ under $F$. This contradicts the maximality of $(\wt S, F)$ and hence, $F$ is the desired choice function.
