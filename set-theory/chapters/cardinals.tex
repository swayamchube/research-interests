\begin{definition}
    Sets $A$ and $B$ are said to be \emph{equinumerous} if there is a bijection $f: A\to B$. This is denoted by $A\approx B$. On the other hand, if there is an injection $f: A\to B$, it is denoted by $A\preceq B$. We write $A\prec B$ if $A\preceq B$ and $B\not\preceq A$.
\end{definition}

\begin{theorem}[Cantor-Schr\"oder-Bernstein]\thlabel{thm:cantor-schroder-bernstein}
    $A\preceq B\wedge B\preceq A\implies A\approx B$.
\end{theorem}

\begin{definition}
    For a set $A$, $|A|$ is the least $\alpha$ such that $\alpha\approx A$.
    $\alpha$ is a \emph{cardinal} if and only if $\alpha = |\alpha|$.
\end{definition}

From \thref{thm:well-ordering}, there is a well ordering $R$ on $A$ and thus an ordinal $\alpha$ with an order preserving bijection between $\langle A,R\rangle$ and $\alpha$, in particular, $A\approx\alpha$. Thus, $|A|$ is defined for every set. Further, note that $\alpha$ is a cardinal if and only if $\forall\beta < \alpha(\beta\not\approx\alpha)$ and for any ordinal $\alpha$, $|\alpha|\le\alpha$.

\begin{lemma}
    If $|\alpha|\le\beta\le\alpha$, then $|\beta| = |\alpha|$.
\end{lemma}
\begin{proof}
    Since $\beta\le\alpha$, we have $\beta\subseteq\alpha$ and thus $\beta\preceq\alpha$. On the other hand, $|\alpha|\subseteq\beta$. Composing this inclusion with the bijection $\alpha\approx|\alpha|$, we have $\alpha\preceq\beta$. We are done due to \thref{thm:cantor-schroder-bernstein}.
\end{proof}

\begin{lemma}
    If $n\in\omega$, then 
    \begin{enumerate}[label=(\alph*)]
        \item $n\not\approx n + 1$. 
        \item $\forall\alpha(\alpha\approx n\implies\alpha = n)$.
    \end{enumerate}
\end{lemma}
\begin{proof}
\begin{enumerate}[label=(\alph*)]
    \item Suppose not. Pick the smallest $n\in\omega$ such that $n\approx n + 1$. Note that $n\ne 0$. We have an injective function $f: n + 1\to n$. Composing appropriately, we may suppose that $f(n) = n - 1$ where $n\in n + 1$ and $n - 1\in n$. The restriction $f\restrict_n$ is an injective function from $n$ to $n - 1$ whence by \thref{thm:cantor-schroder-bernstein}, $n - 1\approx n$, a contradiction. 

    \item If $n < \alpha$, then $n + 1\le\alpha$ whence $n + 1\preceq\alpha$. On the other hand, $\alpha\approx n < n + 1$, consequently $\alpha\approx n + 1$, a contradiction to (a). 

    Now suppose $\alpha < n$. Then, $|n| = |\alpha|\le\alpha\le\alpha + 1\le n$, consequently, $|\alpha + 1| = |n|$. But since $\alpha + 1\approx n + 1$, we have $n + 1\approx n$, a contradiction to (a). Thus $\alpha = n$. \qedhere
\end{enumerate}
\end{proof}

\begin{corollary}
    $\omega$ is a cardinal and so is every ordinal $n < \omega$.
\end{corollary}

\begin{definition}
    $A$ is \emph{finite} if and only if $|A| < \omega$. $A$ is \emph{countable} if and only if $|A|\le\omega$. We use the shorthand \emph{infinite} to mean ``not finite'' and \emph{uncountable} to mean ``not countable''.
\end{definition}

\begin{definition}[Cardinal Arithmetic]
    For cardinals $\kappa$ and $\lambda$, define 
    \begin{equation*}
        \kappa\oplus\lambda := |\kappa\times\{0\}\cup\lambda\times\{1\}|,\quad\kappa\otimes\lambda := |\kappa\times\lambda|.
    \end{equation*}
\end{definition}

Unlike ordinal arithmetic, the operations $\oplus$ and $\otimes$ are commutative, which is obvious from the definition above. Furthermore, note that 
\begin{equation*}
    |\kappa + \lambda| = |\lambda + \kappa| = \kappa\oplus\lambda\quad\text{ and }\quad|\kappa\cdot\lambda| = |\lambda\cdot\kappa| = \kappa\otimes\lambda.
\end{equation*}

\begin{lemma}
    For $m,n\in\omega$, $n\oplus m = n + m < \omega$ and $n\otimes m = n\cdot m < \omega$.
\end{lemma}
\begin{proof}
    
\end{proof}

\begin{proposition}
    Every infinite cardinal is a limit ordinal.
\end{proposition}
\begin{proof}
    Suppose $\kappa = \alpha + 1$ is a cardinal. Then, $\alpha$ is not a finite ordinal, that is, $\omega < \alpha$ and thus there is an ordinal $\beta$ such that $\alpha = \omega + \beta$. Consequently, $1 + \alpha = 1 + \omega + \beta = \omega + \beta$ as we have seen previously that $1 + \omega = \omega$. Consequently, 
    \begin{equation*}
        |\kappa| = |\alpha + 1| = |1 + \alpha| = |\alpha|,
    \end{equation*}
    a contradiction to the fact that $\kappa$ is a cardinal.\todo{if $\alpha\le\beta$ there is an ordinal $\delta$ such that $\beta = \alpha + \delta$.}
\end{proof}

\begin{theorem}[Tarski]
    If $\kappa$ is an infinite cardinal, then $\kappa\otimes\kappa = \kappa$.
\end{theorem}
\begin{proof}
    We shall prove this statement by transfinite induction on $\kappa$. That this statement holds for $\kappa = \omega$ is well known. Suppose now that $\kappa > \omega$ and the statement holds for each cardinal $\lambda < \kappa$.

    Note that for an infinite ordinal $\alpha < \kappa$, we have $|\alpha| < \kappa$ and thus 
    \begin{equation*}
        |\alpha\times\alpha| = |\alpha|\otimes|\alpha| = |\alpha| < \kappa.
    \end{equation*}

    Let $\prec$ denote the strict lexicographic ordering on $\kappa\times\kappa$. Define the relation $\unlhd$ on $\kappa\times\kappa$ by $\langle\alpha,\beta\rangle\unlhd\langle\gamma,\delta\rangle$ if and only if 
    \begin{equation*}
        \max\{\alpha,\beta\} < \max\{\gamma,\delta\}\text{ or }\max\{\alpha,\beta\} = \max\{\gamma,\delta\}\text{ and }\langle\alpha,\beta\rangle\prec\langle\gamma,\delta\rangle.
    \end{equation*}
    That this relation is an ordering is immediate from the definition. We shall now show that this is a well ordering. Let $S\subseteq\kappa\times\kappa$ be nonempty. Using Replacement, construct the set $S'$ which consists of $\max\{\alpha,\beta\}$ for all $\langle\alpha,\beta\rangle\in S$. Since $S'\subseteq\kappa$, it contains a minimum element, say $\alpha_0$. Using Comprehension, construct the set $S''$ consisting of all pairs $\langle\alpha,\beta\rangle$ such that $\max\{\alpha,\beta\} = \alpha_0$. Now, $S''\subseteq\kappa\times\kappa$, and under the lexicographic order, it has a minimum element, which is also the minimum element of $S$ under the ordering $\unlhd$.

    Given any $\langle\alpha,\beta\rangle\in\kappa\times\kappa$, the set of all pairs preceeding it in $\langle\kappa\times\kappa,\unlhd\rangle$ is a subset of 
    \begin{equation*}
        (\max\{\alpha,\beta\} + 1)\times(\max\{\alpha,\beta\} + 1)
    \end{equation*}
    Since $\kappa$ is a limit ordinal, we have $\max\{\alpha,\beta\} + 1 < \kappa$ and due to the induction hypothesis, the cardinality of the above set is strictly smaller than $\kappa$ whence $|\kappa\times\kappa|\le\kappa$. There is an obvious injection from $\kappa$ into $\kappa\times\kappa$, forcing $|\kappa\times\kappa| = \kappa$ due to \thref{thm:cantor-schroder-bernstein}.
\end{proof}

\begin{corollary}
    Let $\kappa,\lambda$ be infinite cardinals. Then, 
    \begin{enumerate}[label=(\alph*)]
        \item $\kappa\oplus\lambda = \kappa\otimes\lambda = \max\{\kappa,\lambda\}$,
        \item $|\kappa^{<\omega}| = \kappa$.
    \end{enumerate}
\end{corollary}
\begin{proof}
    
\end{proof}

\begin{theorem}[Cantor]
    $\forall X\left(X\prec\mathscr P(X)\right)$.
\end{theorem}
\begin{proof}
    Suppose not, then $X\approx\mathscr P(X)$ for some $X$, which follows from \thref{thm:cantor-schroder-bernstein} and the fact that there is a canonical injection from $X$ to $\mathscr P(X)$. Let $f:X\to\mathscr P(X)\to X$ be a bijection. Using Comprehension, construct the set 
    \begin{equation*}
        S := \{x\in X\mid x\notin f(x)\}\subseteq X.
    \end{equation*}
    Let $s\in X$ be the unique element such that $f(s) = S$. Then, 
    \begin{equation*}
        s\in S\iff s\notin S,
    \end{equation*}
    a contradiction.
\end{proof}

\begin{theorem}
    $\forall\alpha\exists\kappa\left(\kappa > \alpha\text{ is a cardinal}\right)$ is true in \textsf{ZF}.
\end{theorem}
If we were to work in \textsf{ZFC} then we could just well order $\mathscr P(\alpha)$ and consider its cardinality.
\begin{proof}
    The statement is obvious for finite cardinals. Suppose now that $\alpha\ge\omega$. Let 
    \begin{align*}
        W := \{R\in\mathscr P(\alpha\times\alpha)\mid R \text{ well orders }\alpha\}
        S := \{\type(\langle\alpha, R\rangle)\mid R\in W\}.
    \end{align*}
    Let $\beta = \sup(S)$. We contend that $\beta$ is a cardinal and $\beta > \alpha$. First, note that if $\delta\in W$, then $S(\delta)\in W$, consequently, $\beta\notin W$. Further, $\beta\not\approx\alpha$ lest one could find a well ordering on $\alpha$ which is in order preserving bijection with $\beta$. Suppose $\beta$ were not a cardinal. Then, there is some $\gamma < \beta$ with $\gamma\approx\beta$. By definition, there is $\eta$ such that $\gamma\le\eta < \beta$ with $\eta\in W$, consequently, $\eta\approx\beta$ but $\alpha\approx\eta$, a contradiction. This completes the proof.
\end{proof}

\begin{definition}[Successor, Limit Cardinals]
    Let $\alpha$ be an ordinal. Denote by $\alpha^+$ the smallest \emph{cardinal} strictly greater than $\alpha$. A cardinal $\kappa$ is said to be a \emph{successor cardinal} if $\kappa = \alpha^+$ for some $\alpha$. On the other hand, if $\kappa > \omega$ and is not a successor cardinal, then $\kappa$ is said to be a \emph{limit cardinal}.
\end{definition}

\begin{definition}[Aleph Numbers]
    Define the numbers $\aleph_\alpha$ by transfinite recursion on $\alpha$.
    \begin{enumerate}[label=(\alph*)]
        \item $\aleph_0 := \omega$.
        \item $\aleph_{\alpha + 1} = (\aleph_\alpha)^+$.
        \item For a limit ordinal $\lambda$, define $\aleph_\lambda := \sup\{\aleph_\alpha\mid \alpha < \lambda\}$.
    \end{enumerate}
\end{definition}

\begin{theorem}
\begin{enumerate}[label=(\alph*)]
    \item Each $\aleph_\alpha$ is a cardinal.
    \item Every infinite cardinal is equal to $\aleph_\alpha$ for some $\alpha$.
    \item If $\alpha < \beta$, then $\aleph_\alpha < \aleph_\beta$. 
    \item $\aleph_\alpha$ is a limit cardinal ifa nd only if $\alpha$ is a limit ordinal. 
    \item $\aleph_\alpha$ is a successor cardinal if and only if $\alpha$ is a successor ordinal.
\end{enumerate}
\end{theorem}
\begin{proof}
    All of these follow immediately from the definition above.
\end{proof}

\begin{remark}
    One often writes $\omega_\alpha$ in place of $\aleph_\alpha$. We adopt both conventions and use them interchangeably.
\end{remark}

\begin{lemma}
    If there is a surjective function $f:X\to Y$, then $|Y|\le|X|$.
\end{lemma}
\begin{proof}
    Consider the set 
    \begin{equation*}
        S = \{f^{-1}(y)\mid y\in Y\},
    \end{equation*}
    which can be constructed using Replacement. Let $g: Y\to S$ be given by $g(y) = f^{-1}(y)$ and $F$ be a choice function on $S$. Then, the composition $F\circ g$ is an injective function from $Y$ to $X$, implying the desired conclusion.
\end{proof}

\begin{definition}[Cardinal Exponentiation]
    For sets $A$ and $B$, define 
    \begin{equation*}
        A^B := {}^BA := \{f\subseteq\mathscr P(B\times A)\mid f \text{ is a function}\}.
    \end{equation*}
    For cardinals $\kappa$ and $\lambda$, define $\kappa^\lambda := |{}^\lambda\kappa|$.
\end{definition}

\begin{theorem}
    Let $2\le\kappa\le\lambda$ and $\lambda$ an infinite cardinal. Then, $\kappa^\lambda = 2^\lambda$.
\end{theorem}
\begin{proof}
    Obviously, ${}^\lambda 2\approx\mathscr P(\lambda)$ which can be seen by looking at the characteristic function of each subset of $\lambda$. Then, we have 
    \begin{equation*}
        {}^\lambda k\preceq{}^\lambda\lambda\preceq\mathscr P(\lambda\times\lambda)\preceq\mathscr P(\lambda)\preceq{}^\lambda 2.
    \end{equation*}
    The conclusion follows from \thref{thm:cantor-schroder-bernstein}.
\end{proof}

\begin{theorem}
    Let $\mathscr B(\R)$ denote the Borel $\sigma$-algebra on $\R$ with the standard topology. Then, $|\mathscr B(\R)| = 2^{\aleph_0}$, the cardinality of the continuum.
\end{theorem}
\begin{proof}
    That $\mathscr B(\R)$ has cardinality at least that of the continuum is straightforward since it contains all singletons. Showing the reverse direction is a bit involved and requires transfinite recursion.

    First, note that $\R$ is second countable and thus has a countable abse for its topology, denote this by $S_0$. For an ordinal $\alpha < \omega_1$, let $S_{\alpha + 1}$ denote the collection of all unions of the form 
    \begin{equation*}
        \bigcup_{i} A_i\cup\bigcup_{j}(\R\backslash B_j)
    \end{equation*}
    where $A_i$ and $B_j$ are chosen from $S_\alpha$. Note that if $|S_\alpha|\le 2^{\aleph_0}$, then the number of these unions that can be formed is at most $(2^{\aleph_0})^{\aleph_0} = 2^{\aleph_0}$ since there is a surjection from the set of all functions $\aleph_0\to S_\alpha$ onto $S_{\alpha + 1}$. 

    On the other hand, if $\alpha$ is a limit ordinal, define 
    \begin{equation*}
        S_{\alpha} = \bigcup_{\lambda < \alpha} S_\lambda.
    \end{equation*}

    We contend that $S = \bigcup_{\alpha < \omega_1}S_\alpha$ is a $\sigma$-algebra. Obviously, $S$ contains $\emptyset$ and $\R$, and is closed under complementation. Let $\{A_n\}_{n = 1}^\infty$ be a sequence in $S$. For each positive integer $n$, let $\alpha(n)$ denote the minimal ordinal $\lambda$ such that $A_n\in S_{\lambda}$. Note that for each $n$, the cardinality $|\alpha(n)|\le\omega$. Hence, if $\beta = \sup_{n < \omega}\alpha(n)$, then $|\beta|\le\omega$, consequently, $\beta < \omega_1$ and $\{A_n\mid n < \omega\}\subseteq S_{\beta}$ implying that $\bigcup_{n = 1}^\infty A_n\in S_{\beta + 1}\subseteq S$. 

    As a result, $S$ contains $\mathscr B(\R)$ but the cardinality of $S$ is at most 
    \begin{equation*}
        |\omega_1|\otimes 2^{\aleph_0} = \aleph_1\otimes 2^{\aleph_0}\le 2^{\aleph_0}\otimes 2^{\aleph_0} = 2^{\aleph_0}.
    \end{equation*}
    This completes the proof.
\end{proof}

\begin{theorem}[Mazurkiewicz, 1914]
    There is a subset $A\subseteq\R^2$ which meets \emph{every} line in the plane at \emph{exactly} $2$ points.
\end{theorem}
\begin{proof}
    Let $\mathscr L$ denote the set of all possible lines in the plane. The cardinality of $\mathscr L$ is at least $2^{\aleph_0}$ and atmost $2^{\aleph_0}\otimes 2^{\aleph_0} = 2^{\aleph_0}$. Since this is in bijection with $2^{\aleph_0}$, it has an induced well ordering, which we denote by $\mathscr L = \{L_\alpha\mid\alpha < 2^{\aleph_0}\}$.

    We shall, using transfinite recursion construct a chain $X_\alpha$ of subsets of $\R^2$ for $\alpha < 2^{\aleph_0}$ such that $|X_\alpha| < 2^{\aleph_0}$ and $|X_\alpha\cap L_\beta|\le 2$ for each $\beta < 2^{\aleph_0}$. 

    Begin with $X_0 = \{x_0\}$ for any $x_0\in \R^2$. Suppose now that the sequence has been constructed for each $\beta < \alpha$ where $\alpha > 0$. Let $Y_\alpha := \bigcup_{\beta < \alpha} X_\beta$. Let $S_\alpha$ denote the set of all lines between two points in $Y_\alpha$. Note that the cardinality of $S_\alpha$ is strictly smaller than $2^{\aleph_0}$. 

    Let $\gamma$ be the smallest ordinal such that $|L_\gamma\cap Y_\alpha|\le 1$. If no such ordinal exists, then $Y_\alpha$ is the desired set. Suppose such a $\gamma$ does exist. Then, the set 
    \begin{equation*}
        L_\gamma\backslash\underbrace{\left(\bigcup_{L\in S_\alpha}L\cup\bigcup_{\beta < \gamma} L_\beta\cup Y_\alpha\right)}_{T}
    \end{equation*}
    which is non empty, since the intersection of $L_\gamma$ with $T$ has cardinality strictly smaller than $2^{\aleph_0}$. Let $x_\alpha$ be one such element in the above set and define $X_\alpha = Y_\alpha\cup\{x_\alpha\}$.

    It is not hard to see that $X_\alpha$ satisfies the desired properties and thus we may continue this procedure and obtain $\{X_\alpha\mid\alpha < 2^{\aleph_0}\}$. Let $X = \bigcup_{\alpha < 2{^\aleph_0}} X_\alpha$. This is the required set.
\end{proof}