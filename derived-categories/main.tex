\documentclass{amsart}

\RequirePackage{
    amsmath, 
    amsthm, 
    amssymb, 
    amsfonts,
    xcolor, 
    geometry, 
    tikz, 
    hyperref, 
    mathrsfs,
    enumitem,
    fancyhdr,
    lipsum,
    nicefrac,
    wasysym,
    mathtools,
    xspace,
    anyfontsize,
    todonotes,
    fancyhdr,
    lastpage,
    stmaryrd,
    commath,
    float       % because vanilla LaTeX loses its mind trying to place my floats
    % titlesec
}
\RequirePackage{theoremref} % For referencing theorems
\RequirePackage[framemethod = tikz]{mdframed}
\usetikzlibrary{automata, arrows.meta, positioning, cd} % good ol' Theoretical CS 
\RequirePackage[all,cmtip]{xy} % For diagrams, praise the Freyd–Mitchell theorem 

% Input and Fonts
\RequirePackage[utf8]{inputenc}
\RequirePackage[T1]{fontenc}
\RequirePackage{mathpazo}
% \RequirePackage{euler}
\SetSymbolFont{stmry}{bold}{U}{stmry}{m}{n}
\SetSymbolFont{wasy}{bold}{U}{wasy}{m}{n}
% \RequirePackage{euler}


\hypersetup {
    colorlinks=true, % because why not?
}

\geometry{
    margin=1in, % uniformity 
}

\newtheoremstyle{thmstyle}%               % Name
  {}%                                     % Space above
  {}%                                     % Space below
  {\itshape}%                             % Body font
  {}%                                     % Indent amount
  {\bfseries}%                            % Theorem head font
  {.}%                                    % Punctuation after theorem head
  { }%                                    % Space after theorem head, ' ', or \newline
  {\thmname{#1}\thmnumber{ #2}\thmnote{ (#3)}}%                                     % Theorem head spec (can be left empty, meaning `normal')

\newtheoremstyle{defstyle}%               % Name
  {}%                                     % Space above
  {}%                                     % Space below
  {}%                                     % Body font
  {}%                                     % Indent amount
  {\bfseries}%                            % Theorem head font
  {.}%                                    % Punctuation after theorem head
  { }%                                    % Space after theorem head, ' ', or \newline
  {\thmname{#1}\thmnumber{ #2}\thmnote{ (#3)}}%                 


\theoremstyle{thmstyle}
\newtheorem{theorem}{Theorem}[section]
\newtheorem{lemma}[theorem]{Lemma}
\newtheorem{proposition}[theorem]{Proposition}
\newtheorem{remark}{Remark}[section]

\theoremstyle{defstyle}
\newtheorem{definition}[theorem]{Definition}
\newtheorem{corollary}[theorem]{Corollary}
\newtheorem{example}[theorem]{Example}

\renewcommand{\qedsymbol}{$\blacksquare$}
\renewcommand{\emptyset}{\varnothing}
\renewcommand{\footrulewidth}{\headrulewidth}
\setlength{\headheight}{15pt}

% Common Algebraic Structures
\newcommand{\R}{\mathbb{R}}
\newcommand{\Q}{\mathbb{Q}}
\newcommand{\Z}{\mathbb{Z}}
\newcommand{\N}{\mathbb{N}}
\newcommand{\bbC}{\mathbb{C}}
\newcommand{\calA}{\mathcal{A}}
\newcommand{\frakM}{\mathfrak{M}}

% Categories
\newcommand{\catTopp}{\mathbf{Top}_*}
\newcommand{\catGrp}{\mathbf{Grp}}
\newcommand{\catTopGrp}{\mathbf{TopGrp}}
\newcommand{\catSet}{\mathbf{Set}}
\newcommand{\catTop}{\mathbf{Top}}
\newcommand{\catRing}{\mathbf{Ring}}
\newcommand{\catCRing}{\mathbf{CRing}} % comm. rings
\newcommand{\catMod}{\mathbf{Mod}}
\newcommand{\catMon}{\mathbf{Mon}}
\newcommand{\catMan}{\mathbf{Man}} % manifolds
\newcommand{\catDiff}{\mathbf{Diff}} % smooth manifolds
\newcommand{\catAlg}{\mathbf{Alg}}
\newcommand{\catRep}{\mathbf{Rep}} % representations 
\newcommand{\catVec}{\mathbf{Vec}}

% Group and Representation Theory
\newcommand{\chr}{\operatorname{char}}
\newcommand{\Aut}{\operatorname{Aut}}
\newcommand{\GL}{\operatorname{GL}}
\newcommand{\im}{\operatorname{im}}
\newcommand{\tr}{\operatorname{tr}}
\newcommand{\id}{\mathbf{id}}
\newcommand{\cl}{\mathbf{cl}}
\newcommand{\Gal}{\operatorname{Gal}}
\newcommand{\Tr}{\operatorname{Tr}}
\newcommand{\sgn}{\operatorname{sgn}}
\newcommand{\Sym}{\operatorname{Sym}}
\newcommand{\Alt}{\operatorname{Alt}}

% Commutative and Homological Algebra
\newcommand{\spec}{\operatorname{spec}}
\newcommand{\mspec}{\operatorname{m-spec}}
\newcommand{\Tor}{\operatorname{Tor}}
\newcommand{\tor}{\operatorname{tor}}
\newcommand{\Ann}{\operatorname{Ann}}
\newcommand{\Supp}{\operatorname{Supp}}
\newcommand{\Hom}{\operatorname{Hom}}
\newcommand{\End}{\operatorname{End}}
\newcommand{\coker}{\operatorname{coker}}
\newcommand{\limit}{\varprojlim}
\newcommand{\colimit}{%
  \mathop{\mathpalette\colimit@{\rightarrowfill@\textstyle}}\nmlimits@
}
\makeatother

\newcommand{\fraka}{\mathfrak a} % ideal
\newcommand{\frakb}{\mathfrak b} % ideal
\newcommand{\frakc}{\mathfrak c} % ideal
\newcommand{\frakm}{\mathfrak m} % maximal ideal
\newcommand{\frakp}{\mathfrak p} % prime ideal
\newcommand{\frakq}{\mathfrak q} % qrime ideal
\newcommand{\frakN}{\mathfrak N} % nilradical 
\newcommand{\frakR}{\mathfrak R} % jacobson radical

% Algebraic Geometry 
\newcommand{\bbA}{\mathbb A}
\newcommand{\calO}{\mathcal O}
\newcommand{\scrF}{\mathscr F}
\newcommand{\scrG}{\mathscr G}
\newcommand{\Top}{\mathfrak{Top}}

% General/Differential/Algebraic Topology 
\newcommand{\scrA}{\mathscr A}
\newcommand{\scrB}{\mathscr B}
\newcommand{\scrP}{\mathscr P}
\newcommand{\scrS}{\mathscr S}
\newcommand{\bbH}{\mathbb H}
\newcommand{\Int}{\operatorname{Int}}

% Miscellaneous
\newcommand{\psimeq}{\simeq_p}
\newcommand{\wt}[1]{\widetilde{#1}}
\newcommand{\wh}[1]{\widehat{#1}}
\newcommand{\calM}{\mathcal{M}}
\newcommand{\onto}{\twoheadrightarrow}
\newcommand{\into}{\hookrightarrow}
\newcommand{\Gr}{\operatorname{Gr}}
\newcommand{\Span}{\operatorname{Span}}
\newcommand{\op}{\text{op}}

\title{Derived Categories}
\author{Swayam Chube}
\date{\today}

\begin{document}
\maketitle

\section{Lecture 1}

\begin{definition}
    $f: X\to Y$ is a \emph{quasi-isomorphism} if $H^n(f): H^n(X)\to H^n(Y)$ is an isomorphism for all $n\in\Z$.
\end{definition}


\newcommand{\Mor}{\operatorname{Mor}}
\newcommand{\ob}{\operatorname{ob}}

\begin{definition}
    Let $\mathcal C$ be a category. We say that $\mathcal C$ is an \emph{additive category} if it satisfies the following: 
    \begin{enumerate}
        \item For all $A,B\in\ob(\mathcal C)$, the sets $\Hom_{\mathcal C}(A,B)$ is an abelian group.
        \item If $f: A\to B$, $g_1,g_2: B\to C$ and $h: C\to D$ are morphisms. Then, 
        \begin{equation*}
            h\circ(g_1 + g_2)\circ f = h\circ g_1\circ f + h\circ g_2\circ f.
        \end{equation*}
        \item There is a zero object in $\mathcal C$, that is, there is an object which is both the initial and terminal object.
        \item Finite products and finite coproducts exist in this category.
    \end{enumerate}
\end{definition}

\begin{definition}[$R$-additive category]
    Let $R$ be a commutative ring and $\mathcal C$ an additive category. We say that $\mathcal C$ is an $R$-category if 
    \begin{enumerate}
        \item for all $A,B\in\ob(\mathcal C)$, $\Hom_{\mathcal C}(A,B)$ is an $R$-module. 
        \item For $\in\Hom(A,B)$, $g_1,g_2\in\Hom(B,C)$ and $h\in\Hom(C,D)$, 
        \begin{align*}
            (r_1g_1 + r_2g_2)\circ(sf) = (r_1s)g_1\circ f + (r_2s)g_2\circ f\\
            sh\circ(r_1g_1 + r_2g_2) = (sr_1)h\circ g_1 + (sr_2)h\circ g_2.
        \end{align*}
    \end{enumerate}
\end{definition}

\begin{definition}
    Let $\mathcal C,\mathcal D$ be $R$-additive categories and $F:\mathcal C\to\mathcal D$ a functor. We say $F$ is $R$-linear if the map 
    \begin{equation*}
        \Hom_{\mathcal C}(A,B)\to\Hom_{\mathcal D}(F(A), F(B))
    \end{equation*}
    is $R$-linear.
\end{definition}


\begin{definition}
    Let $\mathcal A$ be an abelian category and $\mathcal C = \operatorname{ch}(\mathcal A)$, the category of chain complexs on $\mathcal A$. Let $C,D\in\ob(\mathcal C)$ be chain complexes and $f,g: C\to D$ be co-chain maps. We say that $f$ is homotopic to $g$, written $f\simeq g$ if if there are maps $s_n: C^n\to D^{n - 1}$ such that 
    \begin{equation*}
        f_n - g_n = s_{n + 1}\circ d_n^C + d_{n - 1}^D\circ s_n.
    \end{equation*}
    If $f\simeq 0$, then we say $f$ is nulhomotopic. Denote by $I(C,D)$ the collection of nulhomotpic maps.
\end{definition}

\begin{remark}
    If $u: B\to C$ and $v: D\to E$ are co-chain maps and $f: C\to D$ is nulhomotopic, then $f\circ u$ and $v\circ f$ are also nulhomotopic. 
\end{remark}

\begin{definition}[Homotopy Category]
    
\end{definition}

Let $R$ be a commutative ring and $C$ an additive $R$-category with $\Sigma:\mathcal C\to\mathcal C$ is an invertible functor, that is, an equivalence of categories

\begin{definition}
    A candidate triangle in $\mathcal C$ with respect to $\Sigma$ is a diagram of the form 
    \begin{equation*}
        \xymatrix {
            X\ar[r]^u & Y\ar[r]^v & Z\ar[r]^w & \Sigma X
        }
    \end{equation*}
    such that $v\circ u = w\circ v = 0$.

    A morphism $\eta:(X,Y,Z,u,v,w)\to (X',Y',Z',u',v',w')$ is a triple $f,g,h$ such that the following diagram commutes: 
    \begin{equation*}
        \xymatrix {
            X\ar[d]_f\ar[r]^u & Y\ar[d]_g\ar[r]^v & Z\ar[d]_h\ar[r]^w & \Sigma X\ar[d]_{\Sigma f}\\
            X'\ar[r]_{u'} & Y'\ar[r]_{v'} & Z'\ar[r]_{w'} & \Sigma X'
        }
    \end{equation*}
\end{definition}

\begin{definition}[Pre-triangulated Category]
    Let $\mathcal C$ be an additive $R$-category, $\Sigma$ an equivalence of categories $\mathcal C\to\mathcal C$. A \textbf{class} of candidate triangles with respect to $\Sigma$, called \emph{distinguished triangles}.
    \begin{enumerate}
        \item The candidate triangle $X\stackrel{\id}{\longrightarrow}X\to0\to\Sigma X$ is distinguished.
        \item If $(X,Y,Z,u,v,w)$ is distinguished and $(X',Y',Z',u',v',w')$ is a candidate triangle such that there is an isomorphism between them, then the latter is also a distinguished triangle.
        \item Let $u: X\to Y$ be any morphism. Then, there exists a distinguished triangle fo the form 
        \begin{equation*}
            \xymatrix {
                X\ar[r]^u & Y\ar[r] & Z\ar[r] & \Sigma X.
            }
        \end{equation*}
        \item (Rotation of Triangles) If $X\stackrel{u}{to}Y\stackrel{v}{\to} Z\stackrel{w}{\to}\Sigma X$ is a distinguished triangle, then so are 
        \begin{equation*}
            \xymatrix {
                Y\ar[r]^v & Z\ar[r]^w & \Sigma X\ar[r]^{-\Sigma u} & \Sigma Y\\
                \Sigma^{-1}Z\ar[r]^{-\Sigma^{-1}w} & X\ar[r]^u & Y\ar[r]^v & Z
            }
        \end{equation*}
        \item For any commutative diagram of the form 
        \begin{equation*}
            \xymatrix {
                X\ar[r]^u\ar[d]_f & Y\ar[r]^v\ar[d]_g & Z\ar[r]\ar@{.>}[d]_{\exists h} & \Sigma X\ar[d]_{\Sigma f}\\
                X'\ar[r]_{u'} & Y'\ar[r]_{v'} & Z'\ar[r]_{w'} & \Sigma X'
            }
        \end{equation*}
    \end{enumerate}
\end{definition}

\begin{proposition}
    Let $\mathcal C$ be a pre triangulated category and let 
    \begin{equation*}
        \xymatrix {
            X\ar[r]^u & Y\ar[r]^v & Z\ar[r]^w & \Sigma X
        }
    \end{equation*}
    be a triangle and $U\in\ob(\mathcal C)$. Then, the sequence 
    \begin{equation*}
        \xymatrix {
            \Hom(U,X)\ar[r]^{f_\ast} & \Hom(U,Y)\ar[r]^{g_\ast} & \Hom(U,Z)
        }
    \end{equation*}
    is exact.
\end{proposition}
\begin{proof}
    That $\im f_\ast\subseteq\ker g_\ast$ is straightforward. Conversely, suppose $t\in\ker g_\ast$, that is, $g\circ t = 0$. Consider now the commutative diagram 
    \begin{equation*}
        \xymatrix {
            U\ar[r]\ar[d] & 0\ar[r]\ar[d] & \Sigma^{-1}U\ar[r]\ar@{.>}[d] & \Sigma^{-1}U\ar[d]^{\Sigma^{-1}t}\\
            Y\ar[r]_g & Z\ar[r]_h & \Sigma^{-1}X\ar[r]_{\Sigma^{-1}f} & \Sigma^{-1}Y
        }
    \end{equation*}
\end{proof}

A similar result holds for $\Hom(-,U)$.

\begin{proposition}
    Consider the commutative diagram: 
    \begin{equation*}
        \xymatrix {
            X\ar[d]_f\ar[r] & Y\ar[d]_g\ar[r] & Z\ar[d]_h\ar[r] & \Sigma X\ar[d]_{\Sigma f}\\
            X'\ar[r] & Y'\ar[r] & Z'\ar[r] & \Sigma X'
        }
    \end{equation*}
    If $f$ and $g$ are isomorphisms, then so is $h$.
\end{proposition}
\begin{proof}
    Hom the entire thing using $\Hom(Z',-)$ and then use the five lemma to see that $h_\ast: \Hom(Z',Z)\to\Hom(Z',Z')$ is an isomorphism and thus, there is $\theta: Z'\to Z'$ such that $h\circ\theta = \id_{Z'}$.

    Similarly, using $\Hom(-,Z)$, there is $\delta: Z'\to Z$ such that $\delta\circ h = \id_Z$ whence $h$ is an isomorphism.
\end{proof}

\newcommand{\cone}{\operatorname{cone}}

\begin{definition}
    Let $\mathcal A$ be an abelian category and $K(A)$ be the \emph{homotopy category}. Let $f: B\to C$ be a chain map. The \emph{cone of $f$} is defined as 
    \begin{equation*}
        \cone(f)^n := B^{n + 1}\oplus C^n
    \end{equation*}
    and the boundary maps are given by $\partial^n(b,c) = \left(-\partial_{n + 1}^B(b), \partial_n^C(c) - f(b)\right)$. \textcolor{red}{Show that this is a complex}.
\end{definition}

\begin{definition}[Shift of a Complex]
    Let $C$ be a co-chain complex and $m$ an integer. Define the complex $C[m]$ by $C[m]^n = C^{m + n}$ and the map $d: C[m]^n\to C[m]^{n + 1}$ is given by $(-1)^md^{n + m}$.
\end{definition}

\begin{proposition}
    There is a short exact sequence of the form 
\end{proposition}

\end{document}