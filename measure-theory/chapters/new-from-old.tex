\section{Product Measures}

\begin{definition}[Measurable Rectangles, Product $\sigma$-algebra]
    Let $(X,\mathscr A,\mu)$ and $(Y,\mathscr B, \nu)$ be two measure spaces. Subsets of $X\times Y$ of the form $A\times B$ where $A\in\mathscr A$ and $B\in\mathscr B$ are called \textit{measurable rectangles}. Let $\mathcal R$ denote the collection of all measurable rectangles. The $\sigma$-algebra $\mathscr A\otimes\mathscr B := \frakM(\mathcal R)$ is called the \textit{product $\sigma$-algebra}.
\end{definition}

\begin{theorem}
    Let $\eta:\mathcal R\to[0,\infty]$ be defined by $\eta(A\times B) = \mu(A)\times\nu(B)$ where $A\in\mathscr A$ and $B\in\mathscr B$. Then, $\eta$ is a well defined measure on $\mathcal R$. Further, if $\mu$ and $\nu$ are $\sigma$-finite, there is a unique measure $\widetilde\eta$ on $A\otimes B$ that extends $\eta$.
\end{theorem}
\begin{proof}
    Obviously, $\eta(\emptyset) = 0$. We shall now show that $\eta$ is countably additive. Indeed, let $\{A_n\}$ be a sequence of sets in $\mathscr A$ and $\{B_n\}$ a sequence of sets in $\mathscr B$ such that the sequence $\{A_n\times B_n\}$ is disjoint and there is $A\in\mathscr A$ and $B\in\mathscr B$ such that $A\times B = \bigcup\limits_{k = 1}^\infty A_k\times B_k$. 

    Fix some $x\in A$ and define $S_x := \{n\in\N\mid x\in A_n\}$. Then, for every $y\in B$, $x\times y\in A\times B$ and therefore, there is an index $n$ $x\times y\in A_n\times B_n$. Note that this index must be unique lest the collection $\{A_n\times B_n\}$ not be disjoint. As a result, $B = \bigsqcup_{n\in S_x} B_n$. Now, using countable additivity, we have 
    \begin{equation*}
        \nu(B) = \sum_{n\in S_x}\nu(B_n)
    \end{equation*}
    for each $x\in A$. Then, we may write
    \begin{equation*}
        \chi_A\nu(B) = \sum_{n = 1}^\infty\chi_{A_n}\nu(B_n)
    \end{equation*}
    We have 
    \begin{align*}
        \mu(A)\nu(B) = \int_X\chi_A\nu(B)~d\mu = \int_X\left(\sum_{n = 1}^\infty\chi_{A_n}\nu(B_n)\right)~d\mu = \sum_{n = 1}^\infty\int_X\chi_{A_n}\nu(B_n)~d\mu = \sum_{n = 1}^\infty\mu(A_n)\nu(B_n)
    \end{align*}

    Finally, we must show that $\mathcal R$ is a semi-algebra. Obviously, $\emptyset, X\times Y\in\mathcal R$ and $(A_1\times B_1)\cap(A_2\times B_2) = (A_1\cap A_2)\times(B_1\cap B_2)$. Finally, for $A\times B\in\mathcal R$, 
    \begin{equation*}
        (X\times Y)\backslash(A\times B) = A\times(Y\backslash B)\sqcup(X\backslash A)\times B\sqcup(X\backslash A)\times(Y\backslash B)
    \end{equation*}
    The uniqueness now followes from Carath\'eodory Extension Theorem, since the measure $\eta$ is $\sigma$-finite on $\mathcal R$ and therefore $\sigma$-finite on the algebra generated by $\mathcal R$. This completes the proof.
\end{proof}

\begin{definition}
    Let $E\subseteq X\times Y$, $x\in X$ and $y\in Y$. Let 
    \begin{equation*}
        E_x := \{y\in Y\mid x\times y\in E\}\qquad E^y := \{x\in X\mid x\times y\in E\}
    \end{equation*}
    The set $E_x$ is called the $x$-section of $E$ and the set $E^y$ is called the $y$-section of $E$.
\end{definition}

\begin{theorem}\thlabel{thm:product-important}
    Let $E\in\mathscr A\otimes\mathscr B$. Then the following hold: 
    \begin{enumerate}[label=(\alph*)]
    \item $E_x\in\mathscr B$ and $E^y\in\mathscr A$ for every $x\in X$ and $y\in Y$. 
    \item The functions $x\mapsto\nu(E_x)$ and $y\mapsto\mu(E^y)$ are measurable functions on $X$ and $Y$, respectively.
    \item 
    \begin{equation*}
        \int_X\nu(E_x)~d\mu(x) = (\mu\times\nu)(E) = \int_Y\mu(E^y)~d\nu(y)
    \end{equation*}
    \end{enumerate}
\end{theorem}
\begin{proof}
\begin{enumerate}[label=(\alph*)]
\item Consider the set 
\begin{equation*}
    \frakM = \{E\in\mathscr A\otimes\mathscr B\mid E_x\in\mathscr B,~E_y\in\mathscr A,~\forall x\in X,~\forall y\in Y\}
\end{equation*}
It is not hard to see that $\frakM$ is a $\sigma$-algebra. Moreover, $\mathscr A\times\mathscr B\subseteq\frakM$, consequently, $\mathscr A\otimes\mathscr B\subseteq\frakM\subseteq\mathscr A\otimes\mathscr B$, which completes the proof.

\item Define the set 
\begin{equation*}
    \mathcal P = \{E\in\mathscr A\otimes\mathscr B\mid\text{for all $x\in X$ and $y\in Y$, the maps $x\mapsto\nu(E_x)$ and $y\mapsto\mu(E^y)$ are measurable}\}
\end{equation*}
For $E = A\times B$ where $A\in\mathscr A$, $B\in\mathscr B$, we have 
\begin{equation*}
    (x\mapsto\nu(E_x))(t) = \chi_A(t)\nu(B)\qquad(y\mapsto\mu(E^y))(t) = \chi_B(t)\mu(A)
\end{equation*}
and are both measurable. Thus, $\mathcal R\subseteq\mathcal P$. We shall now show that $\mathcal P$ is closed under disjoint union. Let $E_1,E_2$ be disjoint sets in $\mathcal P$. For any $x\in X$, we have 
\begin{equation*}
    \nu((E_1\sqcup E_2)_x) = \nu((E_1)_x\sqcup(E_2)_x) = \nu((E_1)_x) + \nu((E_2)_x)
\end{equation*}
Thus, $(x\mapsto\nu((E_1\cup E_2)_x)) = (x\mapsto\nu((E_1)_x)) + (x\mapsto\nu((E_2)_x))$, consequently, is measurable. A similar result can be established for $y$-sections. Now, since $\mathcal P$ is closed under disjoint unions, $\mathcal A(\mathcal R)\subseteq\mathcal P$, that is, the algebra generated by the semi-algebra $\mathcal R$ is contained in $\mathcal P$.

Finally, we shall show that $\mathcal P$ is a monotone class. Let $\{E_n\}_{n = 1}^\infty$ be an ascending chain of sets from $\mathscr A\otimes\mathscr B$ in $\mathcal P$. Then, it is not hard to argue, using the continuity of measure, that the map $(x\mapsto\nu(E_x))$ is the pointwise limit of the sequence of maps $\{(x\mapsto\nu((E_n)_x))\}$.

Since $\mathcal P$ is a monotone class containing $\mathcal A(\mathcal R)$, it must contain the monotone class generated by $\mathcal A(\mathcal R)$, which due to the Monotone Class Theorem is the same as the $\sigma$-algebra generated by $\mathcal A(\mathcal R)$, which is precisely $\mathscr A\otimes\mathscr B$. This completes the proof.

\item Define the set 
\begin{equation*}
    \mathcal P = \{E\in\mathscr A\otimes\mathscr B\mid (c)\text{ holds}\}
\end{equation*}
Obviously, $\mathcal R\subseteq\mathcal P$. First, we shall show that $\mathcal P$ is closed under finite disjoint union. Indeed, let $E = E_1\sqcup E_2$. Then, for each $x\in X$, $E_x = (E_1)_x\sqcup(E_2)_x$, whence 
\begin{equation*}
    \int_X\nu(E_x)~d\mu = \int_X\nu((E_1)_x\sqcup(E_2)_x)~d\mu = \int_X\nu((E_1)_x)~d\mu + \int_X\nu((E_2)_x)~d\mu = (\mu\times\nu)(E_1) + (\mu\times\nu)(E_2)
\end{equation*}

Finally, we must show that $\mathcal P$ is a monotone class. Let $\{E_n\}_{n = 1}^\infty$ be an ascending chain of sets in $\mathscr A\otimes\mathscr B$. As seen in part (b), the sequence $\{(E_n)_x\}_{n = 1}^\infty$ is increasing the sequence of functions $\{x\mapsto\nu((E_n)_x)\}_{n = 1}^\infty$ converges pointwise to the function $x\mapsto\nu(E_x)$. Then, using the Monotone Convergence Theorem, 
\begin{equation*}
    \int_X\nu(E_x)~d\mu = \lim_{n\to\infty}\int_X\nu((E_n)_x)~d\mu = \lim_{n\to\infty}(\mu\times\nu)(E_n) = (\mu\times\nu)(E)
\end{equation*}
where the last equality follows from the continuity of measure.

The part for descending chains is a bit tricky since the continuity of measure does not apply readily. Let $\{E_n\}_{n = 1}^\infty$ be a descending chain of sets in $\mathcal P$. When $\mu$ and $\nu$ are assumed to be finite, continuity of measure readily applies and the pointwise limit of the sequence of functions $\{x\mapsto\nu((E_n)_x)\}_{n = 1}^\infty$ is the function $x\mapsto\nu(E_x)$. The convergence of integrals would then follow from the Dominated Convergence Theorem. Now, let us consider the case when $\mu$ and $\nu$ are $\sigma$-finite. In which case, there are collections of disjoint sets $\{A_i\}_{i = 1}^\infty$ and $\{B_j\}_{j = 1}^\infty$ in $\mathscr A$ and $\mathscr B$ such that $\mu(A_i) < \infty$, $\nu(B_j) < \infty$, $\bigsqcup\limits_{i = 1}^\infty A_i = X$ and $\bigsqcup\limits_{j = 1}^\infty B_j = Y$. 

Then, due to the previous discussion, we would have 
\begin{equation*}
    (\mu\times\nu)(E\cap(A_i\times B_j)) = \int_X\nu((E\cap(A_i\times B_j))_x)~d\mu
\end{equation*}
Consequently, using countable additivity, 
\begin{align*}
    (\mu\times\nu)(E) &= \sum_{i = 1}^\infty\sum_{j = 1}^\infty(\mu\times\nu)(E\cap(A_i\times B_j))\\
    &= \sum_{i = 1}^\infty\sum_{j = 1}^\infty\int_X\nu((E\cap(A_i\times B_j))_x)~d\mu\\
    &= \int_X\left(\sum_{i = 1}^\infty\sum_{j = 1}^\infty\nu((E\cap(A_i\times B_j))_x)\right)~d\mu\\
    &= \int_X\nu(E_x)~d\mu
\end{align*}
where the second last equality follows from the Monotone Convergence Theorem.
\end{enumerate}
This completes the proof.
\end{proof}

\section{Fubini's Theorem}
Throughout this section, let $(X,\mathscr A,\mu)$ and $(Y,\mathscr B,\nu)$ be $\sigma$-finite measure spaces.

\begin{theorem}[Fubini]\thlabel{thm:fubini-ver1}
    Let $f: (X\times Y,\mathscr A\otimes\mathscr B)\to[0,\infty]$ be a nonnegative measurable function. Then, 
    \begin{enumerate}[label=(\alph*)]
    \item for $x_0\in X$ and $y_0\in Y$, the maps $x\mapsto f(x,y_0)$ and $y\mapsto f(x_0,y)$ are measurable 
    \item the maps $x\mapsto\int_Yf(x,y)~d\nu(y)$ and $y\mapsto\int_Xf(x,y)~d\mu(x)$ are measurable
    \item 
    \begin{equation*}
        \int_X\left(\int_Y f(x,y)~d\nu(y)\right)~d\mu(x) = \int_{X\times Y}f~d(\mu\times\nu) = \int_Y\left(\int_X f(x,y)~d\mu(x)\right)~d\nu(y)
    \end{equation*}
    \end{enumerate}
\end{theorem}
\begin{proof}
Due to \thref{thm:abstract-simple-approximation}, there is an increasing sequence of measurable simple functions $\{\varphi_n\}_{n = 1}^\infty$ converging to $f$ pointwise on $X\times Y$. It follows from \thref{thm:product-important} that the section of a simple measurable function is simple measurable. 

\begin{enumerate}[label=(\alph*)]
\item It is not hard to argue that the sections $\{\varphi_n(x,y_0)\}_{n = 1}^\infty$ converges pointwise to $f(x,y_0)$. Consequently, the map $x\mapsto f(x,y_0)$ is measurable.

\item Due to the Monotone Convergence Theorem, 
\begin{equation*}
    \int_Y f(x,y)~d\nu(y) = \lim_{n\to\infty}\int_Y\varphi_n(x, y)~d\nu(y)
\end{equation*}
We contend that whenever $\varphi$ is a simple function, the map $x\mapsto\int_Y\varphi(x,y)~d\nu(y)$ is a measurable function with canonical representation $\sum_{k = 1}^n a_k\chi_{E_k}$. Then, for each $x\in X$, the $x$-section of $\varphi$ is given by 
\begin{equation*}
    \varphi_x = \sum_{k = 1}^n a_k\chi_{(E_k)_x}
\end{equation*}
Consequently, 
\begin{equation*}
    \int_Y\varphi_x~d\nu(y) = \sum_{k = 1}^n a_k\nu((E_k)_x)
\end{equation*}
Due to \thref{thm:product-important}, each map $x\mapsto\nu((E_k)_x)$ is measurable and therefore, $\int_Y\varphi_x~d\nu(y)$ is a measurable function of $x$.

This now implies that $\int_Yf(x,y)~d\nu(y)$ is the pointwise limit of measurable functions and is therefore measurable.

\item We contend that for a simple measurable function $\varphi$, 
\begin{equation*}
    \int_{X\times Y} \varphi~d(\mu\times\nu) = \int_X\left(\int_Y\varphi~d\nu(y)\right)~d\mu(x)
\end{equation*}
Indeed, this follows from \thref{thm:product-important}, since
\begin{equation*}
    \int_X\left(\int_Y\varphi_x~d\nu(y)\right)~d\mu(x) = \int_X\left(\sum_{k = 1}^na_k\nu((E_k)_x)\right)~d\mu(x) = \sum_{k = 1}^n a_k(\mu\times\nu)(E_k) = \int_{X\times Y}\varphi~d(\mu\times\nu)
\end{equation*}
Finally, from the Monotone Convergence Theorem, we have 
\begin{align*}
    \int_{X\times Y} f~d(\mu\times\nu) &= \lim_{n\to\infty}\int_{X\times Y}\varphi_n~d(\mu\times\nu)\\
    &= \lim_{n\to\infty}\int_X\left(\int_Y\varphi_n~d\nu(y)\right)~d\mu(x)\\
    &= \int_X\left(\lim_{n\to\infty}\int_Y\varphi_n~d\nu(y)\right)~d\mu(x)\\
    &= \int_X\left(\int_Y\lim_{n\to\infty}\varphi_n~d\nu(y)\right)~d\mu(x)\\
    &= \int_X\left(\int_Y f~d\nu(y)\right)~d\mu(x)
\end{align*}
\end{enumerate}
This completes the proof. 
\end{proof}
\begin{corollary}
    Let $f: X\times Y\to[-\infty,\infty]$ be measurable. Then, the following are equivalent: 
    \begin{enumerate}[label=(\alph*)]
    \item $f$ is integrable 
    \item $\displaystyle\int_X\left(\int_Y|f|~d\nu(y)\right)~d\mu(x) < \infty$
    \item $\displaystyle\int_Y\left(\int_X|f|~d\mu(x)\right)~d\nu(y) < \infty$
    \end{enumerate}
\end{corollary}
\begin{proof}
    That $(a)\Longrightarrow(b)\wedge(c)$ is trivial. To see that $(b)\Longrightarrow(a)$, note that $|f|$ is nonnegative and due to \thref{thm:fubini-ver1}, we have 
    \begin{equation*}
        \int_{X\times Y}|f|~d(\mu\times\nu) = \int_X\left(\int_Y|f|~d\nu(y)\right)~d\mu(x) < \infty
    \end{equation*}
    and $f$ is integrable. The proof of $(c)\Longrightarrow(a)$ is analogous.
\end{proof}


\begin{theorem}\thlabel{thm:fubini-ver2}
    Let $f: X\times Y\to[-\infty,\infty]$ be integrable. Then, 
    \begin{enumerate}[label=(\alph*)]
    \item The functions $x\mapsto f(x,y)$ and $y\mapsto f(x,y)$ are integrable for almost all $y\in Y$ and $x\in X$ respectively
    \item The functions 
    \begin{equation*}
        y\mapsto\int_X f(x,y)~d\mu(x)\text{ and } x\mapsto\int_Y f(x,y)~d\nu(y)
    \end{equation*}
    are $\nu$,$\mu$-integrable respectively
    \item 
    \begin{equation*}
        \int_Y\left(\int_X f(x,y)~d\mu(x)\right)~d\nu(y) = \int_X\left(\int_Y f(x,y)~d\nu(y)\right)~d\mu(x)
    \end{equation*}
    \end{enumerate}
\end{theorem}
\begin{proof}
    Since $f$ is integrable, so are $f^+$ and $f^-$. Now, both $f^+$ and $f^-$ are nonnegative functions and therefore, 
    \begin{equation*}
        \int_X\left(\int_Y f^+~d\nu(y)\right)~d\mu(x) = \int_{X\times Y} f^+~d(\mu\times\nu) < \infty
    \end{equation*}
    Consequently, the map $x\mapsto\int_Y f^+~d\nu(y)$ is $\mu$-integrable. Similarly, the map $x\mapsto\int_Y f^-~d\nu(y)$ is $\mu$-integrable. Therefore, the map $x\mapsto\int_Y f~d\nu(y)$ is integrable, since it is their difference.

    Further, in order to be integrable, they must be finite a.e., that is, $x\mapsto\int_Yf^+~d\nu(y)$ and $x\mapsto\int_Y f^-~d\nu(y)$ are finite a.e.. Hence, $y\mapsto f^+$ and $y\mapsto f^-$ are integrable for almost all $x\in X$, which implies $y\mapsto f = f^+ - f^-$ is integrable for almost all $x\in X$.

    Now, we have 
    \begin{align*}
        \int_{X\times Y}f~d(\mu\times\nu) &= \int_{X\times Y}f^+~d(\mu\times\nu) - \int_{X\times Y} f^-~d(\mu\times\nu)\\
        &= \int_X\left(\int_Yf^+~d\nu(y)\right)~d\mu(x) - \int_X\left(\int_Y f^-~d\nu(y)\right)~d\mu(x)\\
        &= \int_X\left(\int_Y f~d\nu(y)\right)~d\mu(x)
    \end{align*}
    This completes the proof.
\end{proof}

% \begin{theorem}[Fubini]
%     Let $(X,\mathscr A, \mu)$ and $(Y,\mathscr B, \nu)$ be $\sigma$-finite measure spaces. Let $f: X\times Y\to[-\infty,\infty]$ be integrable on $X\times Y$ with respect to $\mu\times\nu$ on $\mathscr A\otimes\mathscr B$. Then, for almost all $x\in X$, the $x$-sections $f(x,\cdot)$ are integrable over $Y$. Further, 
%     \begin{equation*}
%         \int_{X\times Y}f~d(\mu\times\nu) = \int_X\left(\int_Yf(x,y)~d\nu(y)\right)~d\mu(x)
%     \end{equation*}
% \end{theorem}
% Note that a similar result holds for $y$-sections, $f(\cdot,y)$. Further, a similar statement is true if $f$ is nonnegative measurable function without assuming $f$ is integrable.

Putting \thref{thm:fubini-ver1} and \thref{thm:fubini-ver2} together, we have the following:

\begin{theorem}[Fubini]\thlabel{thm:fubini-combined}
    Let $(X,\mathscr A,\mu)$ and $(Y,\mathscr B,\nu)$ be $\sigma$-finite measure spaces. Let $f$ be an extended real valued measurable function on $X\times Y$ and $f$ satisfies any one of the following conditions: 
    \begin{enumerate}[label=(\alph*)]
    \item $f$ is non-negative 
    \item $f$ is integrable 
    \item 
    \begin{equation*}
        \int_X\left(\int_Y|f(x,y)|~d\nu(y)\right)~d\mu(x) < \infty
    \end{equation*}
    \item 
    \begin{equation*}
        \int_Y\left(\int_X|f(x,y)|~d\mu(x)\right)~d\nu(y) < \infty
    \end{equation*}
    \end{enumerate}
    Then, 
    \begin{equation*}
        \int_{X\times Y}f(x,y)~d(\mu\times\nu) = \int_X\left(\int_Yf(x,y)~d\nu(y)\right)~d\mu(x) = \int_Y\left(\int_Xf(x,y)~d\mu(x)\right)~d\nu(y)
    \end{equation*}
\end{theorem}

\subsection*{Examples of Fubini's Theorem}

\begin{example}
    Let $f$ be a non-negative measurable function on a $\sigma$-finite measure space $(X,\frakM,\mu)$. Show that 
    \begin{equation*}
        \int_X f~d\mu = \int_0^\infty\mu\left(\{x\in X\mid 0\le t\le f(x)\}\right)~dm(t)
    \end{equation*}
    where $m$ is the Lebesgue Measure.
\end{example}
\begin{proof}
    Define the region $A\subseteq X\times\R$ as
    \begin{equation*}
        A = \{x\times y\in X\times\R\mid 0\le t\le f(x)\}
    \end{equation*}
    Note that both $\mu$ and the Lebesgue measure are $\sigma$-finite and therefore, we may use Fubini's Theorem. Then, we have 
    \begin{align*}
        (\mu\times m)(A) &= \int_X\chi_A~d(\mu\times m)\\
        &= \int_X\left(\int_{\R}\chi_A~dm\right)d\mu\\
        &= \int_X f~d\mu
    \end{align*}
    and similarly, 
    \begin{align*}
        (\mu\times m)(A) &= \int_X\chi_A~d(\mu\times m)\\
        &= \int_{\R}\left(\int_X\chi_A~d\mu\right)~dm\\
        &= \int_{\R}\mu\left(\{x\in X\mid 0\le t\le f(x)\}\right)~dm(t)\\
        &= \int_0^\infty\mu\left(\{x\in X\mid 0\le t\le f(x)\}\right)~dm(t)
    \end{align*}
    The last equality is obvious. This completes the proof.
\end{proof}

\begin{example}
    Let $f:\R^2\to\R$ be defined by 
    \begin{equation*}
        f(x,y) = (\sin x)\chi_{\{x\times y\mid y < x < y + 2\pi\}}
    \end{equation*}
    Then, the iterated integrals are not equal. One of them diverges while the other is equal to $0$. The reason being that the integrand is not integrable over $\R^2$ with the product measure.
\end{example}