\begin{theorem}\thlabel{thm:equivalence-measurable}
    Let the function $f$ have a measurable domain $E$. Then the following statements are equivalent: 
    \begin{enumerate}[label=(\roman*)]
        \item For each real number $c$, the set $E_1(c) = \{x\in E\mid f(x) > c\}$ is measurable 
        \item For each real number $c$, the set $E_2(c) = \{x\in E\mid f(x)\ge c\}$ is measurable 
        \item For each real number $c$, the set $E_3(c) = \{x\in E\mid f(x) < c\}$ is measurable 
        \item For each real number $c$, the set $E_4(c) = \{x\in E\mid f(x)\le c\}$ is measurable
    \end{enumerate}
    Each of the above implies that for each \textbf{extended real number} $c$, the set 
    \begin{equation*}
        \{x\in E\mid f(x) = c\}
    \end{equation*}
    is measurable.
\end{theorem}
\begin{proof}
We shall show that $(i)\Longrightarrow(iv)\Longrightarrow(iii)\Longrightarrow(ii)\Longrightarrow(i)$. This follows from the following equalities: 
\begin{align*}
    E_4(c) &= E\backslash E_1(c)\\
    E_3(c) &= \bigcup_{n = 1}^\infty E_4\left(c - \frac{1}{n}\right)\\
    E_2(c) &= E\backslash E_3(c)\\
    E_1(c) &= \bigcup_{n = 1}^\infty E_2\left(c + \frac{1}{n}\right)
\end{align*}
Now, if $c$ is finite, then 
\begin{equation*}
    \{x\in E\mid f(x) = c\} = E_2(c)\cap E_4(c)
\end{equation*}
and is measurable. Next, if $c = \infty$, then 
\begin{equation*}
    \{x\in E\mid f(x) = c\} = \bigcap_{n\in\N} E_1(n)
\end{equation*}
\end{proof}

We denote the extended real line by $[-\infty,\infty]$. The arithmetic on this real line is well known and we omit its discussion.

\begin{definition}[Measurable Function]
    An extended real-valued function $f: E\to[-\infty,\infty]$ defined on $E$ is said to be \textit{Lebesgue measurable}, or just \textit{measurable} provided $E$ is measurable and it satisfies one of the four statements of \thref{thm:equivalence-measurable}.
\end{definition}

\begin{proposition}
    Let $E$ be a measurable set and $f: E\to[-\infty,\infty]$. Then $f$ is measurable if and only if for each open set $\mathcal O$, $f^{-1}(\mathcal O)$ is measurable.
\end{proposition}
\begin{proof}
    Suppose the inverse image of each open set is measurable, consequently, the inverse image of $(c,\infty)$ is open for all $c\in\R$, and due to \thref{thm:equivalence-measurable}, $f$ is measurable. 

    Now suppose $f$ is measurable. Then, 
    \begin{equation*}
        f^{-1}((a,b)) = f^{-1}((-\infty, b))\cap f^{-1}((a,\infty))
    \end{equation*}
    is measurable. Since $\mathcal O$ is open, it can be written as the union of countably many disjoint open intervals, and thus, the inverse image of $\mathcal O$ can be written as the union of countably many disjoint measurable sets and is measurable. This completes the proof.
\end{proof}

\begin{corollary}
    If $E$ is measurable and $f: E\to\R$\footnote{Note that we are not talking about an extended real valued function here} is continuous, then it is measurable.
\end{corollary}

\begin{proposition}
    A monotone function that is defined on an interval is measurable.
\end{proposition}

\begin{proposition}
    Let $f, g: E\to[-\infty,\infty]$
    \begin{enumerate}[label=(\roman*)]
        \item If $f$ is measurable on $E$ and $f = g$ a.e. on $E$, then $g$ is measurable on $E$
        \item For a measurable subset $D$ of $E$, $f$ is measurable on $E$ if and only if the restrictions of $f$ to $D$ and $E\backslash D$ are measurable.
    \end{enumerate}
\end{proposition}
\begin{proof}
\begin{enumerate}[label=(\roman*)]
\item Let $A = \{x\in E\mid f(x)\ne g(x)\}$. It is known that $m(A) = 0$. Then, for any $c\in\R$, 
\begin{equation*}
    \{x\in E\mid g(x) > c\} = \{x\in A\mid g(x) > c\}\cup\left(\{x\in E\mid f(x) > c\}\cap(E\backslash A)\right)
\end{equation*}

Since $\{x\in A\mid g(x) > c\}\subseteq A$, it has outer measure $0$ and is measurable. Further, since measurable sets are closed under intersection, $\{x\in E\mid f(x) > c\}\cap(E\backslash A)$ is measurable. As a result, their union is measurable.

\item Simply note that 
\begin{equation*}
    \{x\in E\mid f(x) > c\} = \{x\in D\mid f(x) > c\}\cup\{x\in E\backslash D\mid f(x) > c\}
\end{equation*}
\end{enumerate}
\end{proof}

Assertion $(i)$ of the previous theorem allows us to extend a measurable function to another measurable function. Take for example two extended real valued functions $f$ and $g$ finite on $E\backslash E_0$ where $E_0$ has measurae $0$. Then, $f + g$ is defined on $E\backslash E_0$. If we were to show that $f + g$ is measurable on $E\backslash E_0$, any extension of it to $E$ would also be measurable.

\begin{theorem}
    Let $f$ and $g$ be measurable (extended real valued) functions on $E$ that are finite a.e. on $E$. Then 
    \begin{enumerate}[label=(\alph*)]
        \item For any $\alpha$ and $\beta$, $\alpha f + \beta g$ is measurable on $E$ 
        \item $fg$ is measurable on $E$
    \end{enumerate}
\end{theorem}
\begin{proof}
Note that since $f,g$ are finite a.e. on $E$, we may suppose, due to the discussion preceeding the statement of the theorem that $f,g$ are finite everywhere on $E$. Then, it shall suffice to show that both $\alpha f$ and $f + g$ are measurable. Without loss of generality, let $\alpha > 0$. Then 
\begin{equation*}
    \{x\in E\mid \alpha f(x) < c\} = \{x\in E\mid f(x) < c/\alpha\}
\end{equation*}
Thus, $\alpha f(x)$ is measurable. 

Next, we shall show that $f + g$ is measurable for which it suffices to show that $\{x\in E\mid f(x) + g(x) < c\}$ is measurable. We claim that 
\begin{equation*}
    \{x\in E\mid f(x) + g(x) < c\} = \bigcup_{q\in\Q}\{x\in E\mid f(x) < q\}\cap\{x\in E\mid g(x) < c - q\}
\end{equation*}
Indeed, if $f(x) + g(x) < c$, then there is a rational $q$ such that $f(x) < q < c - g(x)$, consequently, $f(x) < q$ and $g(x) < c - q$. Since the rationals are countable, we have the desired conclusion.

Finally, note that
\begin{equation*}
    \{x\in E\mid f(x)^2 > c\} = 
    \begin{cases}
        \{x\in E\mid f(x) > \sqrt{c}\}\cup\{x\in E\mid f(x) < -\sqrt{c}\} & c\ge 0\\
        E & c < 0
    \end{cases}
\end{equation*}
Thus, $f^2$ is also measurable. Now, simply note that 
\begin{equation*}
    fg = \frac{1}{2}\left((f + g)^2 - f^2 - g^2\right)
\end{equation*}
whence $fg$ is measurable.
\end{proof}

\begin{proposition}
    The composition of two measurable functions need not be measurable.
\end{proposition}
\begin{proof}
    Define the function 
    \begin{equation*}
        \psi(x) = 
        \begin{cases}
            \varphi(x) + x & x\in [0,1]\\
            2x & x\notin[0,1]
        \end{cases}
    \end{equation*}
    where $\varphi$ is the Cantor-Lebesgue function. We have already shown that $\psi$ is continuous and strictly increasing, and is therefore a bijection from $\R$ to $\R$. We have also shown that there is $A\subseteq\mathbf C$ such that $\psi(A)$ is not measurable. Let $\chi_A$ be the characteristic function from $\R$ to $\R$ for the set $A$. We claim that $f = \chi_A\circ\psi^{-1}$ is not measurable, which would prove the statement of the proposition, since $\psi^{-1}$ is continuous and thus measurable.

    Indeed, 
    \begin{equation*}
        \{x\in\R\mid f(x) > 0.5\} = \{x\in\R\mid\psi^{-1}(x)\in A\} = \psi(A)
    \end{equation*}
    which is not measurable. This shows that $f$ is not measurable.
\end{proof}

\begin{proposition}
    Let $E$ be measurable, $g: E\to\R$ be measurable and $f:\R\to\R$ be continuous. Then the composition $f\circ g: E\to\R$ is measurable.
\end{proposition}
\begin{proof}
    Let $\mathcal O$ be an open set in $\R$. Then $(f\circ g)^{-1}(\mathcal O) = g^{-1}(f^{-1}(\mathcal O))$. Since $f$ is contiuous, $f^{-1}(\mathcal O)$ is open and thus $g^{-1}(f^{-1}(\mathcal O))$ is measurable, implying that $f\circ g$ is measurable.
\end{proof}

\begin{proposition}
    Let $f: E\to[-\infty,\infty]$ be an extended real valued measurable function on a measurable domain $E$. Then, for every Borel set $B$, $f^{-1}(B)$ is measurable.
\end{proposition}
\begin{proof}
    Let $\mathfrak M = \{A\subseteq\R\mid f^{-1}(A)\text{ is measurable}\}$. It is not hard to show that $\mathfrak M$ is a $\sigma$-algebra. Further, $\mathfrak M$ contains all the open sets in $\R$, consequently, contains all the Borel sets in $\R$. This completes the proof.
\end{proof}

\begin{theorem}
    Let $E$ be measurable and $\{f_n: E\to[-\infty,\infty]\}$ be a sequence of measurable functions. Define 
    \begin{equation*}
        g = \sup f_n\quad\text{and}\quad h = \limsup_{n\to\infty} f_n
    \end{equation*}
    Then $g$ and $h$ are measurable.
\end{theorem}
\begin{proof}
    Let $c\in\R$. Then we have 
    \begin{equation*}
        \{x\in E\mid g(x) > c\} = \bigcup_{n\in\N}\{x\in E\mid f_n(x)>c\}
    \end{equation*}
    and 
    \begin{equation*}
        \{x\in E\mid h(x) > c\} = \bigcap_{n\in\N}\bigcup_{m = n}^\infty\{x\in E\mid f_m(x) > c\}
    \end{equation*}
    both of which are measurable. This completes the proof.
\end{proof}

Similarly, one can show that $\inf f_n$ and $\liminf f_n$ are measurable.

\begin{corollary}
    Let $E$ be measurable and $f,g: E\to[-\infty,\infty]$ be measurable functions. Then $\min\{f,g\}$ and $\max\{f,g\}$ are measurable.
\end{corollary}
\begin{proof}
    Note that 
    \begin{align*}
        \max\{f,g\} &= \limsup\{f,g,f,g,f,g,\ldots\}\\
        \min\{f,g\} &= \liminf\{f,g,f,g,f,g,\ldots\}
    \end{align*}
\end{proof}

For a function $f: E\to[-\infty,\infty]$, define $f^+ = \max\{f, 0\}$ and $f^- = -\min\{f, 0\}$. It is obious that $f^+$ and $f^-$ are nonnegative functions and $f = f^+ - f^-$ on $E$.


\section{Pointwise Limits and Simple Approximation}

We begin by reacalling some definitions from Real Analysis.

\begin{definition}
    For a sequence $\{f_n\}$ of functions with common domain $E$ and a function $f$ on $E$ and a subset $A$ of $E$, we say that 
    \begin{enumerate}[label=(\roman*)]
        \item The sequence $\{f_n\}$ converges to $f$ pointwise on $A$ provided $\lim\limits_{n\to\infty} f_n(x) = f(x)$ for all $x\in A$
        \item The sequence $\{f_n\}$ covnerges to $f$ pointwise a.e. on $A$ provided it converges to $f$ pointwise on $A\backslash B$ where $m(B) = 0$ 
        \item The sequence $\{f_n\}$ converges to $f$ uniformly on $A$ provided for each $\varepsilon > 0$ there is an index $N$ for which 
        \begin{equation*}
            |f - f_n| < \varepsilon \text{ on $A$ for all $n\ge N$}
        \end{equation*}
    \end{enumerate}
\end{definition}

Recall that the pointwise limit of a Riemann integrable function need not be Riemann integrable. Measurable functions on the other hand are better behaved under pointwise limits.

\begin{proposition}
    Let $\{f_n\}$ be a sequence of measurable functions on $E$ that converges pointwise a.e. on $E$ to the function $f$. Then $f$ is measurable.
\end{proposition}
\begin{proof}
    Let $E_0$ be such that $\{f_n\}$ converges pointwise on $E\backslash E_0$. If we show that $f$ is measurable on $E\backslash E_0$, then every extension of $f$ to $E$ is measurable. Therefore, without loss of generality, we may suppose that $\{f_n\}$ converges pointwise on $E$. 

    If $\{f_n\}$ converges pointwise, then $f = \limsup f_n$ and is measurable.
\end{proof}

The \textbf{characteristic function} of a set $A$, denoted $\chi_A$ is defined as 
\begin{equation*}
    \chi_A = 
    \begin{cases}
        1 & x\in A\\
        0 & x\notin A
    \end{cases}
\end{equation*}

\begin{definition}[Simple Function]
    A real-valued function $\varphi$ defined on a measurable set $E$ is called \textit{simple} if it is measurable and takes only a finite number of distinct values.
\end{definition}

It is not hard to infer from the definition that a simple function can be represented as 
\begin{equation*}
    \varphi = \sum_{k = 1}^n c_k\chi_{E_k}
\end{equation*}
where $E_k$'s are disjoint and measurable. This is known as the \textbf{canonical representation of the simple function}.

Note that the simple functions that we consider are real valued and not extended real valued.

\begin{lemma}[Simple Approximation Lemma]\thlabel{lem:simple-approximation}
    Let $f$ be measurable real-valued bounded function on $E$. Then for each $\varepsilon > 0$, there are simple functions $\varphi_\varepsilon$ and $\psi_\varepsilon$ defined on $E$ such that 
    \begin{equation*}
        \varphi_\varepsilon\le f\le\psi_\varepsilon\text{ and } 0\le\psi_\varepsilon - \varphi_\varepsilon < \varepsilon
    \end{equation*}
\end{lemma}
\begin{proof}
    Since $f$ is bounded, there is an open interval $(c,d)$ containing $f(E)$. Consider a partition 
    \begin{equation*}
        c = y_0 < y_1 < \cdots < y_{n - 1} < y_n = d
    \end{equation*}
    such that $y_k - y_{k - 1} < \varepsilon$. Define $E_k = f^{-1}([y_{k - 1}, y_k))$ and the functions 
    \begin{equation*}
        \psi = \sum_{k = 1}^n y_k\chi_{E_k}\quad\text{and}\quad\varphi = \sum_{k = 1}^{n}y_{k - 1}\chi_{E_k}
    \end{equation*}

    Obviously, for all $x$, $\psi - \varphi < \varepsilon$, further, there is a unique $k$ such that $x\in E_k$. Then $\varphi(x) = y_{k - 1} < f(x) < y_k = \psi(x)$. This completes the proof.
\end{proof}

\begin{theorem}[Simple Approximation Theorem]
    Let $E$ be measurable and $f: E\to[-\infty,\infty]$ is measurable if and only if there is a sequence $\{\varphi_n\}$ of simple functions on $E$ which converge pointwise on $E$ to $f$ and has the property that $|\varphi_n|\le|f|$ on $E$ for all $n$. If $f$ is nonnegative, we may choose $\{\varphi_n\}$ to be increasing.
\end{theorem}
\begin{proof}
    If there is a sequence of simple functions that converge to $f$, then $f$ is measurable since it is the pointwise limit of measurable functions.

    Now suppose $f$ is measurable. We shall first prove the statement in the case when $f$ is nonnegative. Let $E_n = \{x\in E\mid f(x)\le n\}$, which is measurable. Due to \thref{lem:simple-approximation}, there are simple functions $\varphi_n$ and $\psi_n$ on $E_n$ such that 
    \begin{equation*}
        0\le\varphi_n\le f\le\psi_n\quad\text{and}\quad 0\le\psi_n - \varphi_n < 1/n\text{ on $E_n$}
    \end{equation*}

    Extend both $\varphi_n$ and $\psi_n$ to $E$ by defining $\varphi_n(x) = \psi_n(x) = n$ when $x\in E\backslash E_n$. It is not hard to show that the sequences $\{\varphi_n\}$ and $\{\psi_n\}$ converge pointwise to $f$ on $E$. Now, let $\varphi_n' = \max\{\varphi_1,\ldots,\varphi_n\}$. Then, $\{\varphi_n'\}$ is an increasing sequence of functions that converge to $f$.

    Now, we shall prove the problem statement for a general function $f = f^+ - f^-$. Since $f^+$ and $f^-$ are nonnegative functions, there are sequences $\varphi_n^+$ and $\varphi_n^-$ of simple functions satisfying the assertion of the theorem. Note that the functions $\varphi^+_n$ and $\varphi_n^-$ are defined in such a way that they do not both take nonzero values at the same point. Then, 
    \begin{equation*}
        |\varphi_n^+ - \varphi_n^-| = \varphi_n^+ + \varphi_n^-\le f^+ + f^- = |f|
    \end{equation*}
    which completes the proof.
\end{proof}

\subsection{Step Functions}

A step function is a special kind of simple function. It is of the form $\sum_{k = 1}^n \alpha_k\chi_{E_k}$ where each $E_k$ is an interval. One must note that step functions are not as strong as simple functions, that is to say that they do not approximate measurable functions as well as simple functions do.

\begin{theorem}[Step Approximation Theorem]\thlabel{thm:step-approximation}
    Let $I$ be a closed, bounded interval and $f: I\to\R$ a bounded measurable function. Let $\varepsilon, \delta > 0$. Then there is a step function $h$ on $I$ and a measurable subset $F$ of $I$ for which 
    \begin{equation*}
        |h - f| < \varepsilon \text{ and } m(I\backslash F) < \delta
    \end{equation*}
\end{theorem}

To prove the above theorem, we require a series of lemmas. 

\begin{lemma}
    Let $E\subseteq I$ be measurable. Let $\varepsilon > 0$. Show that there is a step function $h:I\to\R$ and a measurable subset $F$ of $I$ for which 
    \begin{equation*}
        h = \chi_E\text{ on F, and } m(I\backslash F) < \varepsilon
    \end{equation*}
\end{lemma}
\begin{proof}
    Let $U$ be an open set containing $U$ such that $m(U\backslash E) < \varepsilon/2$. Now, $U$ may be written as as disjiont union of open intervals $\{I_k\}_{k = 1}^\infty$ where some intervals may be empty. Using the continuity of measure, there is a positive integer $N$ such that 
    \begin{equation*}
        \sum_{k = N + 1}^\infty\ell(I_k) < \varepsilon/2
    \end{equation*}
    Define $\mathcal O = \bigcup\limits_{k = 1}^{N} I_k$ and $F = (\mathcal O\cap E)\cup(I\backslash U)$. Then, 
    \begin{equation*}
        I\backslash F = (U\backslash E)\cup(\mathcal U\backslash O)
    \end{equation*}
    and thus, $m(I\backslash F)\le m(U\backslash E) + m(\mathcal U\backslash O) < \varepsilon$. Finally, define $h = \chi_{\mathcal O}$. Since $F\cap\mathcal O = \mathcal O\cap E\subseteq E$, the restrictions on $h$ are satisfied.
\end{proof}

\begin{lemma}
    Let $\psi: I\to\R$ be a simple function, $E\subseteq I$ be measurable and $\varepsilon > 0$. Then there is a step function $h: I\to\R$ and a measurable subset $F$ of $I$ for which 
    \begin{equation*}
        h = \psi\text{ on F, and } m(I\backslash F) < \varepsilon
    \end{equation*}
\end{lemma}
\begin{proof}
    Follows from the previous lemma by taking linear combinations.
\end{proof}

\begin{proof}[Proof of \thref{thm:step-approximation}]
    Due to \thref{lem:simple-approximation}, there is a simple function $\psi: I\to\R$ such that $0\le f - \psi < \varepsilon$. Due to the preceeding lemma, there is a step function $h: I\to\R$ and a measurable subset $F$ of $I$ with $m(I\backslash F) < \delta$ such that $h = \psi$ on $F$. The conclusion now follows.
\end{proof}


\section{Egoroff and Lusin's Theorems}

\begin{theorem}[Egoroff]\thlabel{thm:egoroff}
    Let $E$ be a measurable set with finite measure. Let $\{f_n\}$ be a sequence of measurable functions on $E$ that converge pointwise a.e. on $E$ to the extended real-valued function $f$ which is finite a.e. Then for each $\varepsilon > 0$, there is a closed set $F\subseteq E$ for which 
    \begin{equation*}
        \{f_n\}\rightrightarrows f\text{ on $F$ and }m(E\backslash F) < \varepsilon
    \end{equation*}
\end{theorem}

Since $\{f_n\}\to f$ pointwise a.e. on $E$, there is a subset $E_0$ such that $m(E_0) = 0$ and $\{f_n\}\to f$ pointwise on $E\backslash E_0$. Similarly, there is $E_0'$ such that $f$ is finite on $E\backslash E_0'$ and $m(E_0') = 0$. Let $E' = E\backslash(E_0\cup E_0')$. Then $m(E') = m(E)$ and $\{f_n\}\to f$ pointwise on $E'$ and $f$ is finite on all of $E'$. Then, if we prove Egoroff's Theorem on $E'$, we woud have a closed subset $F$ of $E'$ such that $\{f_n\}\rightrightarrows f$ on $F$ and $m(E\backslash F) = m(E'\backslash F) < \varepsilon$, and hence would prove it for the general case. Therefore, without loss of generality, we may suppose that $\{f_n\}\to f$ pointwise on $E$ and is finite on all of $E$.

In order to prove the reduced statement of Egoroff's Theorem, we require the following lemma,

\begin{lemma}
    Let $E$ have finite measure and $\{f_n\}\to f$ pointwise on $E$ where $f$ is finite on all of $E$. Then, for every $\eta > 0$ and $\delta > 0$, there is a measurable subset $A\subseteq E$ and index $N\in\N$ such that 
    \begin{equation*}
        |f_n - f|\le\eta\text{ $\forall~n\ge N$ on $A$ and }m(E\backslash A) < \delta
    \end{equation*}
\end{lemma}
\begin{proof}
    Define the collection of sets $\{F_n\}$ as 
    \begin{equation*}
        F_n := \{x\in E\mid |f(x) - f_n(x)| < \eta\}
    \end{equation*}
    Since the function $|f - f_n|$ is measurable, so is the set $F_n$. Now, define 
    \begin{equation*}
        E_n := \bigcap_{k = n}^\infty F_k = \{x\in E\mid |f(x) - f_n(x)| < \eta,~\forall n\ge N\}
    \end{equation*}
    Since $E_n$ is the countable intersection of measurable sets, it is measurable. Furthermore, by definition, the collection $\{E_n\}$ forms an ascending chain satisfying 
    \begin{equation*}
        \bigcup_{n = 1}^\infty E_n = E
    \end{equation*}
    Consequently, $\lim\limits_{n\to\infty} m(E_n) = m(E) < \infty$. Hence, there is $N\in\N$ such that $m(E\backslash E_N) = m(E) - m(E_N) < \delta$. Let $A = E_N$. Then, for all $x\in A$, $|f(x) - f_n(x)| < \eta$ for all $n\ge N$. This completes the proof.
\end{proof}

We can now prove \thref{thm:egoroff}.

\begin{proof}[Proof of \thref{thm:egoroff}]
    Due to our discussion above, we may suppose that $f_n\to f$ pointwise on $E$ and $f$ is finite on all of $E$. Let $A_n$ be such that there is an index $N$ with $|f - f_k|\le 1/n$ for all $k\ge N$ and $m(E\backslash A_n) < \varepsilon/2^{n + 1}$. Define $A = \bigcap\limits_{n = 1}^\infty A_n$. Then 
    \begin{equation*}
        m(E\backslash A)\le \sum_{n = 1}^\infty m(E\backslash A_n) < \varepsilon/2
    \end{equation*}

    We claim that $f_n\rightrightarrows f$ uniformly on $A$. Choose some $\eta > 0$. Then, there is $N\in\N$ such that $1/N < \eta$. Consequently, for all $x\in A_N$, there is an index $M$ such that for all $k\ge M$, $|f - f_k| < \eta$. But since $A\subseteq A_N$, we have that for all $x\in A$, there is an index $M$ such that for all $k\ge M$, $|f(x) - f_k(x)| < \eta$, which implies uniform convergence.

    Now, since $A$ is the countable intersection of measurable sets, it is measurable. As a result, due to \thref{thm:inner-outer-approx-measurable}, there is a closed set $F\subseteq A$ with $m(A\backslash F) < \varepsilon/2$. Thus, $F\subseteq E$ and $m(E\backslash F) < \varepsilon$ and $f_n\rightrightarrows f$ on $F$. This completes the proof.
\end{proof}

\begin{mdframed}
Let us consider the case when $m(E) = \infty$. Consider the sequence of functions $f_n:\R\to\R$ given by 
\begin{equation*}
    f_n(x) = x\left(1 - \frac{1}{n}\right)
\end{equation*}
Then, it is not hard to see that $f_n\to f$ on $E = \R$ and $f$ is finite on $E$. Choose any $\varepsilon > 0$ and let $F\subseteq\R$ be a closed subset of $\R$ such that $m(\R\backslash F) < \varepsilon$. Since $\R\backslash F$ is open in $\R$, it is the disjiont union of open intervals. Further, since it has finite measure, all the disjiont intervals must be bounded. As a result, $F$ is not bounded.

We now claim that $f_n$ may not converge uniformly to $f$ on $F$. Suppose it did, then, pick some $\delta > 0$. Then, there is $N\in\N$ such that for all $n\ge N$, $|f - f_n| < \delta$ on $F$. But this implies $|x/n| < \delta$ on $F$, which is absurd, since $F$ is unbounded.
\end{mdframed}
The above discussion shows that Egoroff's theorem may not hold on dropping the finite measure hypothesis.

Next, we come to Lusin's Theorem, illustrating the third of Littlewood's three principles. We shall state the theorem first, then prove a useful lemma and finally prove the theorem.

\begin{lemma}
    Let $f$ be a simple function defined on $E$. Then for each $\varepsilon > 0$, there is a continuous function $g$ on $\R$ and a closed set $F$ contained in $E$ for which $f = g$ on $F$ and $m(E\backslash F) < \varepsilon$.
\end{lemma}
\begin{proof}
    Let $f = \sum_{i = 1}^n a_i\chi_{E_i}$ where the sets $E_i$ are disjoint. Due to \thref{thm:inner-outer-approx-measurable}, there are closed sets $F_i$ with $m(E_i\backslash F_i) < \varepsilon/n$ and $F_i\subseteq E_i$. Let $F = \bigsqcup_{k = 1}^n F_i$. Then $F$ is closed. Define the function $g$ on $F$ by $g(x) = a_i$ if $x\in F_i$. Note that this function is well defined because the $F_i$ are disjoint. Then, using Tietze's Extension Theorem, we may extend $g$ to a continuous function on all of $\R$.

    Finally, note that 
    \begin{equation*}
        m(E\backslash F) = m\left(\bigsqcup_{k = 1}^n(E_k\backslash F_k)\right) = \sum_{k = 1}^n m(E_k\backslash F_k) < \varepsilon
    \end{equation*}
    This completes the proof.
\end{proof}

Lusin's Theorem essentially extends the above proposition to general measurable functions.

\begin{theorem}[Lusin]\thlabel{thm:lusin}
    Let $f: E\to\R$ be real valued measurable. Then for each $\varepsilon > 0$, there is a continuous function $g:\R\to\R$ and a closed set $F\subseteq E$ such that $f = g$ on $F$ and $m(E\backslash F) < \varepsilon$.
\end{theorem}
\begin{proof}
We divide the proof into two cases, one for when $m(E) < \infty$ and the other for when $m(E) = \infty$.
\begin{description}
\item[\underline{$m(E) < \infty$}: ] Due to the Simple Approximation Theorem, there is a sequence $\varphi_n: E\to\R$ of simple functions that converge to $f$ on $E$. Using the preceeding lemma, there is a sequence of continuous functions $g_n: \R\to\R$ and closed sets $F_n$ such that the restriction of $g_n$ to $F_n$ is $\varphi_n$ and $m(E\backslash F_n) < \varepsilon/2^{n + 2}$. Further, due to Egoroff's Theorem, there is a closed set $F_0$ that is contained in $E$ such that $\{f_n\}$ converges to $f$ uniformly on $F_0$ and $m(E\backslash F_0) < \varepsilon/2^2$. Define $F = \bigcap\limits_{n = 0}^\infty$. Then $m(E\backslash F)\le\varepsilon/2 < \varepsilon$ and since $\varphi_n$ converge uniformly on $F_0$ and thus on $F$, so do $g_n$ and their pointwise limit $g$ is continuous on $F$. Finally, due to the Tietze Extension Theorem, this function may be extended to a continuous function $g:\R\to\R$.

\item[\underline{$m(E) = \infty$}: ] Consider the collection $\{E\cap[k,k + 1)\}_{k\in\Z}$, which is a countable collection of disjoint sets whose union is $E$. Reindex this set as $\{E_n\}$. For each $E_n$, there is a closed subset $F_n$ such that $m(E_n\backslash F_n) < \varepsilon/2^{n + 1}$ and there is a continuous function $g_n$ on $E_n$ which agrees with $f$ on $F_n$. Since the collection $F_n$ is locally finite, the Pasting Lemma holds and there is a continuous function $g$ on $\bigcup F_n$ which agrees with $f$. Again, since $F_n$ is locally finite, the union $\bigcup F_n$ is closed and thus, we may use the Tietze Extension Theorem.
\end{description}
\end{proof}
