\documentclass[12pt]{report}
\usepackage[]{amsmath, amsthm, amssymb, xcolor, geometry, tikz, hyperref, mathrsfs}
\usepackage[utf8]{inputenc}
\usepackage[]{enumitem}
\usepackage{fancyhdr}
\usepackage{lipsum}
\pagestyle{fancy}
\fancyhf{}
\rhead{Combinatorics}
\lhead{\leftmark}
\cfoot{\thepage}
\lfoot{Swayam Chube}
\renewcommand{\footrulewidth}{\headrulewidth}
\setlength{\headheight}{15pt}

\usepackage[T1]{fontenc}
\usepackage{mathpazo}

\hypersetup {
    colorlinks=true,
}

\usetikzlibrary{automata, arrows.meta, positioning}

\usepackage[framemethod = tikz]{mdframed}

\newcounter{theorem}[chapter]
\newenvironment{theorem}[1][]{
	\stepcounter{theorem}
	\ifstrempty{#1}{
		\begin{mdframed}[linewidth=1pt,linecolor=blue!10, backgroundcolor=blue!10,]
		\bfseries Theorem~\thechapter.\thetheorem.\normalfont
	}{
		\begin{mdframed}[linewidth=1pt,linecolor=blue!10, backgroundcolor=blue!10,]
		\bfseries Theorem~\thechapter.\thetheorem~(#1).\normalfont
	}
}{\end{mdframed}}

\newenvironment{definition}[1][]{
	\stepcounter{theorem}
	\ifstrempty{#1}{
		\begin{mdframed}[linewidth=1pt, linecolor=gray!15, backgroundcolor=gray!15,]
		\bfseries Definition~\thechapter.\thetheorem.\normalfont
	}{
		\begin{mdframed}[linewidth=1pt, linecolor=gray!15, backgroundcolor=gray!15,]
		\bfseries Definition~\thechapter.\thetheorem~(#1).\normalfont
	}
}{\end{mdframed}}

\newenvironment{proposition}[1][]{
	\stepcounter{theorem}
	\ifstrempty{#1}{
		\begin{mdframed}[linewidth=1pt, linecolor=blue!10, backgroundcolor=blue!10,]
		\bfseries Proposition~\thechapter.\thetheorem.\normalfont
	}{
		\begin{mdframed}[linewidth=1pt, linecolor=blue!10, backgroundcolor=blue!10,]
		\bfseries Proposition~\thechapter.\thetheorem~(#1).\normalfont
	}
}{\end{mdframed}}

\newcounter{corollary}
\newenvironment{corollary}{
	% \stepcounter{corollary}
	\begin{mdframed}[linecolor=blue!10, backgroundcolor=blue!10, linewidth=1pt,]
	\bfseries Corollary.\normalfont
}{\end{mdframed}}

\newenvironment{lemma}[1][]{
	\stepcounter{theorem}
	\ifstrempty{#1}{
		\begin{mdframed}[linewidth=1pt, linecolor=yellow, backgroundcolor=yellow!30,]
		\bfseries Lemma~\thechapter.\thetheorem.\normalfont
	}{
		\begin{mdframed}[linewidth=1pt, linecolor=yellow, backgroundcolor=yellow!30,]
		\bfseries Lemma~\thechapter.\thetheorem~(#1).\normalfont
	}
}{\end{mdframed}}

\newenvironment{example}[1][]{
	\ifstrempty{#1}{
		\begin{mdframed}[linewidth=1pt, linecolor=red, backgroundcolor=red!20,]
		\bfseries Example.\normalfont
	}{
		\begin{mdframed}[linewidth=1pt, linecolor=red, backgroundcolor=red!20,]
		\bfseries Example~(#1).\normalfont
	}
}{\end{mdframed}}

\newenvironment{exercise}{
	\begin{mdframed}[linewidth=1pt, linecolor=green, backgroundcolor=green!10,]
	\bfseries Exercise.\normalfont
}{\end{mdframed}}

\renewcommand{\qedsymbol}{$\blacksquare$}

\title{Results in Combinatorics}
\author{Swayam Chube}
\date{\today}

\begin{document}
    \maketitle 
    \tableofcontents
    \chapter{Results on Families of Sets}
    \begin{lemma}[Lubell-Yamamoto-Meshalkin]
    Let $n$ be a positive integer and $\mathcal{F}$ be a family of subsets of $\{1,\ldots,n\}$ such that no set in $\mathcal{F}$ is contained in some other set in $\mathcal{F}$. Then, 
    \begin{equation*}
        \sum_{A\in\mathcal{F}}\frac{1}{\binom{n}{|A|}}\le1
    \end{equation*}
\end{lemma}
\begin{proof}
    Let $\mathcal{F} = \{A_1,\ldots,A_m\}$ and $\pi_i$ be the set of all permutations of $\{1,\ldots,n\}$ such that the first $|A_i|$ elements of $\pi_i$ are the elements of $A_i$. It is not hard to see that $|\pi_i| = |A_i|!(n - |A_i|)!$. Further, we note that any permutation $\sigma$ of $\{1,\ldots,n\}$ may be in at most one of the $\pi_i$'s. Thus, double counting the pairs $(\pi_i,\sigma)$, we obtain:
    \begin{equation*}
        \sum_{i=1}^m|A_i|!(n - |A_i|)!\le n!
    \end{equation*}
    and we have the desired conclusion.
\end{proof}

\begin{theorem}[Bollob\'as, 1965]
    Let $\{A_1,\ldots, A_m\}$ and $\{B_1,\ldots,B_m\}$ be two families of subsets of $\{1,\ldots,n\}$ such that 
    \begin{itemize}
        \item $A_i\cap B_i = \emptyset$ for all $1\le i\le m$
        \item $A_i\cap B_j = \emptyset$ for all $1\le i, j\le m$ and $i\ne j$
    \end{itemize}
    Then 
    \begin{equation*}
        \sum_{i=1}^m\frac{1}{\binom{|A_i| + |B_i|}{|A_i|}}\le 1
    \end{equation*}
\end{theorem}
\begin{proof}
    Let $\pi_i$ be the set of all permutations of $\{1,\ldots,n\}$ such that the elements of $A_i$ occur before the elements of $B_i$ (note that this is possible since $A_i\cap B_i = \emptyset$). Then 
    \begin{equation*}
        |\pi_i| = \binom{n}{|A_i| + |B_i|}\cdot|A_i|!\cdot|B_i|!\cdot(n - |A_i| - |B_i|)! = \frac{n!}{\binom{|A_i| + |B_i|}{|A_i|}}
    \end{equation*}

    Finally, we note that for $i\ne j$, $\pi_i\cap\pi_j=\emptyset$, which is not hard to show. This implies that 
    \begin{equation*}
        \sum_{i=1}^m|\pi_i|\le n!
    \end{equation*}
    giving us the desired conclusion.
\end{proof}

\begin{lemma}
    Let $\mathcal{F}$ be an $r$-uniform family of sets. Such that the intersection of any $k$ sets, with $k\le r + 1$ is non-empty. Then, the intersection of all sets in $\mathcal{F}$ is non-empty.
\end{lemma}
\begin{proof}
    Suppose not. Let $\mathcal{F} = \{A_1,\ldots,A_m\}$ be a minimal counter-example to the statement. Obviously, $m\ge r + 2$ and for each $i$, $\bigcup_{j\ne i} A_j$ is non-empty (since $\mathcal{F}$ is a minimal counter-example) and thus, let $b_i$ be one such element in said intersection. Suppose all $b_i$'s are distinct. Then, $\{b_2,\ldots,b_m\}\subseteq A_1$, implying that $|A_1|\ge m - 1\ge r + 1$, which is not possible. Thus, there must exist indices $i$ and $j$ with $i\ne j$ such that $b_i = b_j$. It is not hard to conclude from here that all sets in $\mathcal{F}$ must contain $b_i$.
\end{proof}

\begin{theorem}[Erd\"os-Ko-Rado, 1961]
    Let $n$ be a positive integer, $X$ be an $n$-element set and $k\le n/2$ be a positive integer. Further, let $\mathcal{F}$ be a $k$-uniform, intersecting family of subsets of $X$. Then 
    \begin{equation*}
        |\mathcal{F}|\le\binom{n - 1}{k - 1}
    \end{equation*}
\end{theorem}
\begin{proof}
    Let $\mathcal{F} = \{A_1,\ldots,A_m\}$. Without loss of generality, let $X = \{1,\ldots,n\}$ and $\pi_i$ be the set of all cyclic permutations of $X$ in which the elements of $A_i$ occur consecutively. Of course, $|\pi_i| = k!(n - k)!$ for each $1\le i\le m$. Further, we note that any cyclic permutation of $X$ may occur in at most $k$ of the $\pi_i$'s. (This is an interesting argument). Then, double counting the pair $(C,\pi_i)$ where $C$ is a cyclic permutation, we have 
    \begin{equation*}
        m\cdot k!(n - k)!\le k\cdot(n - 1)!
    \end{equation*}
    this completes the proof.
\end{proof}

\begin{theorem}[Benny Sudakov]
    Let $\mathscr{A} = \{A_1,\ldots,A_m\}$ and $\mathscr{B} = \{B_1,\ldots,B_p\}$ be families of distinct subsets of $\{1,\ldots,n\}$ such that $|A_i\cap B_j|$ is an odd number for all permissible $i$ and $j$. Then $mp\le 2^{n - 1}$.
\end{theorem}
    \chapter{Combinatorial Nullstellensatz}
    \begin{theorem}[Alon-Tarsi, 1992]
Let $\mathbb{F}$ be a field and $f\in\mathbb{F}[x_1,\ldots,x_n]$. Suppose $\deg(f) = d = \sum_{i=1}^nd_i$ and the coefficient of $\prod_{i=1}^nx_i^{d_i}$ is non-zero. Let $L_1,\ldots,L_n$ be subsets of $\mathbb{F}$ with $|L_i| > d_i$. Then there exist $a_i\in L_i$ for each $i$ such that $f(a_1,\ldots,a_n)\ne0$.
\end{theorem}
\begin{proof}
    The proof is by induction on $n$. The base case with $n = 1$ follows from the fact that a polynomial of degree $n$ may have at most $n$ zeros in a field, counting multiplicity. Suppose now that $n > 1$. Let us assume that $|L_n| = d_n + 1$. Define the polynomial $h(x)$ as follows:
    \begin{equation*}
        \prod_{t\in L_n}(x_n - t) = x^{d_n + 1} - h(x)
    \end{equation*}
    One notes that for all $t\in L_n$, $h(t) = t^{d_n + 1}$. Let us now define $\tilde{f}$ as the remainder obtained on dividing $f$ by $x_n^{d_n + 1} - h(x_n)$. We first note that the coefficient of $\prod_{i=1}^nx_i^{d_i}$ does not change. Further, the degree of $x_n$ in any term in $\tilde{f}$ is at most $d_n$. We may now group the terms of $\tilde{f}$ with respect to powers of $x_n$, that is 
    \begin{equation*}
        \tilde{f}(x_1,\ldots,x_{n - 1}) = \sum_{i=0}^{d_n}g_i(x_1,\ldots,x_{n-1})x_n^i
    \end{equation*}
    Let us now focus on $g_{d_n}(x_1,\ldots,x_{n-1})$, in which the coefficient of the term $\prod_{i=1}^{n - 1}x_i^{d_i}$ is equal to the coefficient of the term $\prod_{i=1}^{n}x_i^{d_i}$ in $\tilde{f}(x_1,\ldots,x_n)$, which we have concluded to be non-zero. Therefore, due to the induction hypothesis, there must exist $a_1\in L_1,\cdots,a_{n-1}\in L_{n - 1}$ such that $g_{d_n}(a_1,\ldots,a_{n - 1})\ne0$. Finally, we can choose a suitable $x_n$ in $L_n$ such that the polynomial 
    \begin{equation*}
        \sum_{i=0}^{d_n}g_i(a_1,\ldots,a_{n-1})x_n^i
    \end{equation*}
    is non-zero, since its leading coefficient is non-zero. This finishes the proof.
\end{proof}

\begin{theorem}[Cauchy-Davenport]
    Let $p$ be a prime and $A,B\subseteq\mathbb{Z}_p$. Then 
    \begin{equation*}
        |A + B|\ge\min\{p, |A| + |B| - 1\}
    \end{equation*}
    where 
    \begin{equation*}
        A + B = \{a + b\mid a\in A,~b\in B\}
    \end{equation*}
\end{theorem}
\begin{proof}
    First, suppose $|A| + |B| > p$. Then, for any $g\in\mathbb{Z}_p$, the sets $A$, $g - B$ must intersect. Suppose now that $|A| + |B|\le p$. Assume for the sake of contradiction that the cardinality of $C = A + B$ is less than $|A| + |B| - 1$. Consider the polynoimal:
    \begin{equation*}
        f(x, y) = \prod_{c\in C}(x + y - c)
    \end{equation*}
    the degree of this polynoimal is $|C|\le |A| + |B| - 2$, as a result, there exist $t_A, t_B\in\mathbb{Z}$ such that $t_A + t_B = |C|$, $t_A\le|A| - 1$, $t_B\le|B| - 1$ and the coefficient of $x^{t_A}y^{t_B}$ is non-zero, this is because any binomial coefficient of the form $\binom{|C|}{x}$ is not divisible by $p$. Then, due to the Combinatorial Nullstellensatz, there exists $(a,b)\in A\times B$ such that $f(a,b)\ne0$, a contradiction. This finishes the proof.
\end{proof}

\begin{example}[IMO 2007/6]
    Let $n$ be a positive integer. Consider 
    \begin{equation*}
        S = \{(x, y, z)\mid x,y,z\in\{0,1,\ldots,n\},~x + y + z > 0\}
    \end{equation*}
    as a set of $(n + 1)^3 - 1$ points in the three-dimensional space. Determine the smallest possible number of planes, the union of which contains $S$ but does not include $(0,0,0)$.
\end{example}
\begin{proof}
    The answer is $3n$, with the planes being $x + y + z = i$ for $1\le i\le 3n$. Suppose $k < 3n$ is achievable by the set of planes $\{a_ix + b_iy + c_iz - d_i = 0\}$. Consider now the polynomial:
    \begin{equation*}
        P(x, y, z) = \alpha\prod_{i=1}^n(x - i)(y - i)(z - i) - \prod_{i=1}^k(a_ix + b_iy + c_iz - d_i)
    \end{equation*}
    with $\alpha$ chosen such that $P(0, 0, 0) = 0$. First, note that $\deg(P) = 3n$, since we have assumed $k < 3n$ and the coefficient of $x^ny^nz^n$ is $\alpha\ne 0$ since none of the $d_i$'s can be zero. Then, due to the Combinatorial Nullstellensatz on the three sets $L_x = L_y = L_z = \{0,\ldots,n\}$, we may conclude that there is a solution to $P(x, y, z)\ne 0$ in $L_x\times L_y\times L_z$, a contradiction.
\end{proof}

\begin{theorem}[Erd\"os-Heillbronn]
    Let $p$ be a prime and $A\subseteq\mathbb{Z}_p$. Then 
    \begin{equation*}
        |A \widehat{+} A|\ge\min\{p, 2|A| - 3\}
    \end{equation*}
\end{theorem}
\begin{proof}
    We may suppose that $2a - 3 < p$ where $a = |A|$ and let $C = A \widehat{+} A$ with $|C| = m$. Consider the polynoimal 
    \begin{equation*}
        f(x,y) = (x - y)\prod_{c\in C}(x + y - c)
    \end{equation*}
    of degree $m + 1$. The coefficient of $x^{a - 1}y^{m - a + 2}$ is 
    \begin{equation*}
        \binom{m}{a - 2} - \binom{m}{a - 1} = \left(\frac{m - 2a + 3}{m - a + 2}\right)\binom{m}{a - 2}
    \end{equation*}

    Now, suppose that $m < 2a - 3$, then the coefficient is non-zero and $m - a + 2 < a - 1 < a$. As a result, there is a solution $(a_1, a_2)$ such that $f(a_1, a_2)\ne0$ and thus $a_1\ne a_2$. A contradiction. This finishes the proof.
\end{proof}
	\chapter*{Miscellaneous}
	\begin{example}
    Let $\sigma\in\operatorname{Sym}(\{1,\ldots,n\})$, and let $\varepsilon(\sigma) = 1$ if $\sigma$ is even and $-1$ otherwise. Let $f(\sigma)$ be the number of fixed points of $\sigma$. Prove that 
    \begin{equation*}
        \sum_\sigma\frac{\varepsilon(\sigma)}{1 + f(\sigma)} = (-1)^{n + 1}\frac{n}{n + 1}
    \end{equation*}
\end{example}
\begin{proof}
    Let $A_{n\times n}$ be the matrix such that 
    \begin{equation*}
        a_{ij} = \begin{cases}
            x & i = j\\
            1 & \text{otherwise}
        \end{cases}
    \end{equation*}
    We see that 
    \begin{equation*}
        \det(A) = \sum_{\sigma}\varepsilon(\sigma)x^{f(\sigma)}
    \end{equation*}
    but we know that $\det(A) = (x - 1)^{n - 1}(x + n - 1)$. Then, we may write 
    \begin{align*}
        \int_0^1\sum_{\sigma}\varepsilon(\sigma)x^{f(\sigma)}~\text{d}x &= \int_0^1(x - 1)^{n - 1}(x + n - 1)~\text{d}x\\
        \sum_\sigma\frac{\varepsilon(\sigma)}{1 + f(\sigma)} &= (-1)^{n + 1}\frac{n}{n + 1}
    \end{align*}
    This finishes the proof.
\end{proof}

\begin{example}[Putnam 2016/B4]
   	Let $A$ be a $2n\times 2n$ matrix, with entries chosen independently at random. Every entry is chosen to be $0$ or $1$, each with probability $1/2$. Find the expected value of $\det(A-A^T)$ (as a function of $n$), where $A^T$ is the transpose of $A.$ 
\end{example}
\end{document}