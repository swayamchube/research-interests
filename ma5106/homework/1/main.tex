\documentclass[12pt]{amsart}

\usepackage{amsmath, amsthm, amssymb, xcolor, geometry, hyperref, enumitem, todonotes}

\title{MA5106: Homework 1}
\author{Swayam Chube (200050141)}

\newcommand{\half}{\frac{1}{2}}

%  Common Algebraic Structures
\newcommand{\R}{\mathbb{R}}
\newcommand{\Q}{\mathbb{Q}}
\newcommand{\Z}{\mathbb{Z}}
\newcommand{\N}{\mathbb{N}}
\newcommand{\bbC}{\mathbb{C}}
\newcommand{\calA}{\mathcal{A}}
\newcommand{\frakM}{\mathfrak{M}}

% Categories
\newcommand{\catTopp}{\mathbf{Top}_*}
\newcommand{\catGrp}{\mathbf{Grp}}
\newcommand{\catTopGrp}{\mathbf{TopGrp}}
\newcommand{\catSet}{\mathbf{Set}}
\newcommand{\catTop}{\mathbf{Top}}
\newcommand{\catRing}{\mathbf{Ring}}
\newcommand{\catCRing}{\mathbf{CRing}} % comm. rings
\newcommand{\catMod}{\mathbf{Mod}}
\newcommand{\catMon}{\mathbf{Mon}}
\newcommand{\catMan}{\mathbf{Man}} % manifolds
\newcommand{\catDiff}{\mathbf{Diff}} % smooth manifolds
\newcommand{\catAlg}{\mathbf{Alg}}
\newcommand{\catRep}{\mathbf{Rep}} % representations 
\newcommand{\catVec}{\mathbf{Vec}}

% Group and Representation Theory
\newcommand{\chr}{\operatorname{char}}
\newcommand{\Aut}{\operatorname{Aut}}
\newcommand{\GL}{\operatorname{GL}}
\newcommand{\im}{\operatorname{im}}
\newcommand{\tr}{\operatorname{tr}}
\newcommand{\id}{\mathbf{id}}
\newcommand{\cl}{\mathbf{cl}}
\newcommand{\Gal}{\operatorname{Gal}}
\newcommand{\Tr}{\operatorname{Tr}}
\newcommand{\sgn}{\operatorname{sgn}}
\newcommand{\Sym}{\operatorname{Sym}}
\newcommand{\Alt}{\operatorname{Alt}}

% Commutative and Homological Algebra
\newcommand{\spec}{\operatorname{spec}}
\newcommand{\mspec}{\operatorname{m-spec}}
\newcommand{\Tor}{\operatorname{Tor}}
\newcommand{\tor}{\operatorname{tor}}
\newcommand{\Ann}{\operatorname{Ann}}
\newcommand{\Supp}{\operatorname{Supp}}
\newcommand{\Hom}{\operatorname{Hom}}
\newcommand{\End}{\operatorname{End}}
\newcommand{\coker}{\operatorname{coker}}
\newcommand{\limit}{\varprojlim}
\newcommand{\colimit}{%
  \mathop{\mathpalette\colimit@{\rightarrowfill@\textstyle}}\nmlimits@
}
\makeatother


\newcommand{\fraka}{\mathfrak a} % ideal
\newcommand{\frakb}{\mathfrak b} % ideal
\newcommand{\frakc}{\mathfrak c} % ideal
\newcommand{\frakf}{\mathfrak f} % face map
\newcommand{\frakm}{\mathfrak m} % maximal ideal
\newcommand{\frakp}{\mathfrak p} % prime ideal
\newcommand{\frakq}{\mathfrak q} % qrime ideal
\newcommand{\frakN}{\mathfrak N} % nilradical 
\newcommand{\frakP}{\mathfrak P} % nilradical 
\newcommand{\frakR}{\mathfrak R} % jacobson radical

% General/Differential/Algebraic Topology 
\newcommand{\scrA}{\mathscr A}
\newcommand{\scrB}{\mathscr B}
\newcommand{\scrP}{\mathscr P}
\newcommand{\scrS}{\mathscr S}
\newcommand{\bbH}{\mathbb H}
\newcommand{\Int}{\operatorname{Int}}
\newcommand{\psimeq}{\simeq_p}
\newcommand{\wt}[1]{\widetilde{#1}}
\newcommand{\RP}{\mathbb{R}\text{P}}
\newcommand{\CP}{\mathbb{C}\text{P}}

% Miscellaneous
\newcommand{\wh}[1]{\widehat{#1}}
\newcommand{\calM}{\mathcal{M}}
\newcommand{\calP}{\mathcal{P}}
\newcommand{\onto}{\twoheadrightarrow}
\newcommand{\into}{\hookrightarrow}
\newcommand{\Gr}{\operatorname{Gr}}
\newcommand{\Span}{\operatorname{Span}}

\begin{document}
\maketitle 

\section{Problem 1}

\begin{enumerate}[label=(\alph*)]
    \item Let $n\in\Z$ be non-zero. The $n$-th Fourier coefficient is given by 
    \begin{align*}
        a_n = \frac{1}{2\pi}\int_{-\pi}^\pi f(x)e^{-inx}~dx &= \frac{1}{2\pi}\int_{-\pi}^\pi |x|e^{-inx}~dx\\
        &= \frac{1}{2\pi}\left(\int_{-\pi}^0 (-x)e^{-inx}~dx + \int_{0}^\pi xe^{-inx}~dx\right)\\
        &= \frac{1}{2\pi}\left(\int_{0}^\pi xe^{inx}~dx + \int_{0}^\pi xe^{-inx}~dx\right)\\
        &= \frac{1}{2\pi}\int_{0}^\pi x\left(e^{inx} + e^{-inx}\right)~dx\\
        &= \frac{1}{2\pi}\int_0^\pi 2x\cos(nx)~dx = \frac{1}{\pi}\int_{0}^\pi x\cos(nx)~dx\\
        &= \frac{1}{\pi}\left[\frac{nx\sin(nx) + \cos(nx)}{n^2}\right]_0^{\pi}\\
        &= \frac{(-1)^n - 1}{n^2\pi} = 
        \begin{cases}
            0 & \text{$n$ is even}\\
            \frac{-2}{n^2\pi} & \text{$n$ is odd}.
        \end{cases}
    \end{align*}

    On the other hand, if $n = 0$, then 
    \begin{equation*}
        a_0 = \frac{1}{2\pi}\int_{-\pi}^\pi |x|~dx = \frac{\pi}{2}.
    \end{equation*}

    \item Let $n\in\Z$ be non-zero. The $n$-th Fourier coefficient is given by 
    \begin{align*}
        a_n &= \frac{1}{2\pi}\int_{-\pi}^\pi f(x)e^{-inx}~dx\\
        &= \frac{1}{2\pi}\int_{-\pi}^0 f(x)e^{-inx}~dx + \frac{1}{2\pi}\int_{0}^\pi f(x)e^{-inx}~dx.
    \end{align*}
    Note that $f$ is an odd function on $[-\pi, \pi]$, that is, $f(-x) = -f(x)$. Making the substitution $x = -y$ in the first integral, we have 
    \begin{align*}
        a_n &= \frac{1}{2\pi}\int_{0}^\pi f(-y)e^{iny}~dy + \frac{1}{2\pi}\int_{0}^\pi f(x)e^{-inx}~dx\\
        &= \frac{1}{2\pi}\int_{0}^{\pi}-f(y)e^{iny}~dy + \frac{1}{2\pi}\int_{0}^{\pi} f(x)e^{-inx}~dx\\
        &= \frac{1}{2\pi}\int_{0}^\pi \frac{\pi - x}{2}\left(e^{-inx} - e^{inx}\right)~dx.
    \end{align*}
    Perform the substitution $y = \pi - x$ to get 
    \begin{align*}
        a_n &= \frac{1}{4\pi}\int_{0}^\pi y\left(e^{-in(\pi - y)} - e^{in(\pi - y)}\right)~dy\\
        &= \frac{(-1)^n}{4\pi}\int_0^\pi y(e^{iny} - e^{-iny})~dy\\
        &= \frac{(-1)^n 2i}{4\pi}\int_{0}^\pi y\sin(ny)~dy\\
        &= \frac{(-1)^ni}{2\pi}\left[\frac{\sin(ny) - ny\cos(ny)}{n^2}\right]_0^\pi\\
        &= \frac{(-1)^ni}{2\pi}\frac{(-n\pi)\cdot(-1)^n}{n^2} = \frac{-i}{2n}.
    \end{align*}

    As for $n = 0$, we have 
    \begin{equation*}
        f(0) = \frac{1}{2\pi}\int_{-\pi}^{\pi} f(x)~dx = 0,
    \end{equation*}
    since $f$ is an odd function.
\end{enumerate}

\section*{Problem 2}

Note that 
\begin{equation*}
    f(x) = e^{i(\pi - x)/2} = e^{i\pi/2}e^{-ix/2} = ie^{-ix/2}.
\end{equation*}

First, we obtain the Fourier coefficients, that is, 
\begin{align*}
    a_n &= \frac{1}{2\pi}\int_{0}^{2\pi} ie^{-ix/2}e^{-inx}~dx\\
    &= \frac{i}{2\pi}\int_{0}^{2\pi} e^{-i\left(n + \frac{1}{2}\right)x}~dx\\
    &= \frac{i}{2\pi}\frac{1}{(-i)\left(n + \frac{1}{2}\right)}\left[e^{-i\left(n + \frac{1}{2}\right)x}\right]_0^{2\pi}\\
    &= \frac{i}{2\pi}\frac{-2}{(-i)\left(n + \frac{1}{2}\right)} = \frac{2}{(2n + 1)\pi}
\end{align*}

Using Parseval's Formula, we have 
\begin{equation*}
    \sum_{n\in\Z}|a_n|^2 = \frac{1}{2\pi}\int_{0}^{2\pi}\left|e^{-i\frac{\pi - x}{2}}\right|^2~dx = 1.
\end{equation*}

Therefore, 
\begin{equation*}
    1 = \frac{4}{\pi^2}\sum_{n\in\Z}\frac{1}{(2n + 1)^2} = \frac{8}{\pi^2}\sum_{n = 0}^\infty\frac{1}{(2n + 1)^2}.
\end{equation*}

This gives,
\begin{equation*}
    \sum_{n = 0}^\infty\frac{1}{(2n + 1)^2} = \frac{\pi^2}{8}.
\end{equation*}

\section*{Problem 3}

\begin{enumerate}[label=(\alph*)]
    \item This is an exercise in change of variables. Set $y = x + \xi$ to obtain 
    \begin{equation*}
        \int_{-\pi}^\pi f(x + \xi)~dx = \int_{-\pi + \xi}^{\pi + \xi}f(y)~dy.
    \end{equation*}

    Note that 
    \begin{equation*}
        \int_{-\pi + \xi}^{\pi + \xi}f(y)~dy + \int_{-\pi}^{-\pi + \xi} f(y)~dy = \int_{-\pi}^{\pi + \xi} f(y)~dy = \int_{-\pi}^{\pi}f(y)~dy + \int_{\pi}^{\pi + \xi}f(y)~dy.
    \end{equation*}
    But, using the periodicity of $f$, we have, using the substitution $y = z + 2\pi$,
    \begin{equation*}
        \int_{\pi}^{\pi + \xi}f(y)~dy = \int_{-\pi}^{-\pi + \xi}f(z + 2\pi)~dz = \int_{-\pi}^{-\pi + \xi} f(z)~dz.
    \end{equation*}

    Hence, 
    \begin{equation*}
        \int_{-\pi + \xi}^{\pi + \xi}f(y)~dy = \int_{-\pi}^{\pi} f(y)~dy.
    \end{equation*}

    \item We have 
    \begin{equation*}
        \wh f(n) = \frac{1}{2\pi}\int_{-\pi}^\pi f(x)e^{-inx}~dx
    \end{equation*}
    Invoke the substitution $y = x - \frac{\pi}{n}$. Then, 
    \begin{equation*}
        \wh f(n) = \frac{1}{2\pi}\int_{-\pi - \frac{\pi}{n}}^{\pi - \frac{\pi}{n}}f\left(y + \frac{\pi}{n}\right)e^{-in\left(y + \frac{\pi}{n}\right)}~dy = \frac{-1}{2\pi}\int_{-\pi}^{\pi}f\left(y + \frac{\pi}{n}\right)e^{-iny}~dy,
    \end{equation*}
    where the last equality follows from part (a). Now, simply add the two equivalent formulations of $\wh f(n)$ and divide by $2$. This would give 
    \begin{equation*}
        \wh f(n) = \frac{1}{4\pi}\int_{-\pi}^\pi\left[f(y) - f\left(y + \frac{\pi}{n}\right)\right]e^{-iny}~dy.
    \end{equation*}
\end{enumerate}

\section*{Problem 4}

I shall prove (b), from which (a) would follow. Using integration by parts, we have 
\begin{equation*}
    \int_{-\pi}^\pi f(x)e^{-inx}~dx = \left[f(x)\frac{e^{-inx}}{-in}\right]_{-\pi}^\pi + \int_{-\pi}^\pi f'(x)\frac{e^{-inx}}{in}~dx = \frac{1}{in}\wh{f'}(n).
\end{equation*}

Iterating this $k$-times, we obtain 
\begin{equation*}
    \wh f(n) = \frac{1}{(in)^k}\wh{f^{(k)}}(n).
\end{equation*}

Therefore, 
\begin{equation*}
    \lim_{|n|\to\infty}|n^k\wh f(n)| = \lim_{|n|\to\infty} \wh{f^{(k)}}(n) = 0
\end{equation*}
from the Riemann-Lebesgue lemma. The conclusion follows.

\section*{Problem 5}

\begin{enumerate}[label=(\alph*)]
    \item Let $\omega$ denote $e^{ix}$. We are computing 
    \begin{align*}
        \sum_{n = -N}^N \omega^n &= \omega^{-N}\sum_{n = 0}^{2N}\omega^n\\ 
        &= \omega^{-N}\frac{\omega^{2N + 1} - 1}{\omega - 1}\\
        &= \frac{\omega^{N + 1} - \omega^{-N}}{\omega - 1}\\
        &= \frac{\omega^{N + 1/2} - \omega^{-(N + 1/2)}}{\omega^{1/2} - \omega^{-1/2}}\\
        &= \frac{2i\sin\left(N + \frac{1}{2}\right)x}{2i\sin\left(\frac{x}{2}\right)} = \frac{\sin\left(N + 1/2\right)x}{\sin(x/2)}.
    \end{align*}

    \item We have 
    \begin{align*}
        \frac{1}{N}\sum_{n = 0}^{N - 1}D_n(x) &= \frac{1}{N}\sum_{n = 0}^{N - 1}\frac{2\sin(x/2)\sin(n + 1/2)x}{2\sin^2(x/2)}\\
        &= \frac{1}{N}\sum_{n = 0}^{N - 1}\frac{\cos(nx) - \cos(n + 1)x}{2\sin^2(x/2)}\\
        &= \frac{1}{N}\frac{1 - \cos(Nx)}{2\sin^2(x/2)} = \frac{\sin^2(Nx/2)}{N\sin^2(x/2)}.
    \end{align*}

    \item We may suppose that $N > 1$ since $\log 1 = 0$. Note that $D_N$ is an even function and hence, 
    \begin{align*}
        \int_{-\pi}^{\pi}|D_N(x)|~dx &= 2\int_{0}^{\pi} |D_N(x)|~dx\\
        &\ge\sum_{n = 0}^{N - 1}\int_{\frac{n\pi}{N + \frac{1}{2}}}^{\frac{(n + 1)\pi}{N + \half}}\left|\frac{\sin(N + 1/2)x}{\sin(x/2)}\right|~dx
    \end{align*}
    Therefore, it suffices to show that $\int_{0}^{\pi}|D_N(x)|\ge c\log N$ for some constant $c > 0$. On the interval $\left[\frac{n\pi}{N + \half}, \frac{(n + 1)\pi}{N + \half}\right]$, 
    \begin{equation*}
        \sin(x/2)\le\sin\left(\frac{(n + 1)\pi}{2N + 1}\right)\le\frac{(n + 1)\pi}{2N + 1}
    \end{equation*}
    since $\sin(x/2)$ is an increasing function on $[0,\pi]$ and $\sin x\le x$ on $[0,\infty)$. Hence, 
    \begin{align*}
        \int_{\frac{n\pi}{N + \frac{1}{2}}}^{\frac{(n + 1)\pi}{N + \half}}\left|\frac{\sin(N + 1/2)x}{\sin(x/2)}\right|~dx &\ge\frac{2N + 1}{(n + 1)\pi}\int_{\frac{n\pi}{N + \frac{1}{2}}}^{\frac{(n + 1)\pi}{N + \half}}\left|{\sin(N + 1/2)x}\right|~dx\\
        &= \frac{2 N + 1}{(n + 1)\pi}\frac{1}{N + \frac{1}{2}}\int_{n\pi}^{(n + 1)\pi}|\sin(y)|~dy\\
        &= \frac{2}{(n + 1)\pi}\times 2 = \frac{4}{(n + 1)\pi}.
    \end{align*}

    Consequently, 
    \begin{equation*}
        \int_{0}^{\pi} |D_N(x)|~dx\ge\frac{4}{\pi}\sum_{n = 0}^{N - 1}\frac{1}{n + 1} = \frac{4}{\pi}H_N,
    \end{equation*}
    where $H_N$ is the $N$-th Harmonic number. For $N\ge 2$, it is well known that 
    \begin{equation*}
        H_N\ge\int_{1}^{N + 1}\frac{1}{x}~dx = \log(N + 1)\ge\log N.
    \end{equation*}
    This completes the proof.

    \item This is immediate, since 
    \begin{equation*}
        \int_{-\pi}^{\pi}D_N(x)~dx = \sum_{n = -N}^{N}\int_{-\pi}^{\pi}e^{inx}~dx.
    \end{equation*}
    But for non-zero $n$, we have 
    \begin{equation*}
        \int_{-\pi}^{\pi}e^{inx}~dx = \frac{1}{in}\left[e^{inx}\right]_{-\pi}^{\pi} = 0
    \end{equation*}
    and for $n = 0$, we have $\int_{-\pi}^{\pi}1~dx = 2\pi$. Therefore, 
    \begin{equation*}
        \int_{-\pi}^{\pi}D_N(x)~dx = 2\pi.
    \end{equation*}
\end{enumerate}
\end{document}