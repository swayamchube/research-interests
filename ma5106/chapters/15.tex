\begin{definition}[Convolution of Distributions]
    Let $T\in D'(\R^n)$ and $\phi\in D(\R^n)$. Their convolution is defined as a function in $C^\infty(\R^n)$, given by 
    \begin{equation*}
        (T\ast\phi)(x) = \langle T, \phi(x - \cdot)\rangle.
    \end{equation*}
\end{definition}

We have 
\begin{equation*}
    \partial^\alpha (T\ast\phi)(x) = \langle T, \partial^\alpha(\phi(x - \cdot))\rangle = T\ast\partial^\alpha\phi(x) = \partial^\alpha T\ast\phi(x).
\end{equation*}
Be a bit careful here, the derivative is with respect to $x$ instead of $y$. As a result, $T\ast\phi$ is a smooth function on $\R^n$.

\begin{lemma}
    Let $T\in D'(\R^n)$. Consider the standard mollifiers $\eta_\varepsilon\in C^\infty_c(\R^n)$, $\eta_\varepsilon > 0$, $\Supp\eta_\varepsilon\subseteq B(0,\varepsilon)$ and $\int_{\R^n}\eta_\varepsilon = 1$.

    Then, $T\ast\eta_\varepsilon\to T$ as $\eta\to 0$ in $D'(\R^n)$. That is, $\langle T\ast\eta_\varepsilon,\phi\rangle\to\langle T,\phi\rangle$ as $\varepsilon\to 0$.

    Equivalently, 
    \begin{equation*}
        \lim_{\varepsilon\to 0}\int_{\R^n} (T\ast\eta_\varepsilon)(x)\phi(x)~dx = \langle T, \phi\rangle.
    \end{equation*}
\end{lemma}
\begin{proof}
    Let $\wt\phi(x) = \phi(-x)$. Then, 
    \begin{equation*}
        \langle T,\phi\rangle =(T\ast\wt\phi)(0).
    \end{equation*}
    Using this,
    \begin{equation*}
        \langle T\ast\eta_\varepsilon, \phi\rangle = (T\ast\eta_\varepsilon)\ast\wt\eta(0) = T\ast(\eta_\varepsilon\ast\wt\phi)(0) = \langle T, \wt{\eta_\varepsilon\ast\wt\phi}\rangle = \langle T, \wt{\eta_\varepsilon}\ast\phi\rangle = \langle T, \eta_\varepsilon\ast\phi\rangle.
    \end{equation*}
    Now, note that $\eta_\varepsilon\ast\phi\to\phi$ as $\varepsilon\to 0$. This completes the proof. We have cheated a lot in this proof btw.
\end{proof}

\begin{example}
    We define the translation operator $\tau_\xi$ as 
    \begin{equation*}
        \tau_\xi\phi(x) = \phi(x - \xi).
    \end{equation*}

    Then, 
    \begin{equation*}
        \tau_\xi\phi(x) = (\delta_\xi\ast\phi)(x)
    \end{equation*}
    where $\delta_\xi$ is the dirac distribution centered at $\xi$. In particular, $\phi = \delta_0\ast\phi$.

    Also, 
    \begin{equation*}
        \partial^\alpha\phi = \partial^\alpha\delta_0\ast\phi.
    \end{equation*}
    Therefore, derivatives can also be treated as convolutions with certain distributions (derivatives of dirac in fact).
\end{example}

\section*{Tempered Distributions}

Let 
\begin{equation*}
    \|\phi\|_{\alpha,\beta} := \sup_{\R^n} \left|x^\alpha\partial^\beta\phi\right|.
\end{equation*}

In the Schwarz class, this supremum is always finite (the definition). The Topology on the Schwarz class is the Frechet topology generated by the above seminorms.

There is a more workable definition of convergence (which determines the topology). $\phi_m\to\phi$ if for every pair of $\alpha, \beta$, the seminorms tend to $0$ blah blah.

Denote $\mathscr S'(\R^n)$ as the space of Tempered Distributions, as we saw earlier.

\begin{definition}
    Let $T\in\mathscr S'(\R^n)$. We define the Fourier Transform of $T$ as $\wh T\in\mathscr S'(\R^n)$ as 
    \begin{equation*}
        \langle\wh T,\phi\rangle = \langle T, \wh\phi\rangle.
    \end{equation*}
    The right hand side is well defined since the Fourier transform of an element in the Schwarz class is in the Schwarz class.
\end{definition}

\begin{definition}
    The inverse Fourier transform of $T\in\mathscr S'(\R^n)$ is defined as 
    \begin{equation*}
        \langle\wt T,\phi\rangle = \langle T, \wt\phi\rangle
    \end{equation*}
    where $\wt\phi(x) = \phi(-x)$.
\end{definition}

\begin{example}
    Let $\mathbf 1$ denote the constant distribution in the Tempered Class. Then, 
    \begin{equation*}
        \wh{\mathbf 1} = \sqrt{2}\delta_0,
    \end{equation*}
    where, we are working over $\R$ instead of $\R^n$.
\end{example}