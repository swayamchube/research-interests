\begin{proposition}
    Let $f\in C^2(S^1)$. Then, there is a constant $C > 0$ such that 
    \begin{equation*}
        |\wh f(n)|\le\frac{C}{n^2}
    \end{equation*}
    for all $n\in\Z\backslash\{0\}$.
\end{proposition}
\begin{proof}
    This is a standard integration by parts application. Indeed, 
    \begin{align*}
        2\pi\wh f(n) &= \int_{-\pi}^{\pi}f(\theta)e^{-in\theta}~d\theta\\
        &= -(-in)^{-1}\int_{-\pi}^{\pi}f'(\theta)e^{-in\theta}~d\theta\\
        &= (-in)^{-2}\int_{-\pi}^{\pi}f''(\theta)e^{-in\theta}~d\theta.
    \end{align*}
    We know that $f''$ is bounded by $M$ on $[-\pi, \pi]$ (owing to it being continuous). Then, 
    \begin{equation*}
        2\pi|\wh f(n)|\le \frac{1}{n^2}\times 2\pi\times M.
    \end{equation*}
    This completes the proof.
\end{proof}

\begin{corollary}
    If $f\in C^2(S^1)$, the Fourier series of $f$ converges uniformly to $f$.
\end{corollary}

\begin{definition}
    A function $f$ on the circle is said to be H\"older continuous of class $\alpha > 0$. if there is a $K > 0$ such that $|f(x) - f(y)|\le K|x - y|^\alpha$ for all $x,y\in S^1$. This is denoted by $f\in C^{0,\alpha}(S^1)$.
\end{definition}

\begin{remark}
    The uniform convergence of $S_Nf$ to $f$ holds for $f\in C^{0,\alpha}(S^1)$ where $\alpha > 1/2$. This is due to Bernstein.
\end{remark}

\section{Convolutions}

We have 
\begin{align*}
    S_Nf(x) &= \sum_{k = -N}^N \wh f(k)e^{ikx}\\
    &= \sum_{k = -N}^N \frac{1}{2\pi}\int_{-\pi}^\pi f(y)e^{ik(x - y)}~dy\\
    &= \frac{1}{2\pi}\int_{-\pi}^{\pi}f(y)\sum_{k = -N}^Ne^{ik(x - y)~dy}\\
    &= \frac{1}{2\pi}\int_{-\pi}^{\pi}f(y)D_N(x - y)~dy\\
    &= (f\ast D_N)(x).
\end{align*}
This is called a convolution.

\begin{definition}
    Given $f,g\in L^1(S^1)$, define 
    \begin{equation*}
        (f\ast g)(x) = \frac{1}{2\pi}\int_{-\pi}^{\pi}f(y)g(x - y)~dy.
    \end{equation*}
    It is not hard to argue that $f\ast g\in L^1(S^1)$ using Fubini's Theorem.
\end{definition}

\begin{proposition}
    Let $u,v,w\in L^1(S^1)$. Then, 
    \begin{enumerate}[label=(\alph*)]
        \item $u\ast v = v\ast u$.
        \item $u\ast(v + w) = u\ast v + u\ast w$.
        \item $(\lambda u)\ast v = \lambda(u\ast v)$.
        \item $u\ast (v\ast w) = (u\ast v)\ast w$.
        \item $\wh{u\ast v}(k) = \wh u(k)\wh v(k)$.
        \item $u\ast v$ is continuous if $u,v\in L^1(S^1)$ and bounded.
    \end{enumerate}
\end{proposition}
\begin{proof}
    We have 
    \begin{align*}
        2\pi (v\ast u)(x) = \int_{-\pi}^\pi v(y)u(x - y)~dy = \int_{-\pi + x}^{\pi + x} v(x - z)u(z)~dz = \int_{-\pi}^\pi v(x - z)u(z)~dz = 2\pi(u\ast v)(x).
    \end{align*}

    For (e), 
    \begin{align*}
        \wh{u\ast v}(k) &= \frac{1}{2\pi}\int_{-\pi}^\pi(u\ast v)(x)e^{-ikx}~dx\\
        &= \frac{1}{(2\pi)^2}\int_{-\pi}^\pi \int_{-\pi}^\pi u(y)v(x - y)e^{-ikx}~dy dx\\
        &= \frac{1}{(2\pi)^2}\int_{-\pi}^\pi\int_{-\pi}^\pi u(y)e^{-iky}v(x - y)e^{-ik(x - y)}~dydx\\
        &= \frac{1}{2\pi}\int_{-\pi}^\pi\wh u(k)v(x -y)e^{-ik(x - y)}~dx\\
        &= \wh u(k)\wh v(k).
    \end{align*}
\end{proof}