\section*{Heat equation}

$u: [0,\infty)\times\R^n\to\R$ which satisfies 
\begin{equation*}
    \begin{cases}
        \partial_t u = \Delta u = \sum_{i = 1}^n \partial_{ii}u\\
        u(0,\cdot) = u_0(\cdot).
    \end{cases}
\end{equation*}

The heat equation has a regularization effect, that is, even if $u_0$ is a terrible function, the solution $u$ for positive time will be smooth. It is not symmetric with respect to time. 

\section{Fourier Transform}

\begin{definition}
    Let $f\in L^1(\R^n)$, define, for $\xi\in\R^n$,
    \begin{equation*}
        \wh f(\xi) = \frac{1}{(2\pi)^{n/2}}\int_{\R^n}e^{-i\xi\cdot x}f(x)~dx.
    \end{equation*}
    This yields a function $\wh f:\R^n\to\R$.
\end{definition}

Note that we shall restrict our attention to Schwarz Space instead of $L^1$.

\begin{equation*}
    \mathscr S(\R^n) = \{f\in C^\infty(\R^n)\colon \lim_{x\to\infty}(1 + |x|^2)^{m/2}|\partial^\alpha f(x)| = 0,~\forall m\in\N\}
\end{equation*}

Obviously, $C_c^\infty(\R^n)\subseteq\mathscr S(\R^n)$. Note that the Schwartz class is equivalent to the class of rapidly decreasing functions 
\begin{equation*}
    \lim_{|x|\to\infty}|x|^m|\partial^\alpha f| = 0
\end{equation*}

\begin{example}
    $f(x) = e^{-|x|^2}\in\mathscr S(\R^n)$. Note that the partial derivatives of this function are of the form $p(x)f(x)$ where $p$ is a multivariate polynomial. Therefore, it suffices to show that $|p(x)f(x)|\to 0$ as $x\to\infty$ for all polynomials $p$, whence, it suffices to do this for monomials. Consider a monomial $x_1^{m_1}\dots x_r^{m_n}$. Then, 
    \begin{equation*}
        |x_1^{m_1}\dots x_n^{m_n}e^{-|x|^2}| = \prod_{i = 1}^n |x_ie^{-x_i^2}|.
    \end{equation*}
    This completes the proof.
\end{example}

We now contend that the Fourier Transform of a function in the Schwartz space lies in the Schwatz space.

First, we show that $\mathscr S(\R^n)\subseteq L^1(\R^n)$. Indeed, note that there is a positive constant $M > 0$ such that 
\begin{equation*}
    |f(x)|\le\frac{M}{(1 + |x|^2)^n}\le\frac{M}{x_1^2\dots x_n^2}
\end{equation*}
Choose some $R > 0$ and let $A$ denote the closed cube of length $R$ centered at the origin. Then,
\begin{equation*}
    \int_{\R^n}|f(x)|~dx = \int_A |f(x)|~dx + \int_{\R^n\backslash A}f(x)~dx.
\end{equation*}
The first term is bounded. For the second, use Fubini's theorem and integrate $M/x_1^2\dots x_n^2$ instead. Another way to see this is to integrate over the product manifold $(1,\infty)\times S^{n - 1}$, which is diffeomorphic to $\R^n\backslash\overline B(0, 1)$ and invoke Fubini's Theorem for differential forms.

Now, we examine the differentiation properties of the fourier transform.

\begin{align*}
    \wh{\partial_j f}(\xi) = i\xi_j\wh f(\xi)\qquad 1\le j\le n\\
    \partial_j\wh f(\xi) = -i\wh{x_j f}(\xi)\qquad 1\le j\le n
\end{align*}