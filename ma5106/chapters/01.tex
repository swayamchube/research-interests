The origins of Fourier analysis lie in solving the heat equation:
\begin{equation*}
    \Delta u = \partial_t u
\end{equation*}
where $\Delta$ denotes the Laplacian.

In order to solve this, Fourier believed for a long time that one could expand a function as a series 
\begin{equation*}
    f\sim \sum_k a_k\sin kx + \sum b_k\cos kx.
\end{equation*}

This is not true. In 1876, Paul Du Bois-Reymond gave an example of a continuous function whose Fourier series does not converge. But in 1966, Carleson showed that given an $L^2$ function on $[0,1]$, the points at which the Fourier series does not converge has measure $0$.

There are many applications to PDEs, in solving the 
\begin{description}
    \item[Laplace Equation:] $\Delta u = 0$,
    \item[Heat Equation:] $\partial_t u = \Delta u$,
    \item[Wave Equation:] $\partial_{tt} u = \Delta u$.
\end{description}


\begin{definition}[Fourier Series]
    Given $f\in L^1[a, b]$, its $k$-th \emph{Fourier coefficient} is defined as 
    \begin{equation*}
        \wh f(k) := \frac{1}{L}\int_a^b f(x)\exp\left(-\frac{2\pi i k}{L}x\right)~dx.
    \end{equation*}
    where $L = b - a$.

    The \emph{Fourier series} of $f$ is given formally by 
    \begin{equation*}
        f\sim\sum_{k\in\Z}\wh f(k)\exp\left(\frac{2\pi i k}{L}x\right).
    \end{equation*}
\end{definition}

The question is whether 
\begin{equation*}
    \lim_{n\to\infty}\sum_{k= -n}^n \wh f(k)\exp\left(\frac{2\pi ik}{L}x\right) = f(x)
\end{equation*}
in the following cases: 
\begin{itemize}
    \item if $f\in L^1[a,b]$. Here we cannot expect pointwise convergence because one can just change the value of $f$ at a single point without affecting its Fourier series.
    \item if $f\in C[a,b]$. This is not true because of an example by Paul Du Bois-Reymond.
    \item if $f\in C^1[a,b]$ then this is true.
    \item if $f\in L^2[a,b]$, then there may not be pointwise convergence but there is convergence in the $L^2$-norm.
\end{itemize}

There are notions of convergence other than pointwise and uniform. For example Ces\`aro and Abel. Fej\'er had proved that for continuous functions, the Ces\`aro sums converge uniformly to the function, whatever that means.

\begin{example}
    Consider the function $f:[-\pi,\pi]\to\R$ given by $f(x) = x$. Then, 
    \begin{equation*}
        \wh f(k) = \frac{1}{2\pi}\int_{-\pi}^\pi x\exp(-ikx)~dx = 
        \begin{cases}
            0 & k = 0\\
            \frac{(-1)^ki}{k} & k\ne 0.
        \end{cases}
    \end{equation*}
    The Fourier series is then given by 
    \begin{equation*}
        \sum_{k\in\Z\backslash\{0\}}(-1)^{k + 1}\frac{\sin kx}{k}.
    \end{equation*}
\end{example}