Fix a compact set $K$. Let $\varphi, \psi$ be elements of $D(\R^n)$. Then, we define 
\begin{equation*}
    d_K(\varphi, \psi) = \max_{i\in\N}\frac{1}{2^i}\frac{\|\varphi - \psi\|_{K, i}}{1 + \|\varphi - \psi\|_{K, i}}.
\end{equation*}
where 
\begin{equation*}
    \|\varphi\|_{K, i} = \sum_{j = 0}^{i} \|\varphi^{(j)}\|_K.
\end{equation*}
In this metric, $D(K)$ is a complete metric space. This is called a Fr\'echet space.

A local base for this topology is $\|\varphi\|_{K, l}\le\frac{1}{l}$ for all $l\in\N$. Therefore, $D(K)$ forms a locally convex topological vector space.

\hrulefill 

Take an exhaustion of $\R^n$, the natural choice is closed balls of integer radius. Let us denote this sequence by $\{K_n\}$ which are compact sets in $\R^n$. Further, $K_n\subseteq\mathrm{int}~K_{n + 1}$ and $\bigcup K_n = \R^n$.

\begin{equation*}
    p_{l, a}(\varphi) = \sup_{m\ge 1}\sup_{\R\backslash K_m}\frac{1}{a_m}\sum_{|\alpha| < l_m}|\partial^\alpha\varphi|
\end{equation*}
where $a_m\to 0$ and $l_m\to\infty$. Note that $l$ and $a$ are sequences. This gives a family of seminorms. These generate a topology on $D(\R^n)$, with which this is a complete metric space.

\hrulefill 

Distributions are dual of $D(\R^n)$ with the weak-$\ast$ topology. $T\in D'(\R^n)$ is a linear functional such that 
\begin{enumerate}
    \item $T$ is continuous.
    \item If $\varphi_m\to\varphi$ in $D(\R^n)$ then $T(\varphi_m)\to T(\varphi)$.
    \item For every compact $K\subseteq\R^n$, there is a positive integer $i_K > 0$ and a constant $C_K$ such that 
    \begin{equation*}
        |T(\varphi)|\le C_K\sum_{|\alpha|\le i_K}|\partial^\alpha\varphi|.
    \end{equation*}
    For every $\varphi$ such that $\Supp\varphi\subseteq K$. 
\end{enumerate}

\begin{example}
    The Dirac Delta is an example of a distribution, $\delta_\xi$ where $\xi\in\R^n$. The action is given by $\delta_\xi(\varphi) = \varphi(\xi)$. It is not hard to see that this is indeed a distribution.
\end{example}

\begin{example}
    Take $f\in L^1_{loc}(\R^n)$, then $f$ generates a distribution $T_f\in D'(\R^n)$ whose action is defined as 
    \begin{equation*}
        T_f(\varphi) = \int_{\R^n} \varphi f.
    \end{equation*}
    This is also a zeroth order distribution, that is $i_K = 0$ works.
\end{example}

\begin{example}
    Fix $\xi\in\R$. Then, $T$ defined as 
    \begin{equation*}
        T(\varphi) = \varphi''(\xi)
    \end{equation*}
    Show that $T\in D'(\R)$, that is, it is a distribution.
\end{example}

\hrulefill 

Now, we define the derivative of a distribution. The idea is to pass the derivative on the action.

\begin{theorem}[Also a Definition]
    Let $T\in D'(\R^n)$. We define the distributional derivative $\partial^\alpha T$ of order $|\alpha|$ of $T$ as 
    \begin{equation*}
        (\partial^\alpha T)(\varphi) = (-1)^\alpha T(\partial^\alpha\varphi).
    \end{equation*}
\end{theorem}
This is obviously a linear functional on $D(\R^n)$. Also, the local boundedness thing is quite obvious.

\hrulefill 

We have $C_c^\infty(\R^n)\subseteq\mathscr S(\R^n)\subseteq C^\infty(\R^n)$ where all the inclusions are strict. We care about the dual of the middle object. The dual of the Schwarz space is the space of Tempered Distributions.

Taking duals, we have 
\begin{equation*}
    \mathcal E'(\R^n)\subseteq\mathscr S'(\R^n)\subseteq\mathcal D'(\R^n).
\end{equation*}
The first class is called the space of distributions with compact support.

Let $\Omega\subseteq\R^n$ be open. Then, we say that $T$ vanishes on $\Omega$ if $T(\varphi) = 0$ for all $\varphi\in D(\Omega)$. 

We define the support of $T$ to be $\R^n\backslash\bigcup\Omega$ where the union is taken over all open sets on which $T$ vanishes. A compactly supported distribution is one whose support is a compact set (quite obvious).

Note that if $T$ has support $K$ and $\psi\in D(\R^n)$, we can define 
\begin{equation*}
    \langle\psi T, \varphi\rangle = T(\varphi\psi).
\end{equation*}

\begin{theorem}
    Let $T\in D'(\R^n)$. Then for any multi-index $\alpha$, there are continuous functions $g_\alpha\in C(\R^n)$ such that any compact set $K$ intersects the suport of $g_\alpha$'s. 
    \begin{equation*}
        \langle T, \varphi\rangle = \sum_{\alpha} (-1)^\alpha\int_{\R^n} g_\alpha\partial^\alpha\varphi.
    \end{equation*}
\end{theorem}
