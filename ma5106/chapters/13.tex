\begin{theorem}[Parseval's Identity]
    If $u,v\in\mathscr S$, then $\langle u, v\rangle_{L^2} = \langle\wh u,\wh v\rangle_{L^2}$.
\end{theorem}

\begin{theorem}[Plancherel's Formula]
    $\|u\|_{L^2} = \|\wh u\|_{L^2}$.
\end{theorem}

Note that the Schwarz space is dense in $L^2$ and hence, one can extend the Fourier transform to all of $L^2$. This would give that the Fourier transform is an isometry from $L^2$ to $L^2$ and is also a bijection.

\section*{Distributions and Weak Derivatives}

We work with locally integrable functions. A function $f$ is said to be locally integrable if $f\in L^1(K)$ for all compact $K\subseteq\R^n$. This class of functions is denoted by $L^1_{loc}$.

\subsection*{Weak Derivative}

\begin{definition}
    If $f\in L^1_{loc}(\R^n)$. We say that $f$ is weakly differentiable with respect $x_i$ if there exists $g_i\in L^1_{loc}$ such that for any ``test function'' $\varphi\in C_c^{\infty}(\R^n)$, 
    \begin{equation*}
        \int_{\R^n} f\partial_i\varphi = -\int_{\R^n} g_i\varphi.
    \end{equation*}

    The function $g_i$ is the $i$-th weak partial derivative of $f$.
\end{definition}

\begin{definition}
    Let $\alpha$ be a multiindex and $f\in L^1_{loc}(\R^n)$. We say that $f$ has weak derivative of order $|\alpha|$ if there is a function $g\in L^1_{loc}(\R^n)$ such that 
    \begin{equation*}
        \int_{\R^n} f\partial^{\alpha}\varphi = (-1)^{|\alpha|}\int_{\R^n} g\varphi.
    \end{equation*}
    For all $\varphi\in C_c^\infty(\R^n)$.
\end{definition}

Note that the weak derivative is unique up to measure $0$ sets. One can also show that mixed weak derivatives are equal upto measure $0$ sets. 

\begin{example}
    \begin{equation*}
        f(x) = 
        \begin{cases}
            x & x > 0\\
            0 & x\le 0.
        \end{cases}
    \end{equation*}

    Consider 
    \begin{equation*}
        g(x) = \begin{cases}
            1 & x\ge 0\\
            0 & x < 0.
        \end{cases}
    \end{equation*}

    We contend that $g$ is the weak derivative of $f$. Note that $f$ is differentiable on $(0,\infty)$ and hence, we can use integration by parts to conclude that $g$ is the required function. I guess this can be used to prove a similar statement for piecewise smooth functions.
\end{example}

\begin{example}
    The function $f = \chi_{[0,\infty)}$ is not weakly differentiable. If it were, then there would exist $g\in L^1_{loc}(\R)$ such that 
    \begin{equation*}
        \int_{0}^\infty\varphi' = -\int_{\R} g\varphi,
    \end{equation*}
    for every test function $\varphi$ on $\R$. The left hand side is $-\varphi(0)$.

    Therefore, 
    \begin{equation*}
        \int_{\R}g\varphi = \varphi(0)
    \end{equation*}
    for all test functions $\varphi$. A standard argument shows that this is not possible. Ezpz nimbu squeezy.
\end{example}

\subsection*{Distributions}

Let $T$ be a linear functional on $C_c^{\infty}(\R^n)$. Let $D(\R^n)$ be the same space with the inductive limit topology. This is also called the space of ``test functions''.

$D(\R^n)$ is a complete metric space consisting of $C_c^\infty$-functions. 

\begin{definition}
    In $D(\R^n)$, $\varphi_m\to\varphi$ if $\Supp\varphi_m\subseteq K$, compact in $\R^n$. And $\Supp(\varphi)\subseteq K$. On this compact set, $\varphi^{(n)}_m\to \varphi^{(n)}$ for all $n\ge 0$ in the sup norm.
\end{definition}

\begin{definition}[Distributions]
    Let $T\in D'(\R^n)$, where $D'(\R^n)$ is the dual of $D(\R^n)$. We write $T(\varphi)$.
\end{definition}

It suffices to check 
\begin{equation*}
    |\langle T, \varphi\rangle|\le C_K\sum_{|\alpha|\le i_K}\sup_{K}|\partial^\alpha\varphi|
\end{equation*}
for every compact $K$ and for all $\varphi$ such that $\Supp(\varphi)\subseteq K$ and $i_K < \infty$.