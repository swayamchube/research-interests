\section{Classification of Singularities}

\begin{definition}
    A function $f$ has an \textit{isolated singularity} at a point $z = a$ if there is $R > 0$ such that $f$ is analytic on $0 < |z - a| < R$. The point $a$ is called a \textit{removable singularity} if there is an analytic function $g: B(a,R)\to\bbC$ such that $f(z) = g(z)$ for $0 < |z - a| < R$.
\end{definition}

\begin{theorem}
    If $f$ has an isolated singularity at $a$, then the point $z = a$ is a removable singularity if and only if 
    \begin{equation*}
        \lim_{z\to a}(z - a)f(z) = 0
    \end{equation*}
\end{theorem}
\begin{proof}
    The forward direction is obvious. We shall show the reverse direction, that is, suppose $\lim\limits_{z\to a}(z - a)f(z) = 0$. There is $R > 0$ such that $f$ is analytic in $0 < |z - a| < R$. Now, define the function $g: B(a,R)\to\bbC$ such that $g(z) = (z - a)f(z)$. It is obvious that $g$ is continuous. It suffices to show that $g$ is analytic, since then, there would exist an analytic function $h$ such that $g(z) = (z - a)h(z)$, implying the desired conclusion.

    To show that $g$ is analytic, we shall use Morera's Theorem. Let $T$ be a triangle in $B(a,R)$. Note that since this region is convex, it suffices to choose any three points $a,b,c$ in the interior and they would form a valid triangle. Let $\Delta$ denote the interior of $T$. If $a\notin\Delta$, then $T$ is nulhomotopic and due to \thref{thm:cauchy-version1}, the integral $\int_T g$ must be zero. 

    Next, if $a$ is a vertex of the triangle, say $[a,b,c,a]$, then for any points $x$ and $y$ on the line segments $[a,b]$ and $[a,c]$, 
    \begin{equation*}
        \int_{[a,b,c,a]} g = \int_{[a,x,y]} g + \int_{[x,b,c,y]} g = \int_{[a,x,y]} g
    \end{equation*}
    where the last equality follows from \thref{thm:cauchy-version1}. Since $g$ is continuous, there is $r > 0$ such that for all $t\in B(a,r)$, $|g(t)| < \varepsilon$. And thus, $|\int_{[a,x,y]} g| < \varepsilon\ell$ where $\ell$ is the permieter of $T$. It is now obvious that the integral must be zero. 

    Finally, suppose $a\in\Delta$ where $T = [b,c,d,b]$. The integral is now given by 
    \begin{equation*}
        \int_{[b,c,d,a]}g = \int_{[a,b,c,a]}g + \int_{[a,c,d,a]}g + \int_{[a,d,b,a]} g = 0
    \end{equation*}
    This completes the proof.
\end{proof}

\begin{definition}[Pole, Essential Singularity]
    If $z = a$ is an isolated singularity of $f$, then $a$ is a \textit{pole} of $f$ if $\lim\limits_{z\to a}|f(z)| = \infty$. If an isolated singularity is niether a pole nor a removable singularity, it is then called an \textit{essential singularity}.
\end{definition}

\begin{theorem}
    Let $f: G\backslash\{a\}\to\bbC$ be analytic with a pole at $z = a$. Then there is an analytic function $g:G\to\bbC$ and a positive integer $m$ such that 
    \begin{equation*}
        f(z) = \frac{g(z)}{(z - a)^m}\quad\text{on $G\backslash\{a\}$}
    \end{equation*}
    and $g(a)\ne 0$.
\end{theorem}
\begin{proof}
    Consider the analytic function $h:G\backslash\{a\}\to\bbC$ given by $h = \frac{1}{f}$. Then it is obvious that $\lim\limits_{z\to a}f(z) = 0$, as a result, $f$ has a removable singularity at $z = a$, and thus, there is an analytic function $\tilde h: G\to\bbC$ such that $h = \tilde h$ on $G$. Now, since $\tilde h(a) = 0$, there is a positive integer $m$ and an analytic function $g:G\to\bbC$ such that $\tilde h(z) = (z - a)^m g(z)$. As a result, we see that 
    \begin{equation*}
        f(z) = \frac{1}{(z - a)^m}\frac{1}{g(z)}
    \end{equation*}
    and the conclusion follows.
\end{proof}

\begin{definition}
    If $f$ has a pole at $z = a$, and $m$ is the smallest positive integer such that $f(z)(z - a)^m$ has a removable singularity at $z = a$, then $f$ is said to have a \textit{pole of order $m$} at $z = a$.
\end{definition}

\begin{definition}
    Let $\{z_n\}_{n\in\Z}$ be a doubly infinite sequence of complex numbers. We say that $\sum\limits_{n = -\infty}^\infty z_n$ is \textit{absolutely convergent} if both $\sum\limits_{n = 0}^\infty z_n$ and $\sum\limits_{n = 1}^\infty z_{-n}$ are absolutely convergent.
\end{definition}

We denote the annular region $R_1 < |z - a| < R_2$ by $\ann(a,R_1,R_2)$.

\begin{theorem}[Laurent Series Development]
    Let $f$ be analytic on $\ann(a,R_1,R_2)$. Then 
    \begin{equation*}
        f(z) = \sum_{n = -\infty}^\infty a_n(z - a)^n
    \end{equation*}
    where the convergence is absolute and uniform over $\overline{\ann}(a,r_1,r_2)$ for $R_1 < r_1 < r_2 < R_2$. Also the coefficients $a_n$ are given by the formula 
    \begin{equation*}
        a_n = \frac{1}{2\pi i}\int_\gamma\frac{f(z)}{(z - a)^{n + 1}}~dz
    \end{equation*}
    where $\gamma$ is the circle $|z - a| = r$ for all $R_1 < r < R_2$. Furthermore, this series is unique.
\end{theorem}
\begin{proof}
    
\end{proof}

\section{Residues}

\begin{definition}
    Let $f$ have an isolated singularity at $z = a$ and let 
    \begin{equation*}
        f(z) = \sum_{n = -\infty}^\infty a_n(z - a)^n
    \end{equation*}
    be its Laurent expansion about $z = a$. Then the \textit{residue} of $f$ at $z = a$ is defined as $a_{-1}$.
\end{definition}

\begin{theorem}[Weak Residue Theorem]\thlabel{thm:weak-residue}
    Let $f$ be analytic in the region $G$ except for isolated \textbf{poles} $a_1,\ldots,a_n\in G$. If $\gamma$ is a closed rectifiable curve in $G$ which does not pass through any of the points $a_k$ and if $\gamma$ is nulhomotopic in $G$, then 
    \begin{equation*}
        \frac{1}{2\pi i}\int_\gamma f = \sum_{k = 1}^n n(\gamma,a_k)\Res(f,a_k)
    \end{equation*}
\end{theorem}
\begin{proof}
    Let $S_j$ denote the singular part of $f$ at $a_j$. Then, $g = f - \sum_{k = 1}^n S_k$ has removable singularities at $a_1,\ldots,a_n$. As a result,
    \begin{equation*}
        0 = \int_\gamma g = \int_\gamma f - \sum_{k = 1}^n\int_\gamma S_k 
    \end{equation*}
    and the conclusion follows.
\end{proof}

There is a stronger version of the above theorem wherein the word \textit{poles} is replaced by \textit{singularities}. We shall prove this later.

\begin{proposition}
    Suppose $f$ has a pole of order $m$ at $z = a$ and let $g(z) = (z - a)^mf(z)$. Then, 
    \begin{equation*}
        \Res(f,a) = \frac{1}{(m - 1)!}g^{(m - 1)}(a)
    \end{equation*}
\end{proposition}
\begin{proof}
    Follows from the definition.
\end{proof}

\subsection*{Evaluating Integrals using the Residue Theorem}

\begin{example}
    Evaluate: 
    \begin{equation*}
        \int_{-\infty}^\infty\frac{x^2}{1 + x^4}~dx
    \end{equation*}
\end{example}
\begin{proof}[Solution]
Define the contour 
\begin{equation*}
    \gamma := [-R,R]\cup\underbrace{\{Re^{it}\mid t\in[0,\pi]\}}_{\Gamma}\qquad R > 1
\end{equation*}
and the function $f(z) = \frac{z^2}{1 + z^4}$, which has poles of order $1$ at 
\begin{equation*}
    \cis\left(\frac{\pi}{4}\right),
    \cis\left(\frac{3\pi}{4}\right),
    \cis\left(\frac{5\pi}{4}\right),
    \cis\left(\frac{7\pi}{4}\right)
\end{equation*}

Within our contour, we have only $a_1 = \cis\left(\frac{\pi}{4}\right)$ and $a_2 = \cis\left(\frac{3\pi}{4}\right)$ and 
\begin{align*}
    \Res(f,a_1) &= \lim_{z\to a_1}(z - a_1)f(z) = \frac{1}{4a_1} = \frac{1}{4}\cis\left(-\frac{\pi}{4}\right)\\
    \Res(f,a_2) &= \lim_{z\to a_2}(z - a_2)f(z) = \frac{1}{4a_2} = \frac{1}{4}\cis\left(-\frac{3\pi}{4}\right)
\end{align*}

\begin{equation*}
    \int_\gamma f(z)~dz =  \frac{\pi i}{2}\left(\cis\left(-\frac{\pi}{4}\right) + \cis\left(-\frac{3\pi}{4}\right)\right) = \frac{\pi}{\sqrt{2}}
\end{equation*}

Now, 
\begin{equation*}
    0\le\int_\Gamma f\le\int_\Gamma\frac{R^2}{|1 + z^4|}~|dz|\le\int_\Gamma\frac{\pi R^3}{R^4 - 1}
\end{equation*}
And in the limit $R\to\infty$, $\int_\Gamma f = 0$. The conclusion follows.
\end{proof}

\begin{example}
    Show that 
    \begin{equation*}
        \int_{-\infty}^\infty\frac{e^{ax}}{1 + e^x}~dx = \frac{\pi}{\sin\pi a}
    \end{equation*}
    for $0 < a < 1$.
\end{example}
\begin{proof}
    Consider the function $f(z) = \frac{e^{az}}{1 + e^z}$, which is analytic except for poles at $(2k + 1)\pi i$ for all $k\in\Z$. Let $\gamma$ denote the rectangular contour:
    \begin{equation*}
        -R\longrightarrow R\longrightarrow R + 2\pi i\longrightarrow -R + 2\pi i\longrightarrow -R
    \end{equation*}
    We note that 
    \begin{equation*}
        n(\gamma, (2k + 1)\pi i) = 
        \begin{cases}
            1 & k = 0\\
            0 & \text{otherwise}
        \end{cases}
    \end{equation*}
    Furthermore, 
    \begin{equation*}
        \lim_{z\to\pi i}(z - \pi i)\frac{e^{az}}{1 + e^z} = -e^{a\pi i}
    \end{equation*}
    Therefore, we have, due to \thref{thm:weak-residue}, that 
    \begin{equation*}
        \frac{1}{2\pi i}\int_\gamma\frac{e^{az}}{1 + e^z}~dz = -e^{a\pi i}
    \end{equation*}
    It is not hard to argue that the integral on the segments $R\to R + 2\pi i$ and $-R + 2\pi i\to -R$ both tend to $0$ as $R\to\infty$. Thus, in the limit $R\to\infty$, we have 
    \begin{equation*}
        \int_{-R}^R f + \int_{R + 2\pi i}^{-R + 2\pi i} f = -e^{a\pi i}
    \end{equation*}
    Further, 
    \begin{equation*}
        \int_{R + 2\pi i}^{-R + 2\pi i } f = e^{2a\pi i}\int_{R}^{-R}\frac{e^{ax}}{1 + e^x}dx
    \end{equation*}
    Thus, 
    \begin{equation*}
        (1 - e^{2a\pi i})\int_{-\infty}^\infty f = (-2\pi i)e^{a\pi i}
    \end{equation*}
    Thus, 
    \begin{equation*}
        \int_{-\infty}^\infty f = \frac{2\pi i}{e^{a\pi i} - e^{-a\pi i}} = \frac{\pi}{\sin\pi a}
    \end{equation*}
\end{proof}

The next example has a rather unmotivated solution but we present it anyways since it is an important result to keep in mind.

\begin{example}
    Let $u\in\R\backslash\Z$. Then, show that 
    \begin{equation*}
        \sum_{n = -\infty}^\infty\frac{1}{(u + n)^2} = \frac{\pi}{\sin^2\pi u}
    \end{equation*}
\end{example}
\begin{proof}
    Consider the meromorphic function
    \begin{equation*}
        f(z) = \frac{\pi\cot\pi z}{(u + z)^2}
    \end{equation*}
    It has poles at $k$ for $k\in\Z$ and $-u$. Let $N$ be an integer such that $N > |u|$ and let $R = N + 1/2$. This contour contains the following poles: 
    \begin{equation*}
        \{-u\}\cup\{k\in\Z\mid -N\le k\le N\}
    \end{equation*}
    The residue at $z = k\in\Z$ is given by 
    \begin{equation*}
        \lim_{z\to k}(z - k)\frac{\pi\cot\pi z}{(u + z)^2} = \frac{\pi}{(u + k)^2}
    \end{equation*}
    On the other hand, the residue at $z = -u$ is the coefficient $a_{-1}$ in the Laurent expansion of $f(z)$ around $z = -u$. Since $u$ is not an integer, $\pi\cot\pi z$ is analytic in a ball around $u$, and the required coefficient is given by $f'(u) = -\frac{\pi^2}{\sin^2\pi u}$.
    Hence, 
    \begin{equation*}
        \sum_{n = -N}^N\frac{\pi}{(u + n)^2} = \int_{|z| = R}f(z)~dz + \frac{\pi^2}{\sin^2\pi u}
    \end{equation*}
    Therefore, it suffices to show that the integral on the circle is zero. \textcolor{red}{TODO: Add in later}
\end{proof}

\section{Argument Principle}

\begin{definition}[Meromorphic]
    A function which is analytic on a region except for poles is said to be \textit{meromorphic} on that region.
\end{definition}

\begin{theorem}[Argument Principle]
    Let $f$ be meromorphic in $G$ with poles $p_1,\ldots,p_m$ and zeros $z_1,\ldots,z_n$ counted according to multiplicity. If $\gamma$ is a closed rectifiable curve which is nulhomotopic and not passing through any of the aforementioned points, then 
    \begin{equation*}
        \frac{1}{2\pi i}\int_\gamma\frac{f'(z)}{f(z)}~dz = \sum_{k = 1}^n n(\gamma,z_k) - \sum_{k = 1}^m n(\gamma,p_k)
    \end{equation*}
\end{theorem}
\begin{proof}
    It is not hard to argue that there is an analytic function $g$ on $G$ that does not vanish anywhere such that 
    \begin{equation*}
        \frac{f'}{f} = \sum_{k = 1}^n\frac{1}{z - z_k} - \sum_{k = 1}^m\frac{1}{z - p_k} + \frac{g'}{g}
    \end{equation*}
    Note that $g'/g$ is an analytic function and due to Cauchy's Theorem, 
    \begin{equation*}
        \int_\gamma\frac{f'}{f} = \sum_{k = 1}^nn(\gamma,z_k) - \sum_{k = 1}^mn(\gamma,p_k)
    \end{equation*}
    This completes the proof.
\end{proof}

\begin{corollary}
    Let $f$ be meromorphic in $G$ with poles $p_1,\ldots,p_m$ and zeros $z_1,\ldots,z_n$ counted according to multiplicity. If $\gamma$ is a closed rectifiable curve which is nulhomotopic and not passing through any of the aforementioned points, then for an analytic function $g$ on $G$, 
    \begin{equation*}
        \frac{1}{2\pi i}\int_\gamma g\frac{f'(z)}{f(z)}~dz = \sum_{k = 1}^n g(z_k)n(\gamma,z_k) - \sum_{k = 1}^m g(p_k)n(\gamma,p_k)
    \end{equation*}
\end{corollary}

\begin{theorem}[Rouch\'e]\thlabel{thm:rouche}
    Suppose $f$ and $g$ are meromorphic in the region $G$ and $\overline B(a,R)\subseteq G$. If $f$ and $g$ have no zeros or poles on the circle $\gamma:= \{z: |z - a| = R\}$ and $|f(z) - g(z)| < |g(z)|$ on $\gamma$, then 
    \begin{equation*}
        Z_f - P_f = Z_g - P_g
    \end{equation*}
    where $Z_f,Z_g$ denote the zeros of $f$ and $g$ in $B(a,R)$ and $P_f,P_g$ denote the poles of $f$ and $g$ in $B(a,R)$.
\end{theorem}
\begin{proof}[First Proof]
    First, note that 
    \begin{equation*}
        \left|1 - \frac{f(z)}{g(z)}\right| < 1
    \end{equation*}
    for all $z\in\{\gamma\}$. Since $(f/g)(\{\gamma\})\subseteq B(1,1)$, there is a neighborhood of $\{\gamma\}$ that is mapped into $B(1,1)$. As a result, on this neighborhood, $\log(f/g)$, the principal branch is a primitive for $(f/g)'/(f/g)$. As a result, we have 
    \begin{equation*}
        0 = \frac{1}{2\pi i}\int_\gamma\frac{(f/g)'}{(f/g)} = \frac{1}{2\pi i}\int_\gamma\left(\frac{f'}{f} - \frac{g'}{g}\right)
    \end{equation*}
    The conclusion follows.
\end{proof}

\begin{proof}[Proof 2]
    Define the function $h_t(z) = tf(z) + (1 - t)g(z)$ for all $t\in[0,1]$. Then, $h_0(z) = g(z)$ and $h_1(z) = f(z)$, further, note that on $\gamma$,
    \begin{equation*}
        |h_t(z)| = |g(z) + t(f(z) - g(z))| > 0.
    \end{equation*}
    Let 
    \begin{equation*}
        n_t = \frac{1}{2\pi i}\int_\gamma\frac{h_t'(z)}{h(z)}~dz
    \end{equation*}
    Then, $n_t$ is obviously an integer. We contend that the map $t\mapsto n_t$ is continuous. Indeed, $h_t'(z)/h_t(z)$ is a joint continuous function of $t$ and $z$ since both the numerator and denominator are continuous in $t$ and $z$, and the denominator does not vanish on $\gamma$ as we have argued above. 

    Now, since $n_t$ only takes integral values, it must be a constant function of $t$ and the conclusion follows.
\end{proof}

We now give an alternate proof of the open mapping theorem using \thref{thm:rouche}

\begin{proof}[Alternate proof of \thref{thm:open-mapping}]
    \todo{Add proof}
\end{proof}

\newpage
\section{Runge's Theorem}

This section is taken from \cite{conway}.

\begin{theorem}\thlabel{thm:runge}
    Let $K\subseteq\bbC$ be compact and $E\subseteq\bbC_\infty\backslash K$ which meets every component of $\bbC_\infty\backslash K$. If $f$ is analytic in an open set $\Omega$ containing $K$ and $\varepsilon > 0$, then there is a rational function $R(z)$ with poles only in $E$ such that 
    \begin{equation*}
        |f(z) - R(z)| < \varepsilon
    \end{equation*}
    for all $z\in K$.
\end{theorem}

We prove this result through a series of lemmas. The setup is as mentioned in the statement of \thref{thm:runge} and shall not be repeated.

\begin{lemma}
    There are straight line segments $\gamma_1,\dots,\gamma_n$ in $\Omega\backslash K$ such that 
    \begin{equation*}
        f(z) = \sum_{k = 1}^n\frac{1}{2\pi i}\int_{\gamma_k}\frac{f(w)}{w - z}~dw
    \end{equation*}
    for all $z\in K$. The line segments form a finite number of closed polygons.
\end{lemma}