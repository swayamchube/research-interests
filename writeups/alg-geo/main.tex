\documentclass{amsart}

\RequirePackage{
    amsmath, 
    amsthm, 
    amssymb, 
    amsfonts,
    xcolor, 
    geometry, 
    tikz, 
    hyperref, 
    mathrsfs,
    enumitem,
    fancyhdr,
    lipsum,
    nicefrac,
    wasysym,
    mathtools,
    xspace,
    anyfontsize,
    todonotes,
    fancyhdr,
    lastpage,
    stmaryrd,
    commath,
    float       % because vanilla LaTeX loses its mind trying to place my floats
    % titlesec
}
\RequirePackage{theoremref} % For referencing theorems
\RequirePackage[framemethod = tikz]{mdframed}
\usetikzlibrary{automata, arrows.meta, positioning, cd} % good ol' Theoretical CS 
\RequirePackage[all,cmtip]{xy} % For diagrams, praise the Freyd–Mitchell theorem 

% Input and Fonts
\RequirePackage[utf8]{inputenc}
\RequirePackage[T1]{fontenc}
\RequirePackage{mathpazo}
% \RequirePackage{euler}
\SetSymbolFont{stmry}{bold}{U}{stmry}{m}{n}
\SetSymbolFont{wasy}{bold}{U}{wasy}{m}{n}
% \RequirePackage{euler}


\hypersetup {
    colorlinks=true, % because why not?
}

\geometry{
    margin=1in, % uniformity 
}

\newtheoremstyle{thmstyle}%               % Name
  {}%                                     % Space above
  {}%                                     % Space below
  {\itshape}%                             % Body font
  {}%                                     % Indent amount
  {\bfseries}%                            % Theorem head font
  {.}%                                    % Punctuation after theorem head
  { }%                                    % Space after theorem head, ' ', or \newline
  {\thmname{#1}\thmnumber{ #2}\thmnote{ (#3)}}%                                     % Theorem head spec (can be left empty, meaning `normal')

\newtheoremstyle{defstyle}%               % Name
  {}%                                     % Space above
  {}%                                     % Space below
  {}%                                     % Body font
  {}%                                     % Indent amount
  {\bfseries}%                            % Theorem head font
  {.}%                                    % Punctuation after theorem head
  { }%                                    % Space after theorem head, ' ', or \newline
  {\thmname{#1}\thmnumber{ #2}\thmnote{ (#3)}}%                 


\theoremstyle{thmstyle}
\newtheorem{theorem}{Theorem}[section]
\newtheorem{lemma}[theorem]{Lemma}
\newtheorem{proposition}[theorem]{Proposition}
\newtheorem{remark}{Remark}[section]

\theoremstyle{defstyle}
\newtheorem{definition}[theorem]{Definition}
\newtheorem{corollary}[theorem]{Corollary}
\newtheorem{example}[theorem]{Example}

\renewcommand{\qedsymbol}{$\blacksquare$}
\renewcommand{\emptyset}{\varnothing}
\renewcommand{\footrulewidth}{\headrulewidth}
\setlength{\headheight}{15pt}

% Common Algebraic Structures
\newcommand{\R}{\mathbb{R}}
\newcommand{\Q}{\mathbb{Q}}
\newcommand{\Z}{\mathbb{Z}}
\newcommand{\N}{\mathbb{N}}
\newcommand{\bbC}{\mathbb{C}}
\newcommand{\calA}{\mathcal{A}}
\newcommand{\frakM}{\mathfrak{M}}

% Categories
\newcommand{\catTopp}{\mathbf{Top}_*}
\newcommand{\catGrp}{\mathbf{Grp}}
\newcommand{\catTopGrp}{\mathbf{TopGrp}}
\newcommand{\catSet}{\mathbf{Set}}
\newcommand{\catTop}{\mathbf{Top}}
\newcommand{\catRing}{\mathbf{Ring}}
\newcommand{\catCRing}{\mathbf{CRing}} % comm. rings
\newcommand{\catMod}{\mathbf{Mod}}
\newcommand{\catMon}{\mathbf{Mon}}
\newcommand{\catMan}{\mathbf{Man}} % manifolds
\newcommand{\catDiff}{\mathbf{Diff}} % smooth manifolds
\newcommand{\catAlg}{\mathbf{Alg}}
\newcommand{\catRep}{\mathbf{Rep}} % representations 
\newcommand{\catVec}{\mathbf{Vec}}

% Group and Representation Theory
\newcommand{\chr}{\operatorname{char}}
\newcommand{\Aut}{\operatorname{Aut}}
\newcommand{\GL}{\operatorname{GL}}
\newcommand{\im}{\operatorname{im}}
\newcommand{\tr}{\operatorname{tr}}
\newcommand{\id}{\mathbf{id}}
\newcommand{\cl}{\mathbf{cl}}
\newcommand{\Gal}{\operatorname{Gal}}
\newcommand{\Tr}{\operatorname{Tr}}
\newcommand{\sgn}{\operatorname{sgn}}
\newcommand{\Sym}{\operatorname{Sym}}
\newcommand{\Alt}{\operatorname{Alt}}

% Commutative and Homological Algebra
\newcommand{\spec}{\operatorname{spec}}
\newcommand{\mspec}{\operatorname{m-spec}}
\newcommand{\Tor}{\operatorname{Tor}}
\newcommand{\tor}{\operatorname{tor}}
\newcommand{\Ann}{\operatorname{Ann}}
\newcommand{\Supp}{\operatorname{Supp}}
\newcommand{\Hom}{\operatorname{Hom}}
\newcommand{\End}{\operatorname{End}}
\newcommand{\coker}{\operatorname{coker}}
\newcommand{\limit}{\varprojlim}
\newcommand{\colimit}{%
  \mathop{\mathpalette\colimit@{\rightarrowfill@\textstyle}}\nmlimits@
}
\makeatother

\newcommand{\fraka}{\mathfrak a} % ideal
\newcommand{\frakb}{\mathfrak b} % ideal
\newcommand{\frakc}{\mathfrak c} % ideal
\newcommand{\frakm}{\mathfrak m} % maximal ideal
\newcommand{\frakp}{\mathfrak p} % prime ideal
\newcommand{\frakq}{\mathfrak q} % qrime ideal
\newcommand{\frakN}{\mathfrak N} % nilradical 
\newcommand{\frakR}{\mathfrak R} % jacobson radical

% Algebraic Geometry 
\newcommand{\bbA}{\mathbb A}
\newcommand{\calO}{\mathcal O}
\newcommand{\scrF}{\mathscr F}
\newcommand{\scrG}{\mathscr G}
\newcommand{\Top}{\mathfrak{Top}}

% General/Differential/Algebraic Topology 
\newcommand{\scrA}{\mathscr A}
\newcommand{\scrB}{\mathscr B}
\newcommand{\scrP}{\mathscr P}
\newcommand{\scrS}{\mathscr S}
\newcommand{\bbH}{\mathbb H}
\newcommand{\Int}{\operatorname{Int}}

% Miscellaneous
\newcommand{\psimeq}{\simeq_p}
\newcommand{\wt}[1]{\widetilde{#1}}
\newcommand{\wh}[1]{\widehat{#1}}
\newcommand{\calM}{\mathcal{M}}
\newcommand{\onto}{\twoheadrightarrow}
\newcommand{\into}{\hookrightarrow}
\newcommand{\Gr}{\operatorname{Gr}}
\newcommand{\Span}{\operatorname{Span}}
\newcommand{\op}{\text{op}}




\title{Algebraic Geometry}
\author{Swayam Chube}
\date{\today}

\begin{document}
\begin{abstract}
  These are terse ``notes'' in algebraic geometry which I've made for fun. I don't intend for these to be useful to anyone but myself. These are mostly drawn from a combination of \cite{vakil-foag} through self-teaching along with some discussions with other students. I do not prove all results and use quite a few blackboxes from commutative algebra. Most results can be found with proof in \cite{AM69} or \cite{Lan02} as I shall reference them throughout the text.
\end{abstract}
\maketitle
\tableofcontents
\newpage

\section{Sheaves}
\subsection{The germ of a smooth function}

Let $X$ be a topological space and $p\in X$. We shall consider elements of $C(X)$, the set (ring) of continuous functions from $X$ to $\R$. Define the relation $\sim$ on $C(X)$ by $f\sim g$ if and only if there is a neighborhood $U$ of $p$ on which $f = g$. That this relation is an equivalence relation is not hard to show. The set of equivalence classes is denoted by $\mathscr O_p$. It is not hard to see that this has the structure of a ring with pointwise addition and multiplication, which is also well defined.

We contend that the ring $\mathscr{O}_p$ is local. Indeed, consider the ideal $\frakm_p$ of all germs that vanish at $p$ (that this is an ideal is trivial to check). Further, if $f\in\mathscr O_p\backslash\frakm_p$, then there is a neighborhood $U$ of $p$ on which $f$ is nonzero, whereby $f$ is a unit in $\mathscr O_p$, implying the desired conclusion. The quotient ring is a field, $\R$.

\subsection{Presheaves}

\begin{definition}[Presheaf]
    Let $X$ be a topological space and $\mathfrak{Top}(X)$ denote the poset category of all open sets in $X$ along with inclusion maps. A \emph{presheaf} on $X$ is a contravariant functor $\mathscr{F}$ from $\mathfrak{Top}(X)$ to $\catCRing$. 
\end{definition}

\begin{definition}[Stalk]
    Define the \emph{stalk} of a presheaf $\mathscr F$ at a point $p\in X$ to be the colimit of the diagram induced by $\mathscr F$. The index category in this case, $\mathfrak{Top}^\op(X)$ is a filtered category since given any two open sets, there is an open set contained in both.

    If $p\in U$ and $f\in\mathscr F(U)$, then the image of $f$ in $\mathscr F_p$ is called the \emph{germ of $f$ at $p$}.
\end{definition}

\begin{definition}[Sheaf]
    A presehaf $\mathscr F$ is a \emph{sheaf} if it satisfies the following two axioms. 
    \begin{description}
        \item[Gluability axiom.] If $\{U_i\}_{i\in I}$ is an open cover of $U$, then given $f_i\in\mathscr F(U_i)$ for all $i\in I$ such that $f_i|_{U_i\cap U_j} = f_j|_{U_i\cap U_j}$ for all $i,j\in I$, then there is some $f\in\mathscr F(U)$ such that $\rho_{U,U_i}(f) = f_i$ for all $i\in I$.
        \item[Identity axiom.] If $\{U_i\}_{i\in I}$ is an open cover of $U$ and $f_1,f_2\in\mathscr F(U)$, and $f_1|_{U_i} = f_2|_{U_i}$ for all $i\in I$, then $f_1 = f_2$. 
    \end{description}
    A presheaf satisfying the identity axiom is called a \emph{separated presheaf}.
\end{definition}

\begin{definition}[Morphisms]
    A \emph{morphism of presheaves} $\mathscr F,\mathscr G:\mathfrak{Top}(X)^\op\to\catCRing$ is a natural transformation between the functors $\mathscr F$ and $\mathscr G$, that is, a collection $\{\phi(U)\}$ of maps for each $U\in\mathfrak{Top}(X)$ such that for each $U\into V$, the diagram 
    \begin{equation*}
        \xymatrix {
            \mathscr F(V)\ar[r]^-{\rho_{V,U}}\ar[d]_{\phi(V)} & \mathscr F(U)\ar[d]^{\phi(U)}\\
            \mathscr G(V)\ar[r]_-{\rho_{V,U}} & \mathscr G(U)
        }
    \end{equation*}
    commutes. Similarly, a \emph{morphism of sheaves} is simply a morphism of presheaves between sheaves, since the category of sheaves on $X$ is a full subcategory of the category of presheaves on $X$.
\end{definition}

\begin{proposition}
    Let $\pi: X\to Y$ be a continuous map and $\mathscr F$ a presheaf on $X$. Define $\pi_*\scrF$ by 
    \begin{equation*}
        \pi_*\scrF(V) = \scrF(\pi^{-1}(V))
    \end{equation*}
    where $V\in\Top(Y)$. Then, $\pi_\ast\scrF$ is a presheaf on $X$.
\end{proposition}
\begin{proof}
    Straightforward, since $\pi^{-1}$ is itself a functor from $\Top(Y)$ to $\Top(X)$ and the composition of a contravariant functor and a covariant functor is a contravariant functor.
\end{proof}

\begin{definition}
    Let $\pi: X\to Y$ be a continuous map of topological spaces and $\scrF$ be a presheaf on $X$. Then, $\pi_\ast\scrF$ is called the \emph{pushforward of $\scrF$ by $\pi$} and is a presheaf on $Y$.
\end{definition}

\newpage
\bibliographystyle{alpha}
\bibliography{../../references.bib}
\end{document}