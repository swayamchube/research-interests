\section{Cell Complexes}

\begin{definition}[Cell Complex]
    Cell complexes are constructed using an inductive procedure. 
    \begin{enumerate}[label=(\alph*)]
        \item Begin with a discrete set $X^0$, whose points are regarded as $0$-cells.
        \item Inductively, form the \emph{$n$-skeleton} $X^n$ from $X^{n - 1}$ by attaching $n$-cells $e_\alpha^{n}$ via maps $\varphi_\alpha: S^{n - 1} = \partial e_\alpha^n\to X^{n - 1}$. 
        \item This inductive process can either be stopped at a finite stage or continued indefinitely, setting $X = \bigcup_{n = 1}^\infty X^n$. In the latter case, $X$ is given the \emph{weak topology}.
    \end{enumerate}
\end{definition}

\begin{example}[Real Projective Space $\RP^n$]
    Recall that $\RP^n$ is defined as the quotient space of $\R^{n + 1}\backslash\{0\}$ under the identification $x\sim\lambda x$. This can equivalently be thought of as the hemisphere $D^n$ with the identification $x\sim -x$ for $\partial D^n = S^{n - 1}$. Under this identification, $S^{n - 1}$ quotients to $\RP^{n - 1}$ whereby, $\RP^n$ is obtained by simply attaching an $n$-cell to $\RP^{n - 1}$ through the quotient map $\varphi: S^{n - 1} = \partial D^n\to\RP^{n - 1}$. Thus, the cell complex structure of $\RP^n$ is $e^0\cup e^1\cup\dots\cup e^{n}$, i.e. one cell in each dimension $0\le i\le n$.
\end{example}

\begin{example}[Complex Projective Plane $\CP^n$]
    
\end{example}

\begin{definition}
    A \emph{subcomplex} of a cell complex $X$ is a closed subspace $A\subseteq X$ that is a union of cells of $x$. A pair $(X,A)$ consisting of a cell complex $X$ and a subcomplex $A$ is called a \emph{CW pair}.
\end{definition}

\begin{remark}
    The property of $A$ being a subcomplex depends on the CW structure of $X$. For example, $S^{n - 1}$ is not a subcomplex of $S^n$ with the natural CW structure obtained by gluing two $D^n$'s. But, we may choose a different CW structure for $S^n$ wherein we begin with the equitorial $S^{n - 1}$ and attach two $D^n$'s to it, via the obvious boundary map. In this case, $S^{n - 1}$ is indeed a subcomplex of $S^n$.
\end{remark}

\section{Homotopy Extension Property}

\begin{definition}[Homotopy Extension Property]
    A pair $(X,A)$ with $A\subseteq X$ is said to have the \emph{homotopy extension property} if for any topological space $Y$, a map $f_0: X\to Y$ and a homotopy $H: A\times I\to Y$ such that $H|_{A\times\{0\}} = f_0|_A$, there is an extension of $H$, $\wt H: X\times I\to Y$ with $H|_{X\times\{0\}} = f_0$.
\end{definition}

\begin{proposition}
    A pair $(X,A)$ with $A$ closed\footnote{This is superfluous} in $X$ has the homotopy extension property if and only if $X\times\{0\}\cup A\times I$ is a \emph{retract} of $X\times I$.
\end{proposition}
\begin{proof}
    Suppose $(X,A)$ has the homotopy extension property. Consider the identity map $\id: X\times\{0\}\cup A\times I\to X\times\{0\}\cup A\times I$. This may be extended to a map $f: X\times I\to X\times\{0\}\cup A\times I$ which restricts to the identity map on $X\times\{0\}\cup A\times I$. This shows that the latter is a retract of the former.
\end{proof}

\begin{proposition}
    If $(X,A)$ is a CW-pair, then $X\times\{0\}\cup A\times I$ is a deformation retract of $X\times I$ whereby $(X,A)$ has the homotopy extension property.
\end{proposition}
\begin{proof}
    \todo{$(X,A)$ CW-pair has homotopy ext property}
\end{proof}

\begin{proposition}
    If the pair $(X,A)$ has the homotopy extension property and $A$ is contractible, then the quotient map $q: X\to X/A$ is a homotopy equivalence.
\end{proposition}
\begin{proof}
    Since $A$ is contractible, there is a homotopy between the inclusion $A\into X$ and the constant map on $A$. Due to the homotopy extension property, this can be extended to a homotopy $F: X\times I\to X$ such that $F|_{X\times\{0\}} = \id_X$. Let $q: X\onto X/A$ denote the quotient map and $\wt q: X\times I\to X/A\times I$ denote the quotient map with the obvious identification.

    Consider the composition $q\circ F$. Then, for $a,a'\in A$, $q\circ F(a,t) = q\circ F(a',t)$ for all $t$ whereby this induces a continuous map $\wt F: X/A\times I\to X/A$. Let $f_1 := F|_{X\times\{1\}}$ and $\wt f_1 = \wt F|_{X\times\{1\}}$. Then, $f_1$ maps all of $A$ to a single point whence it induces a map $g: X/A\to X$ such that $f_1 = g\circ q$.

    We contend that $\wt f_1 = q\circ g$. Indeed, for any $\overline x\in X/A$, there is $x\in X$ such that 
    \begin{equation*}
        q\circ g(\overline x) = q\circ g\circ q(x) = q\circ f_1(x) = \wt f_1\circ q(x) = \wt f_1(\overline x).
    \end{equation*}

    This shows that $g\circ q = f_1\simeq\id_X$ through $F$ while $q\circ g = \wt f_1\simeq\id_{X/A}$ through $\wt F$ whence the conclusion follows.
\end{proof}

\begin{corollary}
    If $(X,A)$ is a CW-pair with $A$ contractible, then the quotient map $q: X\to A/A$ is a homotopy equivalence.
\end{corollary}

\begin{example}
    
\end{example}