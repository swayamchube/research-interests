\section{The Setup}

\begin{definition}[Standard and Singular $n$-simplices]
    The standard $n$-simplex, denoted $\Delta^n\subseteq\R^{n + 1}$ is given by 
    \begin{equation*}
        \Delta^n := \{(x_0,\dots,x_n)\in\R^{n + 1}\mid x_0 + \dots + x_n = 1\}.
    \end{equation*}
    Denote the $n + 1$ vertices of $\Delta^n$ by $v^n_0,\dots,v^n_n$ where $v^n_i = (0,\dots, 1,\dots, 0)$ in which $1$ occurs at the $i$-th position. Further, define the $i$-th face map $\frakf_i^n: \Delta^n\to\Delta^{n + 1}$ for $0\le i\le n + 1$, first on the vertices of $\Delta^n$ by 
    \begin{equation*}
        \frakf_i^n(v_j^n) =
        \begin{cases}
            v_j^{n + 1} & j < i\\
            v_{j + 1}^{n + 1} & j\ge i
        \end{cases}
    \end{equation*}
    and then extend linearly to all of $\Delta^n$.

    Given a topological space $X$, a \emph{singular $n$-simplex} in $X$ is a continuous map $\sigma: \Delta^n\to X$. Denote by $S_n(X)$, the set of all singular $n$-simplices in $X$ and let $C_n(X)$ denote the \emph{free abelian group} on $S_n(X)$.
\end{definition}

\begin{definition}[The Singular Complex]
    Let $X$ be a topological space. Define the map $\partial_n: C_n(X)\to C_{n - 1}(X)$ by first defining it on $S_n(X)$, 
    \begin{equation*}
        \partial_n(\sigma) = \sum_{i = 0}^n(-1)^i\sigma\circ\frakf_i^{n - 1}
    \end{equation*}
    and then extending to all of $C_n(X)$ using the universal property of free modules.
\end{definition}

\begin{remark}
    One can check that if $i\le j\le n$, then $\frakf_i^n\circ\frakf_j^{n - 1} = \frakf_{j + 1}^n\circ\frakf_i^{n - 1}$.
\end{remark}

\begin{proposition}
    $\partial_n\circ\partial_{n + 1} = 0$ for $n\ge 1$.
\end{proposition}
\begin{proof}
    It suffices to check this on $S_{n + 1}(X)$, the generator of $C_{n + 1}(X)$. Indeed, 
    \begin{align*}
        \partial_n\circ\partial_{n + 1}(\sigma) &= \sum_{i = 0}^{n + 1}(-1)^i\partial_n(\sigma\circ\frakf_i^{n})\\
        &= \sum_{i = 0}^{n + 1}(-1)^i\sum_{j = 0}^n(-1)^j\sigma\circ\frakf_i^n\circ\frakf_j^{n - 1}\\
        &= \sum_{i = 0}^{n + 1}\sum_{j = 0}^{i - 1}(-1)^{i + j}\sigma\circ\frakf_i^n\circ\frakf_j^{n - 1} + \sum_{i = 0}^{n + 1}\sum_{j = i}^{n}(-1)^{i + j}\sigma\circ\frakf_i^n\circ\frakf_j^{n - 1}\\
        &= \sum_{i = 0}^{n + 1}\sum_{j = 0}^{i - 1}(-1)^{i + j}\sigma\circ f_i^n\circ f_j^{n - 1} + \sum_{i = 0}^{n}\sum_{j = i}^{n}\sigma\circ f^n_{j + 1}\circ f_i^{n - 1} = 0\qedhere.
    \end{align*}
\end{proof}

\begin{definition}[Singular Homology Groups]
    For a topological space $X$, the homology groups corresponding to the singular chain complex $C_\bullet(X)$ are called the \emph{singular homology groups}.
\end{definition}

\begin{mdframed}
    Let $X$ be a topological space. The standard $0$-simplex is just the point $x = 1$ in $\R^1$. Thus, $S_0(X)$ can be identified with the underlying set of $X$, consequently, $C_0(X)$ can be identified with the free abelian group on $X$. Define the map $\varepsilon: S_0(X)\to\Z$ by $\varepsilon(x) = 1$ for each $x\in X$ and extend this to $C_0(X)$ through the universal property. It is evident that the map $\varepsilon: C_0(X)\to\Z$ is a surjection. Furthermore, $\varepsilon\circ\partial_1 = 0$, and thus, we may augment the singular chain complex as follows: 
    \begin{equation*}
        \xymatrix {
            \cdots \ar[r]^-{\partial_2} & C_1(X)\ar[r]^-{\partial_1} & C_0(X)\ar[r]^-{\varepsilon} & \Z\ar[r] & 0
        }
    \end{equation*}
    which we denote by $\wt C_\bullet(X)$ and the corresponding homology groups by $\wt H_n(X)$ which are called the \emph{reduced homology groups}. Note that $H_n(X) = \wt H_n(X)$ for $n > 0$ therefore, the only difference observed is in $\wt H_0(X)$.
\end{mdframed}

\section{Some Functorial Properties}

Let $f: X\to Y$ and $g: Y\to Z$ be continuous maps. There is an induced map $f_n: S_n(X)\to S_n(Y)$ given by $\sigma\mapsto f\circ\sigma$. This can be extended to a map $f_n: C_n(X)\to C_n(Y)$ through the universal property of free modules as follows.
\begin{equation*}
    \xymatrix {
        S_n(X)\ar@{^{(}->}[d]\ar[r]^{f_n} & S_n(Y)\ar@{^{(}->}[d]\\
        C_n(X)\ar@{.>}[r]_{\exists! f_n} & C_n(Y)
    }
\end{equation*}

We denote the sequence of maps $\{f_n\}_{n = 0}^\infty$ by $f_\sharp$.

\begin{proposition}
    Given the setup as above, 
    \begin{enumerate}[label=(\alph*)]
        \item $f_\sharp: C_\bullet(X)\to C_\bullet(Y)$ is a chain map.
        \item $g_\sharp\circ f_\sharp = (g\circ f)_\sharp$.
        \item If $\id: X\to X$ is the identity map, then $\id_\sharp$ is a collection of identity maps on $C_n(X)$ for each nonnegative integer $n$.
    \end{enumerate}
\end{proposition}
\begin{proof}
\begin{enumerate}[label=(\alph*)]
    \item We need to show that $\partial^Y_{n + 1}\circ f_{n + 1} = f_n\circ\partial^X_n: C_{n + 1}(X)\to C_n(Y)$. It suffices to check the equality on elements of $S_{n + 1}(X)$, owing to the universal property. Indeed, for $\sigma\in S_{n + 1}(X)$, we have 

    \begin{align*}
        \partial_{n + 1}^Y(f\circ\sigma) = \sum_{i = 0}^{n + 1}(-1)^i f\circ\sigma\circ\frakf^n_i\\
        f_n\circ\partial_{n + 1}^{X}(\sigma) = f_n\left(\sum_{i = 0}^{n + 1}(-1)^i\sigma\circ\frakf_i^{n}\right) = \sum_{i = 0}^{n + 1}(-1)^i f\circ\sigma\circ\frakf_i^n.
    \end{align*}
    
    \item Since $g_n\circ f_n = (g\circ f)_n$ on the elements of $S_n(X)$, the equality must hold on all of $C_n(X)$. We are implicitly using the universal property here.

    \item Trivial.\qedhere
\end{enumerate}
\end{proof}

Since $f_\sharp$ is a chain map, it induces a group homomorphism $H_n(X)\to H_n(Y)$ on the homology groups, which we denote by $f_\ast$ or $(f_\ast)_n$. We shall try to avoid the latter for the sake of brevity.

\todo{$f_\ast$ is functorial}.

\begin{mdframed}
    We shall establish some notation to make our life easier. If $p_0,\dots,p_k\in\R^k$ are points, then we denote by $[p_0,\dots,p_k]$ the unique linear map $\tau:\Delta^k\to\R^k$ that maps $v^k_i\mapsto p_i$. In particular, this map is given by 
    \begin{equation*}
        \alpha_0 v^k_0 + \dots + \alpha_kv^k_k\mapsto \alpha_0p_0 + \dots + \alpha_{k}p_k.
    \end{equation*}

    Now, let $A\subseteq\R^n$ be a convex subset. Given a map $\sigma: A\to X$ and $p_0,\dots,p_k\in A$, we denote by $\sigma|_{[p_0,\dots,p_k]}$ the composition $\sigma\circ[p_0,\dots,p_k]$.
\end{mdframed}

\begin{theorem}
    Let $f,g: X\to Y$ be homotopic maps. Then, $f_\ast = g_\ast$.
\end{theorem}
\begin{proof}
    We have a map $F: X\times I\to Y$ such that $F|_{X\times\{0\}} = f$ and $F|_{X\times\{1\}} = g$. We shall construct a chain homotopy $P: C_\bullet(X)\to C_\bullet(Y)$ between the maps $f$ and $g$. \todo{chain homotopy. only computation remains}
\end{proof}

\begin{corollary}
    Let $f: X\to Y$ be a homotopy equivalence. Then, $f_\ast$ is an isomorphism of groups.
\end{corollary}
\begin{proof}
    There is a continuous map $g: Y\to X$ such that $g\circ f\simeq\id_X$ and $f\circ g\simeq\id_Y$. Thus, $g_\ast\circ f_\ast = \id_\ast$ and $f_\ast\circ g_\ast = \id_\ast$. The conclusion follows.
\end{proof}

\begin{definition}[Relative Homology Groups]
    Let $X$ be a topological space and $A\subseteq X$ a subspace. There is a canonical inclusion $\iota_n: C_n(A)\into C_n(X)$. Denote by $C_n(X,A)$ the abelian group $\coker\iota_n$. There is an induced map $\partial_n:\coker\iota_n\to\coker\iota_{n - 1}$ giving us a chain complex $C_\bullet(X,A)$. The homology groups corresponding to this chain complex are called \emph{relative homology groups} and denoted by $H_n(X,A)$.
\end{definition}

We now have a short exact sequence of chain complexes
\begin{equation*}
    \xymatrix {
        0\ar[r] & C_\bullet(A)\ar[r]^\iota & C_\bullet(X)\ar[r] & C_\bullet(X,A)\ar[r] & 0
    }
\end{equation*}
which, due to \thref{thm:ses-to-les} gives us a long exact sequence of homology groups
\begin{equation*}
\xymatrix@R-2pc{
    \cdots\ar[r] & H_n(A)\ar[r] & H_n(X)\ar[r] & H_n(X,A)\ar[r] & H_{n - 1}(A)\ar[r] & \cdots\\
    & & &  \cdots\ar[r] & H_0(X,A)\ar[r] & 0
}
\end{equation*}

\section{Barycentric Subdivision and Excision}

\begin{definition}
    Let $\mathscr U$ be a collection of subspaces of $X$ whose interiors form an open cover of $X$ and let $C_n^{\mathscr U}(X)$ be the subgroup of $X$ generated by singular $n$-simplices $\sigma$ whose image is contained in some set in $\mathscr{U}$. Denote the homology groups of this chain complex by $H_n^\mathscr{U}(X)$.
\end{definition}

\begin{theorem}
    The inclusion $\iota: C_n^{\mathscr U}(X)\into C_n(X)$ is a chain homotopy equivalence, that is, $\iota_\ast: H_n^{\mathscr U}(X)\to H_n(X)$ is an isomorphism for all $n$.
\end{theorem}
\begin{proof}
The proof of this theorem involves the procecure of \emph{Barycentric Subdivision}. There are four steps involved in carrying this out.

\noindent\textbf{Barycentric Subdivision of Simplices}

We perform this subdivision inductively. First, begin with a $0$-simplex, which is just a point. There is nothing to do in this case. 

Suppose now that $[v_0,\dots,v_n]$ is a simplex and that we have subdivided each face $[v_0,\dots,\wh v_i,\dots, v_n]$ into smaller simplices. Let 
\begin{equation*}
    b = \frac{1}{n + 1}\sum_{i = 0}^{n + 1}v_i
\end{equation*}
denote the barycenter of $[v_0,\dots,v_n]$. Then, construct the simplices 
\begin{equation*}
    [b,w_0,\dots,w_{n - 1}]
\end{equation*}
where $[w_0,\dots,w_{n - 1}]$ is an $(n - 1)$-simplex in the barycentric subdivision of some face $[v_0,\dots,\wh v_i,\dots,v_n]$ of $[v_0,\dots,v_n]$.

We now claim that the diameter of any simplex in the subdivision of $[v_0,\dots,v_n]$ is atmost $\frac{n}{n + 1}$ times the diameter of $[v_0,\dots,v_n]$. First, we establish an inequality. Let $v\in [v_0,\dots,v_n]$ and $\sum_i t_iv_i$ be another point. Then, 
\begin{equation*}
    \left|v - \sum_{i}t_iv_i\right| = \sum_{i}\left|t_i(v - v_i)\right|\le\sum_i t_i|v - v_i|\le\max_{i}|v - v_i|.
\end{equation*}

Let $w_i,w_j$ be two vertices of a simplex in the barycentric subdivision of $[v_0,\dots,v_n]$. If none of them is the barycenter $b$ of $[v_0,\dots,v_n]$, then, 
\begin{equation*}
    |w_i - w_j|\le\frac{n - 1}{n}\diam[w_0,\dots,w_{n - 1}]\le\frac{n}{n + 1}\diam[b,w_0,\dots,w_{n - 1}]\le\frac{n}{n + 1}\diam[v_0,\dots,v_n].
\end{equation*}
On the other hand, suppose $w_j = b$ and let $b_i$ denote the barycenter of $[v_0,\dots,\wh v_k,\dots,v_n]$, consequently, 
\begin{equation*}
    b = \frac{n}{n + 1}b_k + \frac{1}{n + 1}v_k.
\end{equation*}
In particular, this means that 
\begin{equation*}
    |b - v_k|\le\frac{n}{n + 1}\diam[v_0,\dots,v_n].
\end{equation*}
As a result, $|b - w_i|\le\frac{n}{n + 1}\diam[v_0,\dots,v_n]$.

\noindent\textbf{Barycentric Subdivision of Linear Chains}

Let $Y\subseteq\R^N$ be a convex set. Let $LC_n(Y)$ denote the free abelian group on the set of all linear maps $\Delta^n\to Y$. Note that the restriction of the boundary map $\partial: C_n(Y)\to C_{n - 1}(Y)$ takes $LC_n(Y)$ to $LC_{n - 1}(Y)$. Therefore, we have chain complex of linear chains, which we denote by $LC_{\bullet}(Y)$. Augment this chain complex by setting $LC_{-1}(Y)$ as $\Z$, generated by the empty simplex $[\emptyset]$. The boundary map is given by $\partial([v_0]) = [\emptyset]$.

For each $b\in Y$, denote again by $b: LC_n(Y)\to LC_{n + 1}(Y)$, the group homomorphism $b([w_0,\dots,w_n]) = [b,w_0,\dots,w_n]$ and extend linearly. Then, 
\begin{equation*}
    \partial(b([w_0,\dots,w_n])) = [w_0,\dots,w_n] - \sum_{i = 0}^n(-1)^i[b,w_0,\dots,\wh w_i,\dots,w_n] = [w_0,\dots,w_n] - b(\partial([w_0,\dots,w_n])).
\end{equation*}
Extending linearly, we have that for all $\alpha\in LC_n(Y)$, 
\begin{equation*}
    \partial(b(\alpha)) + b(\partial(\alpha)) = \alpha\iff \partial\circ b + b\circ\partial = \id.
\end{equation*}
In particular, this means that $b$ is a chain homotopy between the identity mamp and the zero map on the augmented chain complex $LC_\bullet(Y)$.

Next, we define a subdivision homomorphism $S: LC_n(Y)\to LC_n(Y)$. For any linear map $\lambda:\Delta^n\to Y$, denote by $b_\lambda$, the image of the barycenter of $\Delta^n$ under $\lambda$. Define the operator $S$ inductively as 
\begin{equation*}
    S(\lambda) = b_\lambda(S(\partial\lambda)).
\end{equation*}
The base case of the induction is given by $S([\emptyset]) = [\emptyset]$. Consider the map $\lambda:\Delta^0\to Y$ given by $[w_0]$. Then, 
\begin{equation*}
    S([w_0]) = w_0(S([\emptyset])) = [w_0]
\end{equation*}
and thus $S$ is the identity map on both $LC_0(Y)$ and $LC_{-1}(Y)$. 

We contend that $S$ is a chain map, that is, $\partial S = S\partial$. This is easily verified in the case of $LC_0(Y)$. We prove it for all $LC_n(Y)$ by induction on $n$. Indeed, let $\lambda\in LC_n(Y)$ for $n\ge 1$, then 
\begin{align*}
    \partial(S(\lambda)) &= \partial(b_\lambda(S(\partial\lambda)))\\
    &= S(\partial\lambda) - b_\lambda(\partial(S(\partial\lambda)))\\
    &= S(\partial\lambda) - b_\lambda(\partial(\partial(S\lambda)))\\
    &= S(\partial\lambda),
\end{align*}
which completes the induction.

Now that we have a chain map $S$, we shall show that it is chain homotopic to the identity map by defining a chain homotopy $T: LC_n(Y)\to LC_{n + 1}(Y)$. Begin by defining $T: LC_{-1}(Y)\to LC_0(Y)$ as the zero map. Now, for a linear map $\lambda:\Delta^n\to Y$, define
\begin{equation*}
    T(\lambda) := b_\lambda(\lambda - T(\partial\lambda)).
\end{equation*}
We need to verify that $\partial T + T\partial = \id -S$, in order to show that $T$ is indeed a chain homotopy. Again, we show this by induction on $n$. The base case on $LC_{-1}(Y)$ is trivial. Now, suppose $\lambda\in LC_n(Y)$ for $n\ge 0$. Then, 
\begin{align*}
    \partial(T(\lambda)) &= \partial(b_\lambda(\lambda - T(\partial\lambda)))\\
    &= \lambda - T(\partial\lambda) - b_\lambda(\partial\lambda - \partial(T(\partial\lambda)))\\
    &= \lambda - T(\partial\lambda) - b_\lambda\left(S(\partial\lambda) + T(\partial(\partial\lambda))\right)\\
    &= \lambda - T(\partial\lambda) - b_\lambda(S(\partial\lambda))\\
    &= \lambda - T(\partial\lambda) - S(\lambda),
\end{align*}
which completes the induction.

\noindent\textbf{Barycentric Subdivision of General Chains}

Define $S: C_(X)\to C_n(X)$ by 
\begin{equation*}
    S(\sigma) = \sigma_\sharp(S(\Delta^n)),
\end{equation*}
where $\sigma_\sharp: C_n(\Delta^n)\to C_n(X)$. Here, it is important to note that $S(\Delta^n)$ denotes $S(\id_{\Delta^n})$ where $S: LC_n(\Delta^n)\to LC_n(\Delta^n)$ is the barycentric subdivision operator on linear chains as we have seen in the previous part. This is obviously valid usage since $\id$ is a linear map from $\Delta^n$ to $\Delta^n$.

First, we must show that $S$ is a chain map. Indeed, for any $\sigma:\Delta^n\to X$, 
\begin{align*}
    \partial(S(\sigma)) &= \partial(\sigma_\sharp(S(\Delta^n))) = \sigma_\sharp(\partial(S(\Delta^n))) = \sigma_\sharp(S(\partial(\Delta^n)))\\
    &= \sigma_\sharp\left(S\left(\sum_{i = 0}^n(-1)^i\Delta^n_i\right)\right)\qquad\text{where $\Delta^n_i$ denotes the $i$-th face of $\Delta^n$}\\
    &= \sum_{i = 0}^n (-1)^i\sigma_\sharp(S(\Delta^n_i))\\
    &= \sum_{i = 0}^n (-1)^i S(\sigma|_{\Delta^n_i})\\
    &= S\left(\sum_{i = 0}^n (-1)^i\sigma_{\Delta^n_i}\right) = S(\partial\sigma).
\end{align*}

Now that we have shown that $S$ is a chain map, we must construct a chain homotopy between $\id$ and $S$, which is done by extending the definition of the operator $T$ as we had seen in the previous part to $T: C_n(X)\to C_{n + 1}(X)$, by 
\begin{equation*}
    T(\sigma) = \sigma_\sharp(T(\Delta^n)),
\end{equation*}
where $\Delta^n$ again denotes the identity map $\id: \Delta^n\to\Delta^n$. That this is a chain homotopy is seen by computing: 
\begin{align*}
    \partial(T(\sigma)) &= \partial(\sigma_\sharp(T(\Delta^n))) = \sigma_\sharp(\partial(T(\Delta^n))) \\
    &= \sigma_\sharp(\Delta^n - S(\Delta^n) - T(\partial(\Delta^n)))\\
    &= \sigma - S(\sigma) - \sigma_\sharp(T(\partial\Delta^n))\\
    &= \sigma - S(\sigma) - T(\partial\sigma).
\end{align*}
This shows that $T$ is a chain map.

\noindent\textbf{Iterated Barycentric Subdivision}

Let $S^m: C_n(X)\to C_n(X)$ denote the iteration $\underbrace{S\circ\dots\circ S}_{n\text{ times}}$ and let $D_m: C_n(X)\to C_{n + 1}(X)$ be given by 
\begin{equation*}
    D_m = \sum_{i = 0}^{m - 1} T\circ S^i.
\end{equation*}

After a short computation, one immediately sees that $D_m$ is a chain homotopy between $S^m$ and $\id$. Now, using the Lebesgue Number Lemma, it is not hard to argue that for each $\sigma:\Delta^n\to X$, there is an $m$ such that $S^m(\sigma)$ lies in $C_n^{\mathscr U}(X)$. Denote by $m(\sigma)$ the smallest positive integer $m$ such that this $S^m(\sigma)$ is in $C_n^{\mathscr U}(X)$.

Define the map $D:C_n(X)\to C_{n + 1}(X)$ by setting $D(\sigma) = D_{m(\sigma)}(\sigma)$ and extend linearly. Consider the map $\rho: C_n(X)\to C_n(X)$ given by 
\begin{equation*}
    \rho = \id - (\partial D + D\partial).
\end{equation*}

We contend that $\rho$ is a chain map. Indeed, 
\begin{equation*}
    \partial\rho(\sigma) = \partial(\sigma) - \partial(D(\partial\sigma))\quad\text{ and }\quad\rho(\partial\sigma) = \partial(\sigma) - \partial(D(\partial\sigma)).
\end{equation*}

Next, we claim that the image of $\sigma$ under $\rho$ lies in $C_n^{\mathscr U}(X)$. Indeed, 
\begin{align*}
    \rho(\sigma) &= \sigma - \partial(D(\sigma)) - D(\partial(\sigma))\\
    &= \sigma - \partial(D_{m(\sigma)}(\sigma)) - D(\partial(\sigma))\\
    &= S^{m(\sigma)}(\sigma) + D_{m(\sigma)}(\partial\sigma) - D(\partial\sigma).
\end{align*}
By definition, $S^{m(\sigma)}(\sigma)\in C_n^{\mathscr U}(X)$. Note that 
\begin{equation*}
    D_{m(\sigma)}(\partial\sigma) = \sum_{i = 0}^n (-1)^i D_{m(\sigma)}(\sigma_i)
\end{equation*}
and 
\begin{equation*}
    D(\partial\sigma) = \sum_{i = 0}^n (-1)^i D_{m(\sigma_i)}(\sigma_i),
\end{equation*}
where $\sigma_i$ is the restriction of $\sigma$ to the face $[v_0,\dots,\wh v_i,\dots,v_n]$. In particular, this means that $m(\sigma_i)\le m(\sigma)$ for each $0\le i\le n$.

We have 
\begin{equation*}
    D_{m(\sigma)}(\sigma_i) - D_{m(\sigma_i)}(\sigma_i) = \sum_{j > m(\sigma_i)} T\circ S^{j}(\sigma)\in C_n^{\mathscr U}(X),
\end{equation*}
since $T$ takes $C_{n - 1}^{\mathscr U}(X)\to C_{n}^{\mathscr U}(X)$. In particular, this means that $\rho(\sigma)\in C_n^{\mathscr U}(X)$.

We may now view $\rho$ as a chain map $C_\bullet(X)\to C_{\bullet}^{\mathscr U}(X)$. Recall that $\iota: C_n^{\mathscr U}(X)\to C_n(X)$ denotes the inclusion. As we have seen earlier, $\partial D + D\partial = \id - \iota\rho$ and obviously $\rho\iota = 1$. Thus, $\rho$ and $\iota$ are quasi-isomorphisms, which completes the proof.
\end{proof}

\begin{theorem}[Excision Theorem]
    Let $Z\subseteq A\subseteq X$ be such that $\overline Z\subseteq\Int A$. Then, the inclusion $(X\backslash Z, A\backslash Z)\into(X, A)$ induces an isomorphism $H_n(X\backslash Z, A\backslash Z)\stackrel{\sim}{\longrightarrow}H_n(X, A)$.

    Equivalently phrased, if $A, B$ are subspaces of $X$ whose interiors cover $X$, then the inclusion $(B, A\cap B)\into (X, A)$ induces an isomorphism $H_n(B, A\cap B)\stackrel{\sim}{\longrightarrow} H_n(X, A)$ for all $n$.
\end{theorem}
Note that the equivalence can be seen by setting $B = X\backslash Z$. Then, the condition $\overline Z\subseteq\Int A$ is equivalent to saying $X = \Int A\cup\Int B$.
\begin{proof}
    Let $\mathscr U = \{A, B\}$, which is a cover whose interior is also a cover. We use $C_n(A + B)$ to denote $C_n^{\mathscr U}(X)$, which is a more suggestive notation. Recall from the previous proof, the operators $D, \rho$ and $\iota$. We had $\partial D + D\partial = \id - \iota\rho$. Note that all of $\partial, D, \rho$ take $C_n(A)$ to $C_n(A)$ and thus, they induce quotient maps 
    \begin{align*}
        \wt D: C_n(X)/C_n(A)\to C_{n + 1}(X)/C_{n + 1}(A)\\
        \wt\rho: C_n(X)/C_n(A)\to C_n(A + B)/C_n(A)\\
        \wt \iota: C_n(A + B)/C_n(A)\to C_n(X)/C_n(A).
    \end{align*}
    Note that the equality $\partial\wt D + \wt D\partial = \id - \wt\iota\wt\rho$. In particular, $\wt\iota$ is a quasi-isomorphism. In particular, $H_n(A + B, A)\cong H_n(X, A)$.

    Consider now the following diagram.
    \begin{equation*}
        \xymatrix {
            C_n(B)\ar@{^{(}->}[r]\ar@{->>}[d] & C_n(A + B)\ar@{->>}[r] & C_n(A + B)/C_n(A)\\
            C_n(B)/C_n(A\cap B)\ar@{.>}[rru]
        }
    \end{equation*}
    From the first isomorphism theorem, it is evident that the induced map is an isomorphism of abelian groups and thus induces an isomorphism on the homology groups. This, put together with the previous conclusion gives us 
    \begin{equation*}
        H_n(B, A\cap B)\cong H_n(A + B, A)\cong H_n(X, A).\qedhere
    \end{equation*}
\end{proof}

\begin{definition}
    A pair $(X, A)$ is said to be \emph{locally retractive}, if there is a neighborhood $V$ of $A$ such that $A$ is a deformation retract of $V$.
\end{definition}

\begin{lemma}
    For a locally retractive pair $(X, A)$, the quotient map $q: (X, A)\to(X/A, A/A)$ induces isomorphisms $q_\ast: H_n(X, A)\to H_n(X/A, A/A)$. Recall further that $H_n(X/A, A/A)\cong\wt H_n(X/A)$.
\end{lemma}
\begin{proof}
    
\end{proof}