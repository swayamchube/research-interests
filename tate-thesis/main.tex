\documentclass[12pt]{article}

\usepackage{./arxiv}

\title{Tate's Thesis\\VSRP Notes}
\author{Swayam Chube}
\date{\today}

\usepackage[utf8]{inputenc} % allow utf-8 input
\usepackage[T1]{fontenc}    % use 8-bit T1 fonts
\usepackage{hyperref}       % hyperlinks
\usepackage{url}            % simple URL typesetting
\usepackage{booktabs}       % professional-quality tables
\usepackage{amsfonts}       % blackboard math symbols
\usepackage{nicefrac}       % compact symbols for 1/2, etc.
\usepackage{microtype}      % microtypography
\usepackage{graphicx}
\usepackage{natbib}
\usepackage{doi}
\usepackage{amssymb}
\usepackage{amsthm}
\usepackage{amsmath}
\usepackage{xcolor}
\usepackage{theoremref}
\usepackage{enumitem}
\usepackage{mathpazo}
\usepackage{euler}
\usepackage{mathrsfs}
\usepackage{todonotes}
\usepackage{stmaryrd}
\usepackage[all,cmtip]{xy} % For diagrams, praise the Freyd–Mitchell theorem 
\usepackage{marvosym}

\renewcommand{\qedsymbol}{$\blacksquare$}

% Uncomment to override  the `A preprint' in the header
\renewcommand{\headeright}{}
\renewcommand{\undertitle}{}
\renewcommand{\shorttitle}{}

\hypersetup{
    pdfauthor={Lots of People},
    colorlinks=true,
}

\newtheoremstyle{thmstyle}%               % Name
  {}%                                     % Space above
  {}%                                     % Space below
  {}%                             % Body font
  {}%                                     % Indent amount
  {\bfseries}%                            % Theorem head font
  {.}%                                    % Punctuation after theorem head
  { }%                                    % Space after theorem head, ' ', or \newline
  {\thmname{#1}\thmnumber{ #2}\thmnote{ (#3)}}%                                     % Theorem head spec (can be left empty, meaning `normal')

\newtheoremstyle{defstyle}%               % Name
  {}%                                     % Space above
  {}%                                     % Space below
  {}%                                     % Body font
  {}%                                     % Indent amount
  {\bfseries}%                            % Theorem head font
  {.}%                                    % Punctuation after theorem head
  { }%                                    % Space after theorem head, ' ', or \newline
  {\thmname{#1}\thmnumber{ #2}\thmnote{ (#3)}}%                                     % Theorem head spec (can be left empty, meaning `normal')

\theoremstyle{thmstyle}
\newtheorem{theorem}{Theorem}[section]
\newtheorem{lemma}[theorem]{Lemma}
\newtheorem{proposition}[theorem]{Proposition}

\theoremstyle{defstyle}
\newtheorem{definition}[theorem]{Definition}
\newtheorem*{corollary}{Corollary}
\newtheorem{remark}[theorem]{Remark}
\newtheorem{example}[theorem]{Example}

% Common Algebraic Structures
\newcommand{\R}{\mathbb{R}}
\newcommand{\Q}{\mathbb{Q}}
\newcommand{\Z}{\mathbb{Z}}
\newcommand{\N}{\mathbb{N}}
\newcommand{\bbC}{\mathbb{C}}
\newcommand{\calA}{\mathcal{A}}
\newcommand{\frakM}{\mathfrak{M}}

% Categories
\newcommand{\catTopp}{\mathbf{Top}_*}
\newcommand{\catGrp}{\mathbf{Grp}}
\newcommand{\catTopGrp}{\mathbf{TopGrp}}
\newcommand{\catSet}{\mathbf{Set}}
\newcommand{\catTop}{\mathbf{Top}}
\newcommand{\catRing}{\mathbf{Ring}}
\newcommand{\catCRing}{\mathbf{CRing}} % comm. rings
\newcommand{\catMod}{\mathbf{Mod}}
\newcommand{\catMon}{\mathbf{Mon}}
\newcommand{\catMan}{\mathbf{Man}} % manifolds
\newcommand{\catDiff}{\mathbf{Diff}} % smooth manifolds
\newcommand{\catAlg}{\mathbf{Alg}}
\newcommand{\catRep}{\mathbf{Rep}} % representations 
\newcommand{\catVec}{\mathbf{Vec}}

% Group and Representation Theory
\newcommand{\chr}{\operatorname{char}}
\newcommand{\Aut}{\operatorname{Aut}}
\newcommand{\GL}{\operatorname{GL}}
\newcommand{\im}{\operatorname{im}}
\newcommand{\tr}{\operatorname{tr}}
\newcommand{\id}{\mathbf{id}}
\newcommand{\cl}{\mathbf{cl}}
\newcommand{\Gal}{\operatorname{Gal}}
\newcommand{\Tr}{\operatorname{Tr}}
\newcommand{\sgn}{\operatorname{sgn}}
\newcommand{\Sym}{\operatorname{Sym}}
\newcommand{\Alt}{\operatorname{Alt}}

% Commutative and Homological Algebra
\newcommand{\spec}{\operatorname{spec}}
\newcommand{\mspec}{\operatorname{m-spec}}
\newcommand{\Tor}{\operatorname{Tor}}
\newcommand{\tor}{\operatorname{tor}}
\newcommand{\Ann}{\operatorname{Ann}}
\newcommand{\Supp}{\operatorname{Supp}}
\newcommand{\Hom}{\operatorname{Hom}}
\newcommand{\End}{\operatorname{End}}
\newcommand{\coker}{\operatorname{coker}}
\newcommand{\limit}{\varprojlim}
\newcommand{\colimit}{%
  \mathop{\mathpalette\colimit@{\rightarrowfill@\textstyle}}\nmlimits@
}
\makeatother


\newcommand{\fraka}{\mathfrak a} % ideal
\newcommand{\frakb}{\mathfrak b} % ideal
\newcommand{\frakc}{\mathfrak c} % ideal
\newcommand{\frakf}{\mathfrak f} % face map
\newcommand{\frakm}{\mathfrak m} % maximal ideal
\newcommand{\frakp}{\mathfrak p} % prime ideal
\newcommand{\frakq}{\mathfrak q} % qrime ideal
\newcommand{\frakN}{\mathfrak N} % nilradical 
\newcommand{\frakP}{\mathfrak P} % nilradical 
\newcommand{\frakR}{\mathfrak R} % jacobson radical

% General/Differential/Algebraic Topology 
\newcommand{\scrA}{\mathscr A}
\newcommand{\scrB}{\mathscr B}
\newcommand{\scrP}{\mathscr P}
\newcommand{\scrS}{\mathscr S}
\newcommand{\bbH}{\mathbb H}
\newcommand{\Int}{\operatorname{Int}}
\newcommand{\psimeq}{\simeq_p}
\newcommand{\wt}[1]{\widetilde{#1}}
\newcommand{\RP}{\mathbb{R}\text{P}}
\newcommand{\CP}{\mathbb{C}\text{P}}

% Miscellaneous
\newcommand{\wh}[1]{\widehat{#1}}
\newcommand{\calM}{\mathcal{M}}
\newcommand{\calP}{\mathcal{P}}
\newcommand{\onto}{\twoheadrightarrow}
\newcommand{\into}{\hookrightarrow}
\newcommand{\Gr}{\operatorname{Gr}}
\newcommand{\Span}{\operatorname{Span}}
\newcommand{\ev}{\operatorname{ev}}

\begin{document}
\maketitle

\section{Haar Measure}

The main reference for this section is \cite{deitmar}.

\begin{definition}
    A \emph{Radon measure} on a topological space $X$ is a Borel measure $\mu:\scrB\to [0,\infty]$ that is 
    \begin{description}
        \item[locally finite:] every $x\in X$ has an open neighborhood $U$ such that $\mu(U) < \infty$.
        \item[outer regular:] every $S\in\scrB$ satisfies: 
        \begin{equation*}
            \mu(S) = \inf\left\{\mu(U)\mid S\subseteq U\text{ open}\right\}.
        \end{equation*}
        \item[inner regular on open sets:] every open $U\subseteq X$ satisfies 
        \begin{equation*}
            \mu(U) = \sup\left\{\mu(K)\mid U\supseteq K\text{ compact}\right\}.
        \end{equation*}
    \end{description}
\end{definition}

We work under the assumption that all topological groups are Hausdorff.

\begin{definition}
    A \emph{left Haar measure} on a topological group $G$ is a nonzero left-invariant Radon measure $\mu$ on $G$. That is, for each Borel set $E$, and $g\in G$,
    \begin{equation*}
        \mu(E) = \mu(gE).
    \end{equation*}

    Analogously one defines a \emph{right Haar measure}. 
\end{definition}

\begin{theorem}[Existence and Uniqueness of Haar measure]
    Let $G$ be a locally compact topological group. 
    \begin{enumerate}[label=(\alph*)]
        \item There is a left invariant Haar measure $\mu$ on $G$.
        \item Any other left Haar measure on $G$ is a scalar multiple of the above.
    \end{enumerate}

    An analogous statement holds for the right Haar measure. One must note that the left and right Haar measures need not coincide. Indeed, if $\mu_L$ is a left Haar measure, then $\mu_R$ given by $\mu_R(E) = \mu_L(E^{-1})$ is a right Haar measure.
\end{theorem}

\begin{proposition}
    Let $G$ be a locally compact topological group and $\mu$ a left Haar measure on $G$. $G$ is compact if and only if $\mu(G) < \infty$. In this case, $\mu_L = \mu_R$.
\end{proposition}
\begin{proof}
    Suppose $G$ were compact. Then, there is a neighborhood $U$ of $e$ having finite measure. Note that $\{gU\}_{g\in G}$ forms an open cover of $G$ that admits a finite subcover, whence it follows that $\mu(G) < \infty$.
    
    Suppose $G$ were not compact. Using local compactness, there is a neighborhood of the origin with compact closure. Call this closure $K$. Then, $0 < \mu(K) < \infty$. Set $s_1 = e$. We inductively construct a sequence $\{s_n\}$ such that the $s_nK$'s are pairwise disjoint. Note that $\bigcup_{i = 1}^n s_i KK^{-1}$ is a finite union of compact sets and hence, is compact. Since $G$ isn't compact, there is an $s_{n + 1}$ in $G$ but not in the aforementioned union. 

    Finally, note that 
    \begin{equation*}
        \mu(G) > \mu\left(\bigcup_{i = 1}^n s_iK\right) = n\mu(K).
    \end{equation*}
    The conclusion follows.

    For the second part, we introduce the \emph{modular function}. For each $x\in G$, define the measure $\mu_x:\scrB\to[0,\infty]$ by $\mu_x(E) = \mu(Ex)$. This is a left Haar measure and hence, $\mu_x = \Delta(x)\mu$. One can show that $\Delta$ is a continuous group homomorphism $G\to\R^+$. Therefore, the image of $\Delta$ must be a compact subgroup of $\R^+$. The only such subgroup is $\{1\}$. This completes the proof.
\end{proof}

\begin{proposition}
    Let $G$ be a locally compact group and $H$ a closed subgroup. Then, $G/H$ is a locally compact Hausdorff space.
\end{proposition}

\begin{theorem}[Quotient Integration Theorem]
    Let $G$ be a locally compact topological group and $H$ a closed subgroup. Then, there is an invariant Radon measure $\nu\ne 0$ on $G/H$ if and only if the modular functions $\Delta_G$ and $\Delta_H$ agree on $H$. 

    In this case, the measure $\nu$ is unique up to scaling. In particular, given Haar measures on $G$ and $H$, there is a unique choice for $\nu$. Finally, for ay $f\in L^1(G)$, 
    \begin{equation*}
        \int_G f(x)~dx = \int_{G/H} \int_H f(xh)~dh~dx.
    \end{equation*}
\end{theorem}

\section{Pontryagin Duality}
\begin{center}
    \boxed{\textbf{\textcolor{red}{Henceforth, $G$ is a locally compact abelian topological group.}}}
\end{center}

\begin{definition}
    A \emph{character} of $G$ is a continuous group homomorphism $\chi: G\to\bbC^\times$. A \emph{unitary character} of $G$ is a continuous group homomorphism $\chi: G\to S^1$.

    We note that the nomenclature is not uniform. Some authors call these quasi-characters and characters, respectively.
\end{definition}

\begin{proposition}
    If $G$ is compact, then every character of $G$ is unitary.
\end{proposition}
\begin{proof}
    \begin{equation*}
        \xymatrix {
            G\ar[r]^{\chi}\ar[rd]_{\varphi} & \bbC^\times\ar[d]^{|\cdot|}\\
            & \R^+
        }
    \end{equation*}
    Then, $\varphi$ is a continuous group homomorphism. The image must be compact and hence, $\{1\}$.
\end{proof}

\begin{definition}[Compact-Open Topology]
    Let $X$ and $Y$ be topological spaces. The \emph{compact-open topology} on $C(X, Y)$, the set of continuous functions from $X$ to $Y$ is defined to be the topology generated by sets of the form 
    \begin{equation*}
        \{f\in C(X, Y)\mid f(K)\subseteq U\},
    \end{equation*}
    where $K\subseteq X$ is compact and $U\subseteq Y$ is open.
\end{definition}

\begin{definition}[Pontryagin Dual]
    The \emph{Pontryagin dual} of $G$ is the group 
    \begin{equation*}
        \wh G := \Hom_{cts}(G, S^1),
    \end{equation*}
    which is the group of unitary characters of $G$ under pointwise multiplication. We equip $\wh G$ with the \emph{compact-open topology}.
\end{definition}

\begin{proposition}
    Let $G$ be an abelian topological group. 
    \begin{enumerate}[label=(\alph*)]
        \item If $G$ is discrete, then $\wh G$ is compact.
        \item If $G$ is compact, then $\wh G$ is discrete. 
        \item If $G$ is locally compact, then $\wh G$ is locally compact.
    \end{enumerate}
\end{proposition}

\begin{example}
    We compute the Pontryagin dual of $\R$. Let $\chi: \R\to S^1$ be a continuous homomorphism. Since $\R$ is simply connected, this factors through the universal cover $\R\to S^1$ given by $x\mapsto\exp(2\pi i x)$. This gives us 
    \begin{equation*}
        \xymatrix {
            & \R\ar[d]\\
            \R\ar[r]_{\chi}\ar@{-->}[ru]^{\wt\chi} & S^1
        }
    \end{equation*}
    where $\wt\chi$ is a continuous additive group homomorphism $\R\to\R$ and hence, is linear. Thus, $\wh R\cong\R$.
\end{example}

The Pontryagin dual is a contravariant functor from the category of locally compact abelian groups to itself. 

\begin{theorem}[Exactness of Pontryagin Dual]
    If $0\to A\to B\to C\to 0$ is a short exact sequence of locally compact abelian groups, then so is $0\to\wh C\to\wh B\to\wh A\to 0$.
\end{theorem}

For each $g\in G$, there is the evaluation map $\ev_g: \wh G\to S^1$ given by $\chi\mapsto \chi(g)$. This is a continuous group homomorphism and hence, an element of $\wh{\wh G}$.

\begin{theorem}[Pontryagin Duality]
    The canonical map $G\to\wh{\wh G}$ given by $g\mapsto\ev_g$ is an isomorphism of locally compact abelian groups.
\end{theorem}

\begin{definition}[Fourier Transform]
    If $f\in L^1(G)$, define the \emph{Fourier Transform} $\wh f:\wh G\to\bbC$ by 
    \begin{equation*}
        \wh f(\chi) = \int_G f(g)\overline{\chi(g)}~dg.
    \end{equation*}
\end{definition}

\begin{proposition}
    If $f\in L^1(G)$, then $\wh f\in C_0(\wh G)$.
\end{proposition}

\begin{theorem}[Fourier Inversion]
    Let $G$ be a locally compact abelian group and $dg$ a Haar measure on $G$. Then, there is a unique Haar measure $d\chi$ on $\wh G$ called the \emph{dual measure} such that if $f\in L^1(G)$ is such that $\wh f\in L^1(G)$, then 
    \begin{equation*}
        f(g) = \int_{\wh G}\wh f(\chi)\ev_g(\chi)~d\chi = \int_{\wh G}\wh f(\chi)\chi(g)~d\chi
    \end{equation*}
    almost everywhere on $G$.
\end{theorem}

\begin{remark}
    The measure $d\chi$ defined above on $\wh G$ is called the \emph{dual measure}.
\end{remark}

\begin{theorem}[Plancherel]
    Let $dg$ and $d\chi$ be dual measures on $G$ and $\wh G$ respectively. For every $f\in L^1(G)\cap L^2(G)$, $\wh f\in L^2(\wh G)$ and $\|f\|_2 = \|\wh f\|_2$.

    Therefore, the Fourier transform extends uniquely to an isometric automorphism of $L^2(G)$.
\end{theorem}

\section{Restricted Direct Product}

\begin{definition}
    Let $\{(G_\alpha, H_\alpha)\}_{\alpha\in A}$ be a collection of topological groups where $H_\alpha$ is a subgroup of $G_\alpha$. Define the restricted direct product 
    \begin{equation*}
        \wt\prod_{\alpha\in A} G_\alpha := \left\{(x_\alpha)\mid x_\alpha\in G_\alpha,~x_\alpha\in H_\alpha \text{ almost everywhere}\right\}.
    \end{equation*}
    This obviously forms a group under pointwise multiplication.
\end{definition}

\newpage
\bibliographystyle{alpha}
\bibliography{references}
\end{document}