Throughout this chapter, $k$ denotes an algebraically closed field, $A$ denotes the polynomial ring $k[x_1,\dots,x_n]$ in $n$-variables and $R$ denotes the graded ring $k[x_0,\dots,x_n]$ with the standard homogeneous polynomial grading.

\section{Affine Varieties}

\begin{definition}
    For a subset $T\subseteq A$, define 
    \begin{equation*}
        Z(T) := \{p\in\A^n\mid f(p) = 0,~\forall p\in T\}.
    \end{equation*}
    This is called the \emph{zero-set} of $T$. Conversely, for a subset $S\subseteq\A^n$, define 
    \begin{equation*}
        \scrI(S) := \{f\in A\mid f(p) = 0,~\forall p\in S\}.
    \end{equation*}
    This is called the \emph{ideal generated} by $S$.
\end{definition}

\begin{definition}[Algebraic Set]
    A subset $Y$ of $\A^n$ is an \emph{algebraic set} if there is a subset $T\subseteq A$ such that $Y = Z(T)$.
\end{definition}

\begin{theorem}
    Let $T_i\subseteq A$, $\fraka\unlhd A$ an ideal and $Y_i\subseteq\A^n$.
    \begin{enumerate}[label=(\alph*)]
        \item $Z(T) = Z((T))$ where $(T)$ is the ideal generated by $T$ in $A$.
        \item $Z(T_1T_2) = Z(T_1)\cup Z(T_2)$.
        \item $Z(\bigcup T_i) = \bigcap Z(T_i)$. Hence, the collection of all algebraic sets in $\A^n$ can be identified with the collection of closed sets in some topology on $\A^n$. This is called the \emph{Zarkiski Topology} on $\A^n$.
        \item If $T_1\subseteq T_2$, then $Z(T_1)\supseteq Z(T_2)$. 
        \item If $Y_1\subseteq Y_2$, then $\scrI(Y_1)\supseteq\scrI(Y_2)$.
        \item $\scrI(Y_1\cup Y_2) = \scrI(Y_1)\cap\scrI(Y_2)$.
        \item $\scrI(Z(\fraka)) = \sqrt{\fraka}$.
        \item $Z(\scrI(Y)) = \overline Y$, the closure of $Y$ in the Zariski Topology.
        \item There is an inclusion reversing bijection between the radical ideals of $A$ and the algebraic sets in $\A^n$.
    \end{enumerate}
\end{theorem}

\begin{definition}[Irreducible]
    A topological space $X$ is said to be \emph{irreducible} if it cannot be written as the union of two proper closed subspaces.
\end{definition}

\begin{proposition}
    Let $X$ be a topological space and $Y\subseteq X$ an irreducible subspace. Then, $\overline Y$ is irreducible.
\end{proposition}
\begin{proof}
    Suppose $\overline Y = Y_1\cup Y_2$ where $Y_1$ and $Y_2$ are proper closed subspaces of $\overline Y$. Then, $Y = (Y_1\cap Y)\cup(Y_2\cap Y)$. Since $Y$ is irreducible, either $Y\subseteq Y_1$ or $Y\subseteq Y_2$, consequently, $\overline Y\subseteq Y_1$ or $\overline Y\subseteq Y_2$, a contradiction.
\end{proof}

\begin{proposition}
    Let $X$ be an irreducible topological space and $W\subseteq X$ a non-empty open subset. Then, $W$ is dense and irreducible.
\end{proposition}
\begin{proof}

\end{proof}

\begin{definition}[Affine Variety, Quasi-Affine Variety]
    An \emph{affine algebraic variety} (or simply \emph{affine variety}) is an irreducible closed subset of $\A^n$. An open subset of an affine variety is called a \emph{quasi-affine variety}.
\end{definition}

\begin{remark}\thlabel{rem:quasi-affine}
    Let $Y$ be a quasi-affine variety. Then, there is an affine variety $X$ that contains $Y$ and $Y$ is open in $X$. Then, $Y$ is dense in $X$ and hence, $\overline Y = X$, where the closure is taken in $\A^n$.
\end{remark}

\begin{proposition}
    An algebraic set in $\A^n$ is irreducible if and only if its corresponding ideal is prime in $A$.
\end{proposition}
\begin{proof}
\end{proof}

\begin{definition}[Coordinate Ring]
    For an algebraic set $Y\subseteq\A^n$, we define the \emph{affine coordinate ring} $A[Y] := A/\scrI(Y)$. 
\end{definition}

\begin{remark}
    Note that $A[Y]$ is always a reduced, finitely generated $k$-algebra and is an integral domain if and only if $Y$ is irreducible.
\end{remark}

\begin{definition}[Noetherian Space]
    A topological space $X$ is said to be \emph{noetherian} if it has the ascending chain condition on open sets.
\end{definition}

\begin{proposition}
    A subspace of a noetherian topological space is noetherian.
\end{proposition}
\begin{proof}
    Let $X$ be noetherian and $Y\subseteq X$. Let $U_1\subseteq U_2\subseteq\dots$ be an ascending chain of open subsets of $Y$, then there are open $V_i$ in $X$ such that $V_i\cap Y = U_i$. Let $W_i = \bigcup_{1\le j\le i} V_j$. Then, $W_1\subseteq W_2\subseteq\dots$ an $W_i\cap Y = U_i$. Since $X$ is noetherian, the chain $\{W_i\}$ stabilizes, consequently, so does the chain $\{U_i\}$.
\end{proof}

\begin{proposition}
    A noetherian topological space is compact.
\end{proposition}
\begin{proof}
    Let $\{U_\alpha\}$ be an open cover. Let $\mathscr A$ be the collection of all finite unions of $U_\alpha$'s. Then, $\mathscr A$ must contain a maximal element, which must be $X$. Thus, $X$ is compact.
\end{proof}

\begin{corollary}
    A Hausdorff noetherian space $X$ is finite with the discrete topology. 
\end{corollary}
\begin{proof}
    Due to the preceeding result, every subspace of $X$ is compact (and therefore closed) and hence, every subspace of $X$ is open. Thus, $X$ has the discrete topology. A discrete compact set must be finite.
\end{proof}

\begin{proposition}
    A noetherian topological space can be expressed as a finite union $X = Y_1\cup\dots\cup Y_n$ of irreducible closed subspaces. If we require that $Y_i\not\subseteq Y_j$ for $i\ne j$, then the $Y_i$'s are uniquely determined and are caled the \textbf{irreducible components} of $X$.
\end{proposition}
\begin{proof}
    Let $\Sigma$ be the poset of closed subspaces of $X$ that cannot be expressed as a finite union of irreducible subspaces. If this poset is non-empty, then it admits a minimal element, say $Z$. Note that $Z$ cannot be irreducible, hence, $Z = Z_1\cup Z_2$ where $Z_1$ and $Z_2$ are proper closed subspaces of $Z$, whence, are a finite union of irreducible subspaces. Consequently, $Z$ is a finite union of irreducible subspaces.

    Suppose we have two minimal representations, $X = Y_1\cup\dots\cup Y_n = Y_1'\cup\dots\cup Y_m'$. Then, $Y_i = (Y_i\cap Y_1')\cup\dots(Y_i\cap Y_m')$. Since $Y_i$ is irreducible, there is an index $j$ such that $Y_i\subseteq Y_j'$. Similarly, there is an index $l$ such that $Y_j'\subseteq Y_l$. Therefore, $Y_i\subseteq Y_l$, consequently, $i = l$. This completes the proof.
\end{proof}

\begin{definition}[Dimension]
    Let $X$ be a topological space. Then, 
    \begin{equation*}
        \dim X := \sup\{n\mid\exists Y_0\subsetneq\dots\subsetneq Y_n\subseteq X,~\text{ each $Y_i$ is irreducible}\}.
    \end{equation*}
\end{definition}

\begin{lemma}
    Let $X$ be a topological space.
    \begin{enumerate}[label=(\alph*)]
    \item If $Y$ is a subspace of $X$, then $\dim Y\le\dim X$.
    \item $\{U_i\}_{i\in I}$ an open cover of $X$. Then, 
    \begin{equation*}
        \dim X = \sup\limits_{i\in I}\dim U_i.
    \end{equation*}
    \item If $Y$ is a closed subspace of an irreducible finite-dimensional topological space $X$, and if $\dim Y = \dim X$, then $Y = X$.
    \end{enumerate}
\end{lemma}
\begin{proof}
\begin{enumerate}[label=(\alph*)]
    \item Let $Z_0\subsetneq\dots\subsetneq Z_n\subseteq X$ be a chain of closed, irreducible subspaces of $Y$. Then, consider the chain of closures in $X$, 
    \begin{equation*}
        \overline Z_0\subseteq\dots\subseteq\overline Z_n\subseteq X.
    \end{equation*}
    We contend that the inclusion $\overline Z_i\subseteq\overline Z_{i + 1}$ is strict. Indeed, if $\overline Z_i = \overline Z_{i + 1}$, then $\overline Z_i\cap Y = \overline Z_{i + 1}\cap Y$, which is absurd, since the $Z_j$'s are closed in $Y$. Hence, $\dim X\ge n$ and it follows that $\dim X\ge\dim Y$.

    \item From part (a), we know that $\dim X\ge\sup\limits_i \dim U_i$. We shall show the inequality in the other direction. Let $Z_0\subsetneq\dots\subsetneq Z_n\subseteq X$ be a chain of closed, irreducible subspaces of $X$. Pick some point $x_0\in Z_0$ and let $U_i$ be an element of the open cover containing $x_0$. Consider the sequence of closed subspaces $Z_0\cap U_i\subseteq\dots\subseteq Z_n\cap U_i$. Each $Z_j\cap U_i$ is an open subspace of $Z_j$ and hence, is irreducible and dense in $Z_j$. 

    Next, we contend that the inclusions $Z_j\cap U_i\subseteq Z_{j + 1}\cap U_i$ are strict. Indeed, if $Z_j\cap U_i = Z_{j + 1}\cap U_i = Y$, then, $Y$ is dense in both $Z_{j + 1}$ and $Z_j$ but $Z_j$ is a proper closed subspace of $Z_{j + 1}$, a contradiction. Thus, $\dim U_i\ge n$. That is, $\sup\limits_i\dim U_i\ge\dim X$. The conclusion follows. 

    \item \todo{Add in later}
\end{enumerate}
\end{proof}

\begin{proposition}
    If $Y$ is an algebraic set, then $\dim Y = \dim A[Y]$, where the latter is the Krull dimension.
\end{proposition}
\begin{proof}
    Immediate from definition.
\end{proof}

\begin{proposition}
    Let $Y$ be a quasi-affine variety, then $\dim Y = \dim\overline Y$.
\end{proposition}
\begin{proof}
    Note that $Y$ is open in $\overline Y$ as we have argued in \thref{rem:quasi-affine}. Suppose $Z_0\subsetneq\dots\subsetneq Z_n\subseteq Y$ is a sequence of closed irreducible subsets of $Y$, then $\overline Z_0\subsetneq\dots\subsetneq\overline Z_n\subseteq\overline Y$is a sequence of closed irreducible subsets of $\overline Y$. Thus, $\dim\overline Y\ge\dim Y$.

    Conversely, suppose $Z_0\subsetneq\dots\subsetneq Z_n\subseteq\overline Y$ is a chain of closed irreducible subsets of $\overline Y$. Then, each $Z_i\cap Y$ is an open subset of $Z_i$ whence is irreducible. 
    Further, if we have $Z_i\cap Y = Z_{i + 1}\cap Y$ for some $Y$, then $Z_{i  + 1} = (Z_{i + 1}\backslash Y)\cup Z_i$, both of which are closed, a contradiction. Thus, $Z_{i + 1}\cap Y\ne Z_i\cap Y$ and $\dim Y\ge\dim\overline Y$. This completes the proof.
\end{proof}

\begin{proposition}
    The Zariski topology on $\A^2$ is not the same as the product topology on $\A^1\times\A^1$.
\end{proposition}
\begin{proof}
    Note that the diagonal $\Delta$ in $\A^2$ is $Z((x - y))$ and hence, is closed. On the other hand, $\A^1$ is not Hausdorff, whence, the diagonal $\Delta$ in $\A^1\times\A^1$ is not closed. The conclusion follows.
\end{proof}
We shall see how to form the product of two varieties in an upcoming section.

\section{Projective Varieties}

We recall a bit about homogeneous ideals first.

\begin{definition}
    Let $R = \bigoplus_{n\ge 0} R_n$ be a graded ring. An ideal $\fraka\unlhd R$ is said to be \emph{homogeneous} if 
    \begin{equation*}
        \fraka = \bigoplus_{n\ge 0}(\fraka\cap R_n)
    \end{equation*}
    as an abelian group.
\end{definition}

\begin{proposition}
    An ideal $\fraka\unlhd R$ is homogeneous if and only if $\fraka$ can be generated by homogeneous elements.
\end{proposition}
\begin{proof}
    The forward direction is trivial. Conversely, suppose $\fraka$ is generated by $F = \bigcup F_i$ where each $F_i\subseteq R_i$. Obviously, $\bigoplus_{n\ge 0}(\fraka\cap R_n)\subseteq\fraka$. A generic element of $\fraka$ is of the form 
    \begin{equation*}
        a = \sum_{f\in F}r_f f = \sum_{f\in F}\sum_{i = 0}^\infty r_{f,i}f
    \end{equation*}
    where $r_{f, i}\in R_i$. Consequently, $r_{f, i}f$ is a homogeneous element in some $R_j$ and also lies in $\fraka$. Thus, $\fraka\subseteq\bigoplus_{n\ge 0}(\fraka\cap R_n)$. This completes the proof.
\end{proof}

\begin{proposition}
    Homogeneous ideals are closed under sum, product, intersection and radicals.
\end{proposition}
\begin{proof}
    The first three are obvious. Let $\fraka$ be a homogeneous ideal, $\frakb = \sqrt{\fraka}$ and let $x^m\in\fraka$ for some positive integer $m$. We can write $x = x_{i_1} + \dots + x_{i_k}$ where $i_1 < \dots < i_k$ and $x_{i_j}\in R_{i_j}$. Then, $x^m$ has a non-zero component in $R_{mi_k}$, which is $x_{i_k}^m$. Thus, $x_{i_k}^m\in\fraka$, consequently, $x_{i_k}\in\frakb$. Then, we have that $x - x_{i_k}$ also lies in $\frakb$. Using this, we can argue that all the $x_{i_j}$'s lie in $\frakb$, whence, $\frakb\subseteq\bigoplus_{n\ge 0}(\frakb\cap R_n)$ and hence, equality holds. This completes the proof.
\end{proof}

\begin{proposition}
    A homogeneous ideal $\frakp\unlhd R$ is prime if and only if for all homogeneous elements $f,g\in R$, $fg\in\frakp$ implies $f\in\frakp$ or $g\in\frakp$.
\end{proposition}
\begin{proof}
    We shall prove only the reverse direction. Suppose $f,g\in R$, $fg\in\frakp$ but $f,g\notin\frakp$. Let $f = f_1 + \dots + f_n$ and $g = g_1 + \dots + g_m$ where each $f_i$, $g_j$ is homogeneous and are arranged according to increasing homogeneous degree. Let $f_{n_0}$ be the largest such that $f_{n_0}\notin\frakp$, similarly, choose $g_{m_0}$. Then, $fg\in\frakp$ implies 
    \begin{equation*}
        (f_1 + \dots + f_{n_0})(g_1 + \dots + g_{m_0})\in\frakp.
    \end{equation*}
    If we expand the left hand side, $f_{n_0}g_{m_0}$ has the largest homogeneous degree among all the terms and hence, must lie in $\frakp$ (since the latter is a homogeneous ideal). Thus, either $f_{n_0}\in\frakp$ or $g_{m_0}\in\frakp$ according to our assumptions, a contradiction. This completes the proof.
\end{proof}

\begin{definition}
    The \emph{projective $n$-space over $k$}, denote $\bbP^n$ is defined as the set of equivalence classes of the set 
    \begin{equation*}
        \underbrace{k\times\dots\times k}_{n\text{ times}}\backslash\{(0,\dots,0)\},
    \end{equation*}
    under the equivalence relation 
    \begin{equation*}
        (x_0,\dots,x_n)\sim(y_0,\dots,y_n)\iff\exists\lambda\in k^\times,~y_i = \lambda x_i\text{ for every }0\le i\le n.
    \end{equation*}

    Let $S = k[x_0,\dots,x_n]$ with the standard grading $S = \bigoplus_{n\ge 0} S_n$ where $S_n$ is the additive abelian subgroup consisting of homogeneous degree $n$ polynomials in $S$.
\end{definition}

\begin{definition}[Algebraic Set]
    A subset $Y$ of $\bbP^n$ is said to be an \emph{algebraic set} if there is a set $T$ of homogeneous elements of $S$ such that $Y = Z(T)$. Now, let $Y\subseteq\bbP^n$. Define 
    \begin{equation*}
        \scrI(Y) := \{f\in S^h\mid f(p) = 0,~\forall p\in Y\},
    \end{equation*}
    where $S^h = \bigcup_{n\ge 0}S_n$ is the set of all homogeneous polynomials in $S$.
\end{definition}

Henceforth, we endow $\bbP^n$ with the Zariski topology.

\begin{proposition}
    Algebraic sets are closed under finite unions and arbitrary intersections. Therefore, the Zariski topology is defined to be the collection of complements of algebraic sets in $\bbP^n$.
\end{proposition}
\begin{proof}
    Note that $Z(T_1T_2) = Z(T_1)\cup Z(T_2)$ and $Z(\cup T_i) = \bigcap Z(T_i)$. 
\end{proof}

\begin{example}
    Let us consider $\bbP^1$ with the Zariski topology. Note that every homogeneous ideal can be generated by finitely many homogeneous polynomials. It suffices to find $Z(f)$ for a single homogeneous polynomial $f(x,y)$.

    If $[a_0 : a_1]\in Z(f)$, then note that neither of the $a_i$'s can be zero. Thus, $f(1, a_1/a_0) = 0$ and hence, $a_1/a_0$ can take finitely many values. Hence, the Zariski topology on $\A^1$ is precisely the cofinite topology.
\end{example}

\begin{proposition}
    Let $T_i\subseteq S^h$, $\fraka\unlhd A$ and $Y_i\subseteq\bbP^n$.
    \begin{enumerate}[label=(\alph*)]
        \item If $T_1\subseteq T_2\subseteq S^h$, then $Z(T_1)\supseteq Z(T_2)$. 
        \item If $Y_1\subseteq Y_2\subseteq\bbP^n$, then $\scrI(Y_1)\supseteq\scrI(Y_2)$. 
        \item $\scrI(Y_1\cup Y_2) = \scrI(Y_1)\cap\scrI(Y_2)$. 
        \item If $\fraka\unlhd S$ is a homogeneous ideal, then $\scrI(Z(\fraka)) = \sqrt{\fraka}$.
        \item $Z(\scrI(Y)) = \overline Y$.
    \end{enumerate}
\end{proposition}
\begin{proof}
    (a), (b) and (c) are trivial and (e) follows easily. Consider (d). Let $f\in\scrI(Z(\fraka))$. Note that $f$ is a homogeneous polynomial in $S$. Consider the algebraic set  $V\subseteq\A^{n + 1}$ generated by $\fraka$. Since $f$ vanishes on $Z(\fraka)\subseteq\bbP^n$, it must vanish on $V\subseteq\A^{n + 1}$. As a result, there is a positive integer $q$ such that $f^q\in\fraka$. The conclusion follows.
\end{proof}

\begin{proposition}
    The follwoing are equivalent: 
    \begin{enumerate}[label=(\alph*)]
        \item $Z(\fraka) = \emptyset$. 
        \item $\sqrt\fraka$ is either $S$ or $S_+$. 
        \item $S_d\subseteq\fraka$ for some $d > 0$.
    \end{enumerate}
\end{proposition}
\begin{proof}
    $(a)\implies(b)$ Let $V\subseteq\A^{n + 1}$ be the affine algebraic zero set of $\fraka$. We have either $V = \emptyset$ or $V = \{(0,\dots, 0)\}$. Then (b) follows from Hilbert's Nullstellensatz.

    $(b)\implies(c)$. Trivial. 

    $(c)\implies(a)$. Note that $\fraka$ contains $x_i^d$ for every $0\le i\le n$. The conclusion follows.
\end{proof}

\begin{corollary}
    There is a $1-1$ inclusion-reversing bijection between projective algebraic subsets of $\bbP^n$ and homogeneous radical ideals of $S$ not equal to $S_+$.
\end{corollary}

\begin{proposition}
    A projective algebraic set $Y\subseteq\bbP^n$ is irreducible if and only if $\scrI(Y)$ is a prime ideal.
\end{proposition}
\begin{proof}
    Suppose $Y$ is irreducible. Let $f,g$ be homogeneous elements such that $fg\in\scrI(Y)$. Then, $Z(f)\cup Z(g)\supseteq Y$, whence, $Y$ is contained in either $Z(f)$ or $Z(g)$, consequently, either $f\in\scrI(Y)$ or $g\in\scrI(Y)$ whence $\scrI(Y)$ is prime.

    Conversely, suppose $Y = Y_1\cup Y_2$ where $Y_1, Y_2$ are closed subsets of $\bbP^n$. Then, $\scrI(Y) = \scrI(Y_1)\cap\scrI(Y_2)$, consequently, $\scrI(Y) = \scrI(Y_i)$ for some $i\in\{1,2\}$, which follows from the fact that $\scrI(Y)$ is prime.
\end{proof}

\begin{corollary}
    $\bbP^n$ is irreducible.
\end{corollary}

\begin{definition}[Projective Variety, Quasi-Projective Variety]
    A \emph{projective variety} is an irreducible algebraic set in $\bbP^n$. A \emph{quasi-projective variety} is an open subset of a projective variety.
\end{definition}

\begin{theorem}
    Let $U_i$ denote the open set $\bbP^n\backslash Z(\{x_i\})$. The sets $\{U_i\}_{i = 0}^n$ cover $\bbP^n$. Consider the map $\varphi_i: U_i\to\A^n$ given by 
    \begin{equation*}
        \varphi_i((a_0,\dots,a_n)) = \left(\frac{a_0}{a_i},\dots,\widehat{\frac{a_i}{a_i}},\dots,\frac{a_n}{a_i}\right).
    \end{equation*}
    Then, $\varphi_i$ is a homeomorphism.
\end{theorem}
\begin{proof}
    We shall prove this for $i = 0$ and denote $\varphi_0$ by $\varphi: U_0\to\A^n$.
\end{proof}

\begin{theorem}
    Let $Y$ be a projective $n$-variety with homogeneous coordinate ring $S(Y)$. Then, $\dim S(Y) = \dim Y + 1$.
\end{theorem}
\begin{proof}
    Let $U_i = \bbP^n\backslash Z(x_i)$. We have seen that each $U_i$ is homeomorphic to $\A^n$ under the map $\varphi_i: U_i\to\A^n$ as defined above. Let $Y_i = \varphi_i(U_i\cap Y)\subseteq\A^n$. Note further that $Y_i$ is irreducible owing to $U_i\cap Y$ being irreducible (since it is an open subset of $Y$ which is irreducible). Thus, $Y_i$ is an affine $n$-variety. 

    Note that we can identify $A(Y_i)$ with the subring of degree $0$ elements in $S(Y)_{x_i}$. Further, note that 
    \begin{equation*}
        S(Y)_{x_i} = \left(S(Y)_{x_i}\right)_0[x_i, x_i^{-1}]\cong A(Y_i)[x_i, x_i^{-1}].
    \end{equation*}
    Note that $\dim S(Y)_{x_i} = \dim S(Y)$ since they both have the same fraction fields. Then, it follows that 
    \begin{equation*}
        \dim S(Y) = \dim S(Y)_{x_i} = \dim A(Y_i)[x_i, x_i^{-1}] = \dim A(Y_i) + 1 = \dim Y_i + 1.
    \end{equation*}
    The equality $\dim A(Y_i)[x_i, x_i^{-1}] = \dim A(Y_i) + 1$ follows by looking at the transcendence degree of the fraction fields.

    Finally, note that $\dim Y_i = \dim (Y\cap U_i)$ and hence, $\dim Y = \sup\dim Y_i$, consequently, $\dim Y + 1 = \dim S(Y)$. This completes the proof.
\end{proof}

\begin{corollary}
    Let $Y$ be a quasi-projective variety. Then, $\dim Y = \dim\overline Y$.
\end{corollary}
\begin{proof}
    Let $U_i = \bbP^n\backslash Z(x_i)$ for $0\le i\le n$. Note that $Y\cap U_i$ is a quasi-affine variety whose closure is $\overline Y\cap U_i$. Thus, 
    \begin{equation*}
        \dim Y = \sup_{i}\dim(Y\cap U_i) = \sup_i\dim(\overline{Y}\cap U_i) = \dim\overline Y.\qedhere
    \end{equation*}
\end{proof}

\begin{definition}[Cone over a Projective Variety]
    Let $Y\subseteq\bbP^n$ be a projective algebraic set. Let $\varphi:\A^{n + 1}\backslash\{(0,\dots,0)\}\to\bbP^n$ be given by $\theta((a_0,\dots,a_n)) = [(a_0,\dots,a_n)]$. The \emph{affine cone over $Y$} is defined to be 
    \begin{equation*}
        C(Y) := \theta^{-1}(Y)\cup\{(0,\dots,0)\}.
    \end{equation*}
\end{definition}

\begin{definition}[Segre Embedding]
    Define the map $\psi:\bbP^n\times\bbP^m\to\bbP^{N}$ where $N = (m + 1)(n + 1) - 1$ by 
    \begin{equation*}
        \psi\left([a_0 : \dots : a_{n + 1}], [b_0 : \dots : b_{m + 1}]\right) = [[a_ib_j]_{i, j}].
    \end{equation*}
    This is called the \emph{Segre embedding}
\end{definition}

\begin{proposition}
    With notation as above, $\im\psi$ is a subvariety of $\bbP^N$.
\end{proposition}
\begin{proof}
    Consider the $k$-algebra homomorphism $\phi: k[\{z_{ij}\}]\to k[x_0,\dots,x_n,y_0,\dots,y_m]$ given by $\phi(z_{ij}) = x_iy_j$. Let $\frakp = \ker\phi$. This is obviously a prime ideal in $k[\{z_{ij}\}]$. 

    Indeed, if $f\in\frakp$, then $f(\{x_iy_j\}) = 0$. We can write $f = \sum_{d\ge 0} f_d$. Then, $f_d(\{x_iy_j\}) = 0$ for every $d\ge 0$. Thus, every $f_d\in\frakp$ and it follows that $\frakp$ is homogeneous. We may now talk about $Z(\frakp)$ as a projective variety in $\bbP^N$.

    We contend that $Z(\frakp) = \im\psi$. The inclusion $\im\psi\subseteq Z(\frakp)$ is obvious. Now suppose $[\{c_{ij}\}]\in Z(\frakp)$. Without loss of generality, suppose $c_{00}\ne 0$. By normalizing coordinates, we may suppose that $c_{00} = 1$. Define $[a_0:\dots:a_n]$ and $[b_0:\dots:b_m]$ as follows. 
    \begin{equation*}
        a_i = 
        \begin{cases}
            1 & i = 0\\
            c_{i0} & i > 0
        \end{cases}
        \quad\text{ and }\quad
        b_j = 
        \begin{cases}
            1 & j = 0\\
            c_{0j} & j > 0
        \end{cases}.
    \end{equation*}

    Now, note that $z_{ij}z_{00} - z_{i0}z_{0j}\in\frakp$ and hence, $c_{ij} = c_{i0}c_{0j} = a_ib_j$. Thus, 
    \begin{equation*}
        [\{c_{ij}\}] = \psi\left([a_0:\dots:a_n], [b_0:\dots:b_m]\right),
    \end{equation*}
    this completes the proof.
\end{proof}

We shall use the Segre embedding to define the product of two (quasi-)projective varieties.

\section{Morphisms}

We begin by defining regular maps on varieties. The definitions will be different for afine and projective varieties.

\begin{definition}[Regular Map]
    Let $Y$ be a quasi-affine variety in $\A^n$. A function $f: Y\to k$ is \emph{regular at a point $P\in Y$} if there is an open neighborhood $U$ of $P$ in $Y$ and polynomials $g,h\in A$ such that $h$ does not vanish on $U$ and $f = g/h$ on $U$. The map $f$ is said to be \emph{regular on $Y$} if it is regular at every point of $Y$.

    Let $Y$ be a quasi-projective variety in $\bbP^n$. A function $f: Y\to k$ is \emph{regular at a point $P\in Y$} if there is an open neighborhood $U$ of $P$ in $Y$ and homogeneous polynomials $g,h\in S^h$ of equal degree such that $h$ does not vanish on $U$ and $f = g/h$ on $U$.
\end{definition}

\begin{theorem}
    A regular function is continuous when $k$ is identified with $\A^1$ or $\bbP^1$.
\end{theorem}
\begin{proof}
    First, suppose $Y$ is a quasi-affine variety. Identify $k$ with $\A^1$ and let $\varphi: Y\to k$ be a regular map on $Y$. It suffices to show that the inverse image of a closed set in $\A^1$ is closed in $Y$. But closed sets in $\A^1$ are precisely finite subsets of $k$. Hence, it suffices to show that the inverse image of a singleton in $\A^1$ is closed in $Y$. Let $a\in k$.

    There is an open cover of $Y$ such that on every open set of the cover, $\varphi$ is of the form $f/g$. Pick such an open set $U$. Then, $\varphi^{-1}(\{a\})\cap U = Z(f - ag)\cap U$, which is closed in $U$. Thus, $\varphi^{-1}(\{a\})$ is closed in $Y$.

    Now, suppose $Y$ is a quasi-projective variety. Note that when $k$ is identified with $\bbP^1$, it has the cofinite topology and a proof similar to the one in the preceeding paragraphs works.
\end{proof}

Henceforth, a \emph{variety} refers to either a quasi-affine or quasi-projective variety. When a result explicitly depends on the type of variety, we shall mention it.

\begin{corollary}
    Let $Y$ be a variety. If $f,g: Y\to k$ are regular functions that agree on an open subset of $Y$, then $f = g$ on $Y$.
\end{corollary}
\begin{proof}
    Let $X = \{y\in Y\mid f(y) = g(y)\}$. We know that $X$ is closed and contains an open subset $U$ of $Y$. But $U$ is dense in $Y$ (since $Y$ is irreducible) and hence, $X = Y$.
\end{proof}

\begin{definition}[Morphism]
    Let $X$ and $Y$ be varieties over $k$ (can be quasi-affine or quasi-projective). A \emph{morphism} $\varphi: X\to Y$ is a continuous map such that for every open set $V\subseteq Y$ and for every regular function $f: V\to K$, the function $f\circ\varphi:\varphi^{-1}(V)\to k$ is regular on $\varphi^{-1}(V)$.

    \begin{equation*}
        \xymatrix {
            X\supseteq\varphi^{-1}(V)\ar[r]^-{\varphi}\ar@{.>}[rd]_{f\circ\varphi} & V\ar[d]^{f}\\
            & k
        }
    \end{equation*}
\end{definition}

\begin{proposition}
    The composition of two morphisms is a morphism. The identity map on a variety is a morphism.
\end{proposition}
\begin{proof}
    Let $X, Y, Z$ be varieties and $\varphi: X\to Y$ and $\psi: Y\to Z$ be morphisms. Let $V\subseteq Z$ be an open set and $f: V\to k$ be a regular function on $V$.
    \begin{equation*}
        \xymatrix {
            \varphi^{-1}\psi^{-1}V\ar[r]\ar[rrd]_{f\circ\psi\circ\varphi} & \psi^{-1}V\ar[r]\ar[rd]^{f\circ\psi} & V\ar[d]^f\\
            & & k
        }
    \end{equation*}
    Applying the definition of a morphism twice, we see that $f\circ\psi\circ\varphi$ is a regular function. This completes the proof.
\end{proof}

\begin{definition}[Ring of Regular Functions]
    Let $Y$ be a variety. Then, the set of regular functions on $Y$, denoted $\scrO(Y)$ forms a ring known as the \emph{ring of regular functions on $Y$}.

    On the other hand, given any $P\in Y$, there is the \emph{local ring of $P$ on $Y$}, which is the ring of germs of regular functions at $P$. This is denoted by $\scrO_{P, Y}$ or just $\scrO_P$ if the variety is clear from the context.
\end{definition}

\begin{remark}\thlabel{rem:local-ring-p}
    Here, we explicitly define the ring of germs at $P$. Consider the set of all pairs $\langle U, f\rangle$ where $U$ is a neighborhood of $P$ in $Y$. Next, define the relation $\langle U, f\rangle\sim\langle V, g\rangle$ if $f = g$ on $U\cap V$. 

    To see that this is an equivalence relation, we need only verify transitivity. Indeed, suppose $\langle U, f\rangle\sim\langle V,g\rangle$ and $\langle V, g\rangle\sim\langle W, h\rangle$. Then, $f = g = h$ on $U\cap V\cap W$ which is a neighborhood of $P$. Since they agree on an open set, $f = h$ on $U\cap W$.

    Next, we must show that $\scrO_P$ is local. Let $\frakm$ denote the collection of germs that vanish at $P$. This is obviously a maximal ideal in $\scrO_P$. If $[\langle U, f\rangle]\in\scrO_P$ does not vanish at $P$, then there is a neighborhood $V$ of $P$ contained in $U$ on which $f$ does not vanish and $f$ is a quotient of polynomials. Thus, $1/f$ is also a quotient of polynomials on $V$ and is a well-defined inverse of the germ $[\langle V, f\rangle]$. Hence, $\scrO_P$ is local.
\end{remark}

\begin{lemma}\thlabel{lem:regular-function-on-quasi-affine}
    Let $X\subseteq\A^n$ be open and hence, a quasi-affine variety. If $f: X\to k$ is a regular function then $f = g/h$ for some $g,h\in A = k[x_1,\dots,x_n]$.
\end{lemma}
\begin{proof}
    Let $U\subseteq X$ be an open set such that $f = g/h$ (reduced) on $U$ for some polynomials $g,h\in A$. We contend that $f(P) = g(P)/h(P)$ for all $P\in X$. Indeed, there is a neighborhood $V$ of $P$ in $X$ such that $f = g'/h'$ (reduced) on $V$ for some $g', h'\in A$.

    The intersection $U\cap V$ is non-empty and $g/h = g'/h'$ on $U\cap V$. Consequently, $U\cap V\subseteq Z(gh' - g'h)$. But $U\cap V$ is open and dense in $\A^n$, whence $gh = g'h$ as polynomials in $A$. Using the fact that $A$ is a unique factorization domain and $(g, h) = (g', h') = 1$, we have that $g = g'$ and $h = h'$. This completes the proof.
\end{proof}

\begin{corollary}\thlabel{corr:ring-regular-An-pt}
    Let $X = \A^n\backslash\{(0,\dots,0)\}$. Then, $\scrO(X)\cong k[x_1,\dots,x_n]$.
\end{corollary}
\begin{proof}
    Let $f: X\to k$ be a regular function. Due to \thref{lem:regular-function-on-quasi-affine}, $f = g/h$ (reduced) on $X$ for some $g,h\in k[x_1,\dots,x_n]$. Note that $h$ does not vanish on $X$ and hence, must be constant. Therefore, $f\in k[x_1,\dots,x_n]$. The conclusion follows.
\end{proof}

\begin{definition}[Function Field]
    Let $Y$ be a variety. The \emph{function field of $Y$}, denoted $K(Y)$ is defined as the set of equivalence classes of the collection of pairs $\langle U, f\rangle$ where $U\subseteq Y$ is open and $f: U\to k$ is a regular function on $U$. The equivalence relation is defined as: 
    \begin{equation*}
        \langle U, f\rangle\sim\langle V, g\rangle\iff f = g\text{ on } U\cap V.
    \end{equation*}
    The elements of $K(Y)$ are called \emph{rational functions on $Y$}.
\end{definition}
\begin{remark}
    That $K(Y)$ is a well defined ring, one can argue as in \thref{rem:local-ring-p}. We show that this ring is a field. Indeed, suppose $\langle U, f\rangle$ is an element in $K(Y)$ with $f$ not identically $0$ on $U$. Then, there is an open set $V\subseteq U$ such that $f$ is non-zero on $V$. Then, $\langle V, 1/f\rangle$ is an inverse of $\langle U, f\rangle$ in $K(V)$.
\end{remark}

\begin{remark}
    Let $Y$ be a variety. There is a natural map $\scrO(Y)\to\scrO_P$ that sends a regular function on $Y$ to the equivalence class of the pair $\langle Y, f\rangle$, which is obviously an injective map. 

    Next, there is also a map $\scrO_P\to K(Y)$ that sends $[\langle U, f\rangle]\mapsto[\langle U, f\rangle]$. Again, this is obviously injective. Hence, we have an inclusion of rings 
    \begin{equation*}
        \scrO(Y)\subseteq\scrO_{P, Y}\subseteq K(Y)
    \end{equation*}
    and we shall often treat $\scrO(Y)$ and $\scrO_P$ as subrings of $K(Y)$.
\end{remark}

\begin{theorem}
    Let $Y\subseteq\A^n$ be an affine variety. Then, 
    \begin{enumerate}[label=(\alph*)]
        \item $\scrO(Y)\cong A(Y)$.
        \item for each $P\in Y$, $\scrO_P\cong A(Y)_{\frakm_P}$ where $\frakm_P$ is the maximal ideal of functions in $A(Y)$ vanishing at $P$. Consequently, $\dim\scrO_P = \dim Y$.
        \item $K(Y)$ is isomorphic to the fraction field of $A(Y)$. In particular, it is a finitely generated field extension of $k$ of transcendence degree $\dim Y$.
    \end{enumerate}
\end{theorem}
\begin{proof}
    First, note that every maximal ideal in $A(Y)$ is of the form $\frakm_P$ for some $P\in Y$. There is a natural map $A(Y)_{\frakm_P}\to\scrO_P$ that sends $f/g$ to $[\langle Y\backslash Z(g), f/g\rangle]$. This is obviously both injective and surjective and therefore, an isomorphism. This establishes (b). 

    Now, there is a natural map $A\to\scrO(Y)$ that sends a polynomial to the regular function defined by it on $Y$. The kernel of this map is precisely $\scrI(Y)$ and hence, induces an injective ring homomorphism $\alpha: A(Y)\to\scrO(Y)$. Upon identifying $A(Y)$ with a subring of $\scrO(Y)$ and using the fact that $\scrO_P = A(Y)_{\frakm_P}$, we have 
    \begin{equation*}
        A(Y)\subseteq\scrO(Y)\subseteq\bigcap_{P\in Y}\scrO_P\subseteq\bigcap_{\frakm_P}A(Y)_{\frakm_P} = A(Y),
    \end{equation*}
    where the last equality follows from the fact that $A$ is an integral domain. This establishes (a).

    Finally, from (b) and the fact that $\scrO_P\subseteq K(Y)$, it would follow that $K(Y)$ is precisely the fraction field of $A(Y)$.
\end{proof}

\begin{remark}[The Functor $\scrO$]
    We shall quickly see that $\scrO$ is a functor from the category of $k$-varieties, $\mathfrak{Var}_k$ to the category of finitely generated $k$-algebras $\catFinAlg_k$. Let $\varphi: X\to Y$ be a morphism of varieties. Then, if $f\in\scrO(Y)$, then $f\circ\varphi\in\scrO(X)$. This induces a map $\varphi_\ast:\scrO(Y)\to\scrO(X)$. Thus, $\scrO$ is a contravariant functor.
\end{remark}

\begin{theorem}
    Let $X$ be any variety and $Y$ an affine variety. Then, there is a natural bijection of sets
    \begin{equation*}
        \alpha: \Hom_{\catVar}(X, Y)\xrightarrow{\sim}\Hom_{\catAlg}(\scrO(Y),\scrO(X)).
    \end{equation*}
\end{theorem}
\begin{proof}
\end{proof}

\begin{corollary}
    The functor $X\mapsto\scrO(X)\equiv A(X)$ is a contravariant equivalence of categories between $\catVar_k$ and $\catFinAlg_k$.
\end{corollary}

\begin{example}
    We contend that the quasi-affine variety $\A^n\backslash\{(0,\dots,0)\}$ is not isomorphic to an affine variety. Let $X = \A^n\backslash\{(0,\dots,0)\}$, and suppose $X$ is affine. The inclusion morphism $\iota: X\into\A^n$ induces a restriction map $\iota_\ast:\scrO(\A^n)\to\scrO(X) = k[x_1,\dots,x_n]$, which is an isomorphism. From the equivalence of categories, it follows that $\iota$ must be an isomorphism too, which is absurd, since it is not even surjective.
\end{example}

\begin{definition}[Locally Closed]
    A subspace of a topological space is said to be \emph{locally closed} if it is open in its closure. Equivalently, if it is the intersection of an open set and a closed set.
\end{definition}

\begin{definition}
    If $X$ is a quasi-affine (resp. quasi-projective) variety and $Y$ is an irreducible locally closed subset, then $Y$ is also a quasi-affine (resp. quasi-projective) variety and is said to be a \emph{subvariety} of $X$.
\end{definition}

\subsection{Product of Varieties}

\begin{definition}[Product of Affine Varieties]
    If $X\subseteq\A^n$ and $Y\subseteq\A^m$ are affine-varieties, their product is defined to be the topological space $X\times Y\subseteq\A^{n + m}$ in the \emph{subspace topology}.
\end{definition}

\begin{proposition}
    With notation as above, 
    \begin{enumerate}[label=(\alph*)]
        \item $X\times Y\subseteq\A^{n + m}$ is a closed irreducible subspace, and hence, an affine variety.
        \item $A(X\times Y)\cong A(X)\otimes_k A(Y)$.
        \item $\dim X\times Y = \dim X + \dim Y$.
    \end{enumerate}
\end{proposition}
\begin{proof}
\begin{enumerate}[label=(\alph*)]
    \item 
\end{enumerate}
\end{proof}

\begin{definition}[Product of Quasi-Projective Varieties]
    
\end{definition}

\section{Rational Maps}

\begin{lemma}
    Let $X, Y$ be varieties and $\varphi,\psi: X\to Y$ be morphisms. If there is a non-empty open subset $U\subseteq X$ such that $\varphi|_U = \psi|_U$, then $\varphi = \psi$.
\end{lemma}
\begin{proof}
\end{proof}

\begin{definition}[Rational Map]
\end{definition}