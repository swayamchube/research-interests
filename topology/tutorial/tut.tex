\documentclass[12pt]{article}

\usepackage{amsmath, amsthm, amssymb, xcolor, geometry, mathrsfs, hyperref}

\hypersetup{
    colorlinks=true,
}

\title{Tutorials}
\author{Swayam Chube\\ Problems By: Prof. Ronnie Sebastian}
\date{Latest Update: \today}

\newcommand{\T}{\mathcal{T}}
\newcommand{\B}{\mathcal{B}}
\renewcommand{\S}{\mathcal{S}}
\newcommand{\Int}{\operatorname{Int}}
\newcommand{\e}{\epsilon}
\newcommand{\diam}{\operatorname{diam}}

\begin{document}
\maketitle
\tableofcontents
\newpage 

\section{Tutorial 1}
\begin{enumerate}
    \item Since $\emptyset,X\in\T_i$ for all $i$, they are also elements of $\T$. Let $\{A_\alpha\}$ be a collection of open subsets in $\T$. Then, for all $i\in I$, $\{A_\alpha\}\subseteq\T$, therefore $\bigcup_{\alpha} A_\alpha\in\T_i$ for all $i\in I$ and therefore is also an element of $\T$. Finally, let $\{A_1,\ldots,A_n\}$ be a finite collection of elements in $\T$, then by a similar argument, $\bigcup_{i = 1}^nA_i\in\T$ and therefore, $\T$ is a topology.

    Let $X = \{a,b,c,d\}$, $\T_1 = \{\emptyset, \{a,b\}, \{c,d\}, X\}$ and $\T_2 = \{\emptyset, \{a,c\}, \{b,d\}, X\}$. But $\{a,b,c\}\notin\T_1\cup\T_2$, therefore, it is not a topology on $X$.

    \item In the first case, existence is guaranteed by Zorn's lemma. As for uniqueness, suppose $\T_1$ and $\T_2$ are two such distinct topologies, then $\T_1\cap\T_2$ is properly contained in both $\T_1$ and $\T_2$ and also contains $\T_i$ for all $i\in I$, contradicting the minimality of $\T_1$ and $\T_2$ respectively.

    The second case has been answered in the previous problem.

    \item Trivial.
    
    \item 
    \item 
    \item 

    \item Just consider $\bigcup_{x\in A}U_x$.
\end{enumerate}
\end{document}