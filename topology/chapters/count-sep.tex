\section{Countability Axioms}

\begin{definition}[First Countable]
    A topological space $X$ is said to have a \textit{countable basis at $x$} if there is a countable collection $\B$ of neighborhoods of $x$ such that each neighborhood of $x$ contains at least one of the elements of $\B$. A space that has a countable basis at each of its points is said to satisfy the \textit{first countability axiom} or to be \textit{first-countable}.
\end{definition}

Obviously, every metrizable space satisfies this axiom, since we may take 
\begin{equation*}
    \B_x = \left\{B\left(x, \frac{1}{n}\right)\mid n\in\mathbb{N}\right\}
\end{equation*}

\begin{theorem}
    Let $X$ be a first countable topological space. 
    \begin{enumerate}
        \item Let $A$ be a subset of $X$. If $x\in\overline{A}$, then there is a sequence of points of $A$ converging to $x$.

        \item Let $f:X\to Y$. If for every convergent sequence $\{x_n\}_n$ to $x$, the sequence $\{f(x_n)\}_n$ converges to $f(x)$, then $f$ is continuous.
    \end{enumerate}
    \textit{Note that the converses for both do not require the first-countable hypothesis}.
\end{theorem}
\begin{proof}
    \hfill 
    \begin{enumerate}
        \item Suppose $x\in\overline{A}$. We shall construct a sequence of points in $A$ converging to $x$. Let $\B = \{B_1,B_2,\ldots\}$ be a countable basis at $X$. Let $a_1\in A\cap B_1$. If there is no open set $U_2$ containing $x$ that is contained in $B_1$, define $a_j = a_1$ for all $j\ge 2$. Otherwise, let $B_j$ be an element of $\B$ that is contained in $U_2$, and choose $a_2\in A\cap B_j$ and repeat. This gives rise to a convergent sequence.

        \item \textbf{\textcolor{red}{TODO: Fill this up}} 
    \end{enumerate}
\end{proof}

\begin{definition}[Second Countable]
    If a space $X$ has a countable basis for its topology, then $X$ is said to satisfy the \textit{second countability axiom} or to be \textit{second-countable}
\end{definition}

From the definition of the topology generated by a basis, \textit{second countability} implies \textit{first countability}. Further, not every metric space is \textit{second-countable}. To see this, consider the following lemma:
\begin{lemma}
    Let $X$ be a topological space with a countable basis $\B$, then any discrete subspace $A$ of $X$ must be countable.
\end{lemma}
\begin{proof}
    The proof follows quite naturally. Since $A$ is discrete, for each $a\in A$, there is $B_a\in\B$ such that $B_a\cap A = \{a\}$. Then, the function $f:A\to\B$ given by $f(a) = B_a$ is an injection and thus $A$ is countable.
\end{proof}

Now, consider $\R^\omega$ with the uniform topology. The set $A = \{0,1\}^\omega$ is uncountable and under the uniform topology, is discrete, since $\overline{\rho}(a,b) = 1$ for all $a,b\in A$ with $a\ne b$. This immediately implies that $\R^\omega$ under the uniform topology may not have a countable basis and cannot be second-countable.

We now show that the same isn't true for $\R^\omega$ equipped with the product topology. It is well known that the countable collection of all open intervals $(a,b)$ with both $a,b\in\Q$ forms a basis for $\R$. Then, $\R^\omega$ has a \textbf{countable basis} of all open sets of the form $\prod_{n\in\Z^+}U_n$ where $U_n$ is an open interval with rational end points for finitely many values of $n$ and $U_n = \R$ for all others.

\begin{theorem}
    A subspace of a first-countable space is first-countable, and a countable product of first-countable spaces is first-countable. A subspace of a second-countable space is second-countable, and a countable product of a second-countable spaces is second-countable.
\end{theorem}
\begin{proof}
    The assertion about subspaces is trivially true in both cases. As for the second part note that a cross product of countable sets is countable.
\end{proof}

\begin{definition}[Lindel\"of Space]
    A topological space $X$ is said to be Lindel\"of if every open cover has a countable subcover.

    \noindent\textit{Obviously all compact spaces are Lindel\"of}
\end{definition}

\begin{definition}[Dense]
    A subset $A$ of a space $X$ is said to be \textit{dense} in $X$ if $\overline{A} = X$. $X$ is said to be separable if it has a countable dense subset.
\end{definition}

\begin{theorem}
    The following are true: 
    \begin{enumerate}[label=(\alph*)]
    \item The continuous image of a separable space is separable 
    \item An open subspace of a separable space is separable
    \item Let $\{X_\alpha\}_{\alpha\in J}$ be a collection of Hausdorff spaces with at least two points each. Then $\prod_{\alpha\in J}X_\alpha$ is separable if and only if each $X_\alpha$ is separable and $J$ has cardinality at most that of the continuum.
    \end{enumerate}
\end{theorem}
\begin{proof}
\begin{enumerate}[label=(\alph*)]
\item Trivial.
\item Trivial.
\item Suppose $X = \prod_{\alpha\in J}X_\alpha$ is separable. Since projection is a continuous map, each $X_\alpha$ is continuous. We shall now show that the cardinality of $J$ is atmost that of the continuum. Since each $X_\alpha$ is Hausdorff, with at least two points, there are disjoint open sets $U_\alpha, V_\alpha$ in $X_\alpha$. Let $D$ be a countable dense subset of $X$. Define $D_\alpha = D\cap\pi_{\alpha}^{-1}(U_\alpha)$. We claim that the map $\psi: J\to 2^D$ given by $\alpha\mapsto D_\alpha$ is injective. Indeed, for $\alpha\ne\beta$, consider the open sets $U = \pi_{\alpha}^{-1}(U_\alpha)\cap\pi_{\beta}^{-1}(V_\beta)$ and $V = \pi_{\beta}^{-1}(U_\beta)$. Note that $U$ and $V$ are disjoint, further, the points in $U\cap D$ belong to $D_\alpha$ and due to the disjointness of $U$ and $V$, they are not in $D_\beta = V\cap D$. Finally, since $D$ is countable, the cardinality of $2^D$ is at most the cardinality of the continuum and thus $J$ has cardinality at most that of the continuum.

Conversely suppose $J$ has cardinality atmost that of the continuum. Then, we may treat $J$ as a subset of $[0,1]$. Let $D_\alpha = \{d_{\alpha 1}, d_{\alpha 2},\ldots\}$ be a countable dense subset of $X_\alpha$. Let $\mathscr I$ be the countable collection of open intervals with rational endpoints in the order topology on $[0,1]$. Consider the collection of all even length tuples of the form $(I_1,\ldots,I_k;n_1,\ldots,n_k)$ where the $I_j$'s are disjoint intervals in $\mathscr I$ and $n_1,\ldots,n_k$ are positive integers. Define the point $p(I_1,\ldots,I_k;n_1,\ldots,n_k)$ by 
\begin{equation*}
    p_\alpha = 
    \begin{cases}
        d_{\alpha n_i} & \alpha\in J_i\text{ for some $i$}\\
        d_{\alpha 1} & \text{otherwise}
    \end{cases}
\end{equation*}
Obviously, the collection of such points, $D$, is countable. We shall show that this collection is dense. Consider a basic open set in $X$, which is of the form 
\begin{equation*}
    B = \pi_{\alpha_1}^{-1}(U_{\alpha_1})\cap\cdots\cap\pi_{\alpha_m}^{-1}(U_{\alpha_m})
\end{equation*}
Note that each $U_{\alpha_i}$ contains a point $d_{\alpha_in_i}$ of $D_{\alpha_i}$ for some $n_i\in\N$. Since the $\alpha_i$'s are finitely many, there are disjoint intervals $I_1,\ldots,I_m$ in $\mathscr I$ containing $\alpha_1,\ldots,\alpha_m$ respectively. Then, the point $p(J_1,\ldots,J_m;n_1,\ldots,n_m)$ belongs to $B$ and the set $D$ is dense as desired.
\end{enumerate}
\end{proof}

\begin{theorem}
    Suppose that $X$ is second countable. Then 
    \begin{enumerate}
        \item $X$ is Lindel\"of 
        \item $X$ is separable
    \end{enumerate}
\end{theorem}
\begin{proof}
    \hfill 
    \begin{enumerate}
        \item Let $\B = \{B_1,B_2,\ldots\}$ be a countable basis for $X$ and $\mathscr{A}$ be an open cover. For each $x\in X$, let $A_x$ be an element in $\mathscr{A}$ containing $x$. By definition, there must exist a basis element $B_x$ such that $x\in B_x\subseteq A_x$. Let $\mathscr{B} = \{B_x\mid x\in X\}$. Obviously $\mathscr{B}\subseteq\B$ and is therefore countable. Further, for each $B\in\mathscr{B}$, there is $A(B)\in\mathscr{A}$ containing $B$. Therefore, $\{A(B)\mid B\in\mathscr{B}\}$ forms a countable subcover.

        \item Using the Axiom of Choice, choose a set $D = \{x_n\mid x_i\in B_i\}$. For each $x\in X\backslash D$, and an open set $U$ containing $x$, then there is a basis element $B_j$ containing $x$ that is contained in $U$. Therefore, $x_j\in U$. This implies $x\in\overline{D}$.
    \end{enumerate}
\end{proof}

\begin{theorem}
    Let $(X,d)$ be a metric space. Then, the following are equivalent
    \begin{enumerate}
        \item $X$ is second countable 
        \item $X$ is Lindel\"of
        \item $X$ is separable
    \end{enumerate}
\end{theorem}
\begin{proof}
\hfill 
\begin{itemize}
    \item \noindent$\underline{(1)\Longrightarrow(2)\wedge(1)\Longrightarrow(3)}$ Proved above.

    \item \noindent$\underline{(3)\Longrightarrow(1)}$ Let $D = \{x_1,x_2,\ldots\}$ and $\Q = \{q_1,q_2,\ldots\}$. We shall show that the collection $\{B(x_i, q_j)\}_{i,j\in\N\times\N}$ is a basis for the metric topology on $X$.

    \item \noindent$\underline{(2)\Longrightarrow(1)}$ Let $\mathscr A_n$ denote the open cover $\{B(x,\frac{1}{n})\}_{x\in X}$. Since $X$ is Lindel\"of, it has a countable subcover, say $\mathscr B_n$. Define $\B = \bigcup\limits_{n\in\N}\mathscr B_n$ which is countable. We shall show that $\B$ is a basis for the metric topology on $X$. Let $U$ be a neighborhood of $x\in X$. Then, there is $r\in\R^+$ such that $B(x,r)\subseteq U$. Choose $N\in\N$ such that $\frac{1}{N} < r/2$. Let $B$ be the element of $\mathscr B_N$ that contains $x$. Then, for any $y\in B_N$, $d(x,y)\le\frac{2}{N} < r$ and $y\in U$. Consequently, $B\subseteq U$ and we are done.
\end{itemize}
\end{proof}

\section{Separation Axioms}

\begin{definition}[Regular Spaces]
    Suppose one-point sets are closed in $X$. Then $X$ is said to be \textit{regular} or a \textit{$T_3$-space} if for each pair consisting of a point $x$ and a closed set $B$ disjoint from $X$, there exist \underline{disjoint} open sets containing $x$ and $B$, respectively.
\end{definition}

\begin{definition}[Normal Spaces]
    Suppose one-point sets are closed in $X$. Then $X$ is said to be \textit{noraml} or a \textit{$T_4$-space} if for each pair $A,B$ of disjoint closed sets in $X$, there exist \underline{disjoint} open sets containing $A$ and $B$.
\end{definition}

It is not hard to see that 
\begin{equation*}
    \text{Normal}\Longrightarrow\text{Regular}\Longrightarrow\text{Hausdorff}
\end{equation*}

\begin{theorem}
    Let $X$ be a topological space such that one point sets in $X$ are closed. 
    \begin{enumerate}
        \item $X$ is regular if and only if given a point $x\in X$ and a neighborhood $U$ of $x$, there is a neighborhood $V$ of $x$ such that $\overline{V}\subseteq U$

        \item $X$ is normal if and only if given a closed set $A$ and an open set $U$ containing $A$, there is an open set $V$ containing $A$ such that $\overline{V}\subseteq U$.
    \end{enumerate}
\end{theorem}
\begin{proof}
    \hfill 
    \begin{enumerate}
        \item Suppose $X$ is regular and $x\in U\in \T_X$. Since $X\backslash U$ is closed, there are disjoint open sets $V$ and $W$ such that $x\in V$ and $X\backslash U\subseteq W$. It is not hard to see that $\overline{V}\cap W = \emptyset$, therefore $\overline{V}\subseteq U$.

        Conversely, let $x\in U$ and $A\subseteq X$ be a closed set. Then, $X\backslash A$ is open and $x\in X\backslash A$. Therefore, there is an open set $V$ containing $x$ such that $\overline{V}\subseteq X\backslash A$. Then, $A\subseteq X\backslash\overline{V}$ and we are done.

        \item Suppose $X$ is normal. Then, $B = X\backslash U$ is a closed set disjoint from $A$. Therefore, there are open sets $V, W$ containing $A$ and $B$ respectively such that $A\subseteq V$ and $B\subseteq W$. It is not hard to see that $\overline{V}\cap W = \emptyset$, therefore, $\overline{V}\subseteq U$.

        Conversely, let $A$ be closed in $X$. Then, $B = X\backslash U$ is closed, and the sets $V$ and $X\backslash\overline{V}$ contain $A$ and $B$ respectively, and are disjoint, therefore the space is normal.
    \end{enumerate}
\end{proof}

\begin{theorem}
    \hfill 
    \begin{enumerate}
        \item A subspace of a Hausdorff space is Hausdorff; a product of Hausdorff spaces is Hausdorff 
        \item A subspace of a regular space is regular; a product of regular spaces is regular.
    \end{enumerate}
\end{theorem}
\begin{proof}
    \hfill 
    \begin{enumerate}
        \item The subspace part is trivial. Let $(X_\alpha)_\alpha$ be a collection of Hausdorff spaces. Let $\mathbf{x},\mathbf{y}\in\prod_\alpha X_\alpha$. Since $\mathbf{x}\ne\mathbf{y}$, there is an index $\beta$ such that $x_\beta\ne y_\beta$. Therefore, there disjoint are open sets $U,V$ in $X_\beta$ such that $x_\beta\in U$ and $y_\beta\in V$. As a result, $\pi_\beta^{-1}(U)$ and $\pi_\beta^{-1}(V)$ are disjoint and open in $\prod_\alpha X_\alpha$.

        \item The subspace part is trivial. Let $\mathbf{x}\in\prod_\alpha X_\alpha$ where each $X_\alpha$ is regular and $U\subseteq\prod_\alpha X_\alpha$ be an open set containing $\mathbf{x}$. Let $\prod_\alpha U_\alpha$ be a basis element of $\prod_\alpha X_\alpha$ containing $x$ that is also contained in $U$.. For each $x_\alpha$, let $V_\alpha$ be an open set in $X_\alpha$ containing it such that $\overline{V_\alpha}\subseteq U_\alpha$. Note that if $U_\alpha = X_\alpha$, choose $V_\alpha = X_\alpha$ instead. As a result, $\prod_\alpha V_\alpha$ is in the product topology and its closure is contained in $U$. This completes the proof.
    \end{enumerate}
\end{proof}

\section{Normal Spaces}

\begin{theorem}
    Every regular space with a countable basis is normal.
\end{theorem}
\begin{proof}
    Let $X$ be a regular space with countable basis $\B$ and $A,B$ be closed sets in $X$. For each $x\in X$, using regularity, there is an open set $U_x$ containing $x$ and disjoint from $B$. Further, using regularity, there is a neighborhood of $x$, $V_x$ such that $\overline{V}_x\subseteq U_x$. Finally, choose a basis element $B_x$ from $\B$ containing $x$ that is contained in $V$.

    We now have a countable cover $\{U_n\}$ for $A$, such that $\overline{U}_i\cap B = \emptyset$. Similarly, choose a countable open cover $\{V_n\}$ for $B$, such that $\overline{V}_i\cap A = \emptyset$. Let us now define 
    \begin{equation*}
        U_n' = U_n\backslash\bigcup_{i = 1}^n\overline{V}_i 
        \qquad 
        V_n' = V_n\backslash\bigcup_{i = 1}^n\overline{U}_i 
    \end{equation*}

    We shall show that $U_i'$ and $V_j'$ are disjoint for any $i,j$. Without loss of generality, suppose $i\le j$. Suppose $x\in U_i'\cap V_j'$, therefore, $x\in U_i$ and $x\in V_j$, but using the definition of $V_j'$, we must have that $x\notin V_j'$, a contradiction.

    Finally, define 
    \begin{equation*}
        U = \bigcup_{i = 1}^\infty U_i'\qquad V = \bigcup_{j = 1}^\infty V_j'
    \end{equation*}
    These are disjoint open sets contain $A$ and $B$ respectively. This concludes the proof.
\end{proof}

\begin{theorem}
    Every compact Hausdorff space is normal.
\end{theorem}
\begin{proof}
    Let $X$ be a compact Hausdorff space. We shall first show that $X$ is regular. Indeed, let $x\in X$ and $A\subseteq X$ be a closed set. Since $X$ is compact so is $A$. For all $a\in A$, there are disjoint open sets $U_a$ and $V_a$ such that $x\in U_a$ and $a\in V_a$. Note that $\mathscr{A} = \{A_a\mid a\in A\}$ is an open cover for $A$ and therefore has a finte subcover $\{V_{a_1},\ldots,V_{a_n}\}$. Let 
    \begin{equation*}
        U = \bigcap_{i = 1}^nU_{a_i}\qquad V = \bigcup_{i = 1}^nV_{a_i}
    \end{equation*}
    which are disjoint open sets containing $x$ and $A$ respectively.

    Suppose $A$ and $B$ are disjoint closed sets in $X$. For each $a\in A$, there are disjoint open sets $U_a$ and $V_a$ such that $a\in U_a$ and $B\subseteq V_a$. Note that $\mathscr{A} = \{U_a\mid a\in A\}$ is an open cover for $A$, and therefore, has a finite subcover $\{A_{a_1},\ldots,A_{a_n}\}$. Choose 
    \begin{equation*}
        U = \bigcup_{i = 1}^n U_{a_i}\qquad V = \bigcap_{i = 1}^nV_{a_i}
    \end{equation*}
    which are disjoint open sets containing $A$ and $B$ respectively.
\end{proof}

\begin{theorem}
    Every metrizable space is normal.
\end{theorem}
\begin{proof}
    Let $(X,d)$ be a metric space and $A,B$ be two disjoint closed subsets of $X$. 
\end{proof}

\section{Urysohn's Lemma}

\begin{definition}
    Let $X$ be a normal space and $A,B\subseteq X$ be two closed sets. Then there is a continuous function $f:X\to[0,1]$ such that $f(A) = \{0\}$ and $f(B) = \{1\}$.
\end{definition}
\begin{proof}
    Let $P$ be the countable set of all rational numbers in $[0,1]$. First, define $U_1 = X\backslash B$, which is an open set containing $A$. Due to the normality of $X$, there is an open set $U_0$ containing $A$ such that $\overline{U}_0\subseteq U_1$.

    We shall now define an open set $U_p$ for all $p\in P$ such that 
    \begin{equation*}
        p < q \Longrightarrow\overline{U}_p\subseteq U_q
    \end{equation*}
    Let $P_n$ be the set containing the first $n$ rational numbers in some enumeration of $P$ such that the first two enumerated rationals are $0$ and $1$. Let $r$ be the $n + 1$-st rational in the enumeration. Obviously, since $P_n$ is finite and $0,1\in P_n$, there are rationals $p,q\in P$ such that 
    \begin{align*}
        p &= \max\{x\in P_n\mid x < r\}\\
        q &= \min\{x\in P_n\mid x > r\}
    \end{align*}

    Now, due to the induction hypothesis, $\overline{U}_p\subseteq U_q$ and therefore, using the normality of $X$, there is an open set $U_r$ such that $\overline{U}_p\subseteq U_r$ and $\overline{U}_r\subseteq U_q$.

    Let $s\in P_{n + 1}$. If $ s < p$, $\overline{U}_s\subseteq U_p\subseteq\overline{U}_p\subseteq U_r$ and if $q < s$, $\overline{U}_r\subseteq U_q\subseteq\overline{U}_q\subseteq U_s$. Therefore, the induction hypothesis holds.


    Now that we have defined $U_p$ for all $p\in P$, we shall define
    \begin{equation*}
        U_p = 
        \begin{cases}
            \emptyset & p < 0\\
            X & p > 1
        \end{cases}
    \end{equation*}

    Now, for all $x\in X$, define the function $f:X\to[0,1]$ as
    \begin{equation*}
        f(x) = \inf\{p\mid x\in U_p\}
    \end{equation*}

    Note that since for all $p > 1$, $x\in U_p$ and the rationals are dense in the reals, $0\le f(x)\le 1$. For all $a\in A$, note that $a\in U_0$, therefore $f(a) = 0$. Similarly, for all $b\in B$, note that $b\notin U_1$, as a result $b\notin U_p$ for all $p\in[0,1]$, but $b\in U_q$ for all $q > 1$, therefore, $f(b) = \inf\{q\in\mathbb{Q}\mid q > 1\} = 1$

    All that remains is to show that $f$ is continuous. Let $x\in X$ and $(c,d)\in[0,1]$ be an open interval containing $f(x)$. Choose any two rational numbers $p,q$ such that $c < p < f(x) < q < d$. Let us consider the image of the set $Y = U_q\backslash\overline{U}_p$. For all $y\in Y$, $f(y) > p$, while $f(y) < q$, therefore, $f(y)\in(c,d)$, as a result, $f(Y)\subseteq (c,d)$ and $f$ is continuous. This completes the proof.
\end{proof}

\begin{definition}
    If $A$ and $B$ are two subsets of a topological space $X$, and if there is a continuous function $f:X\to[0,1]$ such that $f(A) = \{0\}$ and $f(B) = \{1\}$, we say that $A$ and $B$ \textit{can be separated by a continuous function}.
\end{definition}

\begin{definition}[Completely Regular]
    A space $X$ is \textit{completely regular} or $T_{3\frac{1}{2}}$ if one-point sets are closed in $X$ and for each point $x_0$ and each closed set $A$ not containing $x_0$, there is a continuous function $f:X\to[0,1]$ such that $f(x_0) = 1$ and $f(A) = \{0\}$.
\end{definition}

\begin{theorem}
    A subspace of a completely regular space is regular. A product of completely regular spaces is completely regular.
\end{theorem}
\begin{proof}
    \textbf{\textcolor{red}{TODO: Add in later}}
\end{proof}
\begin{corollary}
    A locally compact Hausdorff space is completely regular.
\end{corollary}

\begin{proposition}
    Let $X$ be locally compact Hausdorff, $K$ a compact subset and $A$ a disjoint closed subset. Then, there is a continuous function $f: X\to[0,1]$ such that $f(K) = 0$ and $f(A) = 1$.
\end{proposition}

\begin{theorem}
    Let $X$ be a locally compact Hausdorff space and $K\subseteq X$ be compact. Let $V$ be an open set containing $K$. Then there is a continuous function $f: X\to[0,1]$ with compact support such that $f(K) = 1$ and $\operatorname{supp}(f)\subseteq V$.
\end{theorem}
\begin{proof}
    First, we show that there is a compact set $K'$ and an open set $U$ such that $K\subseteq U\subseteq K'\subseteq V$. For each $a\in K$, there is a neighborhood $U_a$ with compact closure such that $a\in\overline{U_a}\subseteq V$. Since $\{U_a\}$ forms an open cover for $K$, it has a finite cover, say $\{U_{a_1},\ldots,U_{a_n}\}$. Define $U = U_{a_1}\cup\cdots\cup U_{a_n}$ and $K' = \overline{U_{a_1}}\cup\cdots\cup\overline{U_{a_n}}$. 

    Due to the previous proposition, there is a continuous function such that $f(K) = 1$ and $f(X\backslash U) = 0$. Obviously, $\{x\in X\mid f(x)\ne 0\}\subseteq K'$ and since $K'$ is closed, $\operatorname{supp}(f)\subseteq K'$ whence it is compact. This completes the proof.
\end{proof}

\begin{lemma}
    Let $X$ be normal and $A\subseteq X$. Then, there is a function $f: X\to[0,1]$ such that $f^{-1}(\{0\}) = A$ if and only if $A$ is a closed $G_\delta$ set.
\end{lemma}
\begin{proof}
    Suppose there is a function $f: X\to[0,1]$ such that $f^{-1}(\{0\}) = A$. Then, obviously $A$ is closed, further 
    \begin{equation*}
        A = \bigcap_{n = 1}^\infty f^{-1}\left(\left[0,\frac{1}{n}\right)\right)
    \end{equation*}
    whence $A$ is $G_\delta$.

    Conversely, suppose $A$ is a closed $G_\delta$ subset of $X$. Then there is a countable collection of open sets $\{U_n\}$ such that $A = \bigcap_{n = 1}^\infty U_n$. Using normality and the Urysohn Lemma, there is a continuous function $f_n: X\to[0,1]$ such that $f(A) = \{0\}$ and $f(X\backslash U_n) = \{1\}$. Define the function 
    \begin{equation*}
        f = \sum_{n = 1}^\infty 2^{-n}f_n
    \end{equation*}
    Using the Weierstrass M-test, it is not hard to see that the convergence of the series to $f$ is uniform and thus $f$ is a continuous function satisfying the required prooperties.
\end{proof}

\begin{theorem}
    
\end{theorem}

\section{The Urysohn Metrization Theorem}
\begin{theorem}[Urysohn Metrization Theorem]
    Every regular space $X$ with a countable basis is metrizable.
\end{theorem}
\begin{proof}
    Recall first that every regular space with a countable basis is normal.
    We shall show that $X$ is metrizable by constructing an imbedding of $X$ into $\R^\omega$. We shall first show that there is a countable sequence of functions $\{f_n\}$ from $X$ to $[0,1]$ such that for all $x_0\in X$ and a neighborhood $U$ of $x_0$, there is a function $f_n$ such that $f(x_0) > 0$ and $f(x)=0$ for all $x\in X\backslash U$.

    Let $\B = \{B_n\}$ be a countable basis for $X$. Then, for all pairs $(m, n)$ such that $\overline{B}_m\subseteq B_n$, define the function $g_{m,n}:X\to[0,1]$, using Urysohn's Lemma, such that $g_{m,n}(\overline{B}_m) = \{1\}$ and $g_{m,n}(B_n) = \{0\}$. Obviously, the set $\{g_{m,n}\mid m,n\in\N\}$ is countable and it is not hard to see that this is our desired sequence of functions $\{f_n\}$.

    Define now the map $F:X\to\R^\omega$ given by
    \begin{equation*}
        F(x) = (f_1(x),f_2(x),\ldots)
    \end{equation*}
    We shall now show that $F$ is an imbedding. First, we show that $F$ is injective. Indeed, if $x\ne y$, then we know that there is an open set containing $x$ but not $y$, therefore, there is a function $f_n$ such that $f(x) > 0$ while $f(y) = 0$. As a result, $F(x)\ne F(y)$.

    Since each of the functions $f_i$ are continuous, so is $F$. We need only show now that $F$ maps open sets in $X$ to open sets in $\R^\omega$. Let $Z = F(X)$ and $U$ be an open set in $X$. It suffices to show that for all $x_0\in U$, there is an open set $V$ in $Z$ such that $f(x_0)\in V\subseteq F(U)$. There is $n\in\mathbb{N}$ such that $f_n(x_0) > 0$ and $f_n$ vanishes outside $U$. Let $\pi_n: \R^\omega\to\R$ be the natural projection map. Let $V = \pi_n^{-1}((0,\infty))\cap Z$. Obviously, note that $f(x_0)\in V$. Further, for all $z\in V$, note that there is $x\in X$ such that $z = F(x)$, but since $\pi_n(F(x)) > 0$, we must have that $x\in U$, therefore, $V\subseteq F(U)$. This shows that $F(U)$ is open in $\R^\omega$ and thus, $F$ is an imbedding. This completes the proof.
\end{proof}

\section{Tietze Extension Theorem}

\begin{lemma}
    Let $X$ be a normal space and $A$ be a closed subspace of $X$. Let $f:A\to [-r,r]$ be a continuous map. Then, there is a continuous function $g:X\to [-r,r]$ such that 
    \begin{equation*}
        |f(a) - g(a)|\le 2r/3\qquad |g(x)|\le r/3\quad\text{for all $a\in A$, $x\in X$}
    \end{equation*}
\end{lemma}
\begin{proof}
    Define 
    \begin{equation*}
        I_1 = \left[-r,-\frac{r}{3}\right]\quad
        I_2 = \left[-\frac{r}{3},\frac{r}{3}\right]\quad
        I_1 = \left[\frac{r}{3},r\right]
    \end{equation*}
    and 
    \begin{equation*}
        B = f^{-1}(I_1)\qquad C = f^{-1}(I_3)
    \end{equation*}

    Since $I_1$ and $I_3$ are closed in $[-r,r]$, $B$ and $C$ must be disjoint and closed in $X$. Now, due to Urysohn's Lemma, there is a function $g:X\to [-r/3,r/3]$ such that $g(B) = \{-r/3\}$ and $g(C) = \{r/3\}$, which has a natural extension $g:X\to[-r,r]$. 

    Obviously, $|g(x)|\le r/3$ for all $x\in X$. Further, for all $a\in A$, if $a\in B$, then $g(a) = -r/3$, and $f(a)\in I_1$, similarly, if $a\in C$, then $g(a) = r/3$ and $f(a)\in I_3$. This immediately implies the desired conclusion. 
\end{proof}

\begin{theorem}[Tietze Extension Theorem]
    Let $X$ be a normal space; let $A$ be a closed subspace of $X$ 
    \begin{enumerate}
        \item Any continuous map of $A$ into the closed interval $[-1,1]$ of $\R$ may be extended to a continuous map of all $X$ into $[-1,1]$
        \item Any continuous map of $A$ into $\R$ may be extended to a continuous map of all of $X$ into $\R$
    \end{enumerate}
\end{theorem}
\begin{proof}
    The main idea of the proof is to construct a uniformly convergent sequence of continuous functions to $f$ on $A$. This would immediately imply the continuity of the limiting function over $X$, due to the Uniform Limit Theorem.

    \begin{enumerate}
        \item Using the preceeding lemma, there is a fucntion $g_1: X\to[-1,1]$ such that $|f(a) - g_1(a)|\le 2/3$, while $|g(x)|\le 1/3$ for all $a\in A$ and $x\in X$. Let us define $f_1: A\to[-2/3,2/3]$ as 
        \begin{equation*}
            f_1(x) = f(x) - g_1(x)
        \end{equation*}
        which is a continuous function. Then, we may reuse the previous lemma to define a function $g_2(x): X\to[-1,1]$ such that $|f_1(a) - g_2(a)|\le(2/3)^2$, while $|g(x)|\le(2/3)(1/3)$ and so on. As a result, we define the function $g_n:X\to[-1,1]$ satisfying 
        \begin{equation*}
            |f_{n - 1}(a) - g_n(a)|\le\left(\frac{2}{3}\right)^n\qquad|g(x)|\le\frac{1}{3}\left(\frac{2}{3}\right)^{n - 1}
        \end{equation*}

        Finally, define the functions $s_n:X\to\R$
        \begin{equation*}
            s_n(x) = \sum_{i = 1}^ng_n(x)
        \end{equation*}

        We note that 
        \begin{equation*}
            -1 < -\frac{1}{3}\sum_{i = 1}^{n}\left(\frac{2}{3}\right)^{i - 1}\le s_n(x)\le \frac{1}{3}\sum_{i = 1}^{n}\left(\frac{2}{3}\right)^{i - 1} < 1
        \end{equation*}

        Hence, we may take the restriction of $s_n$ to $[-1,1]$, which would also be continuous since it is the range restriction of a sum of finitely many continuous functions. Now, due to the Weierstrass $M$-test, the sequence of functions $s_n$ are uniformly convergent. Further, since 
        \begin{equation*}
            |f(a) - s_n(a)|\le\left(\frac{2}{3}\right)^n
        \end{equation*}
        we know that the convergent function $s:X\to[-1,1]$ agrees with $f$ on $A$. This completes the proof. 

        \item Recall that the spaces $(-1,1)$ and $\R$ are homeomorphic. Therefore, it suffices to prove the statement for functions of the form $f:A\to(-1,1)$. Using the first part of this theorem, we know that there is a function $g:X\to[-1,1]$. We shall use this function to obtain an extension $h$ of $f$ from $X\to(-1,1)$. Let $D = g^{-1}(\{-1\})\cup g^{-1}(\{1\})$. Since $G$ is continuous, $D$ is closed in $X$ and must be disjoint from $A$. Then, using Urysohn's Lemma, there is a function $\phi:X\to[0,1]$ such that $\phi(A) = \{1\}$ and $\phi(D) = \{0\}$. Then, the function $h(x) = \phi(x)\cdot g(x)$ is a continuous function from $X$ to $(-1,1)$ that agrees with $f$ on $A$. This completes the proof.
    \end{enumerate}
\end{proof}