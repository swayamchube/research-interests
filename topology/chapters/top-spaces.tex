\newcommand{\T}{\mathcal{T}}
\newcommand{\B}{\mathcal{B}}
\renewcommand{\S}{\mathcal{S}}
\newcommand{\Int}{\operatorname{Int}}
\newcommand{\e}{\epsilon}
\newcommand{\diam}{\operatorname{diam}}
\begin{definition}[Topology, Topological Space]
    A topology on a set $X$ is a collection $\T$ of subsets of $X$ having the following properties:
    \begin{itemize}
        \item $\emptyset$ and $X$ are in $\T$ 
        \item The union of the elements of any subcollection of $\T$ is in $\T$ 
        \item The intersection of the elements of any finite subcollection of $\T$ is in $\T$
    \end{itemize}
    A set $X$ for which a topology $\T$ has been specified is called a \textit{topological space}.
\end{definition}

\begin{definition}[Open Set]
    Let $X$ be a topological space with associated topology $\T$. A subset $U$ of $X$ is said to be open if it is an element of $\T$.
\end{definition}
This immediately implies that both $\emptyset$ and $X$ are open. In fact, we shall see that they are also closed. The topology $\T$ of all subsets of $X$ is called the \textbf{discrete topology} while the topology $\T = \{\emptyset, X\}$ is called the \textbf{indiscrete topology} or the \textbf{trivial topology}.

\begin{definition}
    Let $X$ be a set and $\T, \T'$ be two topologies defined on $X$. If $\T'\supseteq\T$, we say that $\T'$ is \textit{finer} than $\T$. Further, if $\T'\subsetneq\T$, then $\T'$ is said to be \textit{strictly finer} than $\T$.
\end{definition}

\begin{definition}[Basis]
    If $X$ is a set, a \textit{basis} for a topology on $X$ is a collection $\B$ of subsets of $X$ (called \textit{basis elements}) such that 
    \begin{itemize}
        \item For each $x\in X$, there is at least one basis element $B$ containing $x$
        \item If $x$ belongs to the intersection of two basis elements $B_1$ and $B_2$, then there is a basis element $B_3$ containing $x$ such that $B_3\subseteq B_1\cap B_2$.
    \end{itemize}
\end{definition}

\begin{definition}[Generated Topology]
    Let $\B$ be a basis for a topology on $X$. The \textit{topology generated by} $\B$ is defined as follows: A subset $U$ of $X$ is said to be open in $X$ if for each $x\in U$, there is $B\in\B$ such that $x\in B\subseteq U$.
\end{definition}

\begin{proposition}
    The collection $\T$ generated by a basis $\B$ is indeed a topology on $X$.
\end{proposition}
\begin{proof}
    Obviously $\emptyset, X\in\T$. Suppose $\{U_\alpha\}$ is a $J$ indexed collection of sets in $\T$. Let $U = \bigcup_{\alpha\in J}U_\alpha$. Then, for each $x\in U$, there is an $\alpha\in J$ such that $x\in U_\alpha$ and thus, there is $B\in\B$ such that $x\in B\subseteq U_\alpha\subseteq U$ and thus $U\in\T$. Let $U_1,U_2\in\T$ and $x\in U_1\cap U_2$. Then, there exist $B_1,B_2\in\B$ such that $x\in B_1\subseteq U_1$ and $x\in B_2\subseteq U_2$ and thus, $x\in B_1\cap B_2\subseteq U_1\cap U_2$. But, by definition, there exists $B_3\in\B$ such that $x\in B_3\subseteq B_1\cap B_2\subseteq U_1\cap U_2$ and consequently $U_1\cap U_2\in\T$. This finishes the proof.
\end{proof}

\begin{lemma}
    Let $X$ be a set and $\B$ be a basis for a topology $\T$ on $X$. Then $\T$ equals the collection of all unions of elements of $\B$.
\end{lemma}
\begin{proof}
    Trivially note that all elements of $\B$ must be in $\T$ and thus, their unions too. Conversely, let $U\in\T$, then for all $x\in U$, there is $B_x\in\B$ such that $x\in B_x\subseteq U$. It is not hard to see that $U = \bigcup_{x\in U}B_x$ and we have the desired conclusion.
\end{proof}

\begin{lemma}
    Let $X$ be a topological space. Suppose $\mathcal{C}$ is a collection of open sets of $X$ such that for each open set $U$ of $X$ and each $x\in U$, there is an element $C$ of $\mathcal{C}$ such that $x\in C\subseteq U$. Then $\mathcal{C}$ is a basis for the topology of $X$.
\end{lemma}
\begin{proof}
    We first show that $\B$ is a basis. Since $X$ is an open set, for each $x\in X$, there is $C\in\mathcal{C}$ such that $x\in C$. Let $C_1,C_2\in\mathcal{C}$. Since both $C_1$ and $C_2$ are given to be open, so is their intersection. Thus, for each $x\in C_1\cap C_2$, there is $C\in\mathcal{C}$ such that $x\in C\subseteq C_1\cap C_2$. Therefore, $\B$ is a basis.

    Let $\T'$ be the topology generated by $\mathcal{C}$ and $\T$ be the topology associated with $X$. Let $U\in\T$, then for each $x\in U$, there is $C\in\mathcal{C}$ such that $x\in C\subseteq U$, and thus $U\in\T'$ by definition. Conversely, let $W\in\T'$. Since $W$ can be written as a union of a collection of sets in $\mathcal{C}$, all of which are open, $W$ must be open too and thus $W\in\T$. This finishes the proof.
\end{proof}

\begin{lemma}
    Let $\B$ and $\B'$ be bases for the topologies $\T$ and $\T'$, respectively, on $X$. Then, the following are equivalent:
    \begin{itemize}
        \item $\T'$ is finer than $\T$
        \item For each $x\in X$ and each basis element $B\in\B$ containing $x$, there is a basis element $B'\in\B'$ such that $x\in B'\subseteq B$
    \end{itemize}
\end{lemma}
\begin{proof}
    Suppose $\T'$ is finer than $\T$. Then $B\in\T$ and thus $B\in\T'$. As a result, there is, by definition $B'\in\T'$ such that $x\in B'\subseteq B$.

    Conversely, let $U\in\T$. Since $\B$ generates $\T$, for each $x\in U$, there is an element $B\in\B$ such that $x\in B\subseteq U$. But due to the second condition, there is an element $B'\in\B'$ such that $x\in B'\subseteq U$, implying that $U$ is in the topology generated by $\B'$, that is $\T'$. This finishes the proof.
\end{proof}

\begin{definition}[Subbasis]
    A \textit{subbasis} $\S$ for a topology on $X$ is a collection of subsets of $X$ whose union equals $X$. The topology generated by the subbasis $\S$ is defined to be the collection $\T$ of all unions of finite intersections of elements of $S$.
\end{definition}

\begin{proposition}
    The topology generated by $\S$ is indeed a topology.
\end{proposition}
\begin{proof}
    For this, it suffices to show that the set $\B$ of all finite intersections of elements of $S$ forms a basis. Since the union of all elements of $\S$ equals $X$, for each $x\in X$, there is $S\in\S$ such that $x\in S$ and note that $S$ must be an element of $\B$. Finally, since the intersection of any two elements of $\B$ can trivially be written as a finite intersection of elements of $\S$, it must be an element of $\B$ and we are done.
\end{proof}

\begin{definition}[Order Topology]
    Let $X$ be a set with a simple order relation an dassume $X$ has more than one element. Let $\B$ be the collection of all sets of the following types: 
    \begin{enumerate}
        \item All open intervals $(a,b)$ in $X$
        \item All intervals of the form $[a_0,b)$ where $a_0$ is the smalest element (if any) of $X$ 
        \item All intervals of the form $(a,b_0]$ where $b_0$ is the largest element (if any) of $X$
    \end{enumerate}
    The collection $\B$ is a basis for a topology on $X$ which is called the \textit{order topology}.
\end{definition}

\begin{definition}[Product Topology]
    Let $X$ and $Y$ be topological spaces. The \textit{product topology} on $X\times Y$ is the topology having as basis the collection $\B$ of all sets of the form $U\times V$ where $U$ and $V$ are open sets in $X$ and $Y$ respectively.
\end{definition}

\begin{proposition}
    The collection $\B$ is indeed a basis.
\end{proposition}
\begin{proof}
    The first condition is trivially satisfied since $X\times Y\in\B$. Suppose $x\in (U_1\times V_1)\cap(U_2\times V_2) = (U_1\cap U_2)\times(V_1\cap V_2) = U_3\times V_3$ for some open sets $U_3$ and $V_3$ in $X$ and $Y$ respectively. This finishes the proof.
\end{proof}

It is important to note here that \textit{every} open set in $X\times Y$ need not be of the form $U\times V$ where $U$ is open in $X$ and $V$ is open in $Y$. For a counterexample, consider $\R^2$ equipped with the standard topology. The unit ball $x^2 + y^2 < 1$ is open in $\R^2$ but cannot be expressed in the form $U\times V$.

\begin{proposition}
    If $\B$ is a basis for the topology of $X$ and $\mathcal{C}$ is a basis for the topology of $Y$, then the collection 
    \begin{equation*}
        \mathcal{D} = \{B\times C\mid B\in\B,~C\in\mathcal{C}\}
    \end{equation*}
    is a basis for the product topology on $X\times Y$.
\end{proposition}
\begin{proof}
    Let $W$ be an open set in $X\times Y$ and $(x,y)\in W$. Then, by definition, there is $B\in\B$ and $C\in\mathcal{C}$ such that $(x,y)\in B\times C\subseteq W$ and we are done due to a preceeding lemma.
\end{proof}

\begin{definition}
    Let $\pi_1:X\times Y\to X$ be defined by the equation $\pi_1(x,y) = x$ and let $\pi_2:X\times Y\to Y$ be defined by the equation $\pi_2(x,y) = y$. The maps $\pi_1$ and $\pi_2$ are called the \textit{projections} of $X\times Y$ onto its first and second factors, respectively.
\end{definition}

Then, by definition if $U$ is an open subset of $X$, then $\pi_1^{-1}(U) = U\times Y$ and similarly, if $V$ is an open subset of $Y$, then $\pi_2^{-1}(V) = X\times V$.

\begin{proposition}
    The collection 
    \begin{equation*}
        \S = \{\pi_1^{-1}(U)\mid \text{$U$ is open in $X$}\}\cup\{\pi_2^{-1}(V)\mid\text{$V$ is open in $Y$}\}
    \end{equation*}
    is a subbasis for the product topology on $X\times Y$.
\end{proposition}
\begin{proof}
    Since $X\times Y\in\S$, the union of all elements of $\S$ is $X\times Y$ and thus $\S$ is a subbasis. Let $\B$ be the basis generated by all finite intersections of $S$. It suffices to show that $\B = \{U\times V\mid\text{$U$ is open in $X$ and $V$ is open in $Y$}\}$. For any $U$ and $V$ open in $X$ and $Y$ respectively, we may write $U\times V = (U\times Y)\cap(X\times V)$ and is therefore a member of $\B$. Conversely, the finite intersection of elements of $S$ is of the form $(U_1\cap\ldots\cap U_m)\times(V_1\cap\ldots\cap V_m)$, which is a product of two open sets and is an element of $\B$, which finishes the proof.
\end{proof}

\begin{definition}
    Let $X$ be a topological space with topology $\T$. If $Y$ is a subset of $X$, the collection 
    \begin{equation*}
        \T_Y = \{Y\cap U\mid U\in\T\}
    \end{equation*}
    is a topology on $Y$, called the \textit{subspace topology}. With this topology, the topological space $Y$ is called a \textit{subspace} of $X$. Its open sets consist of all intersections of open sets of $X$ with $Y$.
\end{definition}

\begin{proposition}
    $\T_Y$ is a topology on $Y$.
\end{proposition}
\begin{proof}
    Since $\emptyset\in\T$, $\emptyset = Y\cap\emptyset\in\T_Y$ and since $X\in\T$, $Y = Y\cap X\in\T_Y$. Further, 
    \begin{equation*}
        \bigcup_{\alpha\in J}(U_\alpha\cap Y) = Y\cap\bigcup_{\alpha\in J}U_\alpha\in\T_Y
    \end{equation*}
    And finally, $(Y\cap U_1)\cap(Y\cap U_2) = Y\cap(U_1\cap U_2)\in\T_Y$. This finishes the proof.
\end{proof}

\begin{lemma}
    If $\B$ is a basis for the topology of $X$ and $Y\subseteq X$. Then the collection 
    \begin{equation*}
        \B_Y = \{B\cap Y\mid B\in\B\}
    \end{equation*}
    is a basis for the subspace topology on $Y$.
\end{lemma}
\begin{proof}
    Let $V$ be an open set in $Y$. Then, there is $U$ in $X$ such that $V = U\cap Y$. Since each $x\in V$ is an element of $U$, there is, by definition $B\in\B$ such that $x\in B\subseteq U$, consequently, $x\in B\cap Y\subseteq V$ and we are done due to a preceeding lemma.
\end{proof}

\begin{proposition}
    Let $Y$ be a subspace of $X$. If $U$ is open in $Y$ and $Y$ is open in $X$, then $U$ is open in $X$.
\end{proposition}
\begin{proof}
    Follows from the fact that $U = V\cap Y$ for some $V$ that is open in $X$.
\end{proof}

\section{Closed Sets and Limit Points}
\begin{definition}[Closed Set]
    A subset $A$ of a topological space $X$ is said to be \textit{closed} if the set $X\backslash A$ is open.
\end{definition}

\begin{theorem}
    Let $X$ be a topological space. Then the following conditions hold:
    \begin{enumerate}
        \item $\emptyset$ and $X$ are closed 
        \item Arbitrary intersections of closed sets are closed 
        \item Finite unions of closed sets are closed
    \end{enumerate}
\end{theorem}
\begin{proof}
    All follow from De Morgan's laws.
\end{proof}

\begin{proposition}
    Let $Y$ be a subspace of $X$. Then a set $A$ is closed in $Y$ if and only if it equals the intersection of a closed set of $X$ with $Y$.
\end{proposition}
\begin{proof}
    If $A$ is closed in $Y$ then $Y\backslash A$ is open and thus, there is an open set $B$ in $X$ such that $Y\backslash A = Y\cap B$. Then,
    \begin{equation*}
        A = Y\backslash(Y\cap B) = Y\cap(X\backslash B)
    \end{equation*}
    which finishes the proof.
\end{proof}

\begin{corollary}
    Let $Y$ be a subspace of $X$. If $A$ is closed in $Y$ and $Y$ is closed in $X$, then $A$ is closed in $X$.
\end{corollary}
\begin{proof}
    Trivial.
\end{proof}

\begin{definition}[Interior, Closure]
    Let $X$ be a topological space and $A\subseteq X$. The \textit{interior} of $A$ is defined as the union of all open sets contained in $A$ and the \textit{closure} of $A$ is defined as the intersection of all closed sets containing $A$. The interior of $A$ is denoted by $\Int A$ and the closure of $A$ is denoted by $\overline{A}$.
\end{definition}

Then, by definition, we have that 
\begin{equation*}
    \Int A\subseteq A\subseteq\overline{A}
\end{equation*}

\begin{theorem}
    Let $Y$ be a subspace of $X$ and $A$ be a subset of $Y$. Let $\overline{A}$ denote the closure of $A$ in $X$. Then, the closure of $A$ in $Y$ is given by $\overline{A}\cap Y$.
\end{theorem}
\begin{proof}
    Let $\mathcal{F}$ be the collection of all closed sets in $X$ containing $A$. Then, by a preceeding theorem, we know that the set of all closed sets in $Y$ containing $A$ is given by $Y\cap\mathcal{F}$. And thus, 
    \begin{equation*}
        \bigcup_{C\in Y\cap\mathcal{F}} C = Y\cap\bigcup_{C\in\mathcal{F}}C = Y\cap\overline{A}
    \end{equation*}
    This finishes the proof.
\end{proof}

\begin{theorem}
    Let $A$ be a subset of the topological space $X$.
    \begin{itemize}
        \item Then $x\in\overline{A}$ if and only if every open set $U$ containing $x$ intersects $A$ 
        \item Supposing the topology of $X$ is given by a basis, then $x\in\overline{A}$ if and only if every basis element $B$ containing $x$ intersects $A$
    \end{itemize}
\end{theorem}
\begin{proof}
    \hfill 
    \begin{itemize}
        \item Suppose $x\in\overline{A}$ and $U$ be an open set containing $x$. Suppose for the sake of contradiction, there is an open set $U$ in $X$ that contains $x$ but does not intersect $A$, in which case $X\backslash U$ is a closed set containing $A$ and not containing $x$. By definition, since $\overline{A}\subseteq X\backslash U$, $x$ may not be an element of $\overline{A}$, a contradiction. Conversely, suppose every open set $U$ containing $x$ intersects $A$ and that $x\notin\overline{A}$. But then, the set $X\backslash\overline{A}$ is open and contains $x$ but does not intersect $A$, a contradiction.
        \item Suppose $x\in\overline{A}$, then every open set containing $x$ intersects $A$. Since all elements of $\B$ are open, they intersect $A$. Conversely, since every open set $U$ containing $x$ has a basis subset $B$ that contains $x$ and therefore intersects $A$, $U$ must intersect $A$. This finishes the proof.
    \end{itemize}
\end{proof}

The stetement ``$U$ is an open set containing $x$'' is often shortened to ``$U$ is a \textbf{neighborhood} of $x$''.

\begin{definition}
    If $A$ is a subset of the topological space $X$ and if $x\in X$, we say that $x$ is a \textit{limit point} or \textit{cluster point} or \textit{accumulation point} of $A$ if every neighborhood of $x$ intersects $A$ in some point other than $x$ itself.
\end{definition}
For example every element of $\R$ is a limit point of $\Q$.

\begin{theorem}
    Let $A$ be a subset of the topological space $X$ and let $A'$ be the set of all limit points of $A$. Then 
    \begin{equation*}
        \overline{A} = A\cup A'
    \end{equation*}
\end{theorem}
\begin{proof}
    If $x\in A'$, due to the preceeding theorem, $x\in\overline{A}$ but since by definition, $A\subseteq\overline{A}$, we have that $A\cup A'\subseteq\overline{A}$.

    Conversely let $x\in\overline{A}$. If $x\in A$, we are done. If not, then $x$ is such that every open set containing $x$ intersects $A$. But since $x\notin A$, the intersection must contain at least one point distinct from $x$, implying that $x\in A'$. This finishes the proof.
\end{proof}
\begin{corollary}
    A subset of a topological space is closed if and only if it contains all its limit points.
\end{corollary}

\begin{definition}[Hausdorff Spaces]
    A topological space $X$ is called a \textit{Hausdorff space} if for each pair $x_1$ and $x_2$ of distinct points of $X$, there exist neighborhoods $U_1$ and $U_2$ of $x_1$ and $x_2$ respectively that are disjoint.
\end{definition}

\begin{theorem}
    Every finite point set in a Hausdorff space $X$ is closed.
\end{theorem}
\begin{proof}
    It suffices to show this for a single point set, say $\{x_0\}$. For any $x\in X$ different from $x_0$, there are open sets $U$ and $V$ such that $x_0\in U$ and $x\in V$ and $U\cap V=\emptyset$. And thus, $x$ may not be in the closure of $\{x_0\}$. This finishes the proof.
\end{proof}

The condition that finite point sets be closed has been given its own name, the \textbf{$T_1$ axiom}.

\begin{theorem}
    Let $X$ be a space satisfying the $T_1$ axiom and $A\subseteq X$. Then the point $x$ is a limit point of $A$ if and only if every neighborhood of $x$ contains infinitely many points of $A$.
\end{theorem}
\begin{proof}
    If every neighborhood of $x$ intersects $A$ at infinitely many points, then it intersects it in at least one point other than $x$ and thus $x\in A'$.
    
    Conversely, suppose $x$ is a limit point of $x$ but there is a neighborhood $U$ of $x$ that intersecs $A$ in only finitely many points. Let $U\cap(A\backslash\{x\}) = \{x_1,\ldots,x_m\}$. Then, the open set $U\cap(X\backslash\{x_1,\ldots,x_m\})$ contains $x$ but does not intersect $A$, which is contradictory to the fact that $x$ is a limit point of $A$.
\end{proof}

\begin{theorem}
    If $X$ is a Hausdorff space, then a sequence of points of $X$ convertes to at most one point of $X$.
\end{theorem}
\begin{proof}
    Suppose the sequence $\{x_n\}$ converges to two distinct points $x$ and $y$. Then, by definition, there exist disjoint neighborhoods $U$ and $V$ of $x$ and $y$ respectively. Since $x_n$ converges to $x$, $U$ contains all but finitely many elements of the sequence but that means $V$ cannot, a contradiction.
\end{proof}


\section{Continuous Functions}
\begin{definition}
    Let $X$ and $Y$ be topological spaces. A function $f:X\to Y$ is said to be continuous if for each open subset $V$ of $Y$, the set $f^{-1}(V)$ is open in $X$.
\end{definition}

We note here that it suffices to check the above condition for just elements of either a \textit{basis} or a \textit{subbasis}.

\begin{theorem}
    Let $X$ and $Y$ be topological spaces; let $f:X\to Y$. Then the following are equivalent 
    \begin{enumerate}
        \item $f$ is continuous 
        \item for every subset $A$ of $X$, one has $f(\overline{A})\subseteq\overline{f(A)}$ 
        \item for every closed set $B$ of $Y$, the set $f^{-1}(B)$ is closed in $X$ 
        \item for each $x\in X$ and each neighborhood $V$ of $f(x)$, there is a neighborhood $U$ of $x$ such that $f(U)\subseteq V$
    \end{enumerate}
\end{theorem}
\begin{proof}
    $(1)\Rightarrow(2)$. Let $x\in\overline{A}$ and $V$ be an open set containing $f(x)$. We know by definition that $f^{-1}(V)$ is open and therefore intersects $A$. As a consequence, $V$ intersects $f(A)$, implying that $f(x)\in\overline{f(A)}$.

    $(2)\Rightarrow(3)$. Let $A = f^{-1}(B)$. Let $x\in\overline{A}$. Then, 
    \begin{equation*}
        f(x)\in f(\overline{A})\subseteq\overline{f(A)}\subseteq\overline{B} = B
    \end{equation*}
    and thus $x\in f^{-1}(B) = A$, implying that $A\subseteq\overline{A}\subseteq A$, finishing the proof.

    $(3)\Rightarrow(1)$. Let $V$ be an open set in $Y$ and let $U = f^{-1}(V)$. Since $Y\backslash V$ is closed, so is $f^{-1}(Y\backslash V) = f^{-1}(Y)\backslash U = X\backslash U$. Then, by definition, $U$ must be open.

    $(1)\Leftrightarrow(4)$. The forward direction is trivial. Conversely, let $V$ be an open set in $Y$ and $U = f^{-1}(V)$. For each $x\in U$, there is an open set $U_x$ such that $U_x\subseteq U$. Then, $U = \bigcup_{x\in U}U_x$ is open. This finishes the proof.
\end{proof}

\begin{definition}[Homeomorphism]
    Let $X$ and $Y$ be topological spaces; let $f:X\to Y$ be a bijection. If both the function $f$ and the inverse function $f^{-1}:Y\to X$ are continuous, then $f$ is a \textit{homeomorphism}.
\end{definition}

As a result, any property of $X$ that is entirely expressed in terms of the topology of $X$ yields, via the correspondence $f$, the corresponding property for the space $Y$. Such a property of $X$ is called a \textbf{topological property}.

If $f:X\to Y$ is an injective, continuous map, where $X$ and $Y$ are topological spaces. Let $Z$ be the image set $f(X)$, considered as a subspace of $Y$; then the function $f':X\to Z$ obtained by restricting the range of $f$ is bijective. If $f'$ happens to be a homeomorphism of $X$ with $Z$, we say that the map $f:X\to Y$ is a \textbf{topological imbedding} or simply an \textbf{imbedding} of $X$ in $Y$.

\begin{theorem}
    Let $X$, $Y$ and $Z$ be topological spaces 
    \begin{enumerate}
        \item (Constant) If $f:X\to Y$ maps all of $X$ to a single point of $Y$, then it is continuous 
        \item (Inclusion) If $A$ is a subspace of $X$, the inclusion function $j:A\to X$ is continuous 
        \item (Composites) If $f:X\to Y$ and $g:Y\to Z$ are ocontinuous, then the map $g\circ f: X\to Z$ is continuous
        \item (Domain Restriction) If $f:X\to Y$ is continuous, and if $A$ is a subspace of $X$, then the restricted function $f|_A:A\to Y$ is continuous.
        \item (Range Restriction/Expansion) Let $f:X\to Y$ be continuous. If $Z$ is a subspace of $Y$ containing the image set $f(X)$, then the function $g:X\to Z$ obtained by restricting the range of $f$ is continuous. If $Z$ is a space having $Y$ ias a subspace, then the function $h:X\to Z$ obtained by expanding the range of $f$ is continuous.
        \item (Local formulation of continuity) The map $f:X\to Y$ is continuous if $X$ can be written as the union of open sets $\{U_\alpha\}$ such that $f|_{U_\alpha}$ is continuous for each $\alpha$.
    \end{enumerate}
\end{theorem}
\begin{proof}
    \hfill 
    \begin{enumerate}
        \item Trivial
        \item Trivial
        \item Let $V$ be an open set in $Z$. Then, $g^{-1}(V)$ is open in $Y$ and $f^{-1}\circ g^{-1}(V)$ is open in $X$ and thus $g\circ f$ is continuous
        \item Notice that $f|_A\equiv f\circ j$
        \item Let $V$ be an open set in $Z$. Then, there is an open set $W$ in $Y$ such that $V = Z\cap W$. Since the range of $f$ is a subset of $Z$, we have
        \begin{equation*}
            g^{-1}(V) = g^{-1}(Z\cap W) = f^{-1}(Z\cap W) = f^{-1}(W)
        \end{equation*}
        which is open in $X$ and thus, $g$ is continuous. A similar argument can be applied in the second case.
        \item Let $V$ be an open set in $Y$, then we may write 
        \begin{equation*}
            f^{-1}(V) = \bigcup_{\alpha}f|_{U_\alpha}^{-1}(V\cup U_\alpha)
        \end{equation*}
        which is a union of a collection of open sets and is therefore open. This finishes the proof.
    \end{enumerate}
\end{proof}

\begin{lemma}[Pasting Lemma]
    Let $X = A\cup B$ where $A$ and $B$ are closed in $X$. Let $f:A\to Y$ and $g:B\to Y$ be continuous If $f(x) = g(x)$ for every $x\in A\cap B$ then $f$ and $g$ combine to give a continuous function $h:X\to Y$ defined as 
    \begin{equation*}
        h(x) =
        \begin{cases}
            f(x) & x\in A\\
            g(x) & x\in B
        \end{cases}
    \end{equation*}
\end{lemma}
\begin{proof}
    Let $C$ be a closed subset of $Y$. We then have $h^{-1}(C) = f^{-1}(C)\cup g^{-1}(C)$. Since $f$ is continuous, we know that $f^{-1}(C)$ is closed in $A$ and therefore in $X$ similarly, so is $g^{-1}(C)$, which finishes the proof.
\end{proof}

\begin{theorem}
    Let $f:A\to X\times Y$ be given by the equation $f(a) = (f_1(a), f_2(a))$ then $f$ is continuous if and only if the functions $f_1:A\to X$ and $f_2:A\to Y$ are continuous. The maps $f_1$ and $f_2$ are called the \textit{coordinate maps} of $f$.
\end{theorem}
\begin{proof}
    We know that the projection maps $\pi_1, \pi_2$ are continuous. We note that $f_1(a) = \pi_1(f(a))$ and $f_2(a) = \pi_2(f_2(a))$. If $f$ is continuous, then so are $f_1$ and $f_2$.

    Conversely, suppose $f_1$ and $f_2$ are continuous and $U\times V$ be a basis element for the product topology on $X\times Y$. We know due to a preceeding result that both $U$ and $V$ are open in $X$ and $Y$ respectively. Then 
    \begin{equation*}
        f^{-1}(U\times V) = f_1^{-1}(U)\cap f_2^{-1}(V)
    \end{equation*}
    which is an intersection of two open sets and is therefore open.
\end{proof}

\section{Product Topology}
\begin{definition}
    Let $J$ be an index set. Given a set $X$, we define a $J$-tuple of elements of $X$ to be a function $x:J\to X$. If $\alpha$ is an element of $J$, we often denote the value of $x$ at $\alpha$ by $x_\alpha$ rather than $x(\alpha)$ and call it the $\alpha$-th coordinate of $x$. We often denote the function $x$ itself by the symbol
    \begin{equation*}
        (x_\alpha)_{\alpha\in J}
    \end{equation*}
\end{definition}

\begin{definition}[Cartesian Product]
    Let $\{A_\alpha\}_{\alpha\in J}$ be an indexed family of sets and let $X = \bigcup_{\alpha\in J}A_\alpha$. The cartesian product of this indexed family, denoted by 
    \begin{equation*}
        \prod_{\alpha\in J}A_\alpha
    \end{equation*}
    is defined to be the set of all $J$-tuples $x$ of elements of $X$ such that $x_\alpha\in A_\alpha$ for each $\alpha\in J$. That is, the set of all functions 
    \begin{equation*}
        x: J\to\bigcup_{\alpha\in J}A_\alpha
    \end{equation*}
    such that $x(\alpha)\in A_\alpha$ for each $\alpha\in J$.
\end{definition}

\begin{definition}[Box Topology]
    Let $\{X_\alpha\}_{\alpha\in J}$ be an indexed family of topological spaces. Let us take as a basis for a topology on the product space 
    \begin{equation*}
        \prod_{\alpha\in J}X_\alpha
    \end{equation*}
    the collection of all sets of the form 
    \begin{equation*}
        \prod_{\alpha\in J}U_\alpha
    \end{equation*}
    where $U_\alpha$ is open in $X_\alpha$ for each $\alpha\in J$. The topology generated by this basis is called the \textit{box topology}.
\end{definition}

\begin{definition}[Product Topology]
    Let $\S_\beta$ denote the collection 
    \begin{equation*}
        S_\beta = \{\pi_\beta^{-1}(U_\beta)\mid \text{$U_\beta$ open in $X_\beta$}\}
    \end{equation*}
    and let $\S$ denote the union of these collections 
    \begin{equation*}
        \S = \bigcup_{\beta\in J}\S_\beta
    \end{equation*}
    The topology generated by the subbasis $\S$ is called the \textit{product topology}. In this topology $\prod_{\alpha\in J}X_\alpha$ is called a \textit{product space}.
\end{definition}

It is not hard to see that $\S$ is indeed a subbasis and therefore defines a topology. Let $\B$ be the basis induced by $\S$. Then, any basis element is a finite intersection of elements of $S$ and eventually would have the form 
\begin{equation*}
    B = \bigcap_{i=1}^n\pi_{\beta_i}^{-1}(U_{\beta_i})
\end{equation*}

It is then obvious that the \textit{box topology} is finer than the \textit{product topology} since it has more open sets. In the case of finite products of topological spaces, obviously the two of them are equal, but this is not the case for infinite products of topological spaces, since the basis of the product topology are only finite intersections of the subbasis, implying that for any basis element of the form $B = \prod_{\alpha\in J}U_\alpha$, there exist infinitely many $\alpha\in J$ such that $U_\alpha$ is the entire space $X_\alpha$ and is therefore strictly coarser than the box topology.

As a rule of thumb:
\begin{quotation}
    Whenever we consider the product $\prod_{\alpha\in J}X_\alpha$, we shall assume it is given the product topology unless we specifically state otherwise.
\end{quotation}

\begin{theorem}
    Suppose the topology on each space $X_\alpha$ is given by a basis $\B_\alpha$. The collection of all sets of the form 
    \begin{equation*}
        \prod_{\alpha\in J}B_\alpha
    \end{equation*}
    where $B_\alpha\in\B_\alpha$ for each $\alpha$ will serve as a basis for the box topology on $\prod_{\alpha\in J}X_\alpha$. The collection of all sets of the same form where $B_\alpha\in\B_\alpha$ for finitely many indices $\alpha$ and $B_\alpha = X_\alpha$ for all the remaining indices will serve as a basis for the product topology $\prod_{\alpha\in J}X_\alpha$.
\end{theorem}
\begin{proof}
    Straightforward.
\end{proof}

\begin{theorem}
    If each space $X_\alpha$ is a Hausdorff space, then $\prod X_\alpha$ is a Hausdorff space in both the box and product topologies.
\end{theorem}

\begin{theorem}
    Let $\{X_\alpha\}$ be an indexed family of spaces and $A_\alpha\subseteq X_\alpha$ for each $\alpha$. If $\prod X_\alpha$ is given by either the product or box topology, then 
    \begin{equation*}
        \prod\overline{A}_\alpha = \overline{\prod A_\alpha}
    \end{equation*}
\end{theorem}
\begin{proof}
    Let $x = (x_\alpha)$ be a point of $\prod\overline{A}_\alpha$ and let $U = \prod U_\alpha$ be a basis element for either the box or product topology that contains $x$. Since $x_\alpha\in\overline{A}_\alpha$, we know that there is $y_\alpha\in U_\alpha\cap A_\alpha$ and thus $y = (y_\alpha)\in U\cap\prod A_\alpha$.

    Conversely, suppose $x = (x_\alpha)\in\overline{\prod A_\alpha}$, and let $V_\alpha$ be an arbitrary open set in $X_\alpha$ containing $x_\alpha$. Since $\pi_\alpha^{-1}(V_\alpha)$ is an open set containing $x$, it must intersect $\prod A_\alpha$, thus, there is $y = (y_\alpha)\in\pi_\alpha^{-1}(V_\alpha)\cap\prod A_\alpha$, consequently, $y_\alpha\in V_\alpha\cap A_\alpha$, and it follows that $x_\alpha\in\overline{A_\alpha}$. This completes the proof.
\end{proof}

\begin{theorem}
    Let $f:A\to\prod_{\alpha\in J}X_\alpha$ be given by the equation 
    \begin{equation*}
        f(a) = (f_\alpha(a))_{\alpha\in J}
    \end{equation*}
    where $f_\alpha:A\to X_\alpha$ for each $\alpha$. Let $\prod X_\alpha$ have the product topology. Then the function $f$ is continuous if and only if each coordinate function $f_\alpha$ is continuous.
\end{theorem}
\begin{proof}
    First, suppose $f$ is continuous. Let $U_\beta$ be open in $X_\beta$. The function $\pi^{-1}_\beta$ maps $U_\beta$ to an open set in $\prod X_\alpha$ and is therefore continuous. As a result, $f_\beta = \pi_\beta\circ f$ is continuous.

    Conversely, suppose each coordinate function $f_\alpha$ is continuous. We remarked earlier that $\pi_\beta^{-1}(U_\beta)$ for some open set $U_\beta$ is a subbasis for the product topology and it suffices to show that the inverse image under $f$ of the same is open to imply continuity. Indeed, 
    \begin{equation*}
        f^{-1}\circ\pi^{-1}_\beta(U_\beta) = f_\beta^{-1}(U_\beta)
    \end{equation*}
    which is obviously open, since $f_\beta$ is known to be continuous. This finishes the proof.
\end{proof}

\begin{mdframed}
    \textbf{\textcolor{red}{Caution.}} It is important to note that the above theorem \textbf{does not} hold for the box topology. As a simple counter example, consider the box topology on $\R^\omega$ and the function $f:\R\to\R^\omega$ given by $f(t)=(t,t,\ldots)$. Suppose $f$ were continuous, then the inverse image of each basis element must be open in $\R^\omega$. Indeed, consider 
    \begin{equation*}
        B = (-1,1)\times(-\frac{1}{2},\frac{1}{2})\times\cdots
    \end{equation*}
    the inverse image would have to contain some open interval $(-\delta,\delta)$ in the standard topology of $\R$, that is, $(-\delta,\delta)\subseteq f^{-1}(B)$, or equivalently, $f((-\delta,\delta))\subseteq B$, which is absurd.
\end{mdframed}

\section{Metric Topology}
\begin{definition}[Metric]
    A \textit{metric} on a set $X$ is a function $d:X\times X\to\R$ such that 
    \begin{enumerate}
        \item $d(x,y)\ge0$ for all $x,y\in X$; equality holds if and only if $x = y$
        \item $d(x, y) = d(y, x)$ for all $x,y\in X$ 
        \item (Triangle Inequality) $d(x, y) + d(y, z)\ge d(x, z)$ for all $x,y,z\in X$
    \end{enumerate}
\end{definition}

For $\epsilon > 0$, define the set 
\begin{equation*}
    B_d(x,\epsilon) = \{y\mid d(x,y) < \epsilon\}
\end{equation*}

\begin{definition}[Metric Topology]
    If $d$ is a metric on the set $X$, then the collection of all $\epsilon$-balls $B_d(x,\epsilon)$ for $x\in X$ and $\epsilon > 0$ is a basis for a topology on $X$, called the \textit{metric topology} induced by $d$.
\end{definition}

\begin{proposition}
    The collection of all $\epsilon$-balls $B_d(x,\epsilon)$ for all $x\in X$ and $\epsilon > 0$ is a basis.
\end{proposition}
\begin{proof}
    The first condition is trivially satisfied. Suppose $z\in B(x,\e)\cap B(y,\e)$. Let $r = \frac{1}{2}\min\{\e - d(x,z), \e - d(y,z)\}$. It is obvious, due to the triangle inequality, that $B(z,r)\subseteq B(x,\e)\cap B(y,\e)$.
\end{proof}

\begin{definition}[Metrizable]
    If $X$ is a topological space, $X$ is said to be \textit{metrizable} if there exists a metric $d$ on the set $X$ that induces the topology of $X$.
\end{definition}

A \textbf{metric space} is a metrizable space $X$ together with a specific metric $d$ that gives the topology of $X$.

\begin{definition}
    Let $X$ be a metric space with metric $d$. A subset $A$ of $X$ is said to be \textit{bounded} if there is some number $M$ such that $d(a_1,a_2)\le M$ for every pair $a_1,a_2$ of points of $A$. if $A$ is bounded and non-empty, the \textit{diameter} of $A$ is defined to be 
    \begin{equation*}
        \diam(A) = \sup\{d(a_1,a_2)\mid a_1,a_2\in A\}
    \end{equation*}
\end{definition}

\begin{proposition}
    Every metric space is Hausdorff.
\end{proposition}
\begin{proof}
    Trivial.
\end{proof}

\begin{theorem}
    Let $X$ be a metric space with metric $d$. Define $\overline{d}:X\times X\to\R$ by the equation 
    \begin{equation*}
        \overline{d}(x,y) = \min\{d(x,y), 1\}
    \end{equation*}
    Then $\overline{d}$ is a metric that induces the same topology as $d$.
\end{theorem}
\begin{proof}
    We need only check the triangle inequality. This is euqivalent to 
    \begin{equation*}
        \overline{d}(x,y) + \overline{d}(y,z)\ge \overline{d}(x,z)
    \end{equation*}
    Obviously if either one of $\overline{d}(x,y)$ or $\overline{d}(y,z)$ is greater than or equal to $1$, then we are done. If not, then 
    \begin{equation*}
        \overline{d}(x,y) + \overline{d}(y,z) = d(x,y) + d(y,z)\ge\overline{d}(x,z)\ge\min\{d(x,z), 1\}
    \end{equation*}

    Let $\T$ be the topology on $X$ induced by $d$, having basis $\B$. Let $\overline{\B}$ be the set of all balls induced by $\overline{d}$ having radius strictly less than $1$. Let $U$ be an open set in $\T$ and $x\in U$, then, by definition, there is $B_d(x,\e)$ in $\B$ such that $x\in B_d(x,\e)\subseteq U$. The ball $B_{\overline{d}}(x,\frac{1}{2}\min\{\e, 1\})$ is contained in $B_d(x,\e)$ and also contains $x$. Thus, $\overline{\B}$ is a basis for $\T$. This finishes the proof.
\end{proof}

\begin{definition}[Euclidean, square Metric]
    Given $x = (x_1,\ldots,x_n), y = (y_1,\ldots,y_n)\in\R^n$, we define the \textit{Euclidean metric} on $\R^n$ by the equation 
    \begin{equation*}
        d(x,y) = \|x - y\| = \left((x_1 - y_1)^2 + \ldots + (x_n - y_n)^2\right)^{1/2}
    \end{equation*}
    and the \textit{square metric} $\rho$ by the equation 
    \begin{equation*}
        \rho(x,y) = \max\{|x_1 - y_1|,\ldots,|x_n - y_n|\}
    \end{equation*}
\end{definition}

\begin{lemma}
    Let $d$ and $d'$ be two metrics on the set $X$; let $\T$ and $\T'$ be the topologies they induce, respectively. Then $\T'$ is finer than $\T$ if and only if for each $x\in X$ and each $\e > 0$, there exists $\delta > 0$ such that 
    \begin{equation}
        B_{d'}(x,\delta)\subseteq B_d(x,\e)
    \end{equation}
\end{lemma}
\begin{proof}
    Suppose the $\e-\delta$ condition holds. Let $B\in\B_d$ be a basis element for the topology induced by $d$ and let $x$ be an arbitrary element of $B$. Then, we can find $\e$ such that $B_d(x,\e)\subseteq B$ and thus, there exists $\delta$ such that $x\in B_{d'}(x,\delta)\subseteq B$. Taking the union of all such $\delta$-balls for $x$, we have an open set in $\T'$ which corresponds to a basis element for $\T$, and thus $\T'$ is finer than $\T$. 

    Conversely, suppose $\T'$ is finer than $\T$, then the condition is trivially satisfied.
\end{proof}

\begin{theorem}
    The topologies on $\R^n$ induced by the Euclidean metric $d$ and the square metric $\rho$ are the same as the product topology on $\R^n$.
\end{theorem}
\begin{proof}
    We shall first show that the topologies induced by $d$ and $\rho$ on $\R^n$ are identical. Indeed, we have, for any two points $x$ and $y$ that 
    \begin{equation*}
        \rho(x,y)\le d(x,y)\le\sqrt{n}\rho(x,y)
    \end{equation*}
    this immediately implies the conclusion due to the preceeding lemma.

    Finally, we shall show that the topology induced by $\rho$ is same as the product topology. Let $B = (a_1,b_1)\times\cdots\times(a_n,b_n)$ be a basis element of the product topology and let $x\in B$, then for each $i$, there is an $\e_i$ such that $(x-\e_i,x+\e_i)\subseteq(a_i,b_i)$. Choosing $\e = \min\{\e_1,\ldots,\e_n\}$, we have that the topology induced by $\rho$ is finer than the product topology. But since every basis element of the $\rho$-topology is inherently an element of the product topology, since it is a cartesian product of open intervals, it must be that the product topology is finer than the $\rho$-topology. This completes the proof.
\end{proof}

\begin{definition}[Uniform Metric]
    Given an index set $J$ and given points $x = (x_\alpha)_{\alpha\in J}$ and $y = (y_\alpha)_{\alpha\in J}$ of $\R^J$, let us define a metric $\overline{\rho}$ given by 
    \begin{equation*}
        \overline{\rho}(x,y) = \sup\{\overline{d}(x_\alpha, y_\alpha)\mid \alpha\in J\}
    \end{equation*}
    where $\overline{d}$ is the standard bounded metric on $\R$. This is called the \textit{uniform metric} on $\R^J$ and the topology it induces is called the \textit{uniform topology}.
\end{definition}

\begin{theorem}[$\e-\delta$ Theorem]
    Let $f:X\to Y$; let $X$ and $Y$ be metrizable with metrics $d_X$ and $d_Y$ respectively. Then continuity of $f$ is equivalent to the requirement that tiven $x\in X$ and given $\e > 0$, there exists $\delta > 0$ such that 
    \begin{equation*}
        d_X(x,y) <\delta \Longrightarrow d_Y(f(x), f(y)) < \e
    \end{equation*}
\end{theorem}
\begin{proof}
    Suppose $f$ is continuous and let $\e > 0$ be given. Consider the set $f^{-1}(B_Y(f(x), \e))$, which is open in $X$ and contains the point $x$. Therefore, there exists a $\delta$-ball centered at $x$. If $y$ is in this $\delta$-ball, then $f(y)$ is in the $\e$-ball centered at $f(x)$ as desired.

    Conversely, suppose the $\e-\delta$ condition holds and let $V$ be open in $Y$ and $x\in f^{-1}(V)$. But since $f(x)\in V$, there exists $\e$ such that $B_Y(f(x),\e)$ is contained in $V$, consequently, there exists $\delta$ such that $B_X(x, \delta)$ is contained in $f^{-1}(V)$ and thus $f^{-1}(V)$ is open. This finishes the proof.
\end{proof}

\begin{lemma}[Sequence Lemma]
    Let $X$ be a topological space; let $A\subseteq X$. If there is a sequence of points of $A$ converging to $x$, then $x\in\overline{A}$; the converse holds if $X$ is metrizable.
\end{lemma}
\begin{proof}
    Suppose there is a sequence of points of $A$ converging to $x$, then each neighborhood of $x$ contains a points of $A$ and thus, by definition, $x\in\overline{A}$.

    Conversely, suppose $X$ is metrizable and $x\in\overline{A}$. For each $n\in\N$, consider $B_X(x,\frac{1}{n})$ which must contain at least one point of $A$, call it $x_n$. It is not hard to see that this sequence converges to $x$. This finishes the proof.
\end{proof}

\begin{theorem}
    Let $f:X\to Y$. If the function $f$ is continuous, then for every convergent sequence $x_n\to x$, the sequence $f(x_n)$ converges to $f(x)$. The converse holds if $X$ is metrizable.
\end{theorem}
\begin{proof}
\end{proof}