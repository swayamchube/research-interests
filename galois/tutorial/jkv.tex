\documentclass[12pt]{article}

\usepackage[]{amsmath, amsthm, amssymb, xcolor, geometry, hyperref}

\newcommand{\F}{\mathbb{F}}
\newcommand{\R}{\mathbb{R}}
\newcommand{\Q}{\mathbb{Q}}
\newcommand{\Z}{\mathbb{Z}}
\newcommand{\N}{\mathbb{N}}

\newcommand{\setItemnumber}[1]{\setcounter{enumi}{\numexpr#1-1\relax}}

\begin{document}
\begin{enumerate}
    \item The prime subfield of $F$ is isomorphic to $\F_p$. Suppose the dimension of $F$ over $\F_p$ is $n > 0$. Then, there is a basis $\{\alpha_1,\ldots,\alpha_n\}$ where each $\alpha_i\in F$. Any element in $F$ may be written as $\sum_{i = 1}^nc_i\alpha_i$ with $c_i\in\F_p$. This immediately implies the conclusion.

    \item The main idea is as follows. Let $f(x)\in\F_p[x]$ be irreducible and $\alpha$ be a root of $f(x)$ in some extension $E$ of $\F_p$. Consider the field $\F_p(\alpha)$, which has degree $\deg f$ over $\F_p$ and therefore is a field of size $p^{\deg f}$.

    \item 
    \begin{enumerate}
        \item $\underline{1 + i}$: $f(x) = x^2 + 2x + 2$ 
        \item $\underline{2 + \sqrt{3}}$: $g(x) = x^2 - 4x + 1$
        \item $\underline{1 + \sqrt[3]{2} + \sqrt[3]{4}}$: $h(x) = x^3 - 3x^2 - 3x - 1$
    \end{enumerate}

    \setItemnumber{5}
    \item Trivial 
    
    \item $2$ since $\sqrt{3 + 2\sqrt{2}} = 1 + \sqrt{2}$.

    \item Obviously, $F(\alpha^2)\subseteq F(\alpha)$. It suffices to show that $[F(\alpha):F(\alpha^2)] = 1$. Indeed, we have the equality $n = [F(\alpha):F] = [F(\alpha):F(\alpha^2)][F(\alpha^2):F]$. But since $\alpha$ is a root of $m_\alpha(x) = x^2 - \alpha^2\in F(\alpha)[x]$, we must have $[F(\alpha):F(\alpha^2)]\le 2$. It cannot be equal to $2$ since $n$ is odd and therefore must be equal to $1$, giving us the desired conclusion.

    \item Obviously $R$ is an integral domain. It suffices to show that every non-zero element in $R$ has an inverse. Indeed, let $a\in R\backslash\{0\}$. Since $K/F$ is algebraic, $a$ must be algebraic over $F$. Then, there is a monic polynomial $m_a(x) = x^n + a_{n - 1}x^{n - 1} + \cdots + a_0$ with $a_0\ne 0$ such that $a_0\ne 0$ and $m_a(a) = 0$. It is now obvious that the inverse of $a$ must lie in $R$.

    \setItemnumber{11}
    \item Note that since both $\beta$ and $\sqrt[3]{2}$ have the same minimal polynomial $m(x) = x^3 - 2$ and therefore $\Q(\beta)\cong\Q(\sqrt[3]{2})$. The isomorphism maps $-1$ to $-1$. And thus if $-1$ is the sum of squares in $\Q(\beta)$ then it is also a sum of squares in $\Q(\sqrt[3]{2})$ a contradiction. This finishes the proof.

    \item Trivial. Just note the parities of $f(2k)$ and $f(2k + 1)$ for $k\in\Z$.


    \item Suppose $\mathbb{C}\subsetneq R$. Let $a\in R\backslash\mathbb{C}$ and let $R$ have dimension $n$. Then, the elements $1,a,\ldots,a^n$ are not linearly independent and therefore there is a polynomial $f\in\mathbb{C}[x]$ such that $f(a) = 0$ but this is contradictory to the fact that $\mathbb{C}$ is algebraically closed.

    \setItemnumber{29}
    \item Trivial using induction.

    \item 
\end{enumerate}
\end{document}