\documentclass{report}

\usepackage{../swayam}
\usepackage{bm}
\newcommand{\sep}{\text{sep}}
\newcommand{\F}{\mathbb F}
\renewcommand{\qedsymbol}{$\blacksquare$}
\newcommand{\disc}{\operatorname{disc}}

\title{Field and Galois Theory}
\author{Swayam Chube}
\date{\today}

\begin{document}
\maketitle

\begin{abstract}
    This is meant to be a rapid introduction to Galois Theory. We shall not provide intuition or comment far too much on any specific result. The main reference followed while making these notes is \cite{Lan02}
\end{abstract}

\tableofcontents

\chapter{Algebraic Extensions}
\begin{definition}[Extension, Degree]
    Let $F$ be a field. If $F$ is a subfield of another field $E$, then $E$ is said to be an \textit{extension} field of $F$. The dimension of $E$ when viewed as a vector space over $F$ is said to be the \textit{degree of the extension} $E/F$ and is denoted by $[E:F]$.
\end{definition}

\begin{definition}[Algebraic Element]
    
\end{definition}


\begin{definition}[Distinguished Class]
    Let $\mathscr{C}$ be a class of extension fields $F\subseteq E$. We say that $\mathscr{C}$ is distinguished if it satisfies the following conditions: 
    \begin{enumerate}
        \item Let $k\subseteq F\subseteq E$ be a tower of fields. The extension $K\subseteq E$ is in $\mathscr{C}$ if and only if $k\subseteq F$ is in $\mathscr{C}$ and $F\subseteq E$ is in $\mathscr{C}$.

        \item If $k\subseteq E$ is in $\mathscr{C}$, if $F$ is any extension of $k$, and $E$, $F$ are both contained in some field, then $F\subseteq EF$ is in $\mathscr{C}$.

        \item If $k\subseteq F$ and $k\subseteq E$ are in $\mathscr{C}$ and $F$, $E$ are subfields of a common field, then $k\subseteq FE$ is in $\mathscr{C}$.
    \end{enumerate}
\end{definition}

\begin{lemma}\thlabel{lem:self-embedding-is-automorphism}
    Let $E/k$ be algebraic and let $\sigma: E\to E$ be an embedding of $E$ over $k$. Then $\sigma$ is an automorphism.
\end{lemma}
\begin{proof}
    Since $\sigma$ is known to be injective, it suffices to show that it is surjective. Pick some $\alpha\in E$ and let $p(x)\in k[x]$ be its minimal polynomial over $k$. Let $K$ be the subfield of $E$ generated by all the roots of $p$ in $E$. Obviously, $[K:k]$ is finite. Since $p$ remains unchanged under $\sigma$, it is not hard to see that $\sigma$ maps a root of $p$ in $E$ to another root of $p$ in $E$. Therefore, $\sigma(K)\subseteq K$. But since $[\sigma(K):k] = [K:k]$ due to obvious reasons, we musth ave that $\sigma(K) = K$, consequently, $\alpha\in K = \sigma(K)$. This shows surjectivity.
\end{proof} 

\chapter{Algebraic Closure}
\begin{theorem}\thlabel{thm:alg-closed-fields-exist}
    Let $k$ be a field. Then there is an algebraicaly closed field containing $k$.
\end{theorem}
\begin{proof}[Proof due to Artin]
\end{proof}

\begin{corollary}
    Let $k$ be a field. Then there exists an extension $k^a$ which is algebraic over $k$ and algebraically closed.
\end{corollary}
\begin{proof}
    
\end{proof}

\begin{lemma}\thlabel{lem:num-extensions-eq-distinct-roots}
    Let $k$ be a field and $L$ and algebraically closed field with $\sigma: k\to L$ an embedding. Let $\alpha$ be algebraic over $k$ in some extension of $k$. Then, the number of extensions of $\sigma$ to an embedding $k(\alpha)\to L$ is precisely equal to the number of distinct roots of the minimal polynomial of $\alpha$ over $k$.
\end{lemma}

\begin{lemma}
    Suppose $E$ and $L$ are algebraically closed fields with $E\subseteq L$. If $L/E$ is algebraic, then $E = L$.
\end{lemma}
\begin{proof}
    Let $\alpha\in L$. Let $p(x)\in E[x]$ be the minimal polynomial of $\alpha$ over $E$. Since $E$ is algebraically closed, $p$ splits into linear factors over $E$, one of them being $(x - \alpha)$, implying that $\alpha\in E$. This completes the proof.
\end{proof}

\begin{theorem}[Extension Theorem]\thlabel{thm:extension-theorem}
    Let $E/k$ be algebraic, $L$ an algebraically closed field and $\sigma: k\to L$ be an embedding of $k$. Then there exists an extension of $\sigma$ to an embedding of $E$ in $L$. If $E$ is algebraically closed and $L$ is algebraic over $\sigma k$, then any such extension of $\sigma$ is an isomorphism of $E$ onto $L$.
\end{theorem}
\begin{proof}
    Let $\mathscr S$ be the set of all pairs $(F,\tau)$ where $F\subseteq E$ and $F/k$ is algebraic and $\tau: F\to L$ is an extension of $\sigma$. Define a partial order $\leqq$ on $\mathscr S$ by $(F_1,\tau_1)\leqq(F_2,\tau_2)$ if and only if $F_1\subseteq F_2$ and $\tau_2\mid_{F_1}\equiv\tau_1$. Note that $\mathscr S$ is nonempty since it contains $(k,\sigma)$. Let $\mathscr C = \{(F_\alpha,\tau_\alpha)\}$ be a chain in $\mathscr S$. Define $F = \bigcup_{\alpha} F_\alpha$. Now, for any $t\in F$, there is $\beta$ such that $t\in F_\beta$; using this, define $\tau(t) = \tau_\beta(t)$. It is not hard to see that this is a valid embedding.

    Now, invoking Zorn's Lemma, there is a maximal element, say $(K,\tau)$. We claim that $K = E$, for if not, then we may choose some $\alpha\in E$ and invoke \thref{lem:num-extensions-eq-distinct-roots}.

    Finally, if $E$ is algebraically closed, so is $\sigma E$, consequently, we are done due to the preceeding lemma.
\end{proof}

\begin{corollary}
    Let $k$ be a field and $E, E'$ be algebraic extensions of $k$. Assume that $E, E'$ are algebraically closed. Then there exists an isomorphism $\tau: E\to E'$ inducing the identity on $k$.
\end{corollary}
\begin{proof}
    Consider the extension of $\sigma: k\to E'$ where $\sigma\mid_k = \mathbf{id}_k$ whence the conclusion immediately follows.
\end{proof}

Since an algebraically closed and algebraic extension of $k$ is determined upto an isomorphism, we call such an extension an \textit{algebraic closure} of $k$ and is denoted by $k^a$.

\begin{definition}[Conjugates]
    Let $E/k$ be an algebraic extension contained in an algebraic closure $k^a$. Then, the distinct roots of the minimal polynomial of $\alpha$ over $k$ are called the \textit{conjugates} of $\alpha$. In particular, two roots of the same minimal polynomial over $k$ are said to be \textit{conjugate} to one another.
\end{definition}

Here's a nice exercise from \cite{DF04}.

\begin{example}
    A field is said to be \emph{formally real} if $-1$ cannot be expressed as a sum of squares in it. Let $k$ be a formally real field with $k^a$ its algebraic closure. If $\alpha\in k^a$ with odd degree over $k$, then $k[\alpha]$ is also formally real.
\end{example}
\begin{proof}
    Suppose not. Let $\alpha\in k^a$ be such that $k[\alpha]$ is not formally real and $[k[\alpha]:k]$ is minimum, greater than $1$. Then, there are elements $\gamma_1,\dots,\gamma_m\in k[\alpha]$ such that $\sum_{i = 1}^m\gamma_i^2 = -1$. We may choose polynomials $p_i(x)\in k[x]$ such that $p_i(\alpha) = \gamma_i$ with $\deg p_i(\alpha) < [k[\alpha]:k]$.

    Let $f(x)\in k[x]$ be the irreducible polynomial of $\alpha$ over $k$. We have 
    \begin{equation*}
        p_1(\alpha)^2 + \cdots + p_m(\alpha)^2 = -1
    \end{equation*}
    and thus, $\alpha$ is a root of the polynomial $p_1(x)^2 + \cdots + p_m(x)^2 + 1$. Thus, there is a polynomial $g(x)\in k[x]$ such that 
    \begin{equation*}
        p_1(x)^2 + \cdots + p_m(x)^2 + 1 = f(x)g(x).
    \end{equation*}
    Notice that the degree of the left hand side is even and less than $2\deg f$ whence $\deg g < \deg f$ and is odd.

    Further, note that $g(x)$ may not have a root in $k$ lest $-1$ be written as a sum of squares in $k$. Consider now the factorization of $g(x)$ as a product of irreducibles: 
    \begin{equation*}
        g(x) = h_1(x)\cdots h_n(x).
    \end{equation*}
    Equating degrees, we see that there is an index $j$ such that $\deg h_j$ is odd. Let $\beta$ be a root of $h_j$ in $k^a$. Then, $[k[\beta]:k] = \deg h_j\le \deg g < \deg f$ and 
    \begin{equation*}
        p_1(\beta)^2 + \cdots + p_m(\beta)^2 + 1 = f(\beta)g(\beta) = 0
    \end{equation*}
    whence $k[\beta]$ is not formally real and contradicts the choice of $\alpha$.
\end{proof}

\chapter{Normal Extensions}
\begin{definition}[Splitting Field]
    Let $k$ be a field and $\{f_i\}_{i\in I}$ be a family of polynomials in $k[x]$. By a \textit{splitting field} for this family, we shall mean an extension $K$ of $k$ such that every $f_i$ splits in linear factors in $K[x]$ and $K$ is generated by all the roots of all the polynomials $f_i$ for $i\in I$ in some algebraic closure $\overline{k}$.
\end{definition}

In particular, if $f\in k[x]$ is a polynomial, then the splitting field of $f$ over $k$ is an extension $K/k$ such that $f$ splits into linear factors in $K$ and $K$ is generated by all the roots of $f$.

\begin{definition}[Normal Extension]
    An algebraic extension $K/k$ is said to be \textit{normal} if whenever an irreducible polynomial $f(x)\in k[x]$ has a root in $K$, it splits into linear factors over $K$.
\end{definition}

\begin{theorem}[Uniqueness of Splitting Fields]\thlabel{thm:uniqueness-splitting-field}
    Let $K$ be a splitting field of the polynomial $f(x)\in k[x]$. If $E$ is another splitting field of $f$, then there exists an isomorphism $\sigma: E\to K$ inducing the identity on $k$. If $k\subseteq K\subseteq\overline{k}$, where $\overline{k}$ is an algebraic closure of $k$, then any embedding of $E$ in $\overline{k}$ inducing the identity on $k$ must be an isomorphism of $E$ on $K$.
\end{theorem}
\begin{proof}
    We prove both assertions together. Due to \thref{thm:extension-theorem}, there is an embedding $\sigma: E\to\overline{k}$ such that $\sigma\mid_k = \mathbf{id}_k$. Therefore, it suffices to prove the second half of the theorem.

    We have two factorizations 
    \begin{align*}
        f(x) &= c(x - \alpha_1)\cdots(x - \alpha_n)\qquad\text{over $E$}\\
        &= c(x - \beta_1)\cdots(x - \beta_n)\qquad\text{over $K$}
    \end{align*}

    Since $\sigma$ induces the identity map on $k$, $f$ must remain invariant under $\sigma$. Further, we have 
    \begin{equation*}
        \sigma f(x) = c(x - \sigma\beta_1)\cdots(x - \sigma\beta_n)
    \end{equation*}
    Due to unique factorization, we must have that $(\sigma\beta_1,\ldots,\sigma\beta_n)$ differs from $(\alpha_1,\ldots,\alpha_n)$ by a permutation. Since $\sigma E = k(\sigma\beta_1,\ldots,\sigma\beta_n)$, we immediately have the desired conclusion.
\end{proof}

\begin{theorem}\thlabel{thm:normal-equivalence}
    Let $K/k$ be algebraic in some algebraic closure $\overline{k}$ of $k$. Then, the following are equivalent: 
    \begin{enumerate}
        \item Every embedding $\sigma$ of $K$ in $\overline{k}$ over $k$ is an automorphism of $K$ 
        \item $K$ is the splitting field of a family of polynomials in $k[x]$
        \item $K/k$ is normal
    \end{enumerate}
\end{theorem}
\begin{proof}
\hfill 

\noindent\underline{$(1)\Longrightarrow(2)\wedge(3)$:} For each $\alpha\in K$, let $m_\alpha(x)$ denote the minimal polynomial for $\alpha$ over $k$. We shall show that $K$ is the splitting field for $\{m_\alpha\}_{\alpha\in K}$. Obviously, $K$ is generated by $\{\alpha\}_{\alpha\in K}$, hence, it suffices to show that $m_\alpha$ splits into linear factors over $K$. Let $\beta$ be a root of $m_\alpha$ in $\overline{k}$. Then, there is an isomorphism $\sigma: k(\alpha)\to k(\beta)$. One may extend this to an embedding $\sigma: K\to\overline{k}$, which by our hypothesis, is an automorphism of $K$, implying that $\beta\in K$ and giving us the desired conclusion.

\noindent\underline{$(2)\Longrightarrow(1)$:} Let $K$ be the splitting field for the family of polynomials $\{f_i\}_{i\in I}$. Let $\alpha\in K$ and $\alpha$ be the root of some polynomial $f_i$ and $\sigma: K\to\overline k$ be an embedding of fields. Since $f_i$ remains invariant under $\sigma$, it must map a root of $f_i$ to another toot of $f_i$, that is, $\sigma\alpha$ is a root of $f_i$. Consequently, $\sigma$ maps $K$ into $K$. Now, due to \thref{lem:self-embedding-is-automorphism}, $\sigma$ is an automorphism and $K/k$ is normal.

\noindent\underline{$(3)\Longrightarrow(1)$:} Let $\sigma: K\to\overline{k}$ be an embedding of fields. Let $\alpha\in K$ and $p(x)\in k[x]$ be its irreducible polynomial over $k$. Since $p$ remains invariant under $\sigma$, it must map $\alpha$ to a root $\beta$ of $p$ in $\overline{k}$. But since $p$ splits into linear factors over $K$, $\beta\in K$ and thus $\sigma(K)\subseteq K$, consequently, $\sigma(K) = K$ due to \thref{lem:self-embedding-is-automorphism}, therefore completing the proof.
\end{proof}


\begin{corollary}
    The splitting field of a polynomial is a normal extension.
\end{corollary}

\begin{theorem}\thlabel{thm:normal-properties}
    Normal extensions remain normal under lifting. If $k\subseteq E\subseteq K$, and $K$ is normal over $k$, then $K$ is normal over $E$. If $K_1,K_2$ are normal over $k$ and are contained in some field $L$, then $K_1K_2$ is normal over $k$ and so is $K_1\cap K_2$.
\end{theorem}
\begin{proof}
    Let $K/k$ be normal and $F/k$ be any extension with $K$ and $F$ contained in some larger extension. Let $\sigma$ be an embedding of $KF$ over $F$ in $\overline{F}$. The restriction of $\sigma$ to $K$ is an embedding of $K$ over $k$ and therefore, is an automorphism of $K$. As a result, $\sigma(KF) = (\sigma K)(\sigma F) = KF$ and thus $KF/F$ is normal.

    Now, suppose $k\subseteq E\subseteq K$ with $K/k$ normal. Let $\sigma$ be an embedding of $K$ in $\overline{k}$ over $E$. Then, $\sigma$ induces the identity on $k$ and is therefore an automorphism of $K$. This shows that $K/E$ is normal.

    Next, if $K_1$ and $K_2$ are normal over $k$ and $\sigma$ is an embedding of $K_1K_2$ over $k$, then its restriction to $K_1$ and $K_2$ respectively are also embeddings over $k$ and consequently are automorphisms. This gives us 
    \begin{equation*}
        \sigma(K_1K_2) = (\sigma K_1)(\sigma K_2) = K_1K_2
    \end{equation*}

    Finally, since any embedding of $K_1\cap K_2$ can be extended to that of $K_1K_2$, we have, due to a similar argument, that $K_1\cap K_2$ is normal over $k$.
\end{proof}

\chapter{Separable Extensions}
Let $E/k$ be a finite extension, and therefore, algebraic. Let $L$ be an algebraically closed field along with an embedding $\sigma: k\to L$. Define $S_\sigma$ to be the set of extensions of $\sigma$ to $\sigma^*: E\to L$.

\begin{definition}[Separable Degree]
    Given the above setup, the \textit{separable degree} of the finite extension $E/k$, denoted by $[E:k]_s$ is defined to be the cardinality of $S_\sigma$.
\end{definition}

\begin{proposition} 
    The separable degree is well defined. That is, if $L'$ is an algebraically closed field and $\tau: k\to L'$ be an embedding, then the cardinality of $S_\tau$ is equal to that of $S_\sigma$
\end{proposition}

\begin{definition}[Separable Extension]
    Let $E/k$ be a finite extension. Then it is said to be \textit{separable} if $[E:k]_s = [E:k]$. Similarly, let $\alpha\in\overline{k}$. Then $\alpha$ is said to be separable over $k$ if $k(\alpha)/k$ is separable.
\end{definition}

\begin{proposition}\thlabel{prop:sep-deg-multiplicative}
    Let $E/F$ and $F/k$ be finite extensions. Then 
    \begin{equation*}
        [E:k]_s = [E:F]_s[F:k]_s
    \end{equation*}
\end{proposition}
\begin{proof}
    Let $L$ be an algebraically closed field and $\sigma: k\to L$ be an embedding. Let $\{\sigma_i\}_{i\in I}$ be the extensions of $\sigma$ to an embedding $F\to L$ and $\{\tau_{ij}\}$ be the extensions of $\sigma$ to an embedding $E\to L$. We have indexed $\tau$ in such a way that the restriction $\tau_i\mid_{E} = \sigma_i$. Using the definition of the separable degree, we have that for each $i$ there are precisely $[E:F]_s$ $j$'s such that $\tau_{ij}$ is a valid extension. This immediately implies the desired conclusion.
\end{proof}

\begin{corollary}
    Let $E/k$ be finite. Then, $[E:k]_s\le [E:k]$.
\end{corollary}
\begin{proof}
    Due to finitness, we have a tower of extensions 
    \begin{equation*}
        k\subsetneq k(\alpha_1)\subsetneq\cdots\subsetneq k(\alpha_1,\ldots,\alpha_n)
    \end{equation*}
    We may now finish using \thref{lem:num-extensions-eq-distinct-roots}.
\end{proof}

\begin{theorem}\thlabel{thm:charzero-finite-separable}
    Let $E/k$ be finite and $\chr k = 0$. Then $E/k$ is separable.
\end{theorem}
\begin{proof}
    Since $E/k$ is finite, there is a tower of extensions as follows: 
    \begin{equation*}
        k\subsetneq k(\alpha_1)\subsetneq\cdots\subsetneq k(\alpha_1,\ldots,\alpha_n)
    \end{equation*}

    We shall show that the extension $k(\alpha)/k$ is separable for some $\alpha\in\overline{k}$. Let $p(x) = m_\alpha(x)$ be the minimal polynomial over $k[x]$. We contend that $p(x)$ does not have any multiple roots. Suppose not, then $p(x)$ and $p'(x)$ share a root, say $\beta$. But since $p(x)$ is the minimal polynomial for $\beta$ over $k$, it must divide $p'(x)$ which is impossible over a field of characteristic $0$. Finally, due to \thref{lem:num-extensions-eq-distinct-roots}, we must have $k(\alpha)/k$ is separable.

    This immediately implies the desired conclusion, since 
    \begin{equation*}
        [E:k]_s = [k(\alpha_1,\ldots,\alpha_n): k(\alpha_1,\ldots,\alpha_{n - 1}]\cdots[k(\alpha_1): k] = [E:k]
    \end{equation*}
\end{proof}

\begin{theorem}\thlabel{thm:nonzerochar-sep-degree}
    Let $E/k$ be finite and $\chr k = p > 0$. Then, there is $m\in\N_0$ such that 
    \begin{equation*}
        [E:k] = p^m[E:k]_s
    \end{equation*}
\end{theorem}
\begin{proof}
    
\end{proof}

\begin{remark}
    From the above proof we obtain that if $\alpha\in E$, then $\alpha^{[E:k]_i}$ is separable over $k$.
\end{remark}

\begin{corollary}
    Let $E/k$ be a finite extension. Then, $[E:k]_s$ divides $[E:k]$. 
\end{corollary}
\begin{proof}
    Follows from \thref{thm:charzero-finite-separable} and \thref{thm:nonzerochar-sep-degree}.
\end{proof}

\begin{definition}[Inseparable Degree]
    Let $E/k$ be finite. Then, we denote 
    \begin{equation*}
        [E:k]_i = \frac{[E:k]}{[E:k]_s}
    \end{equation*}
    as the \textit{inseparable degree}.
\end{definition}

\begin{lemma}\thlabel{lem:separable-element-lifting}
    Let $K/k$ be algebraic and $\alpha\in K$ is separable over $k$. Let $k\subseteq F\subseteq K$. Then, $\alpha$ is separable over $F$.
\end{lemma}
\begin{proof}
    Let $p(x)\in k[x]$ and $f(x)\in F[x]$ be the minimal polynomial of $\alpha$ over $k$ and $F$ respectively. By definition, $f(x)\mid p(x)$ and therefore has distinct roots in the algebraic closure of $k$. Consequently, $\alpha$ is separable over $F$.
\end{proof}

\begin{proposition}\thlabel{prop:finite-sep-iff-elem-sep}
    Let $E/k$ be finite. Then, it is separable if and only if each element of $E$ is separable over $k$.
\end{proposition}
\begin{proof}
    Suppose $E/k$ is separable and $\alpha\in E\backslash k$. Then, there is a tower of extensions 
    \begin{equation*}
        k\subsetneq k(\alpha_1)\subsetneq\cdots\subsetneq k(\alpha_1,\ldots,\alpha_n) = E
    \end{equation*}
    with $\alpha_1 = \alpha$. Recall that $[E:k]_s\le [E:k]$ with equality if and only if there is an equality at each step in the tower. This implies the desired conclusion.

    Conversely, suppose each element of $E$ is separable over $k$. Then, each $\alpha_i$ is separable over $k(\alpha_1,\ldots,\alpha_{i - 1})$ due to \thref{lem:separable-element-lifting}. Consequently, for each step in the tower, 
    \begin{equation*}
        [k(\alpha_1,\ldots,\alpha_i):k(\alpha_1,\ldots,\alpha_{i - 1})]_s = 
        [k(\alpha_1,\ldots,\alpha_i):k(\alpha_1,\ldots,\alpha_{i - 1})]
    \end{equation*}
    implying the desired conclusion.
\end{proof}

\begin{definition}[Infinite Separable Extensions]
    An algebraic extension $E/k$ is said to be \textit{separable} if each finitely generated sub-extension is separable.
\end{definition}

\begin{theorem}\thlabel{thm:sep-if-generators-sep}
    Let $E/k$ be algebraic and generated by a family $\{\alpha_i\}_{i\in I}$. If each $\alpha_i$ is separable over $k$, then $E$ is separable over $k$.
\end{theorem}
\begin{proof}
    Let $k(\alpha_1,\ldots,\alpha_n)/k$ be a finitely generated sub-extension of $E/k$. From our proof of \thref{prop:finite-sep-iff-elem-sep}, we know that $\alpha_i$ is separable over $k(\alpha_1,\ldots,\alpha_{i - 1})$, and therefore, $k(\alpha_1,\ldots,\alpha_n)$ is separable over $k$ and we have the desired conclusion.
\end{proof}

\begin{theorem}\thlabel{thm:sep-iff-elem-sep}
    Let $E/k$ be algebraic. Then, $E/k$ is separable if and only if each element of $E$ is separable over $k$.
\end{theorem}
\begin{proof}
    Suppose $E/k$ is separable, then for each $\alpha\in E$, $k(\alpha)$ is a finitely generated sub-extension of $E$, which is separable by definition. This implies that $\alpha$ is separable over $k$, again by definition.

    Conversely, suppose each element is separable over $k$. Let $k(\alpha_1,\ldots,\alpha_n)$ be a finitely generated sub-extension of $E$. Then, we have the following tower 
    \begin{equation*}
        k\subsetneq k(\alpha_1)\subsetneq\cdots\subsetneq k(\alpha_1,\ldots,\alpha_n)
    \end{equation*}
    From our proof of \thref{prop:finite-sep-iff-elem-sep}, we know that $\alpha_i$ is separable over $k(\alpha_1,\ldots,\alpha_{i-1})$, this immediately implies that $k(\alpha_1,\ldots,\alpha_n)/k$ is separable.
\end{proof}

\begin{theorem}\thlabel{thm:sep-distinguished-class}
    Separable extensions (not necessarily finite) form a distinguished class of extensions.
\end{theorem}
\begin{proof}
    Suppose $E/k$ is separable and $F$ is an intermediate field. Since each element of $F$ is an element of $E$, we have that $F$ must be separable over $K$, due to \thref{thm:sep-iff-elem-sep}. Conversely, suppose both $E/F$ and $F/k$ are separable. Now, if $E/k$ is finite, so is $F/k$ and we are done due to \thref{prop:sep-deg-multiplicative}.

    Now, suppose $E/k$ is not finite. It suffices to show that for all $\alpha\in E$, $\alpha$ is separable over $k$. Let $p(x) = a_nx^n + \cdots + a_0$ be the unique monic irreducible polynomial of $\alpha$ over $F$. Then, $p(x)$ is also the monic irreducible polynomial of $\alpha$ over $k(a_0,\ldots,a_n)$. Since $\alpha$ is separable over $F$, $p(x)$ has no repeated roots and therefore $\alpha$ is also separable over $k(a_0,\ldots,a_n)$. We now have a finite tower 
    \begin{equation*}
        k\subsetneq k(a_0,\ldots,a_n)\subsetneq k(a_0,\ldots,a_n)(\alpha)
    \end{equation*}
    Furthermore, since each $a_i$ is separable over $k$ for $0\le i\le n$, it must be the case that $k(a_0,\ldots,a_n)$ is separable over $k$ and finally so must $\alpha$.

    Next, suppose $E/k$ is separable and $F/k$ is an extension, where both $E$ and $F$ are contained in some algebraically closed field $L$. Since every element of $E$ is separable over $k$, it must be separable over $F$, through a similar argument involving the minimal polynomial as carried out above. Since $EF$ is generated by all the elements of $E$, we may finish using \thref{thm:sep-if-generators-sep}. This completes the proof.
\end{proof}

\begin{definition}[Separable Closure]
    Let $k$ be a field and $k^a$ be an algebrai closure. We define the separable closure $k^\sep$ as 
    \begin{equation*}
        k^\sep = \{a\in k^a\mid\text{$a$ is separable over $k$}\}
    \end{equation*}
\end{definition}

If $\alpha,\beta\in k^\sep$, then $\alpha,\beta\in k(\alpha,\beta)$, which by choice of $\alpha,\beta$ is separable over $k$. Therefore, $\alpha\beta,\alpha/\beta,\alpha + \beta,\alpha - \beta\in k(\alpha,\beta)$ are separable over $k$, and lie in $k^\sep$, from which it follows that $k^\sep$ is a field extension of $k$.

\section*{Primitive Element Theorem}

\begin{definition}[Primitive Element]
    Let $E/k$ be a finite extension. Then $\alpha\in E$ is said to be \textit{primitive} if $E = k(\alpha)$. In this case, the extension $E/k$ is said to be simple.
\end{definition}

\begin{theorem}[Steinitz, 1910]\thlabel{thm:primitive-element-theorem}
    Let $E/k$ be a finite extension. Then, there exists a primitive element $\alpha\in E$ if and only if there exist only a finite number of fields $F$ such that $k\subseteq F\subseteq E$. If $E/k$ is separable, then there exists a primitive element.
\end{theorem}
\begin{proof}
    If $k$ is finite, then so is $E$ and it is known that the multiplicative group of finite fields are cyclic, therefore generated by a single element, immediately implying the desired conclusion. Henceforth, we shall suppose that $k$ is infinite.

    Suppose there are only a finite number of fields intermediate between $k$ and $E$. Let $\alpha,\beta\in E$. We shall show that $k(\alpha,\beta)/k$ has a primitive element. Indeed, consider the intermediate fields $k(\alpha + c\beta)$ for $c\in k$, which are infinite in number. Therefore, there are distinct elements $c_1,c_2\in k$ such that $k(\alpha + c_1\beta) = k(\alpha + c_2\beta)$. Consequently, $(c_1 - c_2)\beta\in k(\alpha + c_1\beta)$, therefore, $\beta\in k(\alpha + c_1\beta)$ and thus $\alpha\in k(\alpha + c_1\beta)$. This implies that $\alpha + c_1\beta$ is a primitive element for $k(\alpha,\beta)/k$. Now, since $E/k$ is finite, it must be finitely generated. We may now use induction to finish.

    Conversely, suppose $E/k$ has a primitive element, say $\alpha\in E$. Let $f(x)$ be the monic irreducible polynomial for $\alpha$ over $k$. Now, for each intermediate field $k\subseteq F\subseteq E$, let $g_F$ denote the monic irreducible polynomial for $\alpha$ over $F$. Using the unique factorization over $\overline{k}[x]$, $g_F\mid f$ for each intermediate field $F$, therefore, there may be only finitely many such $g_F$ and thus, only finitely many intermediate fields $F$.

    Finally, suppose $E/k$ is separable and therefore, finitely generated. Hence, it suffices to prove the statement for $k(\alpha, \beta)/k$. Say $n = [k(\alpha, \beta): k]$ and let $\sigma_1,\ldots,\sigma_n$ be distinct embeddings of $k(\alpha,\beta)$ into $\overline{k}$ over $k$
    \begin{equation*}
        f(x) = \prod_{1\le i\ne j\le n}\left(x(\sigma_i\beta - \sigma_j\beta) + (\sigma_i\alpha - \sigma_j\beta)\right)
    \end{equation*}

    Since $f$ is not identically zero, there is $c\in k$ (due to the infiniteness of $k$), such that $f(c)\ne 0$ and thus, the elements $\sigma_i(\alpha + c\beta)$ are distinct for $1\le i\le n$, and thus
    \begin{equation*}
        n\le[k(\alpha + c\beta): k]_s\le [k(\alpha + c\beta): k]\le[k(\alpha,\beta): k] = n
    \end{equation*}
    Thus, $\alpha + c\beta$ is primitive for $k(\alpha,\beta)/k$ which completes the proof.
\end{proof}

Note that there are finite extension with infinitely many subfields. For example, consider the extension $\F_p(x,y)/\F_p(x^p,y^p)$ which has degree $p^2$. Let $z\in k=\F_p(x^p,y^p)$ and $w = x + zy\in\F_p(x,y)$. We have $w^p = x^p + z^py^p\in\F_p(x^p,y^p)$ and thus, $k(w)/k$ has degree $p$. Furthermore, for $z\ne z'$ and $w' = x + z'y$, it is not hard to see that $k(w,w')$ contains both $x$ and $y$, and is equal to $\F_p(x,y)$, from which it follows that $w\ne w'$. Since we have infinitely many choices of $z$, there are infinitely many subfields of the extension $\F_p(x,y)/\F_p(x^p,y^p)$.

\begin{lemma}
    Let $E/k$ be an algebraic separable extension. Further, suppose that there is an integer $n\ge 1$ such that for every element $\alpha\in E$, $[k(\alpha):k]\le n$. Then $E/k$ is finite and $[E:k]\le n$.
\end{lemma}
\begin{proof}
    Let $\alpha\in E$ such that $[k(\alpha):k]$ is maximal. We claim that $E = k(\alpha)$, for if not, there would be $\beta\in E\backslash k(\alpha)$. Now, since $k(\alpha,\beta)$ is a separable extension and is finite, it must be primitve. Thus, there is $\gamma\in E$ such that $k(\alpha,\beta) = k(\gamma)$ and $[k(\gamma):k] = [k(\alpha,\beta):k] > [k(\alpha):k]$, contradicting the assumed maximality. This completes the proof.
\end{proof}

\chapter{Inseparable Extensions}
\begin{proposition}
    Let $\alpha\in k^a$ and $f(x)\in k[x]$ be the minimal polynomial of $\alpha$ over $k$. If $\chr k = 0$, then all the roots of $f$ have multiplicity $1$. If $\chr k = p > 0$, then there is a non-negative integer $m$ such that every root of $f$ has multiplicity $p^m$. Consequently, we have 
    \begin{equation*}
        [k(\alpha):k] = p^m[k(\alpha):k]_s
    \end{equation*}
    and $\alpha^{p^m}$ is separable over $k$.
\end{proposition}
\begin{proof}
    
\end{proof}

\begin{definition}
    Let $\chr k = p > 0$. An element $\alpha\in k^a$ is said to be \emph{purely inseparable} over $k$ if there is a non-negative integer $n\ge 0$ such that $\alpha^{p^n}\in k$.
\end{definition}

\begin{theorem}\thlabel{thm:pins-equivalence}
    Let $\chr k = p > 0$ and $E/k$ be an algebraic extension. Then the following are equivalent: 
    \begin{enumerate}[label=(\alph*)]
        \item $[E:k]_s = 1$.
        \item Every element $\alpha\in E$ is purely inseparable over $k$. 
        \item For every $\alpha\in E$, the irreducible equation of $\alpha$ over $k$ is of type $X^{p^n} - a = 0$ for some $n\ge 0$ and $a\in k$.
        \item There is a set of generators $\{\alpha_i\}_{i\in I}$ of $E$ over $k$ such that each $\alpha_i$ is purely inseparable over $k$.
    \end{enumerate}
\end{theorem}
\begin{proof}
    $(a)\implies(b)$. Let $\alpha\in E$. From the multiplicativity of the separable degree, we must have $[k(\alpha):k]_s = 1$. Let $f(x)\in k[x]$ be the minimal polynomial of $\alpha$ over $k$. Since $[k(\alpha):k]_s$ is equal to the number of distinct roots of $f$, we see that $f(x) = (x - \alpha)^m$ for some positive integer $m$. Let $m = p^nr$ such that $p\nmid r$. Then, we have 
    \begin{equation*}
        f(x) = \left(x - \alpha\right)^{p^nr} = \left(x^{p^n} - \alpha^{p^n}\right)^r = x^{p^nr} - r\alpha^{p^n}x^{p^n(r - 1)} + \cdots
    \end{equation*}
    Since the coefficients of $f$ lie in $k$, we have $r\alpha^{p^n}\in k$ whence $\alpha^{p^n}\in k$.

    $(b)\implies(c)$. There is a minimal non-negative integer $n$ such that $\alpha^{p^n}\in k$. Consider the polynomial $g(x) = x^{p^n} - \alpha^{p^n}\in k[x]$. Note that $g(x) = (x - \alpha)^{p^n}$, whence the minimal polynomial for $\alpha$ over $k$ divides $g$ and is thus of the form $(x - \alpha)^{m}$ for some positive integer $m\le p^n$. Using a similar argument as in the previous paragraph, we see that there is a non-negative integer $r$ such that $\alpha^{p^r}\in k$. Due to the minimality of $n$, we must have $m = p^n$ and $g$ the minimal polynomial of $\alpha$ over $k$. 

    $(c)\implies(d)$. Trivial. 

    $(d)\implies(a)$. Any embedding of $E$ in $k^a$ must be the identity on the $\alpha_i$'s whence the embedding must be the identity on all of $E$ which completes the proof.
\end{proof}

\begin{definition}
    An algebraic extension $E/k$ is said to be \emph{purely inseparable} if it satisfies the equivalent conditions of \thref{thm:pins-equivalence}.
\end{definition}

\begin{proposition}
    Purely inseparable extensions form a distinguished class of extensions.
\end{proposition}
\begin{proof}
    Let $\chr k = p > 0$. The assertion about the tower of fields follows from the multiplicativity of separable degree. Now, let $E/k$ be purely inseparable. Then there is a set of generators $\{\alpha_i\}_{i\in I}$ generating $E$ over $k$. Then, $\{\alpha_i\}_{i\in I}$ generates $EF$ over $F$. Since the minimal polynomial of $\alpha_i$ over $F$ must divide the minimal polynomial of $\alpha_i$ over $k$, which is of the form $(x - \alpha_i)^{p^{n_i}}$ for some non-negative integer $n$, we see that $\alpha_i$ is purely inseparable over $F$ whence $EF$ is purely inseparable over $F$.

    Finally, let $E/k$ and $F/k$ be purely inseparable extensions. If $\{\alpha_i\}_{i\in I}$ and $\{\beta_j\}_{j\in J}$ generate $E$ and $F$ over $k$ respectively such that each $\alpha_i$ and $\beta_j$ is purely inseparable over $k$, then $EF$ is generated by $\{\alpha_i\}_{i\in I}\cup\{\beta_j\}_{j\in J}$ over $k$ whence is purely inseparable over $k$.
\end{proof}

\begin{proposition}
    Let $E/k$ be an algebraic extension and $E_0$ the separable closure of $k$ in $E$. Then, $E/E_0$ is purely inseparable.
\end{proposition}
\begin{proof}
    If $\chr k = 0$, then $E/k$ is separable and $E_0 = E$ and the conclusion is obvious. On the other hand, if $\chr k = p > 0$, then for every $\alpha\in E$, there is a non-negative integer $m$ such that $\alpha^{p^m}$ is separable over $k$ whence an element of $E_0$. Thus, $E/E_0$ is purely inseparable.
\end{proof}

\begin{proposition}
    Let $K/k$ be normal and $K_0$ the separable closure of $k$ in $K$. Then $K_0/k$ is normal.
\end{proposition}
\begin{proof}
    Let $\sigma: K_0\to k^a$ be an embedding of fields. This extends to an embedding of $K$ and is thus an automorphism of $K$. Note that $\sigma(K_0)$ is separable over $k$ and is thus contained in $k_0$ whence $\sigma(K_0) = K_0$ and $\sigma$ is an automorphism. This completes the proof.
\end{proof}

\begin{lemma}
    Let $K/k$ be normal, $G = \Aut(K/k)$ and $K^G$ the fixed field of $G$. Then $K^G/k$ is purely inseparable and $K/K^G$ is separable. If $K_0$ is the separable closure of $k$ in $K$, then $K = K^GK_0$ and $K^G\cap K_0 = 0$.
\end{lemma}
\begin{proof}
    Let $\alpha\in K^G$ and $\sigma: k(\alpha)\to k^a$ be an embedding over $k$. This can be extended to an embedding $\wt\sigma: K\to k^a$. Since $K$ is normal, this is an automorphism $\wt\sigma: K\to K$ and thus an element of $G$. This must leave $\alpha$ fixed whence $\sigma$ is the identity map, consequently, $\alpha$ is purely inseparable over $k$ and the conclusion follows.


    We shall now show that $K/K^G$ is separable. Pick some $\alpha\in K$ and let $\sigma_1,\ldots,\sigma_n\in G$ such that the elements $\sigma_1(\alpha),\ldots,\sigma_n(\alpha)$ form a maximal set of pairwise distinct elements. Consider the polynomial $f(x)$ in $K[x]$ given by
    \begin{equation*}
        f(x) = \prod_{i = 1}^n(x - \sigma_i(\alpha))
    \end{equation*}
    It is not hard to see that for any $\sigma\in G$, $\sigma(f) = f$, whence $f\in K^G[x]$ and $\alpha$ is separable over $K^G$.

    Note that any element of $K^G\cap K_0$ is both separable and purely inseparable over $k$ whence an element of $k$. Thus $K^G\cap K_0 = k$. 

    Finally, since both purely inseparable and separable extensions form a distinguished class, we have $K/K_0K^G$ is both separable and purely inseparable whence $K = K_0K^G$. This completes the proof.
\end{proof}

\chapter{Finite Fields}
It is well known that every finite field must have prime characteristic. In fact, any integral domain with nonzero characteristic must have prime characteristic.

\begin{theorem}
    Let $F$ be a finite field with characteristic $p > 0$. Then there is a positive integer $n$ such that $F$ has cardinality $p^n$. Further, there is a unique field upto isomorphism of cardinality $p^n$.
\end{theorem}
\begin{proof}
    The prime subfield of $F$ is the subfield generated by $1$ and is isomorphic to $\F_p$. Then $[F:\F_p] = n$, whence the conclusion follows. Now, we show that there is a field with cardinality $p^n$. Consider the polynomial $f(x) = x^{p^n} - x\in\F_p[x]$. First, note that $Df(x) = -1$, and thus $f(x)$ has distinct roots in $\overline\F_p$. It is not hard to see that if $\alpha,\beta$ are roots of $f(x)$ in $\overline F_p$, then $\alpha - \beta$ and $\alpha\beta$ are roots of $f(x)$ in $\overline\F_p$. Therefore, the collection of roots of $f(x)$ in $\overline F_p$ form a field. The cardinality of this field is the number of distinct roots of $f(x)$ in $\overline\F_p$, which is precisely $p^n$.

    As for uniqueness, note that if $F$ is a field of cardinality $p^n$, then every element of $F$ is a root of $f(x) = x^{p^n} - x\in\F_p[x]$ (this is because $F$ contains a copy of $\F_p$ in it). Therefore, $F$ is the splitting field for $f(x)$ over $\F_p[x]$ in some algebraic closure. But since all splitting fields are isomorphic, we have the desired conclusion.
\end{proof}

\begin{theorem}[Frobenius]
    The group of automorphisms of $\F_q$ where $q = p^n$ is cyclic of degree $n$, generated by the Frobenius mapping, $\varphi:\F_q\to\F_q$ given by $\varphi(x) = x^p$.
\end{theorem}
\begin{proof}
    We first verify that $\varphi$ is an automorphism. That $\varphi$ is a ring homomorphism is easy to show, from which it would follow that $\varphi$ is injective. Surjectivity follows from here since $\F_q$ is finite. Next, note that $\varphi$ leaves $\F_p$ fixed, thus, $G = \Aut(\F_q) = \Aut(\F_q/\F_p)$. Furthermore, $|\Aut(\F_q/\F_p)| = [\F_q:\F_p]_s\le[\F_q:\F_p] = n$.

    We now show that the order of $\varphi$ in $G$ is precisely $n$, for if $d$ were the order of $\varphi$, then $\varphi^d(x) = x$ for all $x\in\F_q$ and thus, $x^{p^d} - x = 0$ for all $x\in\F_q$, from which it follows that $p^d\ge q$ and $d\ge n$ and the conclusion follows.
\end{proof}

\begin{theorem}
    Let $m,n\in\N$. Then in an algebraic closure $\overline{\F_p}$ of $\F_p$, the subfield $\F_{p^n}$ is contained in $\F_{p^m}$ if and only if $n\mid m$.
\end{theorem}
\begin{proof}
    If $\F_{p^n}$ is contained in $\F_{p^m}$, then $p^m = (p^n)^d$ where $d = [\F_{p^m}:\F_{p^n}]$. The converse follows from noting that $x^{p^n} - x\mid x^{p^m} - x$.
\end{proof}

\begin{theorem}
    Let $m,n\in\N$ such that $n\mid m$. Then the extension $\F_{p^m}/\F_{p^n}$ is finite Galois.
\end{theorem}
\begin{proof}
    We have $[\F_{p^m}:\F_p] = m$ and $[\F_{p^n}:\F_p] = n$, consequently, $[\F_{p^m}:\F_{p^n}]_s = m/n = [\F_{p^m}:\F_{p^n}]$ and thus the extension is separable. To show that the extension $\F_{p^m}/\F_{p^n}$ is normal, it suffices to show that the extension $\F_{p^m}/\F_p$ is normal but this trivially follows from the fact that $\F_{p^m}$ is the splitting field of $x^{p^m} - x\in\F_p[x]$. This completes the proof.
\end{proof}

\chapter{Galois Extensions}
\begin{definition}[Fixed Field]
    Let $K$ be a field and $G$ be a group of automorphisms of $K$. The \textit{fixed field} of $K$ under $G$, denoted by $K^G$ is the set of all elements $x\in K$ such that $\sigma x = x$ for all $\sigma\in G$.
\end{definition}

That the aforementioned set forms a field is trivial.

\begin{definition}[Galois Extension, Group]
    An extension $K/k$ is said to be \textit{Galois} if it is normal and separable. The group of automorphisms of $K$ over $k$ is known as the \textit{Galois Group} of $K/k$ and is denoted by $\Gal(K/k)$.
\end{definition}

\begin{theorem}
    Let $K$ be a Galois extension of $k$ and $G = \Gal(K/k)$. Then $k = K^G$. If $F$ is an intermediate field, $k\subseteq F\subseteq K$, then $K$ is Galois over $F$ and the map 
    \begin{equation*}
        F\mapsto\Gal(K/F)
    \end{equation*}
    from the intermediate fields to subgroups of $G$ is injective.
    \textcolor{red}{Finiteness is not required in this case.}
\end{theorem}
\begin{proof}
    Let $\alpha\in K^G$ and $\sigma: k(\alpha)\to\overline{K}$ be an embedding over $k$. Due to \thref{thm:extension-theorem}, $\sigma$ may be extended to an embedding of $K$ over $k$ in $\overline{K}$. Since $K/k$ is normal, this is an automorphism and therefore, an element of $G$. As a result, $\sigma$ sends $\alpha$ to itself, therefore, any embedding of $k(\alpha)$ over $k$ is the identity map, implying that $[k(\alpha):k]_s = 1$, or equivalently, $k(\alpha) = k$ whence $\alpha\in k$.

    Let $F$ be an intermediate field. Due to \thref{thm:normal-properties} and \thref{thm:sep-distinguished-class}, we have that $K/F$ is normal and separable, therefore Galois.

    Finally, if $F$ and $F'$ map to the same subgroup $H$ of $G$, then due to the first part, of this theorem, we must have $F = K^H = F'$, establishing injectivity.
\end{proof}

\begin{lemma}\thlabel{lem:artin-lemma}
    Let $E/k$ be algebraic and separable, further suppose that there is an integer $n\ge 1$ such that every element $\alpha\in E$ is of degree at most $n$ over $k$. Then $[E:k]\le n$.
\end{lemma}
\begin{proof}
    Let $\alpha\in E$ such that $[k(\alpha):k]$ is maximized. We shall show that $k(\alpha) = E$. Suppose not, then there is $\beta\in E\backslash k(\alpha)$ and thus, we have a tower $k\subseteq k(\alpha)\subsetneq k(\alpha,\beta)$. Due to \thref{thm:primitive-element-theorem}, there is $\gamma\in E$ such that $k(\alpha,\beta) = k(\gamma)$. But then, 
    \begin{equation*}
        [k(\gamma): k] = [k(\alpha, \beta): k] > [k(\alpha):k]
    \end{equation*}
    a contradiction to the maximality of $\alpha$. Therefore, $E = k(\alpha)$ and we have the desired conclusion.
\end{proof}

\begin{theorem}[Artin]
    Let $K$ be a field and let $G$ be a finite group of automorphisms of $K$, of order $n$. Let $k = K^G$. Then $K$ is a finite Galois extension of $k$, and its Galois group is $G$. Further, $[K:k] = n$.
\end{theorem}
\begin{proof}
    Let $\alpha\in K$. We shall show that $K$ is the splitting field of the family $\{m_\alpha(x)\}_{\alpha\in K}$ and that $\alpha$ is separable over $k$. 

    Let $\{\sigma_1\alpha,\ldots,\sigma_m\alpha\}$ be a maximal set of images of $\alpha$ under the elements of $G$. Define the polynomial: 
    \begin{equation*}
        f(x) = \prod_{i = 1}^m(x - \sigma_i\alpha)
    \end{equation*}
    For any $\tau\in G$, we note that $\{\tau\sigma_1\alpha,\ldots,\tau\sigma_m\alpha\}$ must be a permutation of $\{\sigma_1\alpha,\ldots,\sigma_m\alpha\}$, lest we contradict maximality. As a result, $\alpha$ is a root of $f^\tau$ for all $\tau\in G$ and therefore, the coefficients of $f$ lie in $K^G = k$, i.e. $f(x)\in k[x]$. 

    Since the $\sigma_i\alpha$'s are distinct, the minimal polynomial of $\alpha$ over $k$ must be separable, and thus $K/k$ is separable. Next, we see that the minimal polynomial for $\alpha$ also splits in $K$ and thus, $K$ is the splitting field for the family $\{m_\alpha(x)\}_{\alpha\in K}$. Consequently, $K/k$ is normal and hence, Galois.

    Finally, since the minimal polynomial for $\alpha$ divides $f$, we must have $[k(\alpha):k]\le\deg f\le n$ whence due to \thref{lem:artin-lemma}, $[K:k]\le n$. Now, recall that $n = |G|\le [K:k]_s\le[K:k]$ and we have the desired conclusion.
\end{proof}

\begin{corollary}
    Let $K/k$ be a finite Galois extension and $G = \Gal(K/k)$. Then, every subgroup of $G$ belongs to some subfield $F$ such that $k\subseteq F\subseteq K$.
\end{corollary}

\begin{lemma}
    Let $K/k$ be Galois and $F$ an intermediate field, $k\subseteq F\subseteq K$, and let $\lambda: F\to\overline{k}$ be an embedding. Then, 
    \begin{equation*}
        \Gal(K/\lambda F) = \lambda\Gal(K/F)\lambda^{-1}
    \end{equation*}
\end{lemma}
\begin{proof}
    The embedding $\lambda$ can be extended to an embedding of $K$ due to \thref{thm:extension-theorem} and since $K/k$ is normal, $\lambda$ is an automorphism. As a result, $\lambda F\subseteq K$ and thus, $K/\lambda F$ is Galois. Let $\sigma\in\Gal(K/F)$. It is not hard to see that $\lambda\sigma\lambda^{-1}\in\Gal(K/\lambda F)$ and conversely, for $\tau\in\Gal(K/\lambda F)$, $\lambda^{-1}\tau\lambda\in\Gal(K/F)$. This implies the desired conclusion.
\end{proof}

\begin{theorem}
    Let $K/k$ be Galois with $G = \Gal(K/k)$. Let $F$ be an intermediate field, $k\subseteq F\subseteq K$, and let $H = \Gal(K/F)$. Then $F$ is normal over $k$ if and only if $H$ is normal in $G$. If $F/k$ is normal, then the restriction map $\sigma\mapsto\sigma\mid_F$ is a homomorphism of $G$ onto $\Gal(F/k)$ whose kernel is $H$. This gives us $\Gal(F/k)\cong G/H$.
\end{theorem}
\begin{proof}
    Suppose $F/k$ is normal. To see that the map $\sigma\to\sigma\mid_F$ is surjective, simply recall \thref{thm:extension-theorem}. The kernel of said mapping is obviously $H$ and we have that $H\unlhd G$ and due to the First Isomorphism Theorem, $G/H\cong\Gal(F/k)$.

    On the other hand, if $F/k$ is not normal, then there is an embedding $\lambda: F\to\overline{k}$ such that $F\ne\lambda F$. Note that due to \thref{thm:extension-theorem}, $\lambda F\subseteq K$. Then, we have $\Gal(K/F)\ne\Gal(K/\lambda F) = \lambda\Gal(K/F)\lambda^{-1}$, and equivalently, $\Gal(K/F)$ is not normal in $G$. This completes the proof of the theorem.
\end{proof}

\textcolor{red}{Note that in the proof of the above theorem, while showing $H$ is normal in $G$, we did not use that the Galois extension is finite}. We can now put together all the above results into one all-powerful theorem.

\begin{theorem}[Fundamental Theorem of Galois Theory]\thlabel{thm:ftgt}
    Let $K/k$ be a finite Galois extension with $G = \Gal(K/k)$. There is a bijection between the set of subfields $E$ of $K$ containing $k$ and the set of subgroups $H$ of $G$ given by $E = K^H$. The field $E$ is Galois over $k$ if and only if $H$ is normal in $G$, and if that is the case, then the restriction map $\sigma\mapsto\sigma\mid_E$ induces an isomorphism of $G/H$ onto $\Gal(E/k)$.
\end{theorem}

\begin{definition}
    A Galois extension $K/k$ is said to be \textit{abelian (resp. cyclic)} if its Galois group is \textit{abelian (resp. cyclic)}.
\end{definition}

\begin{theorem}
    Let $K/k$ be finite Galois and $F/k$ an arbitrary extension. Suppose $K, F$ are subfields of some larger field. Then $KF$ is Galois over $F$, and $K$ is Galois over $K\cap F$. Let $H = \Gal(KF/F)$ and $G = \Gal(K/k)$. For all $\sigma\in H$, the restriction of $\sigma$ to $K$ is in $G$ and the restriction map $\sigma\mapsto\sigma\mid_K$ gives an isomorphism of $H$ on $\Gal(K/K\cap F)$.
\end{theorem}
\begin{proof}
    That $KF/F$ and $K/K\cap F$ are Galois follow from \thref{thm:normal-properties} and \thref{thm:sep-distinguished-class}. Let $\chi: H\to G$ denote the restriction map. Note that $\ker\chi$ contains all $\sigma\in H$ such that $\sigma$ fixes $K$. But since $\sigma$ implicitly fixes $F$, it must also fix $KF$ and is therefore the unique identity automorphism. As a result, $\ker\chi$ is trivial and $\chi$ is injective. Let $H' = \chi(H)\subseteq G$. We shall show that $K^{H'} = K\cap F$. Indeed, if $\alpha\in K^{H'}$, then $\alpha$ is also fixed by all elements of $H$, since $\chi$ is only the restriction map. As a result, $\alpha\in F$, consequently $\alpha\in K\cap F$. We are now done due to \thref{thm:ftgt}.
\end{proof}

\section{Normal Basis Theorem}

\begin{definition}[Normal Element]
    Let $K/k$ be a finite Galois extension with $\Gal(K/k) = \{\sigma_1,\dots,\sigma_n\}$. An element $\alpha\in K$ is said to be a \emph{normal element} if $\{\sigma_1(\alpha),\dots,\sigma_n(\alpha)\}$ forms a $k$-basis of $K$.
\end{definition}

\begin{theorem}[Normal Basis Theorem]
    If $K/k$ is a finite Galois extension, then it has a normal element.
\end{theorem}
\begin{proof}
    Let $G = \Gal(K/k) = \{\sigma_1,\dots,\sigma_n\}$. We shall divide the proof into two cases.
\begin{description}
    \item[Case 1.] $G$ is cyclic.

    Let $G = \langle\sigma\rangle$ for some $\sigma\in G$. Let $m_\sigma(x)\in k[x]$ denote the minimal polynomial of $\sigma$. Since $\sigma$ is a root of $x^n - 1\in k[x]$, we must have $m_\sigma(x)\mid x^n - 1$. If $\deg(m_\sigma) = m < n$, then there are $a_0,\dots,a_m\in k$ such that 
    \begin{equation*}
        m_\sigma(x) = a_mx^m + \dots + a_0.
    \end{equation*}
    In particular, $a_m\sigma^m + \dots + a_0\id = 0$, but this is a contradiction to Dedekind's Lemma on the independence of characters. Therefore, $m_\sigma(x) = x^n - 1$, consequently, $m_\sigma(x)$ must also be the characteristic polynomial of $\sigma$ due to a degree argument. Since the minimal polynomial and the characteristic polynomial are the same, there is a $\sigma$-cyclic vector for the extension $K/k$, which is the desired normal element.

    \item[Case 2.] $k$ is infinite. Note that the previous case subsumes the case with $k$ finite.

    Due to \thref{thm:primitive-element-theorem}, $K = k(\alpha)$ for some $\alpha\in K$. Suppose without loss of generality that $\sigma_1 = \id$. Let $\alpha_i = \sigma_i(\alpha)$, which are all pairwise distinct, and define
    \begin{equation*}
        g_i(x) = \frac{\prod_{j\ne i}(x - \alpha_j)}{\prod_{j\ne i}(\alpha_i - \alpha_j)}.
    \end{equation*}
    Denote $g_1$ by simply $g$, then, $g_i = \sigma_i(g)$.

    The polynomial 
    \begin{equation*}
        g_1(x) + \dots + g_n(x)
    \end{equation*}
    attains the value $1$ for $\alpha_1,\dots,\alpha_n$ but since it has degree at most $n - 1$, it must be identically equal to $1$. Further, for $i\ne j$, $f\mid g_ig_j$ and $g_i^2 - g_i$ vanishes at $\alpha_1,\dots,\alpha_n$ whence $f\mid g_i^2 - g_i$.

    Define the matrix 
    \begin{equation*}
        A(x) = 
        \begin{bmatrix}
            \sigma_1\sigma_1(g) & \sigma_1\sigma_2(g) & \dots & \sigma_1\sigma_n(g)\\
            \vdots & \vdots & \ddots & \vdots\\
            \sigma_n\sigma_1(g) & \sigma_n\sigma_2(g) & \dots & \sigma_n\sigma_n(g)
        \end{bmatrix}.
    \end{equation*}

    We contend that $\det A(x)$ is a nonzero polynomial. Suppose not. Consider $M(x) = A(x)^TA(x)$. The $(i,j)$-th entry is given by 
    \begin{equation*}
        \sum_{\sigma\in G}\sigma\sigma_i(g)\sigma\sigma_j(g) = \sum_{\sigma\in G}\sigma(g_ig_j).
    \end{equation*}
    If $i\ne j$, note that $f\mid\sigma(g_ig_j)$ for all $\sigma\in G$. Therefore, $f$ divides all non-diagonal entries of $M(x)$ while the diagonal entries of $M(x)$ are given by 
    \begin{equation*}
        \sum_{\sigma\in G}\sigma(g_i)^2\equiv\sum_{\sigma\in G}\sigma(g_i)\pmod{f}\equiv\sum_{i = 1}^n g_i\pmod{f}\equiv 1\pmod{f}.
    \end{equation*}
    Hence, $\det M(x) = 1$ in $K[x]/(f(x))$, in particular, it is nonzero in $K[x]$, therefore, $\det A(x)\ne 0$ in $K[x]$.

    Since $K$ is infinite, there is some $\theta\in K$ such that $\det A(\theta)\ne 0$. Let $\beta = g(\theta)$. We claim that $\beta$ is the desired normal element. To do so, it suffices to show that $\{\sigma_1(\beta),\dots,\sigma_n(\beta)\}$ is linearly independent over $k$.

    Indeed, suppose there is a linear combination 
    \begin{equation*}
        c_1\sigma_1(\beta) + \dots + c_n\sigma_n(\beta) = 0\iff c_1\sigma_1(g(\theta)) + \dots + c_n\sigma_n(g(\theta)) = 0.
    \end{equation*}
    Applying $\sigma_i$ to the above equation for $1\le i\le n$, we obtain a system of linear equations given by 
    \begin{equation*}
        A(\theta)
        \begin{pmatrix}
            c_1\\\vdots\\c_n
        \end{pmatrix}
        = 0,
    \end{equation*}
    whence $c_1 = \dots = c_n = 0$, since $A(\theta)$ is invertible. This completes the proof.\qedhere
\end{description}
\end{proof}

Once we have a normal element, we can easily find the primitive (and sometimes normal) elements of all intermediate fields.

\begin{theorem}
    Let $K/k$ be a finite Galois extension with $G = \Gal(K/k)$ and $\alpha\in K$ be a normal element.
    \begin{enumerate}[label=(\alph*)]
        \item If $H\le G$, then $\beta_H := \Tr^K_{K^H}(\alpha)$ is a primitive element of $K^H/k$. 
        \item If $H\unlhd G$, then $\beta_H$ is a normal element of $K^H/k$.
    \end{enumerate}
\end{theorem}
\begin{proof}
\begin{enumerate}[label=(\alph*)]
    \item Obviously, $\beta_H\in K^H$. We shall show that $\Gal(K/k(\beta_H))\subseteq H$, which would imply $K^H\subseteq k(\beta_H)$ and the conclusion would follow.

    Let $\tau\in G\backslash H$. Then, 
    \begin{equation*}
        \tau(\beta_H) = \sum_{\sigma\in \tau H}\sigma(\alpha).
    \end{equation*}
    Since $\tau H$ is a coset distinct from $H$, they are disjoint and since the collection $\{\sigma(\alpha)\mid\sigma\in G\}$ is a linearly independent set, we cannot have $\tau(\beta_H) = \beta_H$, consequently, $\Gal(K/k(\beta_H))\subseteq H$.

    \item Let $\tau_1,\dots,\tau_m$ be elements of $G$ whose images under the canonical projection $G\onto G/H$ are all the elements of $G/H$. Note that this projection map is simply the restriction map from $\Gal(K/k)$ to $\Gal(k(\beta_H)/k)$. Suppose 
    \begin{equation*}
        c_1\tau_1(\beta_H) + \dots + c_m\tau_m(\beta_H) = 0,
    \end{equation*}
    then, 
    \begin{equation*}
        0 = \sum_{i = 1}^mc_i\left(\sum_{\sigma\in \tau_i H}\sigma(\alpha)\right).
    \end{equation*}
    By our choice of $\tau_i$'s, the cosets $\tau_i H$ and $\tau_j H$ are pairwise distinct, consequently, the sum written above is essentially of linearly independent elements, $\sigma(\alpha)$ where $\sigma$ ranges over $G$. Therefore, $c_1 = \dots = c_m = 0$. This completes the proof.\qedhere
\end{enumerate}
\end{proof}

\chapter{Cyclotomic Extensions}
\begin{definition}
    Let $k$ be a field. A \textit{root of unity} in $k$ is an element $\zeta\in k$ such that $\zeta^n = 1$ for some $n\in\N$.
\end{definition}

Now, if $n > 1$ is an integer not divisible by$ \chr k$, then the polynomial $x^n - 1$ is separable, and hence, in $k^a$, has $n$ distinct roots. It is not hard to see that these form a group under multiplication. Since this is a finite multiplicative subgroup of a field, it must be cyclic. A generator for this group is called a \textit{primitive $n$-th root of unity}. We use $\bm{\mu}_n$ to denote the group of $n$-th roots of unity in $k^a$.

From the previous paragraph, we see that if $\gcd(\chr k, n) = 1$, then $k(\zeta)$ is a splitting field for $x^n - 1$ and $\zeta$ is separable over $k$, therefore, $k(\zeta)/k$ is Galois.

\begin{proposition}
    Let $\gcd(\chr k, n) = 1$. If $\zeta$ is a primitive $n$-th root of unity, then $k(\zeta)/k$ is an abelian extension.
\end{proposition}
\begin{proof}
    Define the map $\psi:\Gal(k(\zeta)/k)\to\Aut(\bm\mu_n)$ by $\sigma\mapsto\sigma|_{\bm\mu_n}$. Note that $\Aut(\bm\mu_n)\cong\left(\Z/n\Z\right)^\times$, further, it is not hard to see that $\psi$ is injective and the conclusion follows.
\end{proof}

Note that although we have shown $\Gal(k(\zeta)/k)$ to be embeddable into $(\Z/n\Z)^\times$, the map may not be a surjection take for example $k = \R$ and $\zeta = \exp(2\pi i/5)$. Then, $k(\zeta) = \bbC$, and $\Gal(k(\zeta)/k)\cong\{\pm 1\}$.

\begin{proposition}
    Let $\zeta$ be a primitive $n$-th root of unity over $\Q$. Then, 
    \begin{equation*}
        [\Q(\zeta):\Q] = \varphi(n)
    \end{equation*}
    and consequently, the map $\psi:\Gal(\Q(\zeta)/\Q)\to(\Z/n\Z)^\times$ is an isomorphism.
\end{proposition}
\begin{proof}
    \textcolor{red}{TODO: Add in later}
\end{proof}

\chapter{Norm and Trace}
\todo{Rewrite this chapter following what JKV taught}
\begin{definition}
    Let $E/k$ be a finite extension and $[E:k]_s = r$ and let $\sigma_1,\ldots,\sigma_r$ be distinct embeddings of $E$ in an algebraic closure $k^a$ of $k$. We define the \textit{norm} and \textit{trace} of $\alpha\in E$ as 
    \begin{align*}
        N_{E/k}(\alpha) = N^E_k(\alpha) = \left(\prod_{j = 1}^r\sigma_j\alpha\right)^{[E:k]_i}\\
        \Tr_{E/k}(\alpha) = \Tr^E_k(\alpha) = [E:k]_i\sum_{j = 1}^r\sigma_j\alpha
    \end{align*}
\end{definition}

Notice that if $E/k$ were not separable, then $\chr k > 0$ and would be a prime, say $p$. Further, $[E:k]_i = p^\nu$ for some $\nu\ge 1$, consequently, $\Tr^E_k(\alpha) = 0$ (since $\chr E = \chr k = p$).

\begin{proposition}
    Let $E/k$ be a finit extension such that $E = k(\alpha)$ for some $\alpha\in E$. If 
    \begin{equation*}
        p(x) = x^n + a_{n - 1}x^{n - 1} + \cdots + a_0
    \end{equation*}
    is the minimal polynomial of $\alpha$ over $k$, then 
    \begin{equation*}
        N^E_k(\alpha) = (-1)^na_0\qquad \Tr^E_k(\alpha) = -a_{n - 1}
    \end{equation*}
\end{proposition}
\begin{proof}
    This follows from the fact that the minimal polynomial splits as 
    \begin{equation*}
        p(x) = \left((x - \alpha_1)\cdots(x - \alpha_r)\right)^{[E:k]_i}
    \end{equation*}
    whence the conclusion follows.
\end{proof}

\begin{proposition}
    Let $E/k$ be a finite extension. Then the norm $N^E_k: E^\times\to k^\times$ is a multiplicative homomorphism and the trace $\Tr^E_k: E\to k$ is an additive homomorphism. Further, if we have a tower of finit extensions $k\subseteq F\subseteq E$, then 
    \begin{equation*}
        N^E_k = N^F_k\circ N^E_F\qquad \Tr^E_k = \Tr^F_k\circ\Tr^E_F
    \end{equation*}
\end{proposition}
\begin{proof}
    First, we must show that $N^E_k$ is a map $E^\times\to k^\times$ and $\Tr^E_k$ is a map $E\to k$. Recall that for $\alpha\in E$, $\beta = \alpha^{[E:k]_i}$ is separable over $k$ and thus $N^E_k$, which is the product of all the conjugates of $\beta$ is also separable since all conjugates lie in $k^\sep$. Now, let $\sigma: k^a\to k^a$ be a homomorphism fixing $k$. Then, it is not hard to see that $\sigma(\beta) = \beta$ and thus $[k(\beta):k]_s = 1$ but since $\beta$ is separable, we have $[k(\beta):k] = 1$ and $\beta\in k$. A similar argument can be applied to the trace.

    Let $\{\sigma_i\}$ be the set of distinct embeddings of $E$ into $k^a$ fixing $F$ and $\{\tau_j\}$ be the set of distinct embeddings of $F$ into $k^a$ fixing $k$. Extend each $\tau_j$ to a homomorphism $k^a\to k^a$. 
    
    We contend that the set of all distinct embeddings of $E$ into $k^a$ fixing $k$ is precisely $\{\tau_j\circ\sigma_i\}$. Obviously, every element of the aforementioned family is distinct and is an embedding of $E$ into $k^a$ fixing $k$. Now, let $\sigma: E\to k^a$ be an embedding of $E$ into $k^a$. Then, the restriction $\sigma|_F$ is equal to (the restriction of) some $\tau_j$, whereby $\tau_j^{-1}\sigma$ fixes $F$ whereby it is equal to some $\sigma_i$. Thus every embedding of $E$ into $k^a$ over $k$ is of the form $\tau_j\circ\sigma_i$.

    Finally, we have 
    \begin{align*}
        \left(\prod_{i,j}(\tau_j\circ\sigma_i)(\alpha)\right)^{[E:F]_i[F:k]_i} = \left(\prod_{j}\tau_j\left(\prod_{i}\sigma_i(\alpha)\right)^{[E:F]_i}\right)^{[F:k]_i} = N^F_k\circ N^E_F(\alpha)\\
        [E:F]_i[F:k]_i\sum_{i,j}\tau_j\circ\sigma_i(\alpha) = [F:k]_i\sum_j\tau_j\left([E:F]_i\sum_{i}\sigma_i(\alpha)\right)
    \end{align*}
    and the conclusion follows.
\end{proof}

\begin{theorem}
    Let $E/k$ be a finite extension and $\alpha\in E$. Let $m_\alpha: E\to E$ be the linear transformation given by $m_\alpha(x) = \alpha x$. Then, 
    \begin{equation*}
        N^E_k(\alpha) = \det(m_\alpha)\qquad \Tr^E_k(\alpha) = \operatorname{tr}(m_\alpha)
    \end{equation*}
\end{theorem}
Note that we may unambiguously write $\det(m_\alpha)$ and $\operatorname{tr}(m_\alpha)$ since both these quantities do not depend on the choice of a basis, since similar matrices have the same determinant and trace.
\begin{proof}

\end{proof}

\chapter{Cyclic Extensions}
\section{Hilbert's Theorems}

\begin{definition}
    A Galois extension $K/k$ is said to be \emph{cyclic} if $\Gal(K/k)$ is a cyclic group. Similarly, it is said to be \emph{abelian} if $\Gal(K/k)$ is abelian.
\end{definition}

\begin{theorem}[Linear Independence of Characters]\thlabel{thm:lin-ind-characters}
    Let $G$ be a group (monoid) and $K$ a field. If $\sigma_1,\dots,\sigma_n: G\to K^\times$ are distinct group homomorphisms. Then, 
    \begin{equation*}
        c_1\sigma_1 + \cdots + c_n\sigma_n = 0 \iff c_1 = \dots = c_n = 0
    \end{equation*}
\end{theorem}

\begin{corollary}\thlabel{cor:non-zero-trace-exists}
    Let $K/k$ be a Galois extension. Then, there is $\alpha\in K$ such that $\Tr^K_k(\alpha)\ne0$.
\end{corollary}
\begin{proof}
    Suppose not. If $\Gal(K/k) = \{\sigma_1,\dots,\sigma_n\}$, then 
    \begin{equation*}
        \sigma_1 + \dots + \sigma_n = 0
    \end{equation*}
    on $K$, a contradiction to \thref{thm:lin-ind-characters}. 
\end{proof}

\begin{theorem}[Hilbert's Theorem 90]\thlabel{thm:hilbert-90}
    Let $K/k$ be a cyclic degree $n$ extension with galois group $G$. Let $\sigma\in G$ be a generator and $\beta\in K$. The norm $N^K_k(\beta) = 1$ if and only if there is $\alpha\in K^\times$ such that $\beta = \alpha/\sigma(\alpha)$
\end{theorem}
\begin{proof}
    $\implies$ Suppose $N^K_k(\beta) = 1$. We have a set of distinct characters $\{\id,\sigma,\ldots,\sigma^{n - 1}\}$ from $K^\times\to K^\times$. Then, due to \thref{thm:lin-ind-characters}, the set map 
    \begin{equation*}
        \tau = \id + \beta\sigma + (\beta\sigma(\beta))\sigma^2 + \cdots + (\beta\sigma(\beta)\cdots\sigma^{n - 2}(\beta))\sigma^{n - 1}
    \end{equation*}
    is nonzero, whereby, there is $\theta\in K^\times$ such that $\alpha = \tau(\theta)\ne 0$. Notice that 
    \begin{equation*}
        \sigma(\alpha) = \sigma(\theta) + (\sigma(\beta))\sigma^2(\theta) + \cdots + (\sigma(\beta)\sigma^2(\beta)\cdots\sigma^{n - 1}(\beta))\sigma^n(\theta)
    \end{equation*}
    Since $N^K_k(\beta) = 1$, we have 
    \begin{equation*}
        \beta\sigma(\beta)\cdots\sigma^{n - 1}(\beta) = 1
    \end{equation*}
    whence, we have $\sigma(\alpha) = \alpha/\beta$ and the conclusion follows.

    $\impliedby$ This is trivial enough.
\end{proof}

\begin{example}
    Find all rational points on the curve $x^2 + y^2 = 1$.
\end{example}
\begin{proof}
    This reduces to finding all elements $\alpha\in\Q[i]$ with $N^{\Q[i]}_\Q(\alpha) = 1$. Any element $\alpha$ of $\Q[i]$ may be written as $(a + bi)/c$. Due to \thref{thm:hilbert-90}, there is an element $\alpha\in\Q[i]$, such that $N^{\Q[i]}_\Q(\alpha) = 1$. Using the general form of elements in $\Q[i]$, we have 
    \begin{equation*}
        \alpha = \frac{a + bi}{a - bi} = \frac{(a^2 - b^2) + 2abi}{a^2 + b^2}
    \end{equation*}
    this completes the proof.
\end{proof}

\begin{lemma}\thlabel{lem:cyclic-primitive-eigenvalue}
    Let $K/k$ be a cyclic extension of degree $n$ with $\Gal(K/k) = \langle\sigma\rangle$ and suppose $k$ contains a primitive $n$-th root of unity, $\zeta$. Then, $\zeta$ is an eigenvalue of $\sigma$.
\end{lemma}
\begin{proof}
    Note that $N^K_k(\zeta^{-1}) = 1$. Due to \thref{thm:hilbert-90} there is $\alpha\in K$ such that $\alpha/\sigma(\alpha) = \zeta^{-1}$ and the conclusion follows.
\end{proof}

\begin{theorem}[Structure of Cyclic Extensions]\thlabel{thm:structure-cyclic-extension}
    Let $K/k$ be a cyclic extension of degree $n$ and suppose $k$ contains a primitive $n$-th root of unity. Then, $K = k(\alpha)$ for some $\alpha\in K$ such that $\alpha^n\in k$.
\end{theorem}
\begin{proof}
    Let $\Gal(K/k) =\langle\sigma\rangle$. Due to \thref{lem:cyclic-primitive-eigenvalue}, there is $\alpha\in K$ such that $\sigma(\alpha) = \zeta\alpha$. Then, $\alpha$ has $n$-distinct conjugates in $K$ whence $K = k(\alpha)$. Now, 
    \begin{equation*}
        \sigma(\alpha^n) = \sigma(\alpha)^n = \alpha^n.
    \end{equation*}
    Thus, $\alpha^n$ is fixed under the action of $\Gal(K/k)$, that is, $\alpha^n\in k$. This completes the proof.
\end{proof}

\begin{theorem}[Additive Hilbert's Theorem 90]
    Let $K/k$ be a cyclic Galois extension with $\Gal(K/k) = \langle\sigma\rangle$ and $\beta\in K$. Then $\Tr^K_k(\beta) = 0$ iff there is $\alpha\in K$ such that $\beta = \alpha - \sigma(\alpha)$.
\end{theorem}
\begin{proof}
    Due to \thref{cor:non-zero-trace-exists}, there is some $\theta\in K$ with $\Tr^K_k(\theta)\ne0$. Consider $\alpha\in K$ given by 
    \begin{equation*}
        \alpha = \frac{1}{\Tr^K_k(\theta)}\left(\beta\sigma(\theta) + (\beta + \sigma(\beta))\sigma^2(\theta) + \dots + (\beta + \dots + \sigma^{n - 2}(\beta))\sigma^{n - 1}(\theta)\right).
    \end{equation*}
    We have 
    \begin{align*}
        \sigma(\alpha) &= \frac{1}{\Tr^K_k(\theta)}\left(\sigma(\beta)\sigma^2(\theta) + (\sigma(\beta) + \sigma^2(\beta))\sigma^3(\theta) + \dots + (\sigma(\beta) + \dots + \sigma^{n - 1}(\beta))\sigma^n(\theta)\right)\\
        &= \alpha - \beta\frac{1}{\Tr^K_k(\theta)}\left(\sigma(\theta) + \dots + \sigma^n(\theta)\right)\\
        &= \alpha - \beta
    \end{align*}

    The converse is trivial.
\end{proof}

\begin{theorem}[Artin-Schreier]
    Let $k$ be a field of characteristic $p > 0$.
    \begin{enumerate}[label=(\alph*)]
        \item Let $K/k$ be a cyclic extension of degree $p$. Then there is $\alpha\in K$ such that $K = k(\alpha)$ and $\alpha$ is a root of $f(x) = x^p - x - a$ for some $a\in k$. Further, $K$ is the splitting field of $f(x)$ over $k$.

        \item Conversely, if $a\ne b^p - b$ for some $b\in k$, and $K$ is the splitting field of $f(x) = x^p - x - a\in k[x]$, then $f(x)$ is irreducible and $K/k$ is cyclic of degree $p$.
    \end{enumerate}
\end{theorem}
\begin{proof}
\begin{enumerate}[label=(\alph*)]
    \item Let $\Gal(K/k) = \langle\sigma\rangle$, since it is a group of prime order. We have $\Tr^K_k(-1) = p\cdot(-1) = 0$ whence there is $\alpha\in K$ such that $-1 = \alpha - \sigma(\alpha)$, equivalently, $\sigma(\alpha) = \alpha + 1$. Let $a = \alpha^p - \alpha$. Then,
    \begin{equation*}
        \sigma(a) = \sigma(\alpha^p - \alpha) = \sigma(\alpha)^p - (\alpha + 1) = \alpha^p + 1 - (\alpha + 1) = a.
    \end{equation*}
    Thus, $\sigma^n(a) = a$ for $1\le n\le p$, consequently, $a\in K^{\Gal(K/k)} = k$. 

    Note that for $1\le m\ne n\le p$, we have 
    \begin{equation*}
        \sigma^m(\alpha) = \alpha + m\ne\alpha + n = \sigma^n(\alpha).
    \end{equation*}
    Thus, $p\le [k(\alpha):k]_s\le[k(\alpha):k]\le[K:k] = p$ whence $[k(\alpha):k] = p$ and $K = k(\alpha)$.

    \item Let $\alpha\in K$ be a root of $f(x)$. Then, so is $\alpha + 1$. Hence, all the roots of $f(x)$ in $K$ are given by 
    \begin{equation*}
        \{\alpha,\alpha + 1,\dots,\alpha + p - 1\},
    \end{equation*}
    whence $K = k(\alpha)$. Suppose $f(x) = g_1(x)\cdots g_r(x)$ where $g_1,\dots,g_r\in k[x]$ are irreducible polynomials. If $r$ is a root of some $g_i$, then $r$ is a root of $f$ and thus $K = k(r)$. In particular, $\deg g_i = [K:k]$. This gives us $r\deg g_1 = p$ and since $f(x)$ does not have a root in $k$, we must have $r = 1$ and $\deg g_1 = p$. That is, $f(x)$ is irreducible.

    Finally, $\Gal(K/k) = \langle\sigma\rangle$ where $\sigma(\alpha) = \alpha + 1$. This completes the proof.\qedhere
\end{enumerate}
\end{proof}

\subsection{Lagrange Resolvents}

Let $p > 0$ be a prime number and $k$ a field such that $\chr k = 0$ or $\gcd(\chr k, p) = 1$. Suppose further, that $\mu_p\subseteq k$, that is, $k$ contains a primitive $p$-th root of unity. Now let $K/k$ be a cyclic extension of order $p$. Using \thref{thm:structure-cyclic-extension}, there is some $a\in k$ such that $K = k(\sqrt[p]{a})$. We shall explicitly find such an $a\in k$.

Let $\alpha\in K$ be primitive for the extension $K/k$ and $\Gal(K/k) = \langle\sigma\rangle$. If $m_\alpha(x)$ is the minimum polynomial of $\alpha$ over $k$, then the roots of $m_\alpha$ are given by $\{\alpha,\sigma(\alpha),\dots,\sigma^{p - 1}(\alpha)\}$ and of course, are distinct. Let $\mu_p = \{z_1,\dots,z_p\}\subseteq k$. Define 
\begin{equation*}
    (z_i, \alpha) := \sum_{j = 0}^{p - 1}\sigma^j(\alpha)z_i^{j}.
\end{equation*}
These are called the \emph{Lagrange Resolvents}.

Then, 
\begin{equation*}
    \begin{bmatrix}
        (z_1,\alpha)\\\vdots\\(z_p,\alpha)
    \end{bmatrix}
    = 
    \underbrace{
    \begin{bmatrix}
        1 & z_1 & \dots & z_1^{p - 1}\\
        \vdots & \vdots & \ddots & \vdots \\
        1 & z_p & \dots & z_p^{p - 1}
    \end{bmatrix}
    }_{V(z_1,\dots,z_p)}
    \begin{bmatrix}
        \alpha\\\vdots\\\sigma^{p - 1}(\alpha)
    \end{bmatrix}.
\end{equation*}
The Vandermonde determinant, $\det V(z_1,\dots,z_p)$ is nonzero and hence, the matrix is invertible. Note that 
\begin{equation*}
    \sigma((z_i,\alpha)) = z_i^{-1} (z_i,\alpha),
\end{equation*}
whence $(z_i,\alpha)$ is an eigenvector corresponding to the eigenvalue $z_i^{-1}$. In particular, $(z_i,\alpha)^p$ is invariant under $\sigma$ and thus lies in the base field $k$. This shows that $K = k((z_i,\alpha))$.

\section{Solvability by Radicals}

\begin{definition}
    An extension $K/k$ is said to be \emph{radical} if there is a tower 
    \begin{equation*}
        k = F_0\subseteq F_1\subseteq\cdots\subseteq F_n = K
    \end{equation*}
    where $F_{i + 1}/F_i$ is obtained by adjoining an $n_i$-th root of an element in $F_i$. Each $F_{i + 1}/F_i$ is called a \emph{simple radical extension}.
\end{definition}

\begin{definition}
    A polynomial $f(x)\in k[x]$ is said to be \emph{solvable by radicals} if any splitting field $K$ of $f$ over $k$ is contained in a radical extension of $k$.
\end{definition}

\begin{lemma}\thlabel{lem:radical-in-galois-radical}
    Let $E/k$ be a finite separable radical extension. Then, the normal closure, $K$ of $E$ is a radical Galois extension.
\end{lemma}
\begin{proof}
    Fix some algebraically closed field $k^a$ containing $k$ and let 
    \begin{equation*}
        k = F_0\subseteq F_1\subseteq\dots\subseteq F_m = E
    \end{equation*}
    be a tower of simple radical extensions. Let $\{\id = \sigma_1,\dots,\sigma_n\}$ be the distinct $k$-embeddings of $E/k$ into $k^a$. Then, note that $\sigma_j(F_{i + 1})/\sigma_j(F_i)$ is also a simple radical extension. Thus, we have a tower of successive simple radical extensions
    \begin{equation*}
        k = \sigma_1(F_0)\subseteq\dots\subseteq\sigma_1(F_m)\subseteq\sigma_1(F_m)\sigma_1(F_0)\subseteq\dots\subseteq\sigma_1(F_m)\dots\sigma_n(F_m) = K.
    \end{equation*}
    This completes the proof.
\end{proof}

\begin{theorem}[Galois]
    Let $\chr k = 0$ and $f(x)\in k[x]$. Then, $f(x)$ is solvable by radicals over $k$ if and only if $G_f$ is a solvable group.
\end{theorem}
\begin{proof}
    $\implies$ Let $K$ be the splitting field of $f$ over $k$, which is contained in a radical extension $E$. Due to \thref{lem:radical-in-galois-radical}, we may suppose that $E/k$ is Galois. There is a tower of extensions 
    \begin{equation*}
        k = F_0\subseteq\dots\subseteq F_r = E.
    \end{equation*}
    with $F_{i + 1} = F_i\left(\sqrt[n_{i + 1}]{a_{i + 1}}\right)$. Let $n = n_1\cdots n_{r}$ and $\zeta$ a primitive $n$-th root of unity. Note that $E(\zeta) = E\cdot k(\zeta)$, a compositium of two Galois extensions over $k$ whence is a Galois extension of $k$. Denote by $M_i = F_i(\zeta)$. Then, we have 
    \begin{equation*}
        k \subseteq M_0\subseteq\dots\subseteq M_r = E(\zeta).
    \end{equation*}
    Note that $M_i$ contains a primitive $n_{i + 1}$-th root of unity (which is a suitable power of $\zeta$) whence $\Gal(M_{i + 1}/M_i)$ is cyclic. Consider the chain of subgroups 
    \begin{equation*}
        \Gal(M_r/k)\supseteq\Gal(M_r/M_0)\supseteq\dots\supseteq\Gal(M_r/M_{r - 1})\supseteq\{1\}.
    \end{equation*}
    Each successive quotient is 
    \begin{equation*}
        \Gal(M_r/M_i)/\Gal(M_r/M_{i + 1})\cong\Gal(M_{i + 1}/M_i)\quad\text{ and }\quad\Gal(M_r/k)/\Gal(M_r/M_0)\cong\Gal(M_0/k),
    \end{equation*}
    all of which are abelian. Thus, $\Gal(M_r/k)$ is solvable, consequently, 
    \begin{equation*}
        G_f = \Gal(K/k)\cong\Gal(M_r/k)/\Gal(M_r/K),
    \end{equation*}
    is solvable.

    $\impliedby$ Let $|G_f| = n$ and $\zeta$ a primitive $n$-th root of unity in $k^a$. Let $L = K(\zeta)$ and $E = k(\zeta)$. Then, $L/E$ is a Galois extension with Galois group isomorphic to a subgroup of $\Gal(K/k)$, in particular, $\Gal(L/E)$ is solvable. Thus, there is a series 
    \begin{equation*}
        \Gal(L/E) = H_0 \supseteq H_1\supseteq\cdots\supseteq H_m = \{1\}
    \end{equation*}
    with $H_{i}/H_{i + 1}$ abelian. Let $F_i = L^{H_i}$. This gives a filtration 
    \begin{equation*}
        E = F_0\subseteq F_1\subseteq\dots\subseteq F_m = L
    \end{equation*}
    wherein each extension $F_{i + 1}/F_i$ is abelian with degree $n_i$ dividing $n$. Let $\Gal(F_{i + 1}/F_i) = P$, an abelian group whence, due to the structure theorem, admits a filtration 
    \begin{equation*}
        P = Q_0\supseteq Q_1\supseteq\dots\supseteq Q_r = \{1\}.
    \end{equation*}
    such that $Q_i/Q_{i + 1}$ is cyclic. Let $S_i = P^{Q_i}$. Then, we have a filtration 
    \begin{equation*}
        F_i = S_0\subseteq S_1\subseteq\dots\subseteq S_r = F_{i + 1}
    \end{equation*}
    where each extension $S_{j + 1}/S_j$ is cyclic with order dividing $n$. But since $S_j$ contains a primitive $n$-th root of unity, the extension $S_{j + 1}/S_j$ must be a simple radical extension. In particular, $F_{i + 1}/F_i$ is a radical extension. Consequently, $L/E$ is a radical extension. Finally, $E/k$ itself is a simple radical extension and hence, $L/k$ is a radical extension containing $K/k$. This completes the proof.
\end{proof}

\section{Kummer Extensions}

\begin{definition}
    A finite algebraic extension $K/k$ is said to be a \emph{Kummer extension} if $\mu_n\subseteq F$, there is $n\in\N$ and $a_i\in k$ for $1\le i\le m$ such that $K = k(\sqrt[n]{a_1},\dots,\sqrt[n]{a_m})$. A Kummer extension is said to be a \emph{simle Kummer extension} if $m = 1$.
\end{definition}

\begin{theorem}
    Let $\mu_n\subseteq k$ and $a\in k^\times$. Let $b\in k^a$ such that $b^n = a$. Then, $\Gal(k(b)/k)$ is cyclic of order $|\overline a|$ where $\overline a$ is the coset of $a$ in $k^\times/(k^\times)^n$.
\end{theorem}
\begin{proof}

\end{proof}

\begin{remark}
    Due to \thref{thm:structure-cyclic-extension}, every simple Kummer extension $K/k$ with $[K:k] = m$ can be obtained by adjoining th $m$-th root of some element in $k$. This makes our analysis a lot easier.
\end{remark}

\begin{lemma}
    Let $\mu_n\subseteq k$ and $a,b\in k^\times$ such that $[k(\sqrt[n]{a}): k] = [k(\sqrt[n]{b}) : k] = n$. Then, these extensions are $k$-isomorphic if and only if $\langle\overline a\rangle = \langle\overline b\rangle$ in $k^\times/(k^\times)^n$.
\end{lemma}
\begin{proof}
    
\end{proof}

\begin{theorem}
    Let $K/k$ be a finite abelian extension and suppose that $\mu_n\subseteq k$. Then, $\Gal(K/k)$ has exponent $n$ if and only if there are $b_1,\dots,b_m\in k^\times$ such that $K = k(\sqrt[n]{b_1},\dots,\sqrt[n]{b_m})$.
\end{theorem}
\begin{proof}
    $\implies$ Due to the structure thoerem, $\Gal(K/k)\cong\Z/n_1\Z\oplus\dots\oplus\Z/n_r\Z$ where $n_i\mid n$. Let $H_i$ denote the subgroup corresponding to 
    \begin{equation*}
        \Z/n_1\Z\oplus\dots\oplus\widehat{\Z/n_i\Z}\oplus\dots\oplus\Z/n_r\Z
    \end{equation*}
    and $F_i = K^{H_i}$. Then, $\bigcap_{i = 1}^r H_i = \{1\}$ and $\Gal(F_i/k)\cong\Z/n_i\Z$. Due to \thref{thm:structure-cyclic-extension}, there is some $b_i\in k^\times$ such that $F_i = k(\sqrt[n]{b_i})$. Finally, since $K = F_1\cdots F_r$, the conclusion follows.

    $\impliedby$ Let $F_i = k(\sqrt[n]{b_i})$. Then, $\Gal(F_i/k)$ is cyclic of exponent $n$. Let $\rho_i: \Gal(K/k)\onto\Gal(F_i/k)$ denote the restriction map. It is not hard to see that the map $\Phi:\Gal(K/k)\to\prod_{i = 1}^m\Gal(F_i/k)$ given by $\Phi = \rho_1\times\dots\times\rho_m$ is an injection and thus $\Gal(K/k)$ is abelian of exponent $n$. This completes the proof.
\end{proof}

\chapter{Infinite Galois Theory}
\section{Galois Groups as Inverse Limits}
\subsection{Inverse Limit of Topological Groups}

\begin{lemma}
    Let $G$ be a compact topological group. Then, $H\le G$ is open if and only if it is closed with finite index.
\end{lemma}
\begin{proof}
    Since $G$ is compact, the number of cosets of $H$ in $G$ must be finite else we would have an infinite open cover of $G$ with no finite subcover. Further, $H$ is the complement of a disjoint union of cosets of $H$ and hence, is closed, since every coset of $H$ in $G$ is open. 

    Conversely, if $H,\sigma_1H,\dots,\sigma_n H$ are the distinct cosets of $H$ in $G$, then $H = G\backslash(\sigma_1 H\cup\dots\cup\sigma_n H)$, and thus, is open.
\end{proof}

\subsection{Profinite Groups}

\begin{definition}[Profinite Group]
    A profinite group is a topological group that is isomorphic to an inverse limit of finite topological groups with the discrete topology. 

    The \emph{profinite completion} of a topological group $G$ is defined as $\wh G = \limit G/N$ where $N$ ranges over the set of all open normal subgroups of finite index in $G$. If no topology is specified on the group, then $\wh G$ refers to the profinite completion of $G$ with the discrete topology.
\end{definition}
\begin{remark}
    Note that if $N$ is an open normal subgroup of a topological group $G$, then $G/N$ has the discrete topology even if $G$ is not Hausdorff.
\end{remark}

\begin{theorem}
    A profinite group is a compact Hausdorff topological group.
\end{theorem}
\begin{proof}
    
\end{proof}

\begin{theorem}
    Let $G$ be a topological group. Let $\phi: G\to\wh G$ denote the natural map. Then, the image of $\phi$ is dense in $\wh G$. If $G$ is a profinite group, then $\phi$ is an isomorphism of topological groups.
\end{theorem}
\begin{proof}
    Let $X = \prod G/N$, which is a compact topological group containing $\wh G$. Let $U$ be a basic open set in $X$.
\end{proof}

\subsection{The Galois Group}

We shall now show that every profinite group occurs as a Galois group. In order to do so, we shall require the following analogue of Artin's Theorem for profinite groups. 

\begin{theorem}\thlabel{thm:extension-artin-theorem}
    Let $G$ be a profinite group acting faithfully by automorphisms on a field $K$ such that for each $x\in K$, $\stab_G(x)$ is an open subgroup of $G$. Then, $K/K^G$ is Galois with group $G$.
\end{theorem}
\begin{proof}
    
\end{proof}

\begin{theorem}[Waterhouse]
    Let $G$ be a profinite group. Then, it is the Galois group of some field extension.
\end{theorem}
\begin{proof}
    Let $\mathcal H$ denote the set of all open subgroups of $G$. Define
    \begin{equation*}
        X = \bigsqcup_{H\in\mathcal H}G/H
    \end{equation*}
    and let $G$ act on $X$ through left multiplication on cosets. This action is faithful and every element of $X$ has an open stabilizer in $G$. Let $K = \Q(X)$ and extend the action of $G$ on $X$ to an action by field automorphisms on $K$. Due to \thref{thm:extension-artin-theorem}, $G\cong\Gal(K/K^G)$.
\end{proof}

\section{The Krull Topology}

\begin{definition}
    Let $K/k$ be a Galois extension. For $\sigma\in\Gal(K/k)$, a \textit{basic open set} around $\sigma$ is a coset $\sigma\Gal(K/F)$ where $F/k$ is a \textbf{finite Galois} extension.
\end{definition}

\begin{proposition}
    The collection of basic open sets as defined above form a basis for a topology on $\Gal(K/k)$.
\end{proposition}
\begin{proof}
    Since $\Gal(K/F)$ contains the identity element for each $F/k$ finite Galois, the union of all the basic open sets is equal to $\Gal(K/k)$. Consider two basic open sets $\sigma_1\Gal(K/F_1)$ and $\sigma_2\Gal(K/F_2)$ having a nonempty intersection. Let $\sigma$ be an automorphism in that intersection. We shall show that the basic open set $\sigma\Gal(K/F_1F_2)$ is contained in the intersection. Since $\sigma\in\sigma_1\Gal(K/F_1)$, there is $\alpha\in\Gal(K/F_1)$ such that $\sigma = \sigma_1\alpha$. Let $\tau\in\sigma\Gal(K/F_1F_2)$, then there is $\beta\in\Gal(K/F_1F_2)$ such that $\tau = \sigma\beta$. Now, $\sigma_1^{-1}\tau = \alpha\beta\in\Gal(K/F_1)$, whence $\tau\in\sigma_1\Gal(K/F_1)$. This completes the proof.
\end{proof}

The topology defined above is known as the \textbf{Krull Topology}.

\begin{theorem}
    The Krull Topology on $\Gal(K/k)$ makes it a topological group.
\end{theorem}
\begin{proof}
    We must show that the multiplication map and the inversion map are continuous. Let $G = \Gal(K/k)$ and $\varphi: G\times G\to G$ be given by $(x,y)\mapsto xy$. Let $U$ be an open set in $G$ and $(\sigma,\tau)\in\varphi^{-1}(U)$. Then there is a basic open set of the form $\sigma\tau\Gal(K/F)$ for some finite Galois extension $F/k$. Consider the basic open set $\sigma\Gal(K/F)\times\tau\Gal(K/F)$ that contains $(\sigma,\tau)$. I claim that the image of this basic open set lies inside $\sigma\tau\Gal(K/F)$. Indeed, for $(\sigma\alpha,\tau\beta)$ in the basic open set, its image is $\sigma\alpha\tau\beta = \sigma\tau\alpha'\beta = \sigma\tau\gamma$ for some $\gamma\in\Gal(K/F)$. Where we used the normality of $\Gal(K/F)$ in $G$ since the extension is normal. Thus $\varphi$ is continuous.

    Let $\psi: G\to G$ be the inversion map, that is, $x\mapsto x^{-1}$. We use a similar strategy as above. Let $U$ be an open set containing $\sigma^{-1}$ for some $\sigma\in G$. Then, there is a basic open set $\sigma^{-1}\Gal(K/F)$ that is contained in $U$. Thus, $\Gal(K/F)$ is normal in $G$. As a result, under $\psi$, $\sigma\Gal(K/F)$ maps to $\sigma^{-1}\Gal(K/F)$. This completes the proof.
\end{proof}

\begin{proposition}
    $\Gal(K/k)$ under the Krull Topology is Hausdorff.
\end{proposition}
\begin{proof}
    Let $\sigma,\tau\in\Gal(K/k)$ be distinct elements. Then, there is $\alpha\in K$ such that $\sigma(\alpha)\ne\tau(\alpha)$. Let $F$ be the normal closure of $k(\alpha)$ in $K$, which is a finite Galois extension, and note that $\sigma\Gal(K/F)\ne\tau\Gal(K/F)$ and thus must be disjoint (since they are cosets).  
\end{proof}

\begin{proposition}
    Let $K/k$ be a Galois extension and $E$ an intermediate field. Then $\Gal(K/E)$ is a closed subgroup of $\Gal(K/k)$.
\end{proposition}
\begin{proof}
    Let $\sigma\in G\backslash\Gal(K/E)$. Then $\sigma\Gal(K/E)$ is a basic open set containing $\sigma$ and disjoint from $\Gal(K/E)$ (since it is a coset). This implies the desired conclusion.
\end{proof}

\begin{proposition}
    Let $H\le G = \Gal(K/k)$. Then $\Gal(K/K^H)$ is the closure of $H$ in $G$.
\end{proposition}
\begin{proof}
    Obviously, $H\subseteq\Gal(K/K^H)$. Further, since the latter is closed, $\overline{H}\subseteq\Gal(K/K^H)$. We shall show the reverse inclusion. Let $\sigma\in G\backslash\overline H$. As we have seen earlier, there is a finite Galois extension $F/k$ such that the basic open set $\sigma\Gal(F/k)$ is disjoint from $\overline H$. We claim that there is $\alpha\in F$ such that $\alpha$ is fixed under $H$ but not under $\sigma$. Suppose there is no such $\alpha$. Then, $\sigma|_F$ fixes $F^{H\vert_F}$ where $H\vert_F = \{h\vert_F : h\in H\}$. From finite Galois theory, we know that $\sigma|_F\in H|_F$. And thus, there is some $h\in H$ such that $\sigma|_F = h|_F$, consequently, $\sigma\Gal(K/F) = h\Gal(K/F)$, a contradiction. 

    Since there is some $\alpha\in F$ that is not fixed by $\sigma$ but fixed under $H$, we must have that $\sigma\notin\Gal(K/K^H)$. This completes the proof.
\end{proof}

\begin{theorem}[Krull]
    Let $K/k$ be Galois and equip $G = \Gal(K/k)$ with the Krull topology. Then 
    \begin{enumerate}[label=(\alph*)]
        \item For all intermediate fields $E$, $\Gal(K/E)$ is a closed subgroup of $G$.
        \item For all $H\le G$, $\Gal(K/K^H)$ is the closure of $H$ in $G$.
        \item (The Galois Correspondence) There is an inclusion reversing bijection between the intermediate fields of $K/k$ an closed subgroups of $\Gal(K/k)$.
        \item For an arbitrary subgroup $H$ of $G$, $K^H = K^{\overline H}$.
    \end{enumerate}
\end{theorem}
\begin{proof}
    (a) and (b) follow from the previous two propositions. From this, the Galois correspondence is immediate. Finally to see (d), suppose $H\le G$. Then, $\Gal(K/K^H) = \overline H$, whence 
    \begin{equation*}
        K^H = K^{\Gal(K/K^H)} = K^{\overline H}.
    \end{equation*}
    This completes the proof.
\end{proof}

\begin{theorem}
    $\Gal(K/k)$ in the Krull Topology is isomorphic, as topological groups to the inverse limit $G = \limit\Gal(E/k)$ as a subspace of $X = \prod\Gal(E/k)$, each of which is given the discrete topology. 

    In particular, $\Gal(K/k)$ in the Krull Topology is a profinite group.
\end{theorem}
\begin{proof}
    Define the map $\Phi:\Gal(K/k)\to X$ by $\Phi(\sigma) = (\sigma|_E)_{E}$. This is obviously an injective map whose image is $G$. To see that this is a continuous map, it suffices to check that each component of this map is continuous. Let $E/k$ be a finite Galois extension. The component of $\Phi$ along $E$ is given by $\Phi_E:\Gal(K/k)\to\Gal(E/k)$, which is the restriction map. A basic open set in $\Gal(E/k)$ is simply a point, say $\sigma\in\Gal(E/k)$. Then, $\Phi_E^{-1}(\sigma) = \tau\Gal(K/E)$ where $\tau$ is a $k$-automorphism of $K$ whose restriction to $E$ is $\sigma$. This is obviously an open set in $\Gal(K/k)$ whence $\Phi$ is continuous.

    Lastly, we must show that $\Phi$ is an open map with respect to $G$, for which, it suffices to show that the image of a basic open set in $\Gal(K/k)$ is open in $G$. Consider the basic open set $\sigma\Gal(K/E)$ where $E/k$ is a finite Galois extension. Then, 
    \begin{equation*}
        \Phi\left(\sigma\Gal(K/E)\right) = \left(\{\sigma_E\}\times\prod_{\substack{F\ne E\\F/k\text{ is finite Galois}}}\Gal(F/k)\right)\cap G,
    \end{equation*}
    which is open in $G$. This completes the proof.
\end{proof}

\begin{corollary}
    $\Gal(K/k)$ is compact in the Krull topology.
\end{corollary}

\bibliographystyle{alpha}
\bibliography{../references}
\end{document}