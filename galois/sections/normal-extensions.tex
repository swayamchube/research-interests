\begin{definition}[Splitting Field]
    Let $k$ be a field and $\{f_i\}_{i\in I}$ be a family of polynomials in $k[x]$. By a \textit{splitting field} for this family, we shall mean an extension $K$ of $k$ such that every $f_i$ splits in linear factors in $K[x]$ and $K$ is generated by all the roots of all the polynomials $f_i$ for $i\in I$ in some algebraic closure $\overline{k}$.
\end{definition}

In particular, if $f\in k[x]$ is a polynomial, then the splitting field of $f$ over $k$ is an extension $K/k$ such that $f$ splits into linear factors in $K$ and $K$ is generated by all the roots of $f$.

\begin{definition}[Normal Extension]
    An algebraic extension $K/k$ is said to be \textit{normal} if whenever an irreducible polynomial $f(x)\in k[x]$ has a root in $K$, it splits into linear factors over $K$.
\end{definition}

\begin{theorem}[Uniqueness of Splitting Fields]\thlabel{thm:uniqueness-splitting-field}
    Let $K$ be a splitting field of the polynomial $f(x)\in k[x]$. If $E$ is another splitting field of $f$, then there exists an isomorphism $\sigma: E\to K$ inducing the identity on $k$. If $k\subseteq K\subseteq\overline{k}$, where $\overline{k}$ is an algebraic closure of $k$, then any embedding of $E$ in $\overline{k}$ inducing the identity on $k$ must be an isomorphism of $E$ on $K$.
\end{theorem}
\begin{proof}
    We prove both assertions together. Due to \thref{thm:extension-theorem}, there is an embedding $\sigma: E\to\overline{k}$ such that $\sigma\mid_k = \mathbf{id}_k$. Therefore, it suffices to prove the second half of the theorem.

    We have two factorizations 
    \begin{align*}
        f(x) &= c(x - \alpha_1)\cdots(x - \alpha_n)\qquad\text{over $E$}\\
        &= c(x - \beta_1)\cdots(x - \beta_n)\qquad\text{over $K$}
    \end{align*}

    Since $\sigma$ induces the identity map on $k$, $f$ must remain invariant under $\sigma$. Further, we have 
    \begin{equation*}
        \sigma f(x) = c(x - \sigma\beta_1)\cdots(x - \sigma\beta_n)
    \end{equation*}
    Due to unique factorization, we must have that $(\sigma\beta_1,\ldots,\sigma\beta_n)$ differs from $(\alpha_1,\ldots,\alpha_n)$ by a permutation. Since $\sigma E = k(\sigma\beta_1,\ldots,\sigma\beta_n)$, we immediately have the desired conclusion.
\end{proof}

\begin{theorem}\thlabel{thm:normal-equivalence}
    Let $K/k$ be algebraic in some algebraic closure $\overline{k}$ of $k$. Then, the following are equivalent: 
    \begin{enumerate}
        \item Every embedding $\sigma$ of $K$ in $\overline{k}$ over $k$ is an automorphism of $K$ 
        \item $K$ is the splitting field of a family of polynomials in $k[x]$
        \item $K/k$ is normal
    \end{enumerate}
\end{theorem}
\begin{proof}
\hfill 

\noindent\underline{$(1)\Longrightarrow(2)\wedge(3)$:} For each $\alpha\in K$, let $m_\alpha(x)$ denote the minimal polynomial for $\alpha$ over $k$. We shall show that $K$ is the splitting field for $\{m_\alpha\}_{\alpha\in K}$. Obviously, $K$ is generated by $\{\alpha\}_{\alpha\in K}$, hence, it suffices to show that $m_\alpha$ splits into linear factors over $K$. Let $\beta$ be a root of $m_\alpha$ in $\overline{k}$. Then, there is an isomorphism $\sigma: k(\alpha)\to k(\beta)$. One may extend this to an embedding $\sigma: K\to\overline{k}$, which by our hypothesis, is an automorphism of $K$, implying that $\beta\in K$ and giving us the desired conclusion.

\noindent\underline{$(2)\Longrightarrow(1)$:} Let $K$ be the splitting field for the family of polynomials $\{f_i\}_{i\in I}$. Let $\alpha\in K$ and $\alpha$ be the root of some polynomial $f_i$ and $\sigma: K\to k^a$ be an embedding of fields. Since $f_i$ remains invariant under $\sigma$, it must map a root of $f_i$ to another toot of $f_i$, that is, $\sigma\alpha$ is a root of $f_i$. Consequently, $\sigma$ maps $K$ into $K$. Now, due to \thref{lem:self-embedding-is-automorphism}, $\sigma$ is an automorphism and $K/k$ is normal.

\noindent\underline{$(3)\Longrightarrow(1)$:} Let $\sigma: K\to\overline{k}$ be an embedding of fields. Let $\alpha\in K$ and $p(x)\in k[x]$ be its irreducible polynomial over $k$. Since $p$ remains invariant under $\sigma$, it must map $\alpha$ to a root $\beta$ of $p$ in $\overline{k}$. But since $p$ splits into linear factors over $K$, $\beta\in K$ and thus $\sigma(K)\subseteq K$, consequently, $\sigma(K) = K$ due to \thref{lem:self-embedding-is-automorphism}, therefore completing the proof.
\end{proof}


\begin{corollary}
    The splitting field of a polynomial is a normal extension.
\end{corollary}

\begin{theorem}\thlabel{thm:normal-properties}
    Normal extensions remain normal under lifting. If $k\subseteq E\subseteq K$, and $K$ is normal over $k$, then $K$ is normal over $E$. If $K_1,K_2$ are normal over $k$ and are contained in some field $L$, then $K_1K_2$ is normal over $k$ and so is $K_1\cap K_2$.
\end{theorem}
\begin{proof}
    Let $K/k$ be normal and $F/k$ be any extension with $K$ and $F$ contained in some larger extension. Let $\sigma$ be an embedding of $KF$ over $F$ in $\overline{F}$. The restriction of $\sigma$ to $K$ is an embedding of $K$ over $k$ and therefore, is an automorphism of $K$. As a result, $\sigma(KF) = (\sigma K)(\sigma F) = KF$ and thus $KF/F$ is normal.

    Now, suppose $k\subseteq E\subseteq K$ with $K/k$ normal. Let $\sigma$ be an embedding of $K$ in $\overline{k}$ over $E$. Then, $\sigma$ induces the identity on $k$ and is therefore an automorphism of $K$. This shows that $K/E$ is normal.

    Next, if $K_1$ and $K_2$ are normal over $k$ and $\sigma$ is an embedding of $K_1K_2$ over $k$, then its restriction to $K_1$ and $K_2$ respectively are also embeddings over $k$ and consequently are automorphisms. This gives us 
    \begin{equation*}
        \sigma(K_1K_2) = (\sigma K_1)(\sigma K_2) = K_1K_2
    \end{equation*}

    Finally, since any embedding of $K_1\cap K_2$ can be extended to that of $K_1K_2$, we have, due to a similar argument, that $K_1\cap K_2$ is normal over $k$.
\end{proof}