\begin{definition}[Root of Unity]
    Let $k$ be a field. A \textit{root of unity} over $k$ is an element $\zeta\in k^a$ such that $\zeta^n = 1$ for some $n\in\N$.
\end{definition}

\begin{mdframed}
    Consider the polynomial $x^n - 1$ with $\gcd(\chr k, n) = 1$. In this case, the polynomial is separable over $k$ and thus has distinct roots. Let $Z_n = \{z_1,\dots,z_n\}$ denote the distinct roots. It is not hard to see that $Z_n\subseteq k^\times$ forms a multiplicative group. Since every finite multiplicative subgroup of a field is cyclic, so is $Z_n$. A generator for the group $Z_n$ is called a \textbf{primitive $n$-th root of unity}. Obviously, there are $\varphi(n)$ such primitive $n$-th roots of unity.

    Consider now the case $\gcd(\chr k, n)\ne 1$. Let $\chr k = p > 0$. Then, there is a positive integer $r$ such that $n = p^rm$ with $p\nmid m$. Then, 
    \begin{equation*}
        x^n - 1 = \left(x^m - 1\right)^{p^r}
    \end{equation*}
    and thus every $n$-th root of unity is an $m$-th root of unity, whence it suffices to study polynomials of the form $(x^n - 1)$ with $\gcd(\chr k, n) = 1$.
\end{mdframed}

\begin{proposition}
    Every root of unity is a primitive $n$-th root of unity for some positive integer $n$.
\end{proposition}
\begin{proof}
    Let $\zeta$ be a root of unity and let $n$ be the smallest positive integer such that $\zeta^n = 1$. Consider the subgroup $\langle\zeta\rangle\le Z_n$. If the order of this subgroup is $m$, then $\zeta^m = 1$ whence $m\ge n$ and thus $m = n$ and the conclusion follows.
\end{proof}

As a result, need only concern ourselves with primitive $n$-th roots of unity with $\gcd(\chr k, n) = 1$.

\begin{proposition}
    Let $k$ be a field and $\zeta_n\in k^a$ a primitive $n$-th root of unity such that $\gcd(n,\chr k) = 1$. Then, $k(\zeta_n)/k$ is a Galois extension.
\end{proposition}
\begin{proof}
    Since $\zeta_n$ is a generator for $Z_n$, $k(\zeta_n)$ is the splitting field of $x^n - 1$ over $k$ and thus a normal extension of $k$. Further, since $x^n - 1$ is a separable polynomial over $k$, so is the extension $k(\zeta_n)/k$ whence it is Galois.
\end{proof}

\begin{proposition}
    Let $\gcd(\chr k, n) = 1$. If $\zeta$ is a primitive $n$-th root of unity, then $k(\zeta)/k$ is an abelian extension.
\end{proposition}
\begin{proof}
    Define the map $\psi:\Gal(k(\zeta)/k)\to\Aut(\bm\mu_n)$ by $\sigma\mapsto\sigma|_{\bm\mu_n}$. Note that $\Aut(\bm\mu_n)\cong\left(\Z/n\Z\right)^\times$, further, it is not hard to see that $\psi$ is injective and the conclusion follows.
\end{proof}

Note that although we have shown $\Gal(k(\zeta)/k)$ to be embeddable into $(\Z/n\Z)^\times$, the map may not be a surjection take for example $k = \R$ and $\zeta = \exp(2\pi i/5)$. Then, $k(\zeta) = \bbC$, and $\Gal(k(\zeta)/k)\cong\{\pm 1\}$.

\begin{proposition}
    Let $\zeta$ be a primitive $n$-th root of unity over $\Q$. Then, 
    \begin{equation*}
        [\Q(\zeta):\Q] = \varphi(n)
    \end{equation*}
    and consequently, the map $\psi:\Gal(\Q(\zeta)/\Q)\to(\Z/n\Z)^\times$ is an isomorphism.
\end{proposition}
\begin{proof}

\end{proof}