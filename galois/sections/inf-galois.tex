\section{Galois Groups as Inverse Limits}
\subsection{Inverse Limit of Topological Groups}

\begin{lemma}
    Let $G$ be a compact topological group. Then, $H\le G$ is open if and only if it is closed with finite index.
\end{lemma}
\begin{proof}
    Since $G$ is compact, the number of cosets of $H$ in $G$ must be finite else we would have an infinite open cover of $G$ with no finite subcover. Further, $H$ is the complement of a disjoint union of cosets of $H$ and hence, is closed, since every coset of $H$ in $G$ is open. 

    Conversely, if $H,\sigma_1H,\dots,\sigma_n H$ are the distinct cosets of $H$ in $G$, then $H = G\backslash(\sigma_1 H\cup\dots\cup\sigma_n H)$, and thus, is open.
\end{proof}

\subsection{Profinite Groups}

\begin{definition}[Profinite Group]
    A profinite group is a topological group that is isomorphic to an inverse limit of finite topological groups with the discrete topology. 

    The \emph{profinite completion} of a topological group $G$ is defined as $\wh G = \limit G/N$ where $N$ ranges over the set of all open normal subgroups of finite index in $G$. If no topology is specified on the group, then $\wh G$ refers to the profinite completion of $G$ with the discrete topology.
\end{definition}
\begin{remark}
    Note that if $N$ is an open normal subgroup of a topological group $G$, then $G/N$ has the discrete topology even if $G$ is not Hausdorff.
\end{remark}

\begin{theorem}
    A profinite group is a compact Hausdorff topological group.
\end{theorem}
\begin{proof}
    
\end{proof}

\begin{theorem}
    Let $G$ be a topological group. Let $\phi: G\to\wh G$ denote the natural map. Then, the image of $\phi$ is dense in $\wh G$. If $G$ is a profinite group, then $\phi$ is an isomorphism of topological groups.
\end{theorem}
\begin{proof}
    Let $X = \prod G/N$, which is a compact topological group containing $\wh G$. Let $U$ be a basic open set in $X$.
\end{proof}

\subsection{The Galois Group}

We shall now show that every profinite group occurs as a Galois group. In order to do so, we shall require the following analogue of Artin's Theorem for profinite groups. 

\begin{theorem}\thlabel{thm:extension-artin-theorem}
    Let $G$ be a profinite group acting faithfully by automorphisms on a field $K$ such that for each $x\in K$, $\stab_G(x)$ is an open subgroup of $G$. Then, $K/K^G$ is Galois with group $G$.
\end{theorem}
\begin{proof}
    
\end{proof}

\begin{theorem}[Waterhouse]
    Let $G$ be a profinite group. Then, it is the Galois group of some field extension.
\end{theorem}
\begin{proof}
    Let $\mathcal H$ denote the set of all open subgroups of $G$. Define
    \begin{equation*}
        X = \bigsqcup_{H\in\mathcal H}G/H
    \end{equation*}
    and let $G$ act on $X$ through left multiplication on cosets. This action is faithful and every element of $X$ has an open stabilizer in $G$. Let $K = \Q(X)$ and extend the action of $G$ on $X$ to an action by field automorphisms on $K$. Due to \thref{thm:extension-artin-theorem}, $G\cong\Gal(K/K^G)$.
\end{proof}

\section{The Krull Topology}

\begin{definition}
    Let $K/k$ be a Galois extension. For $\sigma\in\Gal(K/k)$, a \textit{basic open set} around $\sigma$ is a coset $\sigma\Gal(K/F)$ where $F/k$ is a \textbf{finite Galois} extension.
\end{definition}

\begin{proposition}
    The collection of basic open sets as defined above form a basis for a topology on $\Gal(K/k)$.
\end{proposition}
\begin{proof}
    Since $\Gal(K/F)$ contains the identity element for each $F/k$ finite Galois, the union of all the basic open sets is equal to $\Gal(K/k)$. Consider two basic open sets $\sigma_1\Gal(K/F_1)$ and $\sigma_2\Gal(K/F_2)$ having a nonempty intersection. Let $\sigma$ be an automorphism in that intersection. We shall show that the basic open set $\sigma\Gal(K/F_1F_2)$ is contained in the intersection. Since $\sigma\in\sigma_1\Gal(K/F_1)$, there is $\alpha\in\Gal(K/F_1)$ such that $\sigma = \sigma_1\alpha$. Let $\tau\in\sigma\Gal(K/F_1F_2)$, then there is $\beta\in\Gal(K/F_1F_2)$ such that $\tau = \sigma\beta$. Now, $\sigma_1^{-1}\tau = \alpha\beta\in\Gal(K/F_1)$, whence $\tau\in\sigma_1\Gal(K/F_1)$. This completes the proof.
\end{proof}

The topology defined above is known as the \textbf{Krull Topology}.

\begin{theorem}
    The Krull Topology on $\Gal(K/k)$ makes it a topological group.
\end{theorem}
\begin{proof}
    We must show that the multiplication map and the inversion map are continuous. Let $G = \Gal(K/k)$ and $\varphi: G\times G\to G$ be given by $(x,y)\mapsto xy$. Let $U$ be an open set in $G$ and $(\sigma,\tau)\in\varphi^{-1}(U)$. Then there is a basic open set of the form $\sigma\tau\Gal(K/F)$ for some finite Galois extension $F/k$. Consider the basic open set $\sigma\Gal(K/F)\times\tau\Gal(K/F)$ that contains $(\sigma,\tau)$. I claim that the image of this basic open set lies inside $\sigma\tau\Gal(K/F)$. Indeed, for $(\sigma\alpha,\tau\beta)$ in the basic open set, its image is $\sigma\alpha\tau\beta = \sigma\tau\alpha'\beta = \sigma\tau\gamma$ for some $\gamma\in\Gal(K/F)$. Where we used the normality of $\Gal(K/F)$ in $G$ since the extension is normal. Thus $\varphi$ is continuous.

    Let $\psi: G\to G$ be the inversion map, that is, $x\mapsto x^{-1}$. We use a similar strategy as above. Let $U$ be an open set containing $\sigma^{-1}$ for some $\sigma\in G$. Then, there is a basic open set $\sigma^{-1}\Gal(K/F)$ that is contained in $U$. Thus, $\Gal(K/F)$ is normal in $G$. As a result, under $\psi$, $\sigma\Gal(K/F)$ maps to $\sigma^{-1}\Gal(K/F)$. This completes the proof.
\end{proof}

\begin{proposition}
    $\Gal(K/k)$ under the Krull Topology is Hausdorff.
\end{proposition}
\begin{proof}
    Let $\sigma,\tau\in\Gal(K/k)$ be distinct elements. Then, there is $\alpha\in K$ such that $\sigma(\alpha)\ne\tau(\alpha)$. Let $F$ be the normal closure of $k(\alpha)$ in $K$, which is a finite Galois extension, and note that $\sigma\Gal(K/F)\ne\tau\Gal(K/F)$ and thus must be disjoint (since they are cosets).  
\end{proof}

\begin{proposition}
    Let $K/k$ be a Galois extension and $E$ an intermediate field. Then $\Gal(K/E)$ is a closed subgroup of $\Gal(K/k)$.
\end{proposition}
\begin{proof}
    Let $\sigma\in G\backslash\Gal(K/E)$. Then $\sigma\Gal(K/E)$ is a basic open set containing $\sigma$ and disjoint from $\Gal(K/E)$ (since it is a coset). This implies the desired conclusion.
\end{proof}

\begin{proposition}
    Let $H\le G = \Gal(K/k)$. Then $\Gal(K/K^H)$ is the closure of $H$ in $G$.
\end{proposition}
\begin{proof}
    Obviously, $H\subseteq\Gal(K/K^H)$. Further, since the latter is closed, $\overline{H}\subseteq\Gal(K/K^H)$. We shall show the reverse inclusion. Let $\sigma\in G\backslash\overline H$. As we have seen earlier, there is a finite Galois extension $F/k$ such that the basic open set $\sigma\Gal(F/k)$ is disjoint from $\overline H$. We claim that there is $\alpha\in F$ such that $\alpha$ is fixed under $H$ but not under $\sigma$. Suppose there is no such $\alpha$. Then, $\sigma|_F$ fixes $F^{H\vert_F}$ where $H\vert_F = \{h\vert_F : h\in H\}$. From finite Galois theory, we know that $\sigma|_F\in H|_F$. And thus, there is some $h\in H$ such that $\sigma|_F = h|_F$, consequently, $\sigma\Gal(K/F) = h\Gal(K/F)$, a contradiction. 

    Since there is some $\alpha\in F$ that is not fixed by $\sigma$ but fixed under $H$, we must have that $\sigma\notin\Gal(K/K^H)$. This completes the proof.
\end{proof}

\begin{theorem}[Krull]
    Let $K/k$ be Galois and equip $G = \Gal(K/k)$ with the Krull topology. Then 
    \begin{enumerate}[label=(\alph*)]
        \item For all intermediate fields $E$, $\Gal(K/E)$ is a closed subgroup of $G$.
        \item For all $H\le G$, $\Gal(K/K^H)$ is the closure of $H$ in $G$.
        \item (The Galois Correspondence) There is an inclusion reversing bijection between the intermediate fields of $K/k$ an closed subgroups of $\Gal(K/k)$.
        \item For an arbitrary subgroup $H$ of $G$, $K^H = K^{\overline H}$.
    \end{enumerate}
\end{theorem}
\begin{proof}
    (a) and (b) follow from the previous two propositions. From this, the Galois correspondence is immediate. Finally to see (d), suppose $H\le G$. Then, $\Gal(K/K^H) = \overline H$, whence 
    \begin{equation*}
        K^H = K^{\Gal(K/K^H)} = K^{\overline H}.
    \end{equation*}
    This completes the proof.
\end{proof}

\begin{theorem}
    $\Gal(K/k)$ in the Krull Topology is isomorphic, as topological groups to the inverse limit $G = \limit\Gal(E/k)$ as a subspace of $X = \prod\Gal(E/k)$, each of which is given the discrete topology. 

    In particular, $\Gal(K/k)$ in the Krull Topology is a profinite group.
\end{theorem}
\begin{proof}
    Define the map $\Phi:\Gal(K/k)\to X$ by $\Phi(\sigma) = (\sigma|_E)_{E}$. This is obviously an injective map whose image is $G$. To see that this is a continuous map, it suffices to check that each component of this map is continuous. Let $E/k$ be a finite Galois extension. The component of $\Phi$ along $E$ is given by $\Phi_E:\Gal(K/k)\to\Gal(E/k)$, which is the restriction map. A basic open set in $\Gal(E/k)$ is simply a point, say $\sigma\in\Gal(E/k)$. Then, $\Phi_E^{-1}(\sigma) = \tau\Gal(K/E)$ where $\tau$ is a $k$-automorphism of $K$ whose restriction to $E$ is $\sigma$. This is obviously an open set in $\Gal(K/k)$ whence $\Phi$ is continuous.

    Lastly, we must show that $\Phi$ is an open map with respect to $G$, for which, it suffices to show that the image of a basic open set in $\Gal(K/k)$ is open in $G$. Consider the basic open set $\sigma\Gal(K/E)$ where $E/k$ is a finite Galois extension. Then, 
    \begin{equation*}
        \Phi\left(\sigma\Gal(K/E)\right) = \left(\{\sigma_E\}\times\prod_{\substack{F\ne E\\F/k\text{ is finite Galois}}}\Gal(F/k)\right)\cap G,
    \end{equation*}
    which is open in $G$. This completes the proof.
\end{proof}

\begin{corollary}
    $\Gal(K/k)$ is compact in the Krull topology.
\end{corollary}