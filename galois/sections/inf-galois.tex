In the infinite case, a Galois extension is defined as usual, that is, an extension which is normal and separable. The Galois group is again defined to be the group of automorphisms that fix a base field. Since our definitions of normal and separable extensions do not assume finiteness, we are in the clear. As we have seen earlier, finite-degree Galois extensions have finite Galois groups. The following proposition establishes the converse.

\begin{proposition}
    If $K/k$ is an infinite-degree Galois extension, then $\Gal(K/k)$ is an infinite group. 
\end{proposition}
\begin{proof}
    We shall prove the contrapositive. If $\Gal(K/k)$ is a finite group with cardinality $M$, then for each $\alpha\in K$, $[k(\alpha):k]\le M$, and it follows from \thref{lem:artin-lemma} that $[K:k]\le M$.
\end{proof}


\begin{definition}
    Let $K/k$ be a Galois extension. For $\sigma\in\Gal(K/k)$, a \textit{basic open set} around $\sigma$ is a coset $\sigma\Gal(K/F)$ where $F/k$ is a \textbf{finite} extension.
\end{definition}

\begin{proposition}
    The collection of basic open sets as defined above form a basis for a topology on $\Gal(K/k)$.
\end{proposition}
\begin{proof}
    Since $\Gal(K/F)$ contains the identity element for each $F/k$ finite, the union of all the basic open sets is equal to $\Gal(K/k)$. Consider two basic open sets $\sigma_1\Gal(K/F_1)$ and $\sigma_2\Gal(K/F_2)$ having a nonempty intersection. Let $\sigma$ be an automorphism in that intersection. We shall show that $\sigma\Gal(K/F_1F_2)$ is contained in the intersection. Since $\sigma\in\sigma_1\Gal(K/F_1)$, there is $\alpha\in\Gal(K/F_1)$ such that $\sigma = \sigma_1\alpha$. Let $\tau\in\sigma\Gal(K/F_1F_2)$, then there is $\beta\in\Gal(K/F_1F_2)$ such that $\tau = \sigma\beta$. Now, $\sigma_1^{-1}\tau = \alpha\beta\in\Gal(K/F_1)$, whence $\tau\in\sigma_1\Gal(K/F_1)$. This completes the proof.
\end{proof}

The topology defined above is known as the \textbf{Krull Topology}.

\begin{theorem}
    The Krull Topology on $\Gal(K/k)$ makes it a topological group.
\end{theorem}
\begin{proof}
    We must show that the multiplication map and the inversion map are continuous. Let $G = \Gal(K/k)$ and $\varphi: G\times G\to G$ be given by $(x,y)\mapsto xy$. Let $U$ be an open set in $G$ and $(\sigma,\tau)\in\varphi^{-1}(U)$. Then there is a basic open set of the form $\sigma\tau\Gal(K/F)$ for some finite extension $F/k$. Since the larger $F$ is, the smaller $\Gal(K/F)$ gets, we may suppose that $F/k$ is Galois. Consider the basic open set $\sigma\Gal(K/F)\times\tau\Gal(K/F)$ that contains $(\sigma,\tau)$. I claim that the image of this basic open set lies inside $\sigma\tau\Gal(K/F)$. Indeed, for $(\sigma\alpha,\tau\beta)$ in the basic open set, its image is $\sigma\alpha\tau\beta = \sigma\tau\alpha'\beta = \sigma\tau\gamma$ for some $\gamma\in\Gal(K/F)$. Where we used the normality of $\Gal(K/F)$ in $G$ since the extension is normal. Thus $\varphi$ is continuous.

    Let $\psi: G\to G$ be the inversion map, that is, $x\mapsto x^{-1}$. We use a similar strategy as above. Let $U$ be an open set containing $\sigma^{-1}$ for some $\sigma\in G$. Then, there is a basic open set $\sigma^{-1}\Gal(K/F)$ that is contained in $U$. We may make $F$ larger to make it a Galois extension of $k$. Thus, $\Gal(K/F)$ is normal in $G$. As a result, under $\psi$, $\sigma\Gal(K/F)$ maps to $\sigma^{-1}\Gal(K/F)$. This completes the proof.
\end{proof}

\begin{proposition}
    $\Gal(K/k)$ under the Krull Topology is Hausdorff.
\end{proposition}
\begin{proof}
    Let $\sigma,\tau\in\Gal(K/k)$ be distinct elements. Then, there is $\alpha\in K$ such that $\sigma(\alpha)\ne\tau(\alpha)$. Let $F = k(\alpha)$, and note that $\sigma\Gal(K/F)\ne\tau\Gal(K/F)$ and thus must be disjiont (since they are cosets). 
\end{proof}

We state the main theorem of this chapter below. We shall prove it in parts and not all at once. It would seem less daunting that way.

\begin{theorem}[Krull]
    Let $K/k$ be Galois and equip $G = \Gal(K/k)$ with the Krull topology. Then 
    \begin{enumerate}[label=(\alph*)]
        \item For all intermediate fields $E$, $\Gal(K/E)$ is a closed subgroup of $G$.
        \item For all $H\le G$, $\Gal(K/K^H)$ is the closure of $H$ in $G$.
        \item (The Galois Correspondence) There is an inclusion reversing bijection between the intermediate fields of $K/k$ an closed subgroups of $\Gal(K/k)$.
        \item For an arbitrary subgroup $H$ of $G$, $K^H = K^{\overline H}$.
    \end{enumerate}
\end{theorem}

\begin{proposition}
    Let $K/k$ be a Galois extension and $E$ an intermediate field. Then $\Gal(K/E)$ is a closed subgroup of $\Gal(K/k)$.
\end{proposition}
\begin{proof}
    Let $\sigma\in G\backslash\Gal(K/E)$. Then $\sigma\Gal(K/E)$ is a basic open set containing $\sigma$ and disjiont from $\Gal(K/E)$ (since it is a coset). This implies the desired conclusion.
\end{proof}

\begin{proposition}
    Let $H\le G = \Gal(K/k)$. Then $\Gal(K/K^H)$ is the closure of $H$ in $G$.
\end{proposition}
\begin{proof}
    Obviously, $H\subseteq\Gal(K/K^H)$. Further, since the latter is closed, $\overline{H}\subseteq\Gal(K/K^H)$. We shall show the reverse inclusion. Let $\sigma\in G\backslash\overline H$. As we have seen earlier, there is a finite Galois extension $F/k$ such that the basic open set $\sigma\Gal(F/k)$ is disjoint from $\overline H$. We claim that there is $\alpha\in F$ such that $\alpha$ is fixed under $H$ but not under $\sigma$. Suppose there is no such $\alpha$. Then, $\sigma|_F$ fixes $F^{H\vert_F}$ where $H\vert_F = \{h\vert_F : h\in H\}$. From finite Galois theory, we know that $\sigma|_F\in H|_F$. And thus, there is some $h\in H$ such that $\sigma|_F = h|_F$, consequently, $\sigma\Gal(K/F) = h\Gal(K/F)$, a contradiction. 

    Since there is some $\alpha\in F$ that is not fixed by $\sigma$ but fixed under $H$, we must have that $\sigma\notin\Gal(K/K^H)$. This completes the proof.
\end{proof}

\section{Galois Groups as Inverse Limits}

Let $K/k$ be a Galois extension, not necessarily finite. Let 
\begin{equation*}
    \Sigma = \left\{\Gal(F/k)\mid F/k\text{ is finite Galois}\right\}
\end{equation*}
be a poset with restriction maps
\begin{equation*}
    \pi^E_F:\Gal(E/k)\onto\Gal(F/k).
\end{equation*}
which are continuous maps between topological groups where $\Gal(E/k)$ and $\Gal(F/k)$ have the discrete topology.

This gives $\Sigma$ the implicit structure of a categorical \emph{diagram}. We contend that $\Gal(K/k)$ is the inverse limit\footnote{This is the categorical limit} over this diagram in the category of topological groups, $\catTopGrp$.

First, we shall show that there is a cone $(\Gal(K/k),\varphi)$ on the diagram $\Sigma$. Indeed, for every finite Galois subextension, define 
\begin{equation*}
    \varphi_F: \Gal(K/k)\onto\Gal(F/k)
\end{equation*}
as the restriction map $\sigma\mapsto\sigma|_F$. Recall that $\Gal(F/k)$ has the discrete topology, whereby the preimage of $\sigma\in\Gal(F/k)$ is $\sigma\Gal(K/F)$ which is a basic open set in $\Gal(K/k)$ whence the restriction map is continuous and thus a morphism in $\catTopGrp$.

It is not hard to see that the diagram 
\begin{equation*}
    \xymatrix {
        \Gal(E/k)\ar@{->>}[rr]^{\pi^E_F} & & \Gal(F/k)\\
        & \Gal(K/k)\ar@{->>}[lu]^{\varphi_E}\ar@{->>}[ru]_{\varphi_F} &
    }
\end{equation*}
commutes.

Now let $(G,\psi)$ be another cone on the diagram $\Sigma$ where $G$ is a topological group we shall show that there is a unique morphism of cones $\Phi: (G,\psi)\to(\Gal(K/k),\phi)$. That is, a unique continuous group homomorphism that makes 
\begin{equation*}
    \xymatrix {
        \Gal(E/k)\ar@{->>}[rr]^{\pi^E_F} & & \Gal(F/k)\\
        & G\ar@{.>}[d]\ar@{->}[lu]^{\psi_E}\ar@{->}[ru]_{\psi_F} &\\
        & \Gal(K/k)\ar@{->>}[luu]^{\varphi_E}\ar@{->>}[ruu]_{\varphi_F} &
    }
\end{equation*}
commute.

Pick some $g\in G$. Let $\alpha\in K$ and $L\subseteq K$ be the normal closure of $k(\alpha)$ in $K$. Then, $L/k$ is finite Galois. Now, define 
\begin{equation*}
    \sigma(\alpha) = \psi_L(g)(\alpha).
\end{equation*}

We shall show that $\sigma$ is indeed an automorphism. Let $\alpha,\beta\in K$ and $L$ be the normal closure of $k(\alpha,\beta)$ in $K$. This is a finite Galois extension of $k$ that contains the normal closures of $k(\alpha)$, $k(\beta)$ and $k(\alpha\beta)$, say $M, N, P$ respectively. Then, 
\begin{align*}
    \sigma(\alpha\beta) &= \psi_P(g)(\alpha\beta)\\
    &= \psi_L(g)(\alpha\beta)\\
    &= \psi_L(g)(\alpha)\psi_L(g)(\beta)\\
    &= \psi_M(\alpha)\psi_N(\beta)\\
    &= \sigma(\alpha)\sigma(\beta).
\end{align*}
and similarly, one may show that $\sigma(\alpha + \beta) = \sigma(\alpha) + \sigma(\beta)$ thus $\sigma\in\Hom(K,K)$ which fixes $k$.

Lastly, we must show that $\sigma$ is surjective. Let $\beta\in K$ and $N$ the normal closure of $k(\beta)$ in $K$. Then, there is some $\alpha\in N$ such that $\psi_N(g)(\alpha) = \beta$. Let $M$ be the normal closure of $k(\alpha)$ in $K$. Then $M\subseteq N$, whence 
\begin{equation*}
    \sigma(\alpha) = \psi_M(g)(\alpha) = \psi_N(g)(\alpha) = \beta.
\end{equation*}
Thus, $\sigma\in\Gal(K/k)$ and set $\Phi(g) = \sigma$.

Let $g,h\in G$, $\Phi(g) = \sigma$, $\Phi(h) = \tau$ and $\alpha\in K$. Let $M$ be the normal closure of $k(\alpha)$ in $K$. Then 
\begin{equation*}
    \Phi(gh)(\alpha) = \psi_M(gh)(\alpha) = \psi_M(g)\circ\psi_M(h)(\alpha) = \sigma\circ\tau(\alpha)
\end{equation*}
and thus $\Phi(gh) = \sigma\circ\tau$ and $\Phi$ is a group homomorphism.

Finally, we must show that $\Phi$ is continuous, for which it suffices to show that the preimage of a basic open set in $\Gal(K/k)$ is open in $G$.

Let $\sigma\in\Gal(K/k)$ and $F/k$ an intermediate finite Galois extension of $k$. We have 
\begin{align*}
    \Phi^{-1}(\sigma\Gal(K/F)) &= \{g\in G\mid \Phi(g)\in\sigma\Gal(K/F)\}\\
    &= \{g\in G\mid \Phi(g)|_F = \sigma|_F\}\\
    &= \{g\in G\mid \psi_F(g) = \sigma|_F\}\\
    &= \psi_F^{-1}(\sigma|_F)
\end{align*}
which is open in $G$ since $\Gal(F/k)$ has the discrete topology whence $\Phi$ is continuous. 

This finishes the proof and shows that $\Gal(K/k)$ is the inverse limit $\varprojlim\Gal(F/k)$, and is a profinite group since every topological group in the inverse limit is a finite group with the discrete topology.

\begin{corollary}
    $\Gal(\overline\F_p/\F_p) = \wh\Z$.
\end{corollary}