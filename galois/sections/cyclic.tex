\begin{definition}
    A Galois extension $K/k$ is said to be \emph{cyclic} if $\Gal(K/k)$ is a cyclic group. Similarly, it is said to be \emph{abelian} if $\Gal(K/k)$ is abelian.
\end{definition}

\begin{theorem}[Linear Independence of Characters]\thlabel{thm:lin-ind-characters}
    Let $G$ be a group (monoid) and $K$ a field. If $\sigma_1,\dots,\sigma_n: G\to K^\times$ are distinct group homomorphisms. Then, 
    \begin{equation*}
        c_1\sigma_1 + \cdots + c_n\sigma_n = 0 \iff c_1 = \dots = c_n = 0
    \end{equation*}
\end{theorem}

\begin{corollary}\thlabel{cor:non-zero-trace-exists}
    Let $K/k$ be a Galois extension. Then, there is $\alpha\in K$ such that $\Tr^K_k(\alpha)\ne0$.
\end{corollary}
\begin{proof}
    Suppose not. If $\Gal(K/k) = \{\sigma_1,\dots,\sigma_n\}$, then 
    \begin{equation*}
        \sigma_1 + \dots + \sigma_n = 0
    \end{equation*}
    on $K$, a contradiction to \thref{thm:lin-ind-characters}. 
\end{proof}

\begin{theorem}[Hilbert's Theorem 90]\thlabel{thm:hilbert-90}
    Let $K/k$ be a cyclic degree $n$ extension with galois group $G$. Let $\sigma\in G$ be a generator and $\beta\in K$. The norm $N^K_k(\beta) = 1$ if and only if there is $\alpha\in K^\times$ such that $\beta = \alpha/\sigma(\alpha)$
\end{theorem}
\begin{proof}
    $\implies$ Suppose $N^K_k(\beta) = 1$. We have a set of distinct characters $\{\id,\sigma,\ldots,\sigma^{n - 1}\}$ from $K^\times\to K^\times$. Then, due to \thref{thm:lin-ind-characters}, the set map 
    \begin{equation*}
        \tau = \id + \beta\sigma + (\beta\sigma(\beta))\sigma^2 + \cdots + (\beta\sigma(\beta)\cdots\sigma^{n - 2}(\beta))\sigma^{n - 1}
    \end{equation*}
    is nonzero, whereby, there is $\theta\in K^\times$ such that $\alpha = \tau(\theta)\ne 0$. Notice that 
    \begin{equation*}
        \sigma(\alpha) = \sigma(\theta) + (\sigma(\beta))\sigma^2(\theta) + \cdots + (\sigma(\beta)\sigma^2(\beta)\cdots\sigma^{n - 1}(\beta))\sigma^n(\theta)
    \end{equation*}
    Since $N^K_k(\beta) = 1$, we have 
    \begin{equation*}
        \beta\sigma(\beta)\cdots\sigma^{n - 1}(\beta) = 1
    \end{equation*}
    whence, we have $\sigma(\alpha) = \alpha/\beta$ and the conclusion follows.

    $\impliedby$ This is trivial enough.
\end{proof}

\begin{example}
    Find all rational points on the curve $x^2 + y^2 = 1$.
\end{example}
\begin{proof}
    This reduces to finding all elements $\alpha\in\Q[i]$ with $N^{\Q[i]}_\Q(\alpha) = 1$. Any element $\alpha$ of $\Q[i]$ may be written as $(a + bi)/c$. Due to \thref{thm:hilbert-90}, there is an element $\alpha\in\Q[i]$, such that $N^{\Q[i]}_\Q(\alpha) = 1$. Using the general form of elements in $\Q[i]$, we have 
    \begin{equation*}
        \alpha = \frac{a + bi}{a - bi} = \frac{(a^2 - b^2) + 2abi}{a^2 + b^2}
    \end{equation*}
    this completes the proof.
\end{proof}

\begin{theorem}[Additive Hilbert's Theorem 90]
    Let $K/k$ be a cyclic Galois extension with $\Gal(K/k) = \langle\sigma\rangle$ and $\beta\in K$. Then $\Tr^K_k(\beta) = 0$ iff there is $\alpha\in K$ such that $\beta = \alpha - \sigma(\alpha)$.
\end{theorem}
\begin{proof}
    Due to \thref{cor:non-zero-trace-exists}, there is some $\theta\in K$ with $\Tr^K_k(\theta)\ne0$. Consider $\alpha\in K$ given by 
    \begin{equation*}
        \alpha = \frac{1}{\Tr^K_k(\theta)}\left(\beta\sigma(\theta) + (\beta + \sigma(\beta))\sigma^2(\theta) + \dots + (\beta + \dots + \sigma^{n - 2}(\beta))\sigma^{n - 1}(\theta)\right).
    \end{equation*}
    We have 
    \begin{align*}
        \sigma(\alpha) &= \frac{1}{\Tr^K_k(\theta)}\left(\sigma(\beta)\sigma^2(\theta) + (\sigma(\beta) + \sigma^2(\beta))\sigma^3(\theta) + \dots + (\sigma(\beta) + \dots + \sigma^{n - 1}(\beta))\sigma^n(\theta)\right)\\
        &= \alpha - \beta\frac{1}{\Tr^K_k(\theta)}\left(\sigma(\theta) + \dots + \sigma^n(\theta)\right)\\
        &= \alpha - \beta
    \end{align*}

    The converse is trivial.
\end{proof}

\begin{theorem}[Artin-Schreier]
    Let $k$ be a field of characteristic $p > 0$.
    \begin{enumerate}[label=(\alph*)]
        \item Let $K/k$ be a cyclic extension of degree $p$. Then there is $\alpha\in K$ such that $K = k(\alpha)$ and $\alpha$ is a root of $f(x) = x^p - x - a$ for some $a\in k$. Further, $K$ is the splitting field of $f(x)$ over $k$.

        \item 
    \end{enumerate}
\end{theorem}
\begin{proof}
\begin{enumerate}[label=(\alph*)]
    \item Let $\Gal(K/k) = \langle\sigma\rangle$, since it is a group of prime order. We have $\Tr^K_k(-1) = p\cdot(-1) = 0$ whence there is $\alpha\in K$ such that $-1 = \alpha - \sigma(\alpha)$, equivalently, $\sigma(\alpha) = \alpha + 1$. Let $a = \alpha^p - \alpha$. Then,
    \begin{equation*}
        \sigma(a) = \sigma(\alpha^p - \alpha) = \sigma(\alpha)^p - (\alpha + 1) = \alpha^p + 1 - (\alpha + 1) = a.
    \end{equation*}
    Thus, $\sigma^n(a) = a$ for $1\le n\le p$, consequently, $a\in K^{\Gal(K/k)} = k$. 

    Note that for $1\le m\ne n\le p$, we have 
    \begin{equation*}
        \sigma^m(\alpha) = \alpha + m\ne\alpha + n = \sigma^n(\alpha).
    \end{equation*}
    Thus, $p\le [k(\alpha):k]_s\le[k(\alpha):k]\le[K:k] = p$ whence $[k(\alpha):k] = p$ and $K = k(\alpha)$.

    \item \qedhere
\end{enumerate}
\end{proof}