It is well known that every finite field must have prime characteristic. In fact, any integral domain with nonzero characteristic must have prime characteristic.

\begin{theorem}
    Let $F$ be a finite field with characteristic $p > 0$. Then there is a positive integer $n$ such that $F$ has cardinality $p^n$. Further, there is a unique field upto isomorphism of cardinality $p^n$.
\end{theorem}
\begin{proof}
    The prime subfield of $F$ is the subfield generated by $1$ and is isomorphic to $\F_p$. Then $[F:\F_p] = n$, whence the conclusion follows. Now, we show that there is a field with cardinality $p^n$. Consider the polynomial $f(x) = x^{p^n} - x\in\F_p[x]$. First, note that $Df(x) = -1$, and thus $f(x)$ has distinct roots in $\overline\F_p$. It is not hard to see that if $\alpha,\beta$ are roots of $f(x)$ in $\overline F_p$, then $\alpha - \beta$ and $\alpha\beta$ are roots of $f(x)$ in $\overline\F_p$. Therefore, the collection of roots of $f(x)$ in $\overline F_p$ form a field. The cardinality of this field is the number of distinct roots of $f(x)$ in $\overline\F_p$, which is precisely $p^n$.

    As for uniqueness, note that if $F$ is a field of cardinality $p^n$, then every element of $F$ is a root of $f(x) = x^{p^n} - x\in\F_p[x]$ (this is because $F$ contains a copy of $\F_p$ in it). Therefore, $F$ is the splitting field for $f(x)$ over $\F_p[x]$ in some algebraic closure. But since all splitting fields are isomorphic, we have the desired conclusion.
\end{proof}

\begin{theorem}[Frobenius]
    The group of automorphisms of $\F_q$ where $q = p^n$ is cyclic of degree $n$, generated by the Frobenius mapping, $\varphi:\F_q\to\F_q$ given by $\varphi(x) = x^p$.
\end{theorem}
\begin{proof}
    We first verify that $\varphi$ is an automorphism. That $\varphi$ is a ring homomorphism is easy to show, from which it would follow that $\varphi$ is injective. Surjectivity follows from here since $\F_q$ is finite. Next, note that $\varphi$ leaves $\F_p$ fixed, thus, $G = \Aut(\F_q) = \Aut(\F_q/\F_p)$. Furthermore, $|\Aut(\F_q/\F_p)| = [\F_q:\F_p]_s\le[\F_q:\F_p] = n$.

    We now show that the order of $\varphi$ in $G$ is precisely $n$, for if $d$ were the order of $\varphi$, then $\varphi^d(x) = x$ for all $x\in\F_q$ and thus, $x^{p^d} - x = 0$ for all $x\in\F_q$, from which it follows that $p^d\ge q$ and $d\ge n$ and the conclusion follows.
\end{proof}

\begin{theorem}
    Let $m,n\in\N$. Then in an algebraic closure $\overline{\F_p}$ of $\F_p$, the subfield $\F_{p^n}$ is contained in $\F_{p^m}$ if and only if $n\mid m$.
\end{theorem}
\begin{proof}
    If $\F_{p^n}$ is contained in $\F_{p^m}$, then $p^m = (p^n)^d$ where $d = [\F_{p^m}:\F_{p^n}]$. The converse follows from noting that $x^{p^n} - x\mid x^{p^m} - x$.
\end{proof}

\begin{theorem}
    Let $m,n\in\N$ such that $n\mid m$. Then the extension $\F_{p^m}/\F_{p^n}$ is finite Galois.
\end{theorem}
\begin{proof}
    We have $[\F_{p^m}:\F_p] = m$ and $[\F_{p^n}:\F_p] = n$, consequently, $[\F_{p^m}:\F_{p^n}]_s = m/n = [\F_{p^m}:\F_{p^n}]$ and thus the extension is separable. To show that the extension $\F_{p^m}/\F_{p^n}$ is normal, it suffices to show that the extension $\F_{p^m}/\F_p$ is normal but this trivially follows from the fact that $\F_{p^m}$ is the splitting field of $x^{p^m} - x\in\F_p[x]$. This completes the proof.
\end{proof}