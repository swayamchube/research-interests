\begin{theorem}\thlabel{thm:alg-closed-fields-exist}
    Let $k$ be a field. Then there is an algebraicaly closed field containing $k$.
\end{theorem}
\begin{proof}[Proof due to Artin]
\end{proof}

\begin{corollary}
    Let $k$ be a field. Then there exists an extension $k^a$ which is algebraic over $k$ and algebraically closed.
\end{corollary}
\begin{proof}
    
\end{proof}

\begin{lemma}\thlabel{lem:num-extensions-eq-distinct-roots}
    Let $k$ be a field and $L$ and algebraically closed field with $\sigma: k\to L$ an embedding. Let $\alpha$ be algebraic over $k$ in some extension of $k$. Then, the number of extensions of $\sigma$ to an embedding $k(\alpha)\to L$ is precisely equal to the number of distinct roots of the minimal polynomial of $\alpha$ over $k$.
\end{lemma}

\begin{lemma}
    Suppose $E$ and $L$ are algebraically closed fields with $E\subseteq L$. If $L/E$ is algebraic, then $E = L$.
\end{lemma}
\begin{proof}
    Let $\alpha\in L$. Let $p(x)\in E[x]$ be the minimal polynomial of $\alpha$ over $E$. Since $E$ is algebraically closed, $p$ splits into linear factors over $E$, one of them being $(x - \alpha)$, implying that $\alpha\in E$. This completes the proof.
\end{proof}

\begin{theorem}[Extension Theorem]\thlabel{thm:extension-theorem}
    Let $E/k$ be algebraic, $L$ an algebraically closed field and $\sigma: k\to L$ be an embedding of $k$. Then there exists an extension of $\sigma$ to an embedding of $E$ in $L$. If $E$ is algebraically closed and $L$ is algebraic over $\sigma k$, then any such extension of $\sigma$ is an isomorphism of $E$ onto $L$.
\end{theorem}
\begin{proof}
    Let $\mathscr S$ be the set of all pairs $(F,\tau)$ where $F\subseteq E$ and $F/k$ is algebraic and $\tau: F\to L$ is an extension of $\sigma$. Define a partial order $\leqq$ on $\mathscr S$ by $(F_1,\tau_1)\leqq(F_2,\tau_2)$ if and only if $F_1\subseteq F_2$ and $\tau_2\mid_{F_1}\equiv\tau_1$. Note that $\mathscr S$ is nonempty since it contains $(k,\sigma)$. Let $\mathscr C = \{(F_\alpha,\tau_\alpha)\}$ be a chain in $\mathscr S$. Define $F = \bigcup_{\alpha} F_\alpha$. Now, for any $t\in F$, there is $\beta$ such that $t\in F_\beta$; using this, define $\tau(t) = \tau_\beta(t)$. It is not hard to see that this is a valid embedding.

    Now, invoking Zorn's Lemma, there is a maximal element, say $(K,\tau)$. We claim that $K = E$, for if not, then we may choose some $\alpha\in E$ and invoke \thref{lem:num-extensions-eq-distinct-roots}.

    Finally, if $E$ is algebraically closed, so is $\sigma E$, consequently, we are done due to the preceeding lemma.
\end{proof}

\begin{corollary}
    Let $k$ be a field and $E, E'$ be algebraic extensions of $k$. Assume that $E, E'$ are algebraically closed. Then there exists an isomorphism $\tau: E\to E'$ inducing the identity on $k$.
\end{corollary}
\begin{proof}
    Consider the extension of $\sigma: k\to E'$ where $\sigma\mid_k = \mathbf{id}_k$ whence the conclusion immediately follows.
\end{proof}

Since an algebraically closed and algebraic extension of $k$ is determined upto an isomorphism, we call such an extension an \textit{algebraic closure} of $k$ and is denoted by $k^a$.

\begin{definition}[Conjugates]
    Let $E/k$ be an algebraic extension contained in an algebraic closure $k^a$. Then, the distinct roots of the minimal polynomial of $\alpha$ over $k$ are called the \textit{conjugates} of $\alpha$. In particular, two roots of the same minimal polynomial over $k$ are said to be \textit{conjugate} to one another.
\end{definition}