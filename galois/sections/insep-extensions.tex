\begin{proposition}
    Let $\alpha\in k^a$ and $f(x)\in k[x]$ be the minimal polynomial of $\alpha$ over $k$. If $\chr k = 0$, then all the roots of $f$ have multiplicity $1$. If $\chr k = p > 0$, then there is a non-negative integer $m$ such that every root of $f$ has multiplicity $p^m$. Consequently, we have 
    \begin{equation*}
        [k(\alpha):k] = p^m[k(\alpha):k]_s
    \end{equation*}
    and $\alpha^{p^m}$ is separable over $k$.
\end{proposition}
\begin{proof}
    
\end{proof}

\begin{definition}
    Let $\chr k = p > 0$. An element $\alpha\in k^a$ is said to be \emph{purely inseparable} over $k$ if there is a non-negative integer $n\ge 0$ such that $\alpha^{p^n}\in k$.
\end{definition}

\begin{theorem}\thlabel{thm:pins-equivalence}
    Let $\chr k = p > 0$ and $E/k$ be an algebraic extension. Then the following are equivalent: 
    \begin{enumerate}[label=(\alph*)]
        \item $[E:k]_s = 1$.
        \item Every element $\alpha\in E$ is purely inseparable over $k$. 
        \item For every $\alpha\in E$, the irreducible equation of $\alpha$ over $k$ is of type $X^{p^n} - a = 0$ for some $n\ge 0$ and $a\in k$.
        \item There is a set of generators $\{\alpha_i\}_{i\in I}$ of $E$ over $k$ such that each $\alpha_i$ is purely inseparable over $k$.
    \end{enumerate}
\end{theorem}
\begin{proof}
    $(a)\implies(b)$. Let $\alpha\in E$. From the multiplicativity of the separable degree, we must have $[k(\alpha):k]_s = 1$. Let $f(x)\in k[x]$ be the minimal polynomial of $\alpha$ over $k$. Since $[k(\alpha):k]_s$ is equal to the number of distinct roots of $f$, we see that $f(x) = (x - \alpha)^m$ for some positive integer $m$. Let $m = p^nr$ such that $p\nmid r$. Then, we have 
    \begin{equation*}
        f(x) = \left(x - \alpha\right)^{p^nr} = \left(x^{p^n} - \alpha^{p^n}\right)^r = x^{p^nr} - r\alpha^{p^n}x^{p^n(r - 1)} + \cdots
    \end{equation*}
    Since the coefficients of $f$ lie in $k$, we have $r\alpha^{p^n}\in k$ whence $\alpha^{p^n}\in k$.

    $(b)\implies(c)$. There is a minimal non-negative integer $n$ such that $\alpha^{p^n}\in k$. Consider the polynomial $g(x) = x^{p^n} - \alpha^{p^n}\in k[x]$. Note that $g(x) = (x - \alpha)^{p^n}$, whence the minimal polynomial for $\alpha$ over $k$ divides $g$ and is thus of the form $(x - \alpha)^{m}$ for some positive integer $m\le p^n$. Using a similar argument as in the previous paragraph, we see that there is a non-negative integer $r$ such that $\alpha^{p^r}\in k$. Due to the minimality of $n$, we must have $m = p^n$ and $g$ the minimal polynomial of $\alpha$ over $k$. 

    $(c)\implies(d)$. Trivial. 

    $(d)\implies(a)$. Any embedding of $E$ in $k^a$ must be the identity on the $\alpha_i$'s whence the embedding must be the identity on all of $E$ which completes the proof.
\end{proof}

\begin{definition}
    An algebraic extension $E/k$ is said to be \emph{purely inseparable} if it satisfies the equivalent conditions of \thref{thm:pins-equivalence}.
\end{definition}

\begin{proposition}
    Purely inseparable extensions form a distinguished class of extensions.
\end{proposition}
\begin{proof}
    Let $\chr k = p > 0$. The assertion about the tower of fields follows from the multiplicativity of separable degree. Now, let $E/k$ be purely inseparable. Then there is a set of generators $\{\alpha_i\}_{i\in I}$ generating $E$ over $k$. Then, $\{\alpha_i\}_{i\in I}$ generates $EF$ over $F$. Since the minimal polynomial of $\alpha_i$ over $F$ must divide the minimal polynomial of $\alpha_i$ over $k$, which is of the form $(x - \alpha_i)^{p^{n_i}}$ for some non-negative integer $n$, we see that $\alpha_i$ is purely inseparable over $F$ whence $EF$ is purely inseparable over $F$.

    Finally, let $E/k$ and $F/k$ be purely inseparable extensions. If $\{\alpha_i\}_{i\in I}$ and $\{\beta_j\}_{j\in J}$ generate $E$ and $F$ over $k$ respectively such that each $\alpha_i$ and $\beta_j$ is purely inseparable over $k$, then $EF$ is generated by $\{\alpha_i\}_{i\in I}\cup\{\beta_j\}_{j\in J}$ over $k$ whence is purely inseparable over $k$.
\end{proof}

\begin{proposition}
    Let $E/k$ be an algebraic extension and $E_0$ the separable closure of $k$ in $E$. Then, $E/E_0$ is purely inseparable.
\end{proposition}
\begin{proof}
    If $\chr k = 0$, then $E/k$ is separable and $E_0 = E$ and the conclusion is obvious. On the other hand, if $\chr k = p > 0$, then for every $\alpha\in E$, there is a non-negative integer $m$ such that $\alpha^{p^m}$ is separable over $k$ whence an element of $E_0$. Thus, $E/E_0$ is purely inseparable.
\end{proof}

\begin{proposition}
    Let $K/k$ be normal and $K_0$ the separable closure of $k$ in $K$. Then $K_0/k$ is normal.
\end{proposition}
\begin{proof}
    Let $\sigma: K_0\to k^a$ be an embedding of fields. This extends to an embedding of $K$ and is thus an automorphism of $K$. Note that $\sigma(K_0)$ is separable over $k$ and is thus contained in $k_0$ whence $\sigma(K_0) = K_0$ and $\sigma$ is an automorphism. This completes the proof.
\end{proof}

\begin{lemma}
    Let $K/k$ be normal, $G = \Aut(K/k)$ and $K^G$ the fixed field of $G$. Then $K^G/k$ is purely inseparable and $K/K^G$ is separable. If $K_0$ is the separable closure of $k$ in $K$, then $K = K^GK_0$ and $K^G\cap K_0 = 0$.
\end{lemma}
\begin{proof}
    Let $\alpha\in K^G$ and $\sigma: k(\alpha)\to k^a$ be an embedding over $k$. This can be extended to an embedding $\wt\sigma: K\to k^a$. Since $K$ is normal, this is an automorphism $\wt\sigma: K\to K$ and thus an element of $G$. This must leave $\alpha$ fixed whence $\sigma$ is the identity map, consequently, $\alpha$ is purely inseparable over $k$ and the conclusion follows.


    We shall now show that $K/K^G$ is separable. Pick some $\alpha\in K$ and let $\sigma_1,\ldots,\sigma_n\in G$ such that the elements $\sigma_1(\alpha),\ldots,\sigma_n(\alpha)$ form a maximal set of pairwise distinct elements. Consider the polynomial $f(x)$ in $K[x]$ given by
    \begin{equation*}
        f(x) = \prod_{i = 1}^n(x - \sigma_i(\alpha))
    \end{equation*}
    It is not hard to see that for any $\sigma\in G$, $\sigma(f) = f$, whence $f\in K^G[x]$ and $\alpha$ is separable over $K^G$.

    Note that any element of $K^G\cap K_0$ is both separable and purely inseparable over $k$ whence an element of $k$. Thus $K^G\cap K_0 = k$. 

    Finally, since both purely inseparable and separable extensions form a distinguished class, we have $K/K_0K^G$ is both separable and purely inseparable whence $K = K_0K^G$. This completes the proof.
\end{proof}