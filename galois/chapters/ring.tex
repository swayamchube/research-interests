\begin{definition}
    A ring $R = (R, +,\cdot)$ is a non-empty set $R$ with two binary operations $+$ (called addition) and $\cdot$ (called multiplication) with the following properties: 
    \begin{enumerate}
        \item $(R, +)$ is an Abelian group 
        \item $(R, \cdot)$ is a monoid 
        \item The multiplication operation distributes over addition, that is, $a\cdot(b + c) = a\cdot b + a\cdot c$
    \end{enumerate}
\end{definition}

\textit{Commutative Rings} are those in which multiplication is commutative. We denote the additive identity by $0$.

\begin{definition}[Integral Domain]
    A commutative ring with unity is called an \textit{integral domain} if 
    \begin{equation*}
        ab = 0 \Longleftrightarrow a = 0 \text{ or } b = 0
    \end{equation*}
\end{definition}

That is, an integral domain does not have any zero divisors. Sometimes we call integral domains just \textit{domains}.

\begin{definition}[Unit]
    An invertible element of a ring with unity is said to be a unit. That is, $u\in R$ is a unit if there is $v\in R$ such that $uv = 1$.

    The set of all units in a ring with unity forms a group, generally denoted by either $R^*$ or $R^\times$.
\end{definition}

\begin{definition}[Ideal]
    A subring $A$ of a ring $R$ is said to be an \textit{ideal} of $R$ if for every $r\in R$ and every $a\in A$ both $ra, ar\in A$.
\end{definition}

\begin{lemma}
    Let $S = \{a_1,\ldots,a_n\}$ be a finite subset of a commutative ring $R$ with unity. Then the set 
    \begin{equation*}
        I = \{r_1a_1 + \cdots + r_na_n\mid r_1,\ldots,r_n\in R\}
    \end{equation*}
    is the smallest ideal of $R$ containing $A$.
\end{lemma}
\begin{proof}
    Straightforward.
\end{proof}

We define the ideal generated by $S$ to be the smallest ideal in $R$ containing $S$, denoted by $\langle S\rangle$. A principle ideal is of the form $\langle a\rangle$ for some $a\in R$.

\begin{theorem}
    Let $R$ be a ring and let $A$ be a subring of $R$. The set of cosets $\{r + A\mid r\in R\}$ is a ring under the operations $(s + A) + (t + A) = (s + t) + A$ and $(s + A)\cdot (t + A) = st + A$ if and only if $A$ is an ideal of $R$.
\end{theorem}
\begin{proof}
    \textcolor{red}{TODO: Add in later}
\end{proof}

\begin{definition}[Prime Ideal, Maximal Ideal]
    A \textit{prime ideal} $A$ of a commutative ring $R$ is a proper ideal of $R$ such that $a,b\in R$ and $ab\in A$ imply $a\in A$ or $b\in A$. A \textit{maximal ideal} of a commutative ring $R$ is a proper ideal of $R$ such that, whenever $B$ is an ideal of $R$ and $A\subseteq B\subseteq R$, then $B = A$ or $B = R$.
\end{definition}

\begin{theorem}
    Let $R$ be a commutative ring with unity and $A$ be an ideal of $R$. Then $R/A$ is an integral domain if and only if $A$ is prime.
\end{theorem}
\begin{proof}
    Suppose $R/A$ is an integral domain.
    Indeed, we have 
    \begin{align*}
        st \in A &\Rightarrow st + A = 0\\
        &\Rightarrow (s + A)(t + A) = 0\\
        &\Rightarrow s + A = 0 \vee t + A = 0\\
        &\Rightarrow s\in A \vee t\in A
    \end{align*}
    and thus $A$ is prime.

    Conversely, if $A$ is prime, then 
    \begin{align*}
        (s + A)(t + A) = 0 &\Rightarrow st + A = 0\\
        &\Rightarrow st\in A\\
        &\Rightarrow s\in A\vee t\in A\\
        &\Rightarrow s + A = 0\vee t + A = 0
    \end{align*}
    and thus $R/A$ is a domain.
\end{proof}

\begin{theorem}
    Let $R$ be a commutative ring with unity and $A$ be an ideal of $R$. Then $R/A$ is a field if and only if $A$ is maximal.
\end{theorem}
\begin{proof}
    Suppose $R/A$ is a field and $B$ be an ideal of $R$ containing $A$. Let $b\in B\backslash A$. By definition, there is $c\in R$ such that $bc + A = (b + A)(c + A) = 1 + A$, or equivalently, $1 - bc\in A$. But since $bc\in B$, $1\in B$ implying that $B = R$. Therefore, $A$ is maximal.

    Conversely, suppose $A$ is maximal. It suffices to show that for all $a\notin A$, $a + A$ has an inverse. This immediately implies that $R/A$ is both a domain and a field. Let $a\notin A$. Then, consider the ideal generated by $\langle A, a\rangle$, which properly subsumes $A$ and thus must be equal to $R$. As a result, $1\in\langle A,a\rangle$ and thus, $1 - ra\in A$ for some $r\in R$, implying that $r + A$ is an inverse of $a + A$ in $R/A$.
\end{proof}

\begin{corollary}
    Every maximal ideal is a prime ideal.
\end{corollary}

\begin{definition}[Ring Homomorphism]
    A \textit{ring homomorphism} $\phi:R\to S$ is a mapping from $R$ to $S$ that preserves the two ring operations; that is, for all $a,b\in R$
    \begin{equation*}
        \phi(a + b) = \phi(a) + \phi(b) \qquad \phi(ab) = \phi(a)\phi(b)
    \end{equation*}

    The \textit{kernel} of a ring homomorphism is defined as 
    \begin{equation*}
        \ker\phi = \{r\in R\mid \phi(r) = 0\}
    \end{equation*}
\end{definition}

\begin{theorem}
    Let $\phi$ be a ring homomorphism from $R$ to $S$. Then the mapping from $R/\ker\phi$ to $\phi(R)$ given by $r + \ker\phi\mapsto\phi(r)$ is an isomorphism. Equivalently, $R/\ker\phi\cong\phi(R)$.
\end{theorem}
\begin{proof}
    Obviously, if $a,b\in\ker\phi$, then $\phi(a - b) = \phi(a) - \phi(b) = 0$ and thus $\ker\phi$ is a subring. Further, for any $r\in R$, $\phi(rs) = \phi(r)\phi(s) = 0$ for all $s\in\ker\phi$, implying that $\ker\phi$ is an ideal.

    Consider now the map $\psi:R/\ker\phi\to\phi(R)$, given by $r\ker\phi\mapsto\phi(r)$. The kernel of this homomorphism is obviously trivial and it is therefore injective. Thus, it is an isomorphism.
\end{proof}

\begin{definition}[Characteristic of a Ring]
    The \textit{characteristic} of a ring $R$ is the least positive integer $n$ such that $nx = 0$ for all $x\in R$. If no such integer exists, we say that $R$ has characteristic $0$. The characteristic of $R$ is denoted by $\operatornamewithlimits{char}(R)$.
\end{definition}

\begin{lemma}
    The characteristic of a domain is 0 or prime.
\end{lemma}
\begin{proof}
    Let $st = n = \operatornamewithlimits{char}(R) > 0$. Then, 
    \begin{equation*}
        (s\cdot 1)(t\cdot 1) = (st)\cdot 1 = 0
    \end{equation*}
    implying $s = n$ or $t = n$ and thus $n$ is prime.
\end{proof}

It is important to note that finite characteristic does not imply that the ring is finite. Consider $\Z_p[x]$ for example.

\begin{definition}[Principal Ideal Domain]
    A \textit{principal ideal domain} is an integral domain $R$ in which every ideal has the form $\langle a\rangle$.
\end{definition}

\begin{definition}[Polynomial Ring]
    Let $R$ be a commutative ring. The set of formal symbols 
    \begin{equation*}
        R[x] = \{a_nx^n + \cdots + a_0\mid a_i\in R,~ n\in\N_0\}
    \end{equation*}
    is called the \textit{ring of polynomials} over $R$ in the indeterminate $x$ with addition and multiplication defined using obviousness.
\end{definition}

\begin{lemma}
    If $R$ is an integral domain, then so is $R[x]$.
\end{lemma}
\begin{proof}
    Straightforward.
\end{proof}

\begin{theorem}[Division Algorithm]
    Let $F$ be a field and let $f(x),g(x)\in F[x]$ with $g(x)\ne 0$. Then there exist unique polynomials $q(x), r(x)\in F[x]$ such that $f(x) = g(x)q(x) + r(x)$.
\end{theorem}
\begin{proof}
    Existence is simple induction and uniqueness is another simple degree argument.
\end{proof}

\begin{corollary}
    Let $F$ be a field, $a\in F$ and $f(x)\in F[x]$. Then $f(a)$ is the remainder in the division of $f(x)$ by $x - a$. Thus, if $f(a) = 0$, then $x - a$ is a factor of $f(x)$.
\end{corollary}

\begin{theorem}
    Let $F$ be a field. Then $F[x]$ is a principal ideal domain.
\end{theorem}
\begin{proof}
    Let $A$ be an ideal in $F[x]$ and $g(x)\in A$ be a polynomial of minimum degree. Then, $\langle g(x)\rangle\subseteq A$. Conversely, let $f(x)\in A$, then $f(x) = q(x)g(x) + r(x)$, implying that $r(x) = 0$ lest $\deg r < \deg g$, contradicting the minimality of $\deg g$. Thus, $A\subseteq\langle g(x)\rangle$ and we are done.
\end{proof}

\begin{theorem}
    Let $F$ be a field and $0\ne p(x)\in F[x]$ have degree $n\ge 0$. Then $p$ has at most $n$ zeros in $F$ counting multiplicity.
\end{theorem}
\begin{proof}
    The proof is by induction on $n$. For $n = 0$, the conclusion is trivial. Now suppose $n > 0$. If $p$ has no roots in $F$ then we are trivially done. Suppose $x = a$ is a root of $p$, then $p(x) = (x - a)q(x)$ due to the factor theorem, and we are done due to the induction hypothesis.
\end{proof}

\begin{definition}[Associates, Irreducibles, Primes]
    Elements $a$ and $b$ of an integral domain $R$ are called associates if $a = ub$ for some unit $u\in R$. A nonzero element $a$ of $R$ is called an \textit{irreducible} if $a$ is not a unit and whenever $b,c\in R$ with $a = bc$ then $b$ or $c$ is a unit. A nonzero element $a$ of $R$ is called a \textit{prime} if $a$ is not a unit and $a\mid bc$ implies $a\mid b$ or $a\mid c$.
\end{definition}

\begin{theorem}
    In an integral domain, every prime is irreducible.
\end{theorem}
\begin{proof}
    Let $R$ be an integral domain and $p\in R$ be a prime with $p = ab$. Then, $p\mid ab$ and thus $p\mid a$ or $p\mid b$. Without loss of generality, let $p\mid a$ and $a = pa'$, consequently, $a'b = 1$ implying that $b$ is a unit.
\end{proof}

\begin{theorem}
    In a principal ideal domain, an element is prime if and only if it is irreducible.
\end{theorem}
\begin{proof}
    Since a principal ideal domain is implicitly an integral domain, each prime is irreducible. Conversely, let $a\in R$ be irreducible and $a\mid bc$. Then, consider the ideal $\langle a,b\rangle$ which must be of the form $\langle d\rangle$ for some $d\in R$. Thus, there is $r\in R$ such that $a = dr$, consequently, one of $d$ or $r$ must be a unit. Suppose $d$ is a unit. Then $\langle d\rangle = R$ and thus $1 = ax + by$, implying $c = cax + ay$, i.e. $a\mid c$. On the other hand, if $r$ is a unit, then $\langle a\rangle = \langle d\rangle$, and thus $b\in\langle d\rangle = \langle a\rangle$ i.e. $a\mid b$. This completes the proof.
\end{proof}

\begin{definition}[Unique Factorization Domain]
    An integral domain $D$ is a unique factorization domain if 
    \begin{enumerate}
        \item every nonzero element of $R$ that is not a unit can be written as a product of irreducibles of $R$ 
        \item the factorization into irreducibles is unique up to associates and the order in which the factors appear.
    \end{enumerate}
\end{definition}

\begin{lemma}
    In a principal ideal domain, any strictly increasing chain of ideals $I_1\subsetneq I_2\subsetneq\cdots$ must be finite in length.
\end{lemma}
\begin{proof}
    Let $I = \bigcup_{i = 1}^\infty I_i$. Obviously $I$ is an ideal of $R$. As a result, there is $a\in R$ such that $I = \langle a\rangle$. Let $k = \arg\min_{i\in\N}a\in I_i$. Then, it is obvious that $I_{k + 1} = I_k$. This finishes the proof.
\end{proof}

\begin{theorem}
    Every principal ideal domain is a unique factorization domain.
\end{theorem}
\begin{proof}
    Let $R$ be a principal ideal domain. First, we shall establish that every non-unit in $R$ has at least one irreducible factor. Let $a_0\in R$ be a non-unit. If $a_0$ is irreducible, we are done. If not, then we may write $a_0 = b_1a_1$ where neither $b_1$ nor $a_1$ is a unit. If $a_1$ is irreducible, we are done, if not, then repeat. We now have an ascending chain of ideals $\langle a_0\rangle\subsetneq\langle a_1\rangle\subsetneq\cdots$ and therefore must terminate. As a result, there is some $k$ such that $a_k$ is irreducible. This gives the desired conclusion.

    Using a similar chain argument, one can show that every element can be written as a product of irreducibles. It remains to show uniqueness up to associates. This part of the proof proceeds by induction on $r$, the number of primes in the factorization of the element. For irreducibles, $r = 1$, since each irreducible is a prime in a PID, thus the base case is trivial. Suppose $r > 1$ and 
    \begin{equation*}
        a = p_1\ldots p_r = q_1\ldots q_s
    \end{equation*}
    Since $p_1\mid q_1\ldots q_s$, it must divide one of the $q_i$'s. Without loss of generality, suppose $p_1\mid q_1$. Then there is a unit $u$ such that $q_1 = up_1$. We are now left with $p_2\ldots p_r = uq_2\ldots q_s$, and applying the induction hypothesis, we are done.
\end{proof}

\begin{corollary}
    Let $F$ be a field. Then $F[x]$ is a unique factorization domain.
\end{corollary}

Note that a field is trivially a UFD since every element is a unit.


\hrulefill 

\textcolor{red}{TODO: Place this}

\begin{lemma}
    Let $R$ be a commutative ring with unity and $I\subseteq R$ be an ideal (in this case it is implicitly two sided). Then there is a maximal ideal $\mathfrak{m}$ containing $I$ that is proper in $R$.
\end{lemma}
\begin{proof}
    Define the poset $(P,\subseteq)$ where
    \begin{equation*}
        P = \{A\mid I\subseteq A\subsetneq R,~A\text{ is an ideal}\}
    \end{equation*}
    Let $C$ be any chain in $P$ and define $U_C = \bigcup_{c\in C}c$. By construction, one can show that $U_C$ is an ideal in $R$ and must contain $I$. To see that $U_C$ is proper, simply note that $1\notin c$ for all $c\in C$ and thus $1\notin U_C$. Further, for all $c\in C$, $c\subseteq U_C$. We have shown that each chain has a maximal element, in this case $U_C$ and therefore due to Zorn's Lemma, there is $\mathfrak{m}\in P$ that is maximal and it is obvious that $\mathfrak{m}$ is a maximal ideal.
\end{proof}

Note that it is known that the above lemma is \textbf{equivalent} to the Axiom of Choice.