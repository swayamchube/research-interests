\begin{definition}[Field]
    A \textit{field} is an integral domain where every non-zero element is a unit.
\end{definition}

\begin{definition}[Prime Subfield]
    The \textit{prime subfield} of a field $F$ is the subfield of $F$ generated by the multiplicative identity $1_F$ of $F$. It is isomorphic to either $\Q$ or $\Z/p\Z$ for some prime $p$.
\end{definition}

\begin{definition}[Extension Field]
    A field $E$ is an \textit{extension field} of a field $F$ if there is a monomorphism of fields $\phi:F\hookrightarrow E$.
\end{definition}

\begin{theorem}[Fundamental Theorem of Field Theory]
    Let $F$ be a field and let $f(x)$ be a nonconstant polynomial in $F[x]$. Then there is an extension field $E$ of $F$ in which $f$ has a zero.
\end{theorem}
\begin{proof}
    Consider the field $E = F[x]/\langle p(x)\rangle$ with the monomorphism $\phi:F\hookrightarrow F[x]/\langle p(x)\rangle$ given by $\phi(a) = a$. One immediately notes that $p(x + \langle p(x)\rangle) = 0$.
\end{proof}

While we are at it, note the following lemma
\begin{lemma}
    Let $F$ and $F'$ be fields, then the homomorphism of fields $\phi:F\to F'$ is either trivial or a monomorphism.
\end{lemma}
\begin{proof}
    $F$ has no proper ideals.
\end{proof}

\begin{definition}[Degree of an Extension]
    Let $L$ be an extension field of a field $K$. Then, there is a basis of $L$ over $K$. The size of this basis is the \textit{degree} of the extension. The degree may be infinite.
\end{definition}

\begin{theorem}
    Let $L:K$ and $M:L$ be field extensions. Then, 
    \begin{equation*}
        [M:L][L:K] = [M:K]
    \end{equation*}
\end{theorem}
\begin{proof}
    \textcolor{red}{TODO: Add in later}
\end{proof}

\begin{definition}
    Suppose $L$ is a field and $K$ is a subfield. Let $S\subseteq L$. Then, we denote by $K(S)$, the \textbf{subfield of $L$ generated over $K$ by $S$}. If $S = \{a_1,\ldots,a_n\}$ is finite, then we denote this field by $K(a_1,\ldots,a_n)$.
\end{definition}

\begin{theorem}
    Let $L$ be a field, let $K$ be a subfield and let $\alpha\in L$. Then either:
    \begin{enumerate}
        \item $K(\alpha)$ is isomorphic to $K(x)$, the field of all rational forms with coefficients in $K$; or 
        \item there exists a unique monic irreducible polynomial $m\in K[x]$ such that 
        \begin{enumerate}
            \item for all $f\in K[x]$, $f(\alpha) = 0$ if and only if $m\mid f$ 
            \item the field $K(\alpha)$ coincides with $K[\alpha]$, the ring of all polynomials in $\alpha$ with coefficients in $K$; and 
            \item $[K[\alpha]:K] = \partial m$
        \end{enumerate}
    \end{enumerate}
\end{theorem}
\begin{proof}
    Suppose there is no non-zero polynomial $f$ in $K[x]$ such that $f(\alpha) = 0$. Then, consider the function $\psi:K(x)\to K(\alpha)$ given by $\psi(f/g) = f(\alpha)/g(\alpha)$. One notes that the denominator must always be non-zero. This mapping is indeed well defined,
    \begin{align*}
        \psi(f/g) = \psi(p/q) &\Leftrightarrow f(\alpha)q(\alpha) - p(\alpha)g(\alpha)\in L\\
        &\Leftrightarrow fq - pg = 0\\
        &\Leftrightarrow f/g = p/q
    \end{align*}
    The fact that $\psi$ is a homomorphism is just a routine computation. It is also surjective and therefore must be an isomorphism. (Recall that a homomorphism between fields must be trivial or a monomorphism).

    Now suppose there is a non-zero polynomial $g$ of minimum degree such that $g(\alpha) = 0$. Let $m = g/a$ where $a$ is the leading coefficient of $g$. Let $f\in K[x]$ such that $f(\alpha) = 0$. Then, $f(x) = q(x)m(x) + r(x)$ with $r = 0$ or $\deg r < \deg m$. Therefore, $r = 0$, lest $r(\alpha) = 0$ with $\deg r < \deg m$, a contradiction to the minimality of $\deg m$. Uniqueness and irreducibility are both obvious due to similar minimality arguments.

    Now consider $f(\alpha)/g(\alpha)\in K(\alpha)$, where $g(\alpha)\ne0$ and $\gcd(f(x), g(x))\in K^\times$. Then there exist polynomials $a,b\in K[x]$ such that $ag + bm = 1$ and thus $a(\alpha)g(\alpha) = 1$. Consequently, $f(\alpha)/g(\alpha) = a(\alpha)f(\alpha)\in K[\alpha]$.

    Finally, note that $\{1,\alpha,\ldots,\alpha^{\deg m - 1}\}$ is a linearly independent set and spans $K[\alpha]$. This finishes the proof.
\end{proof}


\section{Algebraic Extensions}
\begin{definition}[Algebraic]
    If $\alpha$ has a minimal polynomial over$K$, we say that $\alpha$ is \textit{algebraic} over $K$ and that $K[\alpha] = K(\alpha)$ is a \textit{simple} algebraic extension of $K$. Otherwise $\alpha$ is said to be \textit{transcendental} over $K$.
\end{definition}

\begin{theorem}
    Let $K(\alpha)$ be a simple transcendental extension of a field $K$. Then the degree of $K(\alpha)$ over $K$ is infinite.
\end{theorem}

\begin{definition}[Algebraic Extension]
    An extension $L$ of $K$ is said to be an \textit{algebraic extension} if every element of $L$ is algebraic over $K$. Otherwise $L$ is a \textit{transcendental extension}.
\end{definition}

\begin{lemma}
    Every finite extension is algebraic.
\end{lemma}
\begin{proof}
    Let $L$ be a finite extension of $K$, with $[L:K] = n$. Then $\{1,\alpha,\ldots,\alpha^n\}$ is linearly dependent.
\end{proof}

One notes that the converse is not necessarily true. For example, $\mathbb{C}$ is an infinite algebraic extension of $\R$.

\begin{corollary}
    Let $L:K$ and $M:L$ be field extensions and let $\alpha\in M$. If $\alpha$ is algebraic over $K$ then it is also algebraic over $L$.
\end{corollary}

\begin{theorem}
    Let $L$ be a field, $K$ be a subfield, and let $\alpha\in L$ be algebraic over $K$ with minimal polynomial $m\in K[x]$. Then, $K(\alpha) = K[\alpha]\cong K[x]/\langle m(x)\rangle$.
\end{theorem}

Recall from the Fundamental Theorem of Field Theory that any irreducible polynomial in a field has a root in an extension field. This gives rise to the following theorem:
\begin{theorem}
    Let $L$ be a field and $K$ be a subfield of $L$. Let $m\in K[x]$ be a monic, irreducible polynomial and $\alpha\in L$ be a zero of $m$. Then, $K(\alpha) = K[\alpha]\cong K[x]/\langle m(x)\rangle$.
\end{theorem}
\begin{proof}
    Consider the mapping $\psi:K[x]/\langle m(x)\rangle\to K[\alpha]$ that maps $x + \langle m(x)\rangle\mapsto\alpha$. The kernel of this isomorphism is obviously $\langle m(x)\rangle$ due to a preceeding theorem. As a result, we have a monomorphism of fields. Conversely, for any $\beta\in K[\alpha]$, there is a polynomial $f\in K[x]$ such that $\beta = f(\alpha)$, therefore establishing surjection. Thus, we have an isomorphism of fields.
\end{proof}

\begin{corollary}
    Let $L$ be a field, $K$ a subfield and $m\in K[x]$ be a monic irreducible polynomial. Let $\alpha,\beta\in L$ be zeros of $m$. Then, $K(\alpha) = K[\alpha]\cong K[\beta] = K(\beta)$. Furthermore, this isomorphism fixes every element of $K$.
\end{corollary}

The above corollary can be generalized to give the following result: 
\begin{theorem}
    Let $L$ and $L'$ be fields with isomorphic subfields $K$ and $K'$ under the isomorphism $\varphi$ with the canonical extension $\widetilde{\varphi}:K[x]\to K'[x]$. Let $f\in K[x]$ be irreducible and $\widetilde{f} = \widetilde{\varphi}(f)$. Let $\alpha\in L$ be a zero of $f$ and $\alpha'\in L'$ be a zero of $\widetilde{f}$. Then there is an isomorphism $\psi:K[\alpha]\to K'[\alpha']$ that takes $\alpha$ to $\alpha'$ and agrees with $\varphi$ on $K$.
\end{theorem}
\begin{proof}
    It suffices to show that $K[x]/\langle f(x)\rangle\cong K'[x]\langle\widetilde{f}(x)\rangle$. Consider the map $\phi:K[x]/\langle f(x)\rangle\to K'[x]/\langle\widetilde{f}(x)\rangle$ given by $\phi(p(x) + \langle f(x)\rangle) = \widetilde{p}(x) + \langle\widetilde{f}(x)\rangle$. It is routine to check that this is indeed an isomorphism.
\end{proof}

\section{Algebraic Closure}
\begin{definition}
   A field $F$ is said to be \textit{algebtaically closed} if every polynoimal of positive degree has a root in $F$ 
\end{definition}

\begin{theorem}
    Let $k$ be a field. Then there is an algebraically closed field $K$ containing $k$.
\end{theorem}
\begin{proof}
    Define the set $S = \{x_f\mid f\in k[x],~\deg f\ge 1\}$. Let $I$ be the ideal $I$ in $k[S]$ generated by $\{f(x_f)\mid f\in k[x],~\deg f\ge 1\}$. First, we shall show that $I$ is proper. Suppose not, then there are polynomials $g_1,\ldots,g_n\in k[S]$ such that $1 = \sum_{i = 1}^ng_if_i(x_{f_i})$. Note that the polynomials $\{g_i\}$ contain only finitely many distinct indeterminates from $S$. Let us use the shorthand $x_i$ for $x_{f_i}$. Let the indeterminates in $g_i$ and $f_i$ cumulatively be $\{x_1,\ldots,x_n,x_{n + 1},\ldots,x_m\}$. Let $E$ be an extension field containing $\alpha_1,\ldots,\alpha_n$ such that $f_i(\alpha_i) = 0$ for all $1\le i\le n$. Now substituting $x_i = \alpha_i$ for all $1\le i\le n$ and $x_j = 0$ for all $n + 1\le j\le m$, we have a contradiction.


    Now, let $\mathfrak{m}$ be a maximal ideal of $k[S]$ containing $I$. Obviously $k[S]/\mathfrak{m}$ is a field containing $k$ and $x_f + \mathfrak{m}$ is a root of $f(x)$, thus every polynomial in $k[x]$ has a root in $K_1 = k[S]/\mathfrak{m}$. Now repeat this process indefinitely. Define $K = \bigcup_{i = 1}^\infty K_i$. Note that $K$ is a field. Further, if $f(x)\in K[x]$, then $f(x)\in K_n[x]$ for some $n$ and has a root in $K_{n + 1}\subseteq K$. This finishes the proof.
\end{proof}

\begin{corollary}
    Let $F$ be a field. Then there is a field $K\supseteq F$ such that $K$ is algebraically closed and $K$ is algebraic over $F$.
\end{corollary}
\begin{proof}
    Let $L$ be an algebraically closed field containing $F$ and consider 
    \begin{equation*}
        K = \{\alpha\in L\mid [F(\alpha):F] < \infty\}
    \end{equation*}
\end{proof}

\begin{theorem}
    Let $k$ be a field and $E$, an algebraic extension of $k$. Let $\sigma:E\to E$ be a $k$-embedding. Then, $\sigma$ is an automorphism.
\end{theorem}
\begin{proof}
    It suffices to show that $\sigma$ is surjective. Let $\alpha\in E$ and $p(x) = m_\alpha(x)\in k[x]$ and $E'$ be the subfield of $E$ generated by all the roots of $p(x)$ in $E$. It is obvious that $\sigma$ takes the roots of $p$ to other roots of $p$. Therefore, $\sigma(E')\subseteq E'$. Furthermore, if $\alpha_1,\ldots,\alpha_n$ is a basis for $E'$ over $E$, then $\sigma(\alpha_1),\ldots,\sigma(\alpha_n)$ is a basis for $\sigma(E')$ over $E$ and both have the same dimension. Hence, $\sigma(E') = E'$, consequently, $\alpha\in\sigma(E)$. This completes the proof.
\end{proof}

\begin{definition}[Algebraic Closure]
    Let $k$ be a field. Then $K$ is said to be an algebraic closure of $k$ if $K$ is algebraically closed and $K/k$ is algebraic.
\end{definition}

We shall show that the algebraic closure of a field is unique upto isomorphism but there may be multiple algebraically closed fields containing $k$. For example, $\mathbb{A}$ and $\mathbb{C}$ are two algebraically closed fields containing $\Q$.

\begin{corollary}
    Let $k$ be a field. Then there is an algebraically closed extension of $k$ that is algebraic over $k$.
\end{corollary}
\begin{proof}
    Let $L$ be an algebraically closed field containing $k$. Consider the algebraic subfield of $L$ over $k$.
\end{proof}

\begin{lemma}
    Let $k$ be a field and $\sigma:k\to L$ be a homomorphism of fields (also known as an embedding) where $L$ is algebraically closed. Then there is an extension of $\sigma$ that is an embedding of $k(\alpha)$ into $L$ when $\alpha$ is algebraic over $k$.
\end{lemma}
\begin{proof}
    Since $\alpha$ is algebraic over $k$, $k(\alpha) = k[\alpha]$. Therefore, for all $x\in k(\alpha)$, there is $f(X)\in k[X]$ such that $x = f(\alpha)$. Let $\beta$ be any root of $m_\alpha^\sigma$ in $L$. Define $\overline{\sigma}:k(\alpha)\to k^\sigma(\beta)$ such that $f(\alpha)\mapsto f^\sigma(\beta)$. To see that this is well defined, note that if $f(\alpha) = g(\alpha)$, then $f(X) - g(X) = m_\alpha(X)q(X)$. As a result, 
    \begin{equation*}
        f(\beta) - g(\beta) = m_\alpha^\sigma(\beta)q^\sigma(\beta) = 0
    \end{equation*}
    this completes the proof.
\end{proof}

\begin{theorem}[Extension Theorem]
    Let $k$ be a field and $E$ an algebraic extension of $k$, and there exists an extension of $\sigma$ to an embedding of $E$ in $L$. If $E$ is algebraically closed and $L$ is algebraic over $k^\sigma$, then any such extension of $\sigma$ is an isomorphism of $E$ onto $L$.
\end{theorem}
\begin{proof}
    Let $S$ be the set of all pairs $(F,\tau)$ where $F$ is a subfield of $E$ containing $k$, and $\tau$ is an extension of $\sigma$ to an embedding of $F$ in $L$. Note that $S$ is nonempty since $(k,\sigma)\in S$. Let $\{(F_i,\tau_i)\}_{i\in I}$ be a chain in $(S,\leqq)$ where $(F,\tau)\leqq(F',\tau')$ if $F\subseteq F'$ and $\tau'|_{F} = \tau$. Let $F = \bigcup_{i\in I}F_i$ and define $\tau$ on $F$ to be equal to $\tau_i$ on some $F_i$. Obviously, $F$ is a field and $(F,\tau)$ is an upper bound for the chain. Then, using Zorn's Lemma, there is a maximal element $(K,\lambda)\in S$ where $\lambda$ is an extension of $\sigma$. We shall now show that $K = E$. Supopse not, then there is $\alpha\in E$ such that $\alpha\notin K$. But due to the previous theorem, there is an extension of $\lambda$ to $K(\alpha)$. This proves the first part of the theorem.

    Finally, if $E$ is algebraically closed, then so is $E^\sigma$, further, $L$ is algebraic over $E^\sigma$. As a result, $L = E^\sigma$ and $\sigma$ is an isomorphism.
\end{proof}

\begin{corollary}
    Let $k$ be a field and $E$, $E'$ be algebraic and algebraically closed extensions of $k$. Then there is an isomorphism $\sigma: E\to E'$.
\end{corollary}

\begin{theorem}
    Let $K$ be a splitting field of the polynomial $f(X)\in k[X]$. If $E$ is another splitting field of $K$, then there is an isomorphism $\sigma:E\to K$ inducing the identity on $k$. If $k\subseteq K\subseteq k^a$ where $k^a$ is an algebraic closure of $k$, then ay embbedding of $E$ in $k^a$ inducing the identity on $k$ must be an isomorphism of $E$ onto $K$.
\end{theorem}
\begin{proof}
    Let $K^a$ be the algebraic closure of $K$. Then, $K^a$ is algebraic over $k$ and further, is algebraically closed. Then, due to a preceeding theorem, we may extend this to an embedding $\sigma: E\to K^a$ which induces an identity on $K$. We shall now show that $\sigma$ is an isomorphism. It suffices to show that $\sigma$ is surjective since injectivity is implicit.

    Over $E$, we have the factorization $f(X) = (X - \beta_1)\cdots(X- \beta_n)$. Then, $f^\sigma(X) = c(X-\sigma\beta_1)\cdots(X-\sigma\beta_n)$. But since we have a unique factorization in $K^a[X]$, we must have that $\alpha_i$ are a permutation of $\sigma\beta_i$. Therefore, $\sigma\beta_i\in K$ for all $i$ and therefore, $\sigma E\subseteq K$. But since $K = k(\alpha_1,\ldots,\alpha_n) = k(\sigma\beta_1,\ldots,\sigma\beta_n)$, we have that $K\subseteq\sigma E$ and therefore, $K = \sigma E$.
\end{proof}

\section{Splitting Fields}

\begin{definition}[Splitting Field]
    Let $L$ be a field, $K$ a subfield of $L$ and $f\in K[x]$. We say that an extension $M\subseteq L$ of $K$ is a splitting field for $f$ over $K$ if 
    \begin{enumerate}
        \item $f$ splits completely over $M$ 
        \item $f$ does not split completely over any subfield $E$ such that $K\subseteq E\subsetneq M$.
    \end{enumerate}
\end{definition}

It is obvious from definition, if $f(x) = a(x - a_1)\cdots(x - a_n)$ in some extension $L$ of $K$, then the splitting field for $f$ is given by $F(a_1,\ldots,a_n)$.

The next theorem establishes the existence of splitting fields
\begin{theorem}
    Let $K$ be a field and $f\in K[x]$ be nonconstant. Then there exists a splitting field $L$ for $f(x)$ over $K$
\end{theorem}
\begin{proof}
    Straightforward induction.
\end{proof}


\begin{theorem}
    Let $K$ and $K'$ be fields and let $\varphi: K\to K'$ be an isomorphism with the canonical extension $\widetilde{\varphi}:K[x]\to K'[x]$ and let $L$, $L'$ be splitting fields of $f$ over $K$ and $\widetilde{\varphi}(f)$ over $K'$. Then there is an isomorphism $\phi:L\to L'$ that agrees with $\varphi$ on $K$.
\end{theorem}
\begin{proof}
    Induct on $\deg f$. The base case is trivial. Now suppose $\deg f > 1$ and let $p(x)$ be an irreducible factor of $f$ over $K$ and let $a\in L$ be a zero of $p(x)$. Similarly, let $b\in L'$ be a root of $\widetilde{\varphi}(p)$. Due to a preceeding theorem, there is an isomorphism $\alpha: K(a)\to K'(b)$ that agrees with $\varphi$ on $K$. Let us now write $f(x) = (x - a)g(x)$ where $g\in K(a)[x]$ and therefore, $\alpha(g)\in K'(b)[x]$. We already know that $L$ is a splitting field for $g$ and $L'$ for $\alpha(g)$. Due to the inductive hypothesis, there is an isomorphism $\phi:L\to L'$ that agrees with $\alpha$ on $K(a)$ and thus with $\varphi$ on $K$. This finishes the proof.
\end{proof}

\begin{corollary}
    Let $K$ be a field and $f(x)\in K[x]$. Then any two splitting fields of $f(x)$ over $K$ are isomorphic.
\end{corollary}
\begin{proof}
    In the previous theorem, take $K = K'$.
\end{proof}

\section{Finite Fields}
\begin{definition}
    A field $K$ of characteristic $p$ is called perfect if $K^p = K$.
\end{definition}

\begin{theorem}
    Any finite field $F$ has prime power cardinality. Let $F$ be a field of cardinality $p^n$ where $p$ is a prime and $n$ is a positive integer. Then $F$ is unique upto isomorphism.
\end{theorem}
\begin{proof}
    The first fact follows from Cauchy's Theorem on the additive subgroup of $F$. To show the existence of a field of cardinality $p^n$, consider the splitting field of the polynomial $f(x) = x^{p^n} - x\in\mathbb{F}_p[x]$. Let $K$ be the splitting field. I claim that for all $a\in K$, $f(a) = 0$. First note that $K$ must have characteristic $p$. Let $\alpha,\beta$ be two roots of $f$ in $K$, then it is easy to show that $(\alpha - \beta)$ and $\alpha^{-1}\beta$ must be roots of $f$ in $K$, thus the roots of $f(x)$ form a field. Further, note that $D_x f(x) = p^nx^{p^n - 1} - 1 = -1$ and thus does not share any root with $f(x)$. This implies that the roots of $f$ are distinct in $K$. As a result, the $K$ is the field composed of roots of $f(x)$. Thus $K$ has size $p^n$.

    Let $K$ be any field of cardinality $p^n$. Then, for all non-zero elements $x\in K\backslash\{0\}$, we must have $x^{p^n - 1} = 1$ and thus $x^{p^n} - x = 0$ for all $x\in K$. And thus $K$ is isomorphic the field of roots for the polynomial $x^{p^n} - x$, immediately implying that all such fields are isomorphic since splitting fields are unique upto isomorphism.
\end{proof} 

Such fields are denoted by $\GF(p^n)$, the Galois Field of order $p^n$.

\begin{theorem}
    Every finite field is perfect.
\end{theorem}
\begin{proof}
    Consider the map $\GF(p^n)\stackrel{\phi}{\longrightarrow}\GF(p^n)$. Using the fact that $\operatorname{char}\GF(p^n) = p$, it is not hard to show that $\phi$ is a homomorphism of fields. Further, for all $a\in\GF(p^n)$, we have that $a = (a^{p^{n - 1}})^p$ and thus $\phi$ is surjective, implying that $\phi$ is an automorphism and thus finite fields are perfect.
\end{proof}

The map $\phi$ is known as the Frobenius Automorphism.

\section{Normal Extensions}
\begin{definition}[Normal Extension]
    An algebraic extension $E/F$ is called a \textit{normal extension} if whenever $f(x)\in F[x]$, is irreducible and has a root in $E$, then $f(x)$ splits into linear factors in $E[x]$.
\end{definition}

As an example, we note that each extension of degree $2$ is normal. First, note that since $[E:F]$ is finite, it is algebraic. Indeed, let $f(x)$ be an irreducible polynomial with some root $\alpha\in E$. If $\alpha\in F$, then $f(x)$ is linear and the conclusion is trivial. If $\alpha\in E\backslash F$, then $1,\alpha,\alpha^2$ are not linearly independent, as a result, $f(x)$ must have degree $2$ and withoutl loss of generality, let $f$ be monic. Then, we may write $f(x) = (x - \alpha)(x - \beta)$ for some $\beta\in E$. As a result, $f$ splits in $E$ and $E$ is a normal extension of $F$.

\begin{definition}[$F$-embedding]
    Let $K$ and $E$ be fields with a common subfield $F$. Then an $F$-embedding from $K$ to $E$ is an injective homomorphism $\sigma:K\hookrightarrow E$ such that $\sigma(x) = x$ for all $x\in F$.
\end{definition}

\begin{theorem}
    Let $E/F$ be an algebraic extension such that $E\subseteq\overline{F}$. Then the following are equivalent: 
    \begin{enumerate}
        \item Every embedding of $K$ in $k^a$ over $k$ induces an automorphsim of $K$
        \item $K$ is the splitting field of a family of polynomials in $k[X]$ 
        \item Every irreducible polynomial of $k[X]$ which has a root in $K$ splits into linear factors in $K$
    \end{enumerate}
\end{theorem}
\begin{proof}
    \hfill 
    \begin{enumerate}
        \item Let $\alpha\in K$ and $p(x)$ be the minimal polynomial. There is an isomorphism $k(\alpha)\to k(\beta)$ that fixes $k$. This is an embedding of $k(\alpha)$ in $k^a$ and therefore can be extended to an embeddingd of $K$ in $k^a$. But due to the hypothesis, this must be an automorphism of $K$, therefore, $\beta\in K$ and $p(x)$ splits in $K[X]$. In conclusion, $K$ is the splitting field of the collection of polynomials $m_{\alpha}(x)\in k[X]$ where $\alpha\in K$. Note that we have also shown that if $p(x)$ is irreducible and has a root in $K$, then it splits in $K$. This shows that $(1)\Longrightarrow (2)\wedge(3)$.

        \item Now suppose $K$ is the splitting field of the collection of polynomials $\{f_i\}_{i\in I}$. Let $\alpha$ and $\sigma$ be an embedding of $K$ in $k^a$. Let $\alpha\in K$. Let $\alpha\in K$ be the root of some polynomial $f_i\in K[x]$, then $\sigma\alpha$ is also a root of the same polynomial and therefore, an element of $K$. As a result, $\sigma K\subseteq K$. We have shown previously that every such embedding must be an automorphism of $K$. This shows that $(2)\Longrightarrow(1)$.

        \item Now it suffices to show that $(3)\Longrightarrow(1)$. Let $\alpha\in K$, then there is an irreducible polynomial $p(x)\in k[x]$ for $\alpha$. Obviously, $\sigma\alpha$ is also a root of $p$ and due to the hypothesis of $(3)$, we know that $\sigma\alpha\in K$, therefore, $\sigma K\subseteq K$, which immediately implies that $\sigma$ is an automorphism.
    \end{enumerate}
\end{proof}

\begin{theorem}
    Normal extensions remain normal under lifting. If $k\subseteq E\subseteq K$ and $K$ is normal over $k$, then $K$ is normal over $E$. If $K_1$ and $K_2$ are normal over $k$, and are contained in some field $L$, then $K_1K_2$ is normal over $k$ and so is $K_1\cap K_2$.
\end{theorem}
\begin{proof}
    \textcolor{red}{I have no idea what a lifting is}. 

    Let $\sigma$ be an embedding of $K$ over $E$, then it is also an embedding of $K$ over $k$ and is therefore an automorphism of $K$. It now follows that $K$ is normal over $E$.
    
    Let $\sigma$ be an embedding of $K_1K_2$ in $k^a$ over $k$. As a result, $\sigma(K_1K_2) = \sigma(K_1)\sigma(K_2) = K_1K_2$ since the restriction of $\sigma$ to $K_1$ and $K_2$ are both embeddings over $k$. Therefore, it follows that $\sigma$ is an automorhism and $K_1K_2$ is normal over $k$.

    Similarly, let $p(x)\in k[x]$ have a root in $K_1\cap K_2$, then it has all roots in $K_1$ and all roots in $K_2$ and therefore in $K_1\cap K_2$. This completes the proof.
\end{proof}

\begin{theorem}
    Let $E/k$ be a finite extension. Let $\sigma_1,\ldots,\sigma_n$ be the distinct embeddings of $E$ in $E^a$, then the extension 
    \begin{equation*}
        K = (\sigma_1E)\cdots(\sigma_nE)
    \end{equation*}
    is the smallest normal extension of $k$ containing $E$.
\end{theorem}
\begin{proof}
    It is obvious that $E\subseteq K$. Then, for any embedding $\tau$ of $K$ in $E^a$ over $k$, the restriction of $\tau\circ\sigma_i$ is also an embedding of $E$ in $E^a$. Therefore, $(\tau\sigma_i)_{i\in [n]}$ is a permutation of $(\sigma_i)_{i\in[n]}$. Thus, $\tau(K)\subseteq K$ and is an automorphism.

    Let $L/k$ be a normal extension and $\tau$ be an embedding of $L$ in $E^a$. The restriction of $\tau$ to $E$ must be one of the $\sigma_i$'s, therefore, $L$ must contain $\sigma_iE$ and hence the compositum $(\sigma_1E)\cdots(\sigma_nE)$.
\end{proof}

\section{Separable Extensions}
Let $E/F$ be an algebraic extension, $L$ be an algebraicaly closed field and $\sigma: F\to L$ be an embedding of $F$ such that $L$ is algebraic over $\sigma F$, therefore is equal to the algebraic closure of $\sigma F$, which is unique up to isomorphism. 

Let $S_\sigma$ be the set of extensions of $\sigma$ to an embedding of $E$ in $L$. Now, let $L'$ be an algebraically closed field and $\tau: F\to L'$ be an embedding such that $L'/\tau F$ is an algebraic extension. We shall now show that $S_\sigma$ and $S_\tau$ are in bijection.

Due to preceeding results, we know that $L$ and $L'$ are isomorphic. Let $\lambda: L\to L'$ be a field isomorphism which extends the map $\tau\circ\sigma^{-1}$ on the field $\sigma F$. Now, for all $\sigma^*\in S_\sigma$, we know that $\lambda\circ\sigma^*$ is an extension of $\tau$ to an embedding of $E$ into $L'$. It is not hard to see that $\lambda\circ\sigma^*$ is an extension of $\tau$, therefore, $\lambda$ induces a bijection from $S_\sigma$ to $S_\tau$, and they have the same cardinality.

\begin{definition}[Separable Degree]
    The cardinality of $S_\sigma$ is denoted by $[E:F]_s$ and called the \textit{separable degree} of $E$ over $F$.
\end{definition}


\begin{theorem}
    Let $k\subseteq F\subseteq E$ be a tower. Then 
    \begin{equation*}
        [E:k]_s = [E:F]_s[F:k]_s
    \end{equation*}
    Furthermore, if $E$ is finite over $k$, then $[E:k]_s$ is finite and $[E:k]_s\le[E:k]$. 
\end{theorem}
\begin{proof}
    Let $\sigma: k\to L$ be an embedding of $k$ in an algebraically closed field $L$. Let $\{\sigma_i\}_{i\in I}$ be the family of distinct extensions of $\sigma$ to $F$ and for each $i$, let $\{\tau_{ij}\}$ be the family of distinct extensions of $\sigma_i$ to $E$. Note that $\tau_{ij}$ contains precisely $[E:F]_s[F:k]_s$ elements. Moreover, the restriction of any embedding of $E$ into $L$ to $F$ is one of the $\sigma_i$'s and therefore the embedding must be one of the $\tau_{ij}$'s and we have the desired conclusion.

    Since $E/k$ is finite, we have a tower of fields as follows: 
    \begin{equation*}
        k\subseteq k(\alpha_1)\subseteq k(\alpha_1,\alpha_2)\subseteq\cdots\subseteq k(\alpha_1,\ldots,\alpha_n) = E
    \end{equation*}

    Recall that for any field $K$, the number of extensions of an embedding of $K$ into an algebraically closed field $L$ to $K(\alpha)$ is equal to the number of distinct roots of the minimal polynomial of $\alpha$ over $K[X]$.

    As a result, we have $[K(\alpha):K]_s\le [K(\alpha):K]$ and working inductively, we have the desired conclusion.
\end{proof}

\begin{mdframed}[]
    It is important to note that in the last part of the previous proof, the equality holds if and only if the equality holds at each step in the tower, that is, if and only if the minimal polynomial for $\alpha_i$ has distinct roots.
\end{mdframed}

\begin{definition}[Separable]
    Let $E/k$ be a finite extension. Then $E$ is said to be \textit{separable} over $k$ if $[E:k]_s = [E:k]$. An element $\alpha$ algebraic over $k$ is said to be separable over $k$ if $k(\alpha)$ is separable over $k$. A polynomial $f(X)\in k[X]$ is called separable if it has no multiple roots.
\end{definition}

\begin{theorem}
    Let $E$ be a finite extension of $k$. Then $E$ is separable over $k$ if and only if each element of $E$ is separable over $k$.
\end{theorem}
\begin{proof}
    Suppose $E$ is separable over $k$ and $\alpha\in E$. Then we have a tower of finite extensions $k\subseteq k(\alpha)\subseteq E$. Due to a conclusion we made through the proof of the previous theorem, we must have $[k(\alpha):k]_s = [k(\alpha):k]$. Therefore, $\alpha$ is separable over $k$.

    Conversely, suppose each element of $E$ is separable over $k$. Let $E = k(\alpha_1,\ldots,\alpha_n)$. Then, we have the following tower: 
    \begin{equation*}
        k\subseteq k(\alpha_1)\subseteq\cdots\subseteq k(\alpha_1,\ldots,\alpha_n) = E
    \end{equation*}
    We shall inductively show $[k(\alpha_1,\ldots,\alpha_i):k]_s = [k(\alpha_1,\ldots,\alpha_i):k]$. The base case with $i = 1$ is trivial. Now, we have that 
    \begin{equation*}
        [k(\alpha_1,\ldots,\alpha_i):k(\alpha_1,\ldots,\alpha_{i - 1})]_s = [k(\alpha_1,\ldots,\alpha_i):k]_s/[k(\alpha_1,\ldots,\alpha_{i - 1}):k]_s
    \end{equation*}
    But since $\alpha_i$ is separable over $k$, it must be the case that $\alpha_i$ is separable over $k(\alpha_1,\ldots,\alpha_{i - 1})$, therefore, $k(\alpha_1,\ldots,\alpha_i)/k$ is separable. This completes the proof.
\end{proof}

\begin{theorem}
    Let $E/k$ be a finite separable extension and $K\supseteq E\supseteq k$ be the smallest normal extension of $k$ containing $E$. Then $K/k$ is separable.
\end{theorem}
\begin{proof}
    Follows from the fact that the compositum of separable extensions is separable.
\end{proof}

\begin{definition}[Infinite Separability]
    Let $E$ be an arbitrary algebraic extension of $k$. We say $E$ is \textit{separable} over $k$ if every finitely generated subextension is separable over $k$.
\end{definition}

\begin{definition}
    The compositum of all separable extensions of $k$ in a given algebraic closure $k^a$ is a separable extension, which is denoted by $k^s$, and called the \textit{separable closure} of $k$.

    As a matter of terminology, if $E$ is an algebraic extension of $k$, and $\sigma$ any embeding of $E$ in $k$ over $k$< then we call $\sigma E$ a conjugate of $E$ in $k^a$.
\end{definition}

\begin{corollary}
    Let $E/k$ be finite. Then the smallest normal extension of $k$ containing $E$ is the compositum of all the conjugates of $E$ in $E^a$.
\end{corollary}

\begin{definition}
    Let $\alpha$ be algebraic over $k$. If $\sigma_1,\ldots,\sigma_r$ are the distincts embeddings of $k(\alpha)$ in $k^a$ over $k$, then we call $\sigma_1\alpha,\ldots,\sigma_r\alpha$ the \textit{conjugates} of $\alpha$ in $k^a$. Note that these elements are simpy the distinct roots of the irreducible polynomial of $\alpha$ over $k$.
\end{definition}

\begin{theorem}[Primitive Element Theorem]
    Let $E/k$ be finite. There exists an element $\alpha\in E$ such that $E = k(\alpha)$ if and only if there exists only a finite number of fields $F$ such that $k\subseteq F\subseteq E$. If $E/k$ is finite separable, then there exists such an element $\alpha$.
\end{theorem}
\begin{proof}
    If $k$ is finite, then we know that the multiplicative group of $E$ is cyclic, which will therefore also generate $E$ over $k$. Henceforth, suppose $k$ is infinite.

    Suppose now that there are only finitely many intermediate fields. Consider $k(\alpha + c\beta)$ for $c\in K$. Obviously this properly contains $k$ and due to the Pigeon Hole Principle, there must be $c_1\ne c_2\in k$ such that $k(\alpha + c_1\beta) = k(\alpha + c_2\beta)$. As a result, $\beta = (c_1 - c_2)^{-1}(\alpha + c_1\beta - (\alpha + c_2\beta))\in k(\alpha + c_1\beta)$. Thus, $\alpha$ is also in that field. Hence, we see that $k(\alpha,\beta) = k(\alpha + c_1\beta)$. Proceeding inductively, we have the desired conclusion.

    Conversely, suppose $E = k(\alpha)$. Let $f(x) = m_\alpha(x)\in k[x]$ be the minimal polynomial for $\alpha$ over $k$. Let $F$ be an intermediate field. Then $g(x) = m_\alpha(x)\in F[x]$ divides $f(x)$. Let $\alpha_1,\ldots,\alpha_n$ be the roots of $f$ in an algebraic closure $k^a$ containing $E$. Then $g(x)$ is a monic polynomial with leading coefficient $1$, dividing $f(x)$ and is equal to a product of a subset of factors $(x - \alpha_i)$. Therefore, we may have only a finite number of intermediate fields.

    We shall show the statement about separable extensions using induction. Let $E = k(\alpha,\beta)$ be a separable extension. Let $\sigma_1,\ldots,\sigma_n$ be the distinct embeddings of $k(\alpha,\beta)$ in $k^a$ over $k$. Consider the polynomial: 
    \begin{equation*}
        P(x) = \prod_{i\ne j}\left((\sigma_i\beta - \sigma_j\beta)x + (\sigma_i\alpha - \sigma_j\alpha)\right)
    \end{equation*}
    Since we may never have both $\sigma_i\alpha = \sigma_j\alpha$ and $\sigma_i\alpha = \sigma_j\alpha$, $P(x)$ is a non-zero polynomial. Therefore, there is $c\in k(\alpha,\beta)$ such that $P(c)\ne 0$. As a result, the elements $\sigma_i(\alpha + c\beta)$ are all distinct. Hence, $\sigma_i$'s are a subset of the embeddings of $k(\alpha + c\beta)$ over $k$ in $k^a$, consequently, 
    \begin{equation*}
        n\le [k(\alpha + c\beta):k]_s\le[k(\alpha + c\beta):k]\le[k(\alpha,\beta):k] = n
    \end{equation*}
    and we have the desired conclusion.
\end{proof}

\begin{definition}
    If $E/k$ is algebraic and there is $\alpha\in E$ such that $E = k(\alpha)$, then $\alpha$ is a \textit{primitive element} of $E$ over $k$.
\end{definition}