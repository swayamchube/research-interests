\begin{definition}[Integral Extension]
    Let $A\subseteq B$ be a subring. Then, $\alpha\in B$ is said to be \textit{integral} over $A$ if it satisfies a monic polynomial in $A[x]$. The extension $A\hookrightarrow B$ is said to be integral if every element of $B$ is integral over $A$.

    Similarly, if $\fraka\subseteq A$ is an ideal, then $\alpha\in B$ is said to be \textit{integral} over $\fraka$ if it satisfies a monic polynomial in $A[x]$ with coefficients in $\fraka$.
\end{definition}

\begin{theorem}\thlabel{thm:equivalence-integral-extension}
    Let $A\subseteq B$ be a subring and $\alpha\in B$. Then, the following are equivalent: 
    \begin{enumerate}[label=(\alph*)]
        \item $\alpha$ is integral over $A$
        \item $A[\alpha]$ is a finitely generated $A$-module 
        \item $A[\alpha]$ is contained in a subring $C$ of $B$ such that $C$ is a finitely generated $A$-module 
        \item There is a faithful $A[\alpha]$-module $M$ which is finitely generated as an $A$-module.
    \end{enumerate}
\end{theorem}
\begin{proof}
\begin{description}
    \item[$(a)\Longrightarrow(b)$:] If $\alpha^n + a_{n - 1}\alpha^{n - 1} + \cdots + a_0 = 0$. Then, it is not hard to argue that $\{1,\alpha,\ldots,\alpha^{n - 1}\}$ generated $A[\alpha]$ over $A$.
    \item[$(b)\Longrightarrow(c)$:] Take $C = A[\alpha]$ 
    \item[$(c)\Longrightarrow(d)$:] $C$ is a faithful $A[\alpha]$ module which is a finitely generated $A$-module.
    \item[$(d)\Longrightarrow(a)$:] Let $\phi: M\to M$ be the map $m\mapsto\alpha\cdot m$. We have $\phi(M)\subseteq AM$, consequently, due to \thref{prop:CH-type} (since $\fraka = A$ is an ideal in $A$), there are $a_i\in A$ such that 
    \begin{equation*}
        (\alpha^n + a_{n - 1}\alpha^{n - 1} + \cdots + a_0)\cdot m = 0
    \end{equation*}
    for each $m\in M$. But since $M$ is a faithful $A[\alpha]$-module, we must have $\alpha^{n} + a_{n - 1}\alpha^{n - 1} + \cdots + a_0 = 0$, whereby $\alpha$ is integral over $A$.\qedhere
\end{description}
\end{proof}

\begin{corollary}
    If $B$ is a finite $A$-algebra, then $B/A$ is an integral extension. In particular, every element of $B$ is integral over $A$.
\end{corollary}

\begin{proposition}
    Let $G$ be a finite group of ring automorphisms of $A$ and let 
    \begin{equation*}
        A^G := \{a\in A\mid g\cdot a = a,~\forall g\in G\}.
    \end{equation*}
    Then, $A/A^G$ is an integral extension.
\end{proposition}
\begin{proof}
    It is easy to see that $A^G$ is a subring of $A$. For any $a\in A$, consider the monic polynomial 
    \begin{equation*}
        f(x) = \prod_{\sigma\in G}(x - \sigma(a)).
    \end{equation*}
    This is obviously a polynomial with coefficients in $A^G$ and has $x$ as a root. Thus, $x$ is integral over $A^G$.
\end{proof}

\begin{proposition}\thlabel{prop:integral-finite-type}
    Let $\{\alpha_i\}_{i = 1}^n$ be elements of $B$, each integral over $A$. Then the ring $A[\alpha_1,\ldots,\alpha_n]$ is a finitely generated $A$-module, equivalently, a finite $A$-algebra.
\end{proposition}
\begin{proof}
    Let $A_k$ denote the subring $A[\alpha_1,\dots,\alpha_k]$ for $k\ge 1$. We shall induct on $k$ with the convention $A_0 = A$. Obviously $A_0$ is a finite $A$-algebra. We have $A_{k + 1} = A_k[\alpha_{k + 1}]$ and thus is a finite $A_k$-algebra. But since $A_k$ is a finite $A$-algebra, so is $A_{k + 1}$, thereby completing the proof.
\end{proof}

\begin{corollary}
    The set $C$ of elements of $B$ which are integral over $A$ is a subring of $B$ containing $A$.
\end{corollary}
\begin{proof}
    Let $\alpha,\beta\in C$. Then, $A[\alpha,\beta]$ is a finite $A$-algebra. Now, $A\subseteq A[\alpha - \beta]\subseteq A[\alpha,\beta]$ and $A\subseteq A[\alpha\beta]\subseteq A[\alpha,\beta]$ whereby both $\alpha - \beta,\alpha\beta\in C$ and $C$ is a ring.
\end{proof}

The set $C$ as defined above is called the \textit{integral closure of $A$ in $B$}. If $C = A$, then $A$ is said to be \textit{integrally closed in $B$}.

\begin{theorem}\thlabel{thm:integral-extension-transitive}
    Let $A\subseteq B\subseteq C$ such that $B/A$ and $C/B$ are integral extensions. Then $C/A$ is an integral extension.
\end{theorem}
\begin{proof}
    Let $\alpha\in C$. Then, 
    \begin{equation*}
        \alpha^n + b_{n - 1}\alpha^{n - 1} + \cdots + b_0 = 0
    \end{equation*}
    for some $b_i\in B$. Then, $\alpha$ is integral over $B' = A[b_0,\ldots,b_{n - 1}]$, consequently, $B'[\alpha]$ is a finite $B'$-algebra. But since $B'$ is a finite $A$-algebra, $B'[\alpha]$ is a finite $A$-algebra and $\alpha$ is integral over $A$.
\end{proof}

\begin{corollary}
    Let $A\subseteq B$ and $C$ be the integral closure of $A$ in $B$. Then, $C$ is integrally closed in $B$.
\end{corollary}
\begin{proof}
    Let $\alpha\in B$ be integral over $C$. Then, $C[\alpha]$ is integral over $C$, whereby $C[\alpha] = C$.
\end{proof}

\begin{proposition}\thlabel{prop:int-ext-localization}
    Let $A\subseteq B$ be an integral extension. Then, 
    \begin{enumerate}[label=(\alph*)]
        \item if $\frakb\subseteq B$ is an ideal and $\pi: B\to B/\frakb$ is the canonical surjection, then $B/\frakb$ is integral over $\pi(A)$. In particular, due to the First Isomorphism Theorem, we see that $B/\frakb$ is integral over a copy of $A/\fraka$ where $\fraka = \frakb\cap A$.
        \item if $S\subseteq A$ is multiplicatively closed, then $S^{-1}B$ is integral over $S^{-1}A$.
    \end{enumerate}
\end{proposition}
\begin{proof}
\begin{enumerate}[label=(\alph*)]
    \item Let $\beta\in B/\frakb$, then there is some $\alpha\in B$ such that $\pi(\alpha) = \beta$. Then, there are $a_0,\ldots,a_{n - 1}\in A$ such that 
    \begin{equation*}
        \alpha^n + a_{n - 1}\alpha^{n - 1} + \cdots + a_0 = 0
    \end{equation*}
    whereby 
    \begin{equation*}
        \beta^n + \pi(a_{n - 1})\beta^{n - 1} + \cdots + \pi(a_0) = 0
    \end{equation*}
    and the conclusion follows. 
    \item Let $\alpha/s\in S^{-1}B$. Since $\alpha$ is integral over $A$, there are $a_0,\ldots,a_{n - 1}\in A$ such that 
    \begin{equation*}
        \alpha^n + a_{n - 1}\alpha^{n - 1} + \cdots + a_0 = 0
    \end{equation*}
    then 
    \begin{equation*}
        (\alpha/s)^n + (a_{n - 1}/s)(\alpha/s)^{n - 1} + \cdots + a_0/s^n = 0
    \end{equation*}
    which completes the proof.
\end{enumerate}
\end{proof}

\section{The Cohen-Seidenberg Theorems}

\subsection{Going Up Theorem}

\begin{proposition}\thlabel{prop:int-ext-field-field}
    Let $A\subseteq B$ be an integral extension of integral domains. Then $A$ is a field if and only if $B$ is a field.
\end{proposition}
\begin{proof}
    $\implies$ If $x\in B\backslash\{0\}$ is integral over $A$, then 
    \begin{equation*}
        x^n + a_{n - 1}x^{n - 1} + \dots + a_0 = 0
    \end{equation*}
    for some $a_i\in A$. Then, $x(x^{n - 1} + a_{n - 1}x^{n - 2} + \dots + a_1) = -a_0$, in particular, $x$ is a unit in $B$.

    $\impliedby$ Let $x\in A\backslash\{0\}$. Then, $x^{-1}\in B$ is integral over $A$ and satisfies an equation of the form 
    \begin{equation*}
        x^{-n} + a_{n - 1}x^{-(n - 1)} + \dots + a_0 = 0.
    \end{equation*}
    Multiplying this equation by $x^{n - 1}$, we have 
    \begin{equation*}
        x^{-1} = -(a_{n - 1} + a_{n - 2}x + \dots + a_0x^{n - 1})\in A,
    \end{equation*}
    whence $A$ is a field.
\end{proof}

\begin{proposition}\thlabel{prop:q-max-iff-p-max}
    Let $A\subseteq B$ be an integral extension, $\frakq\subseteq B$ a prime ideal and $\frakp = \frakq^c = \frakq\cap A$. Then $\frakq$ is maximal if and only if $\frakp$ is maximal.
\end{proposition}
\begin{proof}
    Due to \thref{prop:int-ext-localization}, $B/\frakq$ is integral over a copy of $A/\frakp$. The conclusion now follows from the above proposition.
\end{proof}

\begin{proposition}
    Let $A\subseteq B$ be an integral extension. Let $\frakq,\frakq'\subseteq B$ be prime ideals of $B$ such that $\frakq\subseteq\frakq'$. If $\frakq\cap A = \frakq'\cap A = \frakp$, then $\frakq = \frakq'$.
\end{proposition}
\begin{proof}
    Let $S = A\backslash\frakp$ and treat all rings and ideals as $A$-modules. Then, $S^{-1}A\subseteq S^{-1}B$ is an integral extension and since $\frakq\cap S = \frakq'\cap S = \emptyset$, the ideals $S^{-1}\frakq$ and $S^{-1}\frakq'$ are prime ideals in $B$ such that 
    \begin{equation*}
        S^{-1}\frakq\cap S^{-1}A = S^{-1}(\frakq\cap A) = S^{-1}\frakp = S^{-1}(\frakq'\cap A) = S^{-1}\frakq'\cap S^{-1}A
    \end{equation*}
    where all the above equalities follow from treating $\frakp,\frakq,\frakq',A$ as $A$-submodules of $B$, in particular, due to \thref{prop:localization-commutes-modules}.

    But note that $S^{-1}\frakp$ is maximal in $A$ whence $S^{-1}\frakq = S^{-1}\frakq'$ due to the previous proposition. But recall that under localization, the contraction after extension of prime ideals is the prime ideal itself, whereby the contraction of $S^{-1}\frakq$ is $\frakq$ whence $\frakq = \frakq'$.
\end{proof}

\begin{lemma}
    Let $A\subseteq B$ be rings, $B$ integral over $A$, and let $\frakp$ be a prime ideal of $A$. Then there is a prime ideal $\frakq$ of $B$ such that $\frakq\cap A = \frakp$.
\end{lemma}

\subsection{Going Down Theorem}

\begin{definition}
    An integral domain is said to be \textit{normal} if it is integrally closed in its field of fractions.
\end{definition}

For example, $\Z$ is integrally closed since the only algebraic integers in $\Q$ are the integers.

\begin{lemma}\thlabel{lem:localization-maps-closure-to-closure}
    Let $A\subseteq B$ be rings and $C$ the integral closure of $A$ in $B$. Let $S\subseteq A$ be multiplicatively closed. Then $S^{-1}C$ is the integral closure of $S^{-1}A$.
\end{lemma}
\begin{proof}
    Since $C$ is integral over $A$, we have that $S^{-1}C$ is integral over $S^{-1}A$. It remains to show that any element that is integral over $S^{-1}A$ is contained in $S^{-1}C$. Indeed, let $b/s\in S^{-1}B$ be an element in $S^{-1}A$ that is contained in the integral closure. Then, there are $a_i/s_i$ such that 
    \begin{equation*}
        (b/s)^n + a_{n - 1}/s_{n - 1}(b/s)^{n - 1} + \cdots + a_0/s_0 = 0
    \end{equation*}
    Let $t = s_1\cdots s_{n - 1}$ and multiply the equation throughout by $(st)^n$ to obtain 
    \begin{equation*}
        \frac{(bt)^n + b_{n - 1}(bt)^{n - 1} + \cdots + b_0}{1} = 0.
    \end{equation*}
    Thus, there is $u\in S$ such that 
    \begin{equation*}
        u\left[(bt)^n + b_{n - 1}(bt)^{n - 1} + \cdots + b_0\right] = 0
    \end{equation*}
    Again, multiply the equation by $u^{n - 1}$ to obtain 
    \begin{equation*}
        (ubt)^n + c_{n - 1}(ubt)^{n - 1} + \cdots + c_0 = 0,
    \end{equation*}
    consequently, $ubt$ is integral over $A$, therefore, lies in $C$. As a result, $b/s = (ubt)/(sut)\in S^{-1}C$. This completes the proof.
\end{proof}

\begin{lemma}
    Let $A$ be an integral domain and $S\subseteq A$ a multiplicatively closed subset. If $A$ is normal, then $S^{-1}A$ is normal.
\end{lemma}
\begin{proof}
    Let $K$ denote the field of fractions of $A$. Since $A$ is an integral domain, the natural map $A\to S^{-1}A$ is an inclusion. Moreover, the inclusion $A\to K$ maps every element of $A$ to a unit and thus induces an inclusion $S^{-1}A\to K$. We can now treat $A\subseteq S^{-1}A\subseteq K$. Since $K$ is a field, the field of fractions of $S^{-1}A$ must also be contained in $K$. Therefore, it suffices to show that $S^{-1}A$ is integrally closed in $K$. But from \thref{lem:localization-maps-closure-to-closure}, we see that $S^{-1}A$ is the integral closure of $S^{-1}A$ in $S^{-1}K = K$. This completes the proof.
\end{proof}

\begin{proposition}
    Let $A$ be an integral domain. Then, the following are equivalent: 
    \begin{enumerate}[label=(\alph*)]
        \item $A$ is normal 
        \item $A_\frakp$ is normal for all $\frakp\in\spec A$ 
        \item $A_\frakm$ is normal for all $\frakm\in\mspec A$
    \end{enumerate}
\end{proposition}
\begin{proof}
    $(a)\implies(b)$ follows from the previous lemma and $(b)\implies(c)$ is obvious. We shall show that $(c)\implies(a)$. Let $K$ be the field of fractions of $A$ and $C$ denote the integral closure of $A$ in $K$. Let $\iota:A\hookrightarrow C$ be the inclusion map. We shall show that $\iota$ is a surjection. Note that both $A$ and $C$ are integral domains and $C_\frakm$ is the integral closure of $A_\frakm$ in $K$ and therefore, in $Q(A_\frakm)$, consequently, $A_\frakm = C_\frakm$ due to $(c)$. As a result, $\iota_\frakm$ is a surjection for all maximal ideals $\frakm$ implying that $\iota$ is a surjection.
\end{proof}

\begin{lemma}
    Let $C$ be the integral closure of $A$ in $B$ and let $\fraka\subseteq A$ be an ideal. Then, the integral closure of $\fraka$ in $B$ is $\sqrt{\fraka^e}$ where the extension is taken through the inclusion $A\hookrightarrow C$.
\end{lemma}
\begin{proof}
    If $x\in C$ is integral over $\fraka$, then $x$ satisfies an equationo the form 
    \begin{equation*}
        x^r + a_{r - 1}x^{r - 1} + \dots + a_0
    \end{equation*}
    with $a_i\in\fraka$. Thus, $x^r\in\fraka^e$ whence $x\in\sqrt{\fraka^e}$.

    Conversely, suppose $x\in\sqrt{a^e}$, then there is a positive integer $n$ such that $x^n\in\fraka^e$. Then, $x^n = a_1x_1 + \dots + a_mx_m$ where each $a_i\in\fraka$ and $x_i\in C$. Let $M = A[x_1,\dots,x_m]$. Since each $x_i$ is integral over $A$, $M$ is a finitely generated $A$-module. Let $\phi: M\to M$ be the homomorphism given by $\phi(y) = x^ny$. Then, $\phi(M)\subseteq\fraka M$. Thus, $\phi$ satisfies and equation of the form 
    \begin{equation*}
        \phi^r + a_{r - 1}\phi^{r - 1} + \dots + a_0\id = 0
    \end{equation*}
    whre $a_i\in\fraka$. Thus, $x$ is integral over $\fraka$.
\end{proof}

\begin{proposition}
    Let $A\subseteq B$ be integral domains with $A$ integrally closed. Let $\alpha\in B$ be integral over an ideal $\fraka$ of $A$. Then $\alpha$, viewed as an element of $L := Q(B)\supseteq Q(A) =: K$ is algebraic over the field of fractions $K$ of $A$. Further, if the minimal polynomial of $\alpha$ over $K$ is given by $x^n + a_{n - 1}x^{n - 1} + \cdots + a_0$, then each $a_i$ is an element of $\sqrt{\fraka}$.
\end{proposition}
\begin{proof}
    Let $\alpha_1,\dots,\alpha_k$ be the distinct conjugates of $\alpha$ in $\overline K$, an algebraic closure of $K$ containing $L$. Then, each $\alpha_i$ is integral over $\fraka$. The irreducible polynomial of $\alpha$ over $K$ is given by 
    \begin{equation*}
        \prod_{i = 1}^k \left(x - \alpha_i\right)^{e}
    \end{equation*}
    for some exponent $e$. In particular, the coefficients of the non-leading terms are polynomials in th $\alpha_i$'s whence are integral over $\fraka$ and also lie in $A$, whence are elements of $\sqrt{\fraka}$. This completes the proof.
\end{proof}

\begin{theorem}[Going Down Theorem]
    Let $A\subseteq B$ be an integral extension of integral domains with $A$ integrally closed. Suppose $\frakp_1\supseteq\dots\supseteq\frakp_n$ are prime ideals in $A$ and correspondingly $\frakq_1\supseteq\dots\supseteq\frakq_m$ are prime ideals in $B$ with $m < n$ and $\frakq_i\cap A = \frakp_i$ for $1\le i\le m$, then there are prime ideals $\frakq_{m + 1}\supseteq\frakq_{n}$ with $\frakq_m\supseteq\frakq_{m + 1}$ such that $\frakq_i\cap A = \frakp_i$ for $m < i\le n$.
\end{theorem}
\begin{proof}
    We shall prove this in the case $m = 1$ and $n = 2$. This obviously suffices to prove the theorem in its full generality. Consider the composition of maps 
    \begin{equation*}
        A\longrightarrow B\longrightarrow B_{\frakq_1}
    \end{equation*}
    where the composition shall be denoted by $f: A\to B_{\frakq_1}$. It suffices to show that there is a prime in $B_{\frakq_1}$ contracting to $\frakp_2$. Due to \thref{thm:prime-is-contraction}, it suffices to show that $\frakp^{ec} = \frakp$ where the extension and contraction is taken with respect to $f$.

    Let $x\in\frakp_2 B_{\frakq_1}$. Then, $x = y/s$ for some $y\in B\frakp_2$ and $s\in S$. Note that $s$ is integral over the ideal $(1)$ in $A$ and thus its minimal polynomial over $K$ is of the form 
    \begin{equation*}
        t^r + a_{r - 1}t^{r - 1} + \dots + a_0
    \end{equation*}
    where $a_i\in A$. 
    
    Now, $y\in B$ and lies in $\frakp_2B\subseteq\sqrt{\frakp_2B}$ whence is integral over $\frakp_2$ and hence its minimal polynomial over $K$ is of the form 
    \begin{equation*}
        t^{r'} + b_{r' - 1}t^{r' - 1} + \dots + b_0
    \end{equation*}
    with $b_i\in\sqrt{\frakp_2} = \frakp_2$. Since $s = y/x$ in $Q(B)$, the minimal polynomials of $s$ and $y$ over $K$ must have the same degree, that is, $r = r'$ and $b_i = x^{r - i}a_i$ for $0\le i\le r - 1$. If $x\notin\frakp_2$, then $a_i\in\frakp_2$ for $0\le i\le r - 1$, which would imply $s^r\in\frakp_2B\subseteq\frakp_1B\subseteq\frakq_1$, i.e. $s\in\frakq_1$, which is absurd. Thus, $x\in\frakp_2$ whence $\frakp_2^{ec}\subseteq\frakp_2$, which completes the proof.
\end{proof}

\section{Noether's Normalization Lemma}

\begin{lemma}\thlabel{lem:towards-noether-normalization}
    Let $k$ be a field and $F\in k[X_1,\dots,X_n]$ a non-constant polynomial. Then there is a $k$-algebra automorphism 
    \begin{equation*}
        \varphi: k[X_1,\dots,X_n]\to k[X_1,\dots,X_n]
    \end{equation*}
    such that $\varphi(X_n) = X_n$ and 
    \begin{equation*}
        \varphi(F) = aX_n^d + f_{d - 1}X_n^{d - 1} + \cdots + f_1X_n + f_0
    \end{equation*}
    where $f_i\in k[X_1,\dots,X_{n - 1}]$ for $1\le i\le d - 1$.
\end{lemma}
\begin{proof}
    We shall pick an automorphism of the form $\varphi(X_i) = X_i + X_n^{t_i}$ for some positive integer $t_i$ for each $1\le i\le n - 1$. We shall choose these $t_i$'s at the end of the proof.

    First, note that for $1\le i\le n - 1$, $\varphi(X_i - X_n^{t_i}) = X_i$ whence $\varphi$ is a surjection. Since $k[X_1,\dots,X_n]$ is a noethering, $\varphi$ is an isomorphism.

    Let $\Lambda\subseteq\N^n$ be a finite subset such that 
    \begin{equation*}
        F = \sum_{\alpha\in\Lambda}a_\alpha X^\alpha
    \end{equation*}
    where $a_\alpha\in k^\times$ for each $\alpha\in\Lambda$. For each $\alpha\in\Lambda$, define $\omega(\alpha) = t_1\alpha_1 + \cdots + t_{n - 1}\alpha_{n - 1} + \alpha_n$.

    Choose a positive integer $N$ greater than 
    \begin{equation*}
        \max_{\alpha\in\Lambda}\max_{1\le i\le n}\alpha_i
    \end{equation*}
    and set $t_i = N^i$ for $1\le i\le n - 1$. It is not hard to see that all the $\omega(\alpha)$'s are distinct. 

    We have 
    \begin{equation*}
        \varphi(F) = \sum_{\alpha\in\Lambda}\left(a_\alpha\prod_{i = 1}^{n - 1}(X_i + X_n^{t_i})^{\alpha_i}\right)X_n^{\alpha_n}
    \end{equation*}
    and since the $\omega(\alpha)$'s are distinct, there is a unique term in the above expansion that contributes to the term with maximum exponent of $X_n$ whence the coefficient of this term is a constant in $K^\times$. This completes the proof.
\end{proof}

\begin{theorem}[Noether Normalization]\thlabel{thm:noether-normalization}
    Let $k$ be a field and $A$ a finitely generated $k$-algebra. Then, there are $z_1,\dots,z_m\in A$ such that 
    \begin{enumerate}[label=(\alph*)]
        \item $z_1,\dots,z_m$ are algebraically independent over $k$\footnote{$m = 0$ is permitted}. That is, the evaluation map 
        \begin{equation*}
            \mathbf{ev}: k[X_1,\dots,X_m]\onto k[z_1,\dots,z_m]
        \end{equation*}
        from the ring of polynomials in $m$ variables over $k$ is an isomorphism. 
        \item $A$ is integral over $k[z_1,\dots,z_m]$.
    \end{enumerate}
\end{theorem}
\begin{proof}
    We shall prove this statement by induction on the cardinality $n$ of the smallest generating set of $A$ as a $k$-algebra. The base case with $n = 0$ is trivial. Since $A$ is a finitely generated $k$-algebra, there is a surjective ring homomomrphism 
    \begin{equation*}
        \pi: k[X_1,\dots,X_n]\onto A.
    \end{equation*}
    Choose a non-constant polynomial $G\in\ker\pi$. Due to \thref{lem:towards-noether-normalization}, there is an automorphism $\varphi$ of $k[X_1,\dots,X_n]$ which sends $G$ to a polynomial $F$ of the form 
    \begin{equation*}
        aX_n^d + f_{d - 1}X_n^{d - 1} + \cdots + f_1X_n + f_0
    \end{equation*}
    where $a\in k^\times$. We now have the following sequence of ring homomorphisms
    \begin{equation*}
        k[X_1,\dots,X_n]\stackrel{\varphi^{-1}}{\longrightarrow}k[X_1,\dots,X_n]\stackrel{\pi}{\longrightarrow} A
    \end{equation*}
    with $F\in\ker(\pi\circ\varphi^{-1})$. Let $x_i = (\pi\circ\varphi^{-1})(X_i)$, then, $F(x_1,\dots,x_n) = 0$. That is, 
    \begin{equation*}
        x_n^d + a^{-1}f_{d - 1}(x_1,\dots,x_{n - 1})x_n^{d - 1} + \cdots + a^{-1}f_0(x_0,\dots,x_{n - 1}) = 0,
    \end{equation*}
    and thus, $x_n$ is algebraic over $B = k[x_1,\dots,x_{n - 1}]$. Due to the induction hypothesis, there are algebraically independent $z_1,\dots,z_m\in B$ such that $B$ is integral over $k[z_1,\dots,z_m]$. 

    We have shown that $x_n$ is integral over $B$ and thus $B\subseteq A$ is an integral extension whence $k[z_1,\dots,z_m]\subseteq A$ is an integral extension. This completes the proof.
\end{proof}

\subsection{Various Forms of the Nullstellensatz}

\begin{lemma}[Zariski's Lemma]
    Let $K/k$ be an extension of fields such that $K$ is a finitely generated $k$-algebra. Then, $K/k$ is a finite extension.
\end{lemma}
\begin{proof}
    According to \thref{thm:noether-normalization}, there are $z_1,\dots,z_m\in K$ such that $K$ is integral over $k[z_1,\dots,z_m]$, which is an integral domain whence a field due to \thref{prop:int-ext-field-field}. We note that $m$ may not be positive since a polynomial ring can never be a field. Hence, $K/k$ is algebraic and since $K$ is a finitely generated $k$-algebra, the extension $K/k$ must be finite.
\end{proof}

\begin{theorem}[Hilbert's Nullstellensatz, Weak Form 1]\thlabel{thm:weak-nullstellensatz-1}
    Let $k$ be an algebraically closed field. Then, any maximal ideal in $k[x_1,\dots,x_n]$ is of the form $(x_1 - a_1,\dots,x_n - a_n)$.
\end{theorem}
\begin{proof}
    It suffices to show the converse. Let $\frakm$ be a maximal ideal in $k[x_1,\dots,x_n]$. We now have a commutative diagram 
    \begin{equation*}
        \xymatrix {
            k\ar@{^(->}[r]^-{\iota}\ar[rd]_-{\varphi} & k[x_1,\dots,x_n]\ar@{->>}[d]^-{\pi}\\
            & k[x_1,\dots,x_n]/\frakm = K
        }
    \end{equation*}
    where $\varphi := \pi\circ\iota$.

    The map $\varphi$ gives $K$ the structure of a finitely generated $k$-algebra and thus $\varphi$ is surjective (since $k$ is algebraically closed). Let $\pi(x_i) = a_i$ for $1\le i\le n$. Due to the surjectivity of $\varphi$, for each $a_i$, there is some $a_i'\in k$ with $\varphi(a_i') = a_i$ whence $\pi(x_i - a_i') = 0$ and 
    \begin{equation*}
        (x_1 - a_1',\dots,x_n - a_n')\subseteq\frakm.
    \end{equation*}
    But since the former is a maximal ideal, we must have equality.
\end{proof}

\begin{theorem}[Hilbert's Nullstellensatz, Weak Form 2]\thlabel{thm:weak-nullstellensatz-2}
    Let $k$ be an algebraically closed field and $S\subseteq k^n$. Then, $I(S) = (1)$ if and only if $S = \emptyset$.
\end{theorem}
\begin{proof}
    $(\implies)$ If $S\ne\emptyset$, then let $a = (a_1,\dots,a_n)$ be a point in $S$. Then, $I(S)\subseteq\frakm_a = (x_1 - a_1,\dots,x_n - a_n)$.
    $(\impliedby)$ If $I(S)\ne(1)$, then it is contained in some maximal ideal $\frakm = \frakm_a$ for some $a\in k^n$, whence $a\in S$. This completes the proof.
\end{proof}

\begin{theorem}[Hilbert's Nullstellensatz, Strong Form]
    Let $\fraka\subseteq k[x_1,\dots,x_n]$ be an ideal. Then, 
    \begin{equation*}
        I(V(\fraka)) = \sqrt{\fraka}
    \end{equation*}
\end{theorem}
The following proof is due to Rabinowitsch.
\begin{proof}
    First, note that the inclusion $\sqrt{\fraka}\subseteq I(V(\fraka))$ is obvious for if $f\in\sqrt{\fraka}$, then there is a positive integer $r$ such that $f^r\in\fraka$ whence $f^r$ vanishes at all points in $V(\fraka)$ and thus $f^r\in I(V(\fraka))$.

    We shall now prove the other inclusion. Since $\fraka$ is finitely generated, let $f_1,\dots,f_m$ be a set of generators for $\fraka$ and let $f\in I(V(\fraka))$. Consider now the ring $B = k[x_0,x_1,\dots,x_n]$ which contains $A = k[x_1,\dots,x_n]$ as a subring and treat all polynomials as elements of $B$. The polynomials
    \begin{equation*}
        f_1,\dots,f_m, 1 - x_0f
    \end{equation*}
    do not have any common zeros. Let $\frakb\subseteq B$ denote the ideal generated by these polynomials. Due to \thref{thm:weak-nullstellensatz-2} and the fact that the polynomials have no common zeros, we must have $\frakb = B$. Consequently, there are polynomials $g_0,\dots,g_n\in k[x_1,\dots,x_n]$ such that 
    \begin{equation*}
        1 = g_0(1 - x_0f) + g_1f_1 + \dots + g_mf_m.
    \end{equation*}

    Consider now the evaluation map $\mathbf{ev}: B\to k(x_1,\dots,x_n)$ which maps $x_0\mapsto 1/f$ and $x_i\mapsto x_i$ for $1\le i\le n$. It is not hard to see that this is a ring homomorphism. Under this map, the above equality transforms to 
    \begin{equation*}
        1 = g_1(1/f,x_1,\dots,x_n)f_1(x_1,\dots,x_n) + \dots + g_m(1/f,x_1,\dots,x_n)f_m(x_1,\dots,x_m).
    \end{equation*}
    Since all the $g_i$'s and $f_i$'s are polynomials, we may clear out the denominators by multiplying with a suitable power of $f$, say $f^N$. Then, we have 
    \begin{equation*}
        f^N = h_1(x_1,\dots,x_n)f_1 + \dots + h_m(x_1,\dots,x_n)f_m
    \end{equation*}
    whereby $f^N\in\fraka$ for some positive integer $N$ and equivalently, $f\in\sqrt{\fraka}$. This completes the proof.
\end{proof}