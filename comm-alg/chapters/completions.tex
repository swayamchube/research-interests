\section{Filtrations of Rings and Modules}

\begin{definition}[Filtered Ring]
    A \emph{filtered ring} $A$ is a ring $A$ together with a family $(A_n)_{n\ge 0}$ of additive subgroups of $A$ satisfying the conditions: 
    \begin{enumerate}[label=(\alph*)]
        \item $A_0 = A$,
        \item $A_{n + 1}\subseteq A_n$ for all $n\ge 0$,
        \item $A_mA_n\subseteq A_{m + n}$ for all $m,n\ge 0$.
    \end{enumerate}
\end{definition}

Substituting $m = 0$ In the last condition, we get $AA_n\subseteq A_n$ for all $n\ge 0$ whence each $A_n$ is in fact an ideal in $A$.

\begin{example}
\begin{enumerate}[label=(\alph*)]
    \item Let $\fraka\subseteq A$ be an ideal. Then, $A_n = \fraka^n$ for $n\ge 0$ gives the \emph{$\fraka$-adic filtration} on $A$.
    \item Let $B\subseteq A$ be a subring. Then, given any filtration $(A_n)_{n\ge 0}$ on $A$, the sequence $(B\cap A_n)_{n\ge 0}$ is a filtration on $B$, called the \emph{induced filtration on $B$}.
\end{enumerate}
\end{example}

\begin{definition}[Filtered Module]
    Let $A$ be a filtered ring with filtration $(A_n)_{n\ge 0}$. A \emph{filtered $A$-module} $M$ is an $A$-module $M$ together with a family $(M_n)_{n\ge 0}$ of additive subgroups of $M$ satisfying: 
    \begin{enumerate}[label=(\alph*)]
        \item $M_0 = M$,
        \item $M_{n + 1}\subseteq M_n$ for all $n\ge 0$,
        \item $A_mM_n\subseteq M_{m + n}$ for all $m,n\ge 0$.
    \end{enumerate}
\end{definition}

Substituting $m = 0$ in the last condition, we obtain $AM_n\subseteq M_n$ for all $n\ge 0$ whence each $M_n$ is an $A$-submodule of $M$.

\begin{example}
\begin{enumerate}[label=(\alph*)]
    \item A filtered ring is a filtered module over itself (with the filtration being the same).
    \item Let $\fraka\subseteq A$ be an ideal, then the sequence $(\fraka^nM)_{n\ge 0}$ of $A$-submodules of $M$ forms a filtration on $M$, called the \emph{$\fraka$-adic filtration}.
    \item More generally, given a filtration $(A_n)_{n\ge 0}$ on a ring $A$, define $M_n := A_nM$, which gives $M$ the structure of a filtered $A$-module.
    \item Let $M$ be a filtered $A$-module and $N$ an $A$-submodule of $M$. Then, we have an \emph{induced filtration} on $N$ and $M/N$ given by 
    \begin{equation*}
        (N\cap M_n)_{n\ge 0}\quad\text{and}\quad\left(\frac{N + M_n}{N}\right)_{n\ge 0}
    \end{equation*}
    respectively.
\end{enumerate}
\end{example}

\begin{definition}
    Let $M$ and $N$ be filtered $A$-modules (over a filtered ring). A \emph{homomorphism of filtered modules} is an $A$-module homomorphism $f: M\to N$ such that $f(M_n)\subseteq N_n$ for all $n\ge 0$.
\end{definition}

\begin{definition}
    A filtration $(M_n)_{n\ge 0}$ of an $A$-module $M$ is said to be an \emph{$\fraka$-filtration} if $\fraka M_n\subseteq M_{n + 1}$ for all $n\ge 0$. And a \emph{stable $\fraka$-filtration} if there is a positive integer $N$ such that $\fraka M_n = M_{n + 1}$ for $n\ge N$.
\end{definition}

\begin{definition}[Graded Ring]
    A \emph{graded ring} is a ring $A$ together with a family $(A_n)_{n\ge 0}$ of additive subgroups such that $A = \bigoplus_{n\ge 0} A_n$ and $A_mA_n\subseteq A_{m + n}$ for all $m,n\ge 0$. A nonzero element of $A_n$ is said to be a \emph{homogeneous element of degree $n$}.
\end{definition}

\begin{proposition}
    Let $A = (A_n)_{n\ge 0}$ be a graded ring with the specified grading. Then, 
    \begin{enumerate}[label=(\alph*)]
        \item $A_0$ is a subring, 
        \item $A$ is an $A_0$-module, 
        \item $A_n$ is an $A_0$-submodule for all $n\ge 0$.
    \end{enumerate}
\end{proposition}
\begin{proof}
\begin{enumerate}[label=(\alph*)]
    \item Since $A_0 A_0\subseteq A_0$, it is closed under multiplication and obviously under addition. There is some $n\ge 0$ and $a_0,\dots,a_n$ such that $1 = a_0 + \dots + a_n$. Thus, $a_i = a_0a_i + \dots + a_na_i$. Comparing degrees, $a_i = a_0a_i$ for $0\le i\le n$. Thus, 
    \begin{equation*}
        a_0 = a_0\cdot 1 = a_0(a_0 + \dots + a_n) = a_0a_0 + \dots + a_0a_n = a_0 + \dots + a_n = 1.
    \end{equation*}
    Hence, $1\in A_0$ and it is a subring.

    \item Trivial.
    \item Trivial.\qedhere
\end{enumerate}
\end{proof}

\begin{definition}
    Let $A$ be a graded ring. A \emph{graded $A$-module} is an $A$-module $M$ together with a family $(M_n)_{n\ge 0}$ of subgroups of $M$ such that $M = \bigoplus_{n\ge 0} M_n$ and $A_m M_n\subseteq M_{m + n}$. A nonzero element of $M_n$ is said to be a \emph{homogeneous element of degree $n$}.
\end{definition}

\begin{definition}
    If $M$ and $N$ are graded $A$-modules, then a \emph{homomorphism of graded $A$-modules} is an $A$-module homomorphism $f: M\to N$ such that $f(M_n)\subseteq N_n$ for all $n\ge 0$.
\end{definition}


\begin{proposition}
    Let $A = \bigoplus_{n\ge0} A_n$ be a graded ring. Then, the following are equivalent: 
    \begin{enumerate}[label=(\alph*)]
        \item $A$ is a Noetherian ring. 
        \item $A_0$ is noetherian and $A$ is an $A$-algebra of finite type.
    \end{enumerate}
\end{proposition}
\begin{proof}
    $\implies$ Let $A_+ := \bigoplus_{n\ge 1} A_n$. This is obviously an ideal of $A$ and $A/A_+\cong A_0$ and thus is noetherian. Since $A$ is noetherian, $A_+$ is a finitey generated ideal. Suppose it is generated by $x_1,\dots,x_s$, where we may suppose that the $x_i$'s are homogeneous with degrees $1,\dots, s$ respectively for $s > 0$. Let $A'$ denote the subring $A[x_1,\dots,x_s]$ of $A$.

    We shall inductively show that $A_n\subseteq A'$ for $n\ge 0$. The base case with $n = 0$ is trivial to prove. Let $y\in A_n$ for $n > 0$. Then, thre is a linear combination 
    \begin{equation*}
        y = \sum_{i = 1}^s a_ix_i
    \end{equation*}
    where $a_i\in A$. Comparing degrees, we see that $a_i\in A_{n - i}$ with the convention that $A_{k} = 0$ for $k < 0$. Due to the induction hypothesis, for each $i$, there is a polynomial $f_i$ with coefficients in $A_0$ such that $a_i = f_i(x_1,\dots,x_s)$. Let $g = a_1f_1 + \dots + a_sf_s$. Then, $y = g(x_1,\dots,x_s)$, whence, $y\in A'$. Thus, $A_n\subseteq A'$ for $n\ge 0$, consequently, $A = A'$.

    $\impliedby$ Follows from \thref{thm:hilbert-basis}.
\end{proof}

\begin{definition}[Rees Algebra]
    Let $\fraka\subseteq A$ be an ideal. Define the \emph{Rees algebra} to be 
    \begin{equation*}
        A^\ast := \bigoplus_{n\ge 0}\fraka^n
    \end{equation*}
    where element multiplication is the analogue of polynomial multiplication. That is, represent every element of $A^\ast$ as a polynomial
    \begin{equation*}
        a_0 + a_1T + \dots + a_nT^n
    \end{equation*}
    in some indeterminate $T$, where $a_i\in\fraka^i$. It is now easy to see how multiplication is defined. The identity element is simply given by $(1,0,\dots)$ or in the polynomial notation, simply the monomial $1$. This gives $A^\ast$ the structure of a commutative ring.
\end{definition}

\begin{definition}
    Let $M$ be a filtered $A$-module with filtration $(M_n)_{n\ge 0}$ over $A$ with the $\fraka$-adic filtration for some ideal $\fraka\subseteq A$. Define 
    \begin{equation*}
        M^\ast := \bigoplus_{n\ge 0} M_n.
    \end{equation*}
    As in the definition of the Rees algebra, we view elements of $M^\ast$ as formal polynomials 
    \begin{equation*}
        m_0 + m_1T + \dots + m_nT^n
    \end{equation*}
    in some indeterminate $T$, where $m_i\in M_i$. This has a natural action of the Rees algebra, $A^\ast$, by polynomial multiplication, which is well defined, since $\fraka^iM_j\subseteq M_{i + j}$ due to the filtered structure of $M$. This structure also shows that $M^\ast$ is a graded $A^\ast$-module with the above grading.
\end{definition}

\begin{proposition}
    $A$ is a noethering if and only if $A^\ast$ is a noethering.
\end{proposition}
\begin{proof}
    The converse is obvious since $A$ can be realized as a quotient of $A^\ast$. Suppose $A$ is a noethering. Then, $\fraka$ is finitely generated, say $\fraka = (a_1,\dots,a_n)$. Consider the map $\varphi: A[x_1,\dots,x_n]\to A[T]$ mapping $x_i\mapsto x_iT$ (this map exists due to the universal property of the polynomial ring). It is not hard to see that $\im\varphi = A^\ast\subseteq A[T]$, whence we are done due to \thref{thm:hilbert-basis}.
\end{proof}

\begin{proposition}\thlabel{prop:before-artin-rees}
    Let $A$ be a noethering, $M$ a finitely generated $A$-module and $(M_n)_{n\ge 0}$ an $\fraka$-filtration of $M$. Then, the following are equivalent: 
    \begin{enumerate}[label=(\alph*)]
        \item $M^*$ is a finitely generated $A^*$-module. 
        \item The filtration $(M_n)_{n\ge 0}$ is $\fraka$-stable.
    \end{enumerate}
\end{proposition}
\begin{proof}
    Since $M$ is a finitely generated module over a noethering, it is a noetherian $A$-module whence each $M_n$ is finitely generated. Let 
    \begin{equation*}
        Q_n := \bigoplus_{k = 1}^n M_kT^k
    \end{equation*}
    be an $A$-module and $M_n^\ast$ be the $A^\ast$-module generated by it. Note that $M_n^\ast$ is finitely generated since each $M_k$ is finitely generated. Further, these form an ascending chain 
    \begin{equation}\label{eq:ascending-m-ast}
        \left(M_0^\ast\subseteq M_1^\ast\subseteq\cdots\right)\subseteq M^\ast\tag{$\dagger$}
    \end{equation}
    Recall that $A^\ast$ is noetherian. Thus, $M^\ast$ is finitely generated if and only if $M^\ast$ is noetherian if and only if \eqref{eq:ascending-m-ast} stabilizes if and only if $M^\ast = M_{n_0}^\ast$ for some $n_0\in\N$. Now, let $n\ge n_0$. Let $m_{n + 1}\in M_{n + 1}$. Then, $m_{n + 1}T^{n + 1}\in M_{n + 1}^\ast = M_n^\ast$ and thus 
    \begin{equation*}
        m_{n + 1}T^{n + 1} = \sum_{k = 1}^r P^A_k(T)P^M_k(T)
    \end{equation*}
    where each $P^A_k$ is a polynomial in $A^\ast$ while $P^M_k$ is a polynomial in $Q_n$. Looking at the coefficient of $T^{n + 1}$, we see that $m_{n + 1}\in\fraka M_n$ whence $M_{n + 1} = \fraka M_n$ whereby the filtration $(M_n)_{n\ge 0}$ is stable. The converse is obvious and thus this is an equivalence thereby completing the proof.
\end{proof}

\begin{lemma}[Artin-Rees Lemma]\thlabel{lem:artin-rees}
    Let $A$ be a noethering, $\fraka\subseteq A$ an ideal, $M$ a finitely-gennerated $A$-module, $(M_n)_{n\ge 0}$ a stable $\fraka$-filtration of $M$. If $M'$ is an $A$-submodule of $M$, then $(M'\cap M_n)_{n\ge 0}$, the induced filtration on $M'$ is a stable $\fraka$-filtration of $M'$.
\end{lemma}
\begin{proof}
    We have 
    \begin{equation*}
        \fraka(M'\cap M_n)\subseteq\fraka M'\cap\fraka M_n\subseteq M'\cap M_{n + 1}
    \end{equation*}
    whence the induced filtration $(M'\cap M_n)_{n\ge 0}$ is an $\fraka$-filtration. Consider $M'^\ast$ induced by this filtration. This is an $A^\ast$-submodule of $M^\ast$. Due to \thref{prop:before-artin-rees}, $M^\ast$ is a finitely generated $A^\ast$-module whence is noetherian and thus $M'^\ast$ is a finitely generated $A^\ast$-module. Again, \thref{prop:before-artin-rees}, the filtration $(M'\cap M_n)_{n\ge 0}$ is $\fraka$-stable. This completes the proof.
\end{proof}

\begin{corollary}[Krull's Intersection Theorem]
    Let $A$ be a noethering and $\fraka\subseteq\frakR(A)$ a proper ideal. Let $M$ be a finitely generated $A$-module. Then $\bigcap_{n\ge 0}\fraka^nM=0$.
\end{corollary}
\begin{proof}
    Let $N := \bigcap_{n\ge 0}\fraka^n M$. Then, $\fraka^nM\cap N = N$ for all $n\in\N$. The filtration $\fraka^nM$ is $\fraka$-stable and thus, so is the induced filtration on $N$. But this means $(N)_{n\ge 0}$ is a stable $\fraka$-filtration, implying that $\fraka N = N$ and thus $N = 0$ from \thref{lem:nakayama}.
\end{proof}

\begin{definition}[Equivalent Filtrations]
    Let $M$ be a filtered $A$-module. Two filtrations $(M_n)_{n\ge 0}$ and $(M'_n)_{n\ge 0}$ are said to be \emph{equivalent} if there is a positive integer $k$ such that 
    \begin{equation*}
        M_{n + k}\subseteq M_n'\qquad\text{and}\qquad M'_{n + k}\subseteq M_n
    \end{equation*}
    for all $n\ge 0$.
\end{definition}


\section{Completion}

\begin{definition}
    An \emph{inverse system} of $A$-modules is a collection of $A$-modules $(M_n)_{n\ge 0}$ and homomorphisms $(\theta_n)_{n\ge 1}$ where $\theta_n: M_n\to M_{n - 1}$. If $\theta_n$ is surjective for all $n$, then the system is said to be a \emph{surjective system}.

    The \emph{inverse limit} of this system is the \emph{categorical limit} over the diagram 
    \begin{equation*}
        M_0\stackrel{\theta_1}{\longleftarrow} M_1\stackrel{\theta_2}{\longleftarrow} M_2\stackrel{\theta_3}{\longleftarrow}\cdots
    \end{equation*}
    in $A-\catMod$.
\end{definition}

\begin{example}
    Suppose we have a filtration $M = M_0\supseteq M_1\supseteq\cdots$, then we have an inverse system $(M/M_n)_{n\ge 0}$ with 
    \begin{equation*}
        \theta_{n + 1}: M/M_{n + 1}\onto M/M_n
    \end{equation*}
    being the natural map $x + M_{n + 1}\mapsto x + M_n$. Moreover, this is a \emph{surjective system}.
\end{example}

\begin{proposition}
    The inverse limit of an inverse system $((M_n)_{n\ge 0}, (\theta_n)_{n\ge1})$ exists and is unique upto unique isomorphism.
\end{proposition}
\begin{proof}
    It suffices to show existence since the ``unique upto unique isomorphism'' simply follows from the fact that the inverse limit is a ``universal object''.

    Let $N := \prod_{i\ge 0} M_i$ and $\pi_i: N\to M_i$ denote the projection. Let 
    \begin{equation*}
        M := \{(x_i)_{i\ge 0}\in N\mid \theta_{i + 1}(x_{i + 1}) = x_i\text{ for all }i\ge 0\}.
    \end{equation*}
    That this is a submodule is easy to verify. This is called the submodule of \emph{coherent sequences}. Next, define $f_i: M\to N_i$ by the restriction $f_i = \pi_i|_M$. We contend that 
    \begin{equation*}
        M = \limit_n M_n.
    \end{equation*}
    Indeed, let $P$ be another $A$-module with maps $g_i: P\to M_i$ such that $\theta_{i + 1}\circ g_{i + 1} = g_{i}$ for all $i\ge 0$. Define the map $h: P\to M$ by $h(p) = (g_0(p),g_1(p),\dots)$. Since this sequence is coherent, it is a valid map into $M$. Morover, for any $a\in A$, and $p,p'\in P$,
    \begin{equation*}
        h(p + ap') = (g_i(p) + ag_i(p')) = (g_i(p)) + a(g_i(p')) = h(p) + ah(p')
    \end{equation*}
    and thus, $h$ is an $A$-module homomorphism. Finally, 
    \begin{equation*}
        f_i\circ h(p) = f_i((g_j(p))_{i\ge 0}) = g_i(p)
    \end{equation*}
    as desired.
\end{proof}

\subsection*{Topological Interlude}

\begin{definition}
    Let $G$ be a topological abelian group. A \emph{fundamental system} of neighborhoods of $\{0\}$ is a descending chain of subgroups 
    \begin{equation*}
        G = G_0\supseteq G_1\supseteq\cdots.
    \end{equation*}
    such that $U\subseteq G$ is a neighborhood of $0$ if and only if it contains some $G_n$.
\end{definition}

\begin{proposition}\thlabel{prop:filtration-induces-topology}
    Let $G$ be an abelian group and $G=G_0\supseteq G_1\supseteq\cdots$ be a descending chain of subgroups of $G$. The collection 
    \begin{equation*}
        \scrB := \{g + G_i\mid g\in G\}
    \end{equation*}
    forms a basis for a topology on $G$. Under this topology, $G$ is a topological group.
\end{proposition}
\begin{proof}
    Let $i < j$ and $h\in (g + G_i)\cap(g' + G_j)$. Then, $g - h\in G_i$ and $g' - h\in G_j\subseteq G_i$ therefore, $g - g'\in G_i$. Consequently, 
    \begin{equation*}
        h + G_j = g' + G_j\subseteq g' + G_i = g + G_i
    \end{equation*}
    whence $h + G_j\subseteq(g + G_i)\cap(g' + G_j)$. This shows that $\mathscr B$ indeed forms a basis for some topology on $G$.

    Let $\varphi: G\times G\to G$ given by $\varphi(x, y) = x - y$. Suppose $(x, y)\in\varphi^{-1}(g + G_n)$. Then, 
    \begin{equation*}
        (x + G_n)\times(y + G_n)\subseteq\varphi^{-1}(g + G_n)
    \end{equation*}
    whence $\varphi^{-1}(g + G_n)$ is open. This completes the proof.
\end{proof}

\begin{definition}
    A sequence $(x_n)$ in a topological abelian group $G$ is said to be \emph{Cauchy} if for every open neighborhood $U$ of $0$, there is a positive integer $N$ such that $x_n - x_m\in U$ for all $m,n\ge N$.
\end{definition}

We shall now construct the completion of a group using Cauchy sequences.
\begin{itemize}
    \item Define a relation on the set of all Cauchy sequences in $G$ by $(x_n)\sim (y_n)$ if and only if $x_n - y_n\to 0$ as $n\to\infty$.
    \item That this is an equivalence relation is easy to see, for if $(x_n)\sim(y_n)$ and $(y_n)\sim(z_n)$, then 
    \begin{equation*}
        \lim_{n\to\infty}(x_n - z_n) = \lim_{n\to\infty}\left((x_n - y_n) + (y_n - z_n)\right) = \lim_{n\to\infty}(x_n - y_n) + \lim_{n\to\infty}(y_n - z_n) = 0.
    \end{equation*}
    \item Let $\wh G$ denote the equivalence classes under the above relation. Define the operation $[(x_n)] + [(y_n)] = [(x_n + y_n)]$. It is not hard to verify that this is well defined and endows $\wh G$ with the structure of an abelian group.
\end{itemize}

\begin{proposition}
    Let $\varphi: G\to\wh G$ denote the map $g\mapsto[(g)]$, the equivalence class of the constant sequence. This is a homomorphism of groups and $\ker\varphi = \bigcap U$ where the intersection ranges over all neighborhoods of $0$.
\end{proposition}
\begin{proof}
\end{proof}

\subsection*{Back to Completions}

Let $M$ be a filtered module with filtration $(M_n)_{n\ge 0}$ over a filtered ring $A$ with filtration $(A_n)_{n\ge 0}$. In accordance with \thref{prop:filtration-induces-topology}, both $M$ and $A$ have the structure of abelian topological groups.

\begin{proposition}
    Under the aforementioned induced topology, $A$ is a topological ring and $M$ is a topological module. This topology is called the \textbf{topology induced by the filtration}.
\end{proposition}
\begin{proof}
    We have seen already that $A$ forms a topological group under addition. It remains to show that multiplication is continuous. Let $\varphi: A\times A\to A$ be the multiplication map and $(a,b)\mapsto x\in A$. Let $A_n$ be a neighborhood $x$. Then, $(a + A_n)\times(b + A_n)$ maps into $x + A_n$ under $\varphi$ (this is where we use properties of the filtration) whence $\varphi^{-1}(x + A_n)$ is open in $A\times A$.

    A similar proof works for the module case.
\end{proof}

\begin{proposition}
    Equivalent filtrations induce the same topology on $M$.
\end{proposition}
\begin{proof}
    Trivial.
\end{proof}

We now have a topology on the module $M$ whence, we can form its completion, $\wh M$, as outlined in the previous (sub)section.
\begin{itemize}
    \item Let $(x_n)$ be Cauchy in $M$ and $a\in A$, in particular, let $a\in A_{m_0}$. Let $U$ be a neighborhood of $0$, which contains $M_{n_0}$ for some positive integer $n_0$. Then, there is a positive integer $n_1$ such that for all $m,n\ge n_1$, $x_m - x_n\in M_{n_0}$, whereby $a(x_m - x_n)\in A_{m_0}M_{n_0}\subseteq M_{n_0}\subseteq U$.

    \item Further, if $(x_n)\sim (y_n)$ and $a\in A$, we must have $(ax_n)\sim (ay_n)$ using a similar argument as above.
\end{itemize}
Thus $\wh M$ is also an $A$-module.

\begin{proposition}
    $\displaystyle\wh M\cong\limit_n M/M_n$ as $A$-modules.
\end{proposition}
\begin{proof}
    We shall define a map $\displaystyle\alpha: \wt M:=\limit_{n}M/M_n\to\wh M$. Let $(y_n)\in\wt M$ be a coherent sequence. For each $n\ge 0$, pick any $x_n\in M_n$ such that $\pi_n(x_n) = y_n$ where $\pi_n: M\to M/M_n$ is the natural projection. First, note that \todo{complete this}
\end{proof}

Now, let $A$ be a filtered ring, which can be regarded as a filtered module over itself. Then, $\wh A$ is an $A$-module. There is a natural product on this module, which can be seen easily using coherent sequences. That is, 
\begin{equation*}
    [(x_n)_{n\ge 0}]\cdot[(y_n)_{n\ge 0}] = [(x_ny_n)_{n\ge 0}].
\end{equation*}
Thus, $\wh A$ is an $A$-algebra, in particular, a ring in its own right.

\begin{definition}
    Given inverse systems $(M_n,\theta_n)$ and $(M_n',\theta_n')$, a \emph{morphism of inverse systems} $f: (M_n')_n\to (M_n)_n$ is a family of maps $f_n: M_n'\to M_n$ for $n\ge 0$ such that the diagram 
    \begin{equation*}
        \xymatrix {
            M_n'\ar[d]_{f_n} & M_{n + 1}'\ar[d]^{f_{n + 1}}\ar[l]_{\theta_{n + 1}'}\\
            M_n & M_{n + 1}\ar[l]_{\theta_{n + 1}}
        }
    \end{equation*}
    commutes for all $n\ge 0$. Exactness of such a sequence has the obvious definition.
\end{definition}

A morphism as above induces a map $\displaystyle f_\ast: \limit_n M_n'\to\limit_n M_n$ given by $(x_n)\mapsto (f_n(x_n))$. The commutativity of the diagram ensures that the sequence on the right is coherent.

\begin{proposition}
    Let $0\to\{A_n\}\to\{B_n\}\to\{C_n\}\to0$ be an exact sequence of inverse systems. Then 
    \begin{equation*}
        0\to\limit A_n\to\limit B_n\to\limit C_n
    \end{equation*}
    is exact. Further, if $\{A_n\}$ is a surjective system, then 
    \begin{equation*}
        0\to\limit A_n\to\limit B_n\to\limit C_n\to 0
    \end{equation*}
    is exact.
\end{proposition}
\begin{proof}
    Define the map $d^A: \prod_{n} A_n\to\prod_n A_n$ as 
    \begin{equation*}
        d^A((a_n)) = (a_n - \theta_{n + 1}(a_{n + 1})).
    \end{equation*}
    Similarly, define $d^B$ and $d^C$. These obviously are morphisms and fit into the following commutative diagram. 
    \begin{equation*}
        \xymatrix {
            0\ar[r] & \prod_n A_n\ar[r]\ar[d]_{d^A} & \prod_n B_n\ar[r]\ar[d]_{d^B} & \prod_n C_n\ar[r]\ar[d]_{d^C} & 0\\
            0\ar[r] & \prod_n A_n\ar[r] & \prod_n B_n\ar[r] & \prod_n C_n\ar[r] & 0
        }
    \end{equation*}
    From the Snake Lemma, we have an exact sequence 
    \begin{equation*}
        0\longrightarrow\ker d^A\longrightarrow\ker d^B\longrightarrow \ker d^C\longrightarrow\coker d^A.
    \end{equation*}
    It remains to show that $d^A$ is surjective when $\{A_n\}$ is a surjective system. Indeed, let $(a_n)\in\prod_n A_n$. Choose any $x_0\in A_0$ and inductively choose $x_{n + 1}$ such that $\theta_{n + 1}(x_{n + 1}) = x_n - a_n$. Then, $d^A((x_n)) = (a_n)$. This completes the proof.
\end{proof}

\begin{corollary}
    Let 
    \begin{equation*}
        0\longrightarrow M'\longrightarrow M\longrightarrow M''\longrightarrow 0
    \end{equation*}
    be an exact sequence and $(M_n)_{n\ge 0}$ a filtration of $M$ with induced filtrations on $M'$ and $M''$. If completions are taken with respect to these filtrations, the sequence 
    \begin{equation*}
        0\longrightarrow\widehat{M'}\longrightarrow\wh{M}\longrightarrow\wh{M''}\longrightarrow 0
    \end{equation*}
    is exact.
\end{corollary}

\begin{corollary}
    Let $M$ be an $A$-module with filtration $(M_n)_{n\ge 0}$ and completion $\wh M$. Then, the completion $\wh M_n$ of $M_n$ with respect to the induced filtration is a submodule of $\wh M$ and $\wh M/\wh M_n\cong M/M_n$ for all $n\ge 0$.
\end{corollary}
\begin{proof}
    The first assertion follows from the exactness of completion. As for the second assertion, again, using the exactness of completion, we have 
    \begin{equation*}
        \frac{\wh M}{\wh M_n}\cong\wh{\left(\frac{M}{M_n}\right)}.
    \end{equation*}
    Note that the induced topology on $M/M_n$ is the discrete topology whence 
    \begin{equation*}
        \wh{\left(\frac{M}{M_n}\right)}\cong\frac{M}{M_n}.\qedhere
    \end{equation*}
\end{proof}

\begin{corollary}
    Let $M$ be an $A$-module with filtration $(M_n)_{n\ge 0}$. This induces a filtration $(\wh M_n)_{n\ge 0}$ on $\wh M$ and $\wh{\wh M}\cong\wh M$.
\end{corollary}
\begin{proof}
    Note that the isomorphism $M/M_n\cong\wh{M/M_n}$. Now, consider the following commutative diagram.
    \begin{equation*}
        \xymatrix {
            0\ar[r] & M_{n + 1}\ar[r]\ar@{^{(}->}[d] & M\ar[r]\ar@{=}[d] & M/M_{n + 1}\ar[r]\ar@{->>}[d] & 0\\
            0\ar[r] & M_n\ar[r] & M\ar[r] & M/M_n\ar[r] & 0
        }
    \end{equation*}
    Taking completions, we obtain another commutative diagram
    \begin{equation*}
        \xymatrix {
            \frac{M}{M_{n + 1}}\ar[r]\ar[d] &\wh{\frac{M}{M_{n + 1}}}\ar[r]\ar[d] &  \frac{\wh M}{\wh M_{n + 1}}\ar[d]\\
            \frac{M}{M_n}\ar[r] &\wh{\frac{M}{M_n}}\ar[r] &\frac{\wh M}{\wh M_n}
        }
    \end{equation*}
    where all the horizontal arrows are isomorphisms. Thus, 
    \begin{equation*}
        \limit M/M_n\cong\limit\wh M/\wh M_n.\qedhere
    \end{equation*}
\end{proof}

Let $M$ be an $A$-module. There is a canonical map $M\to\wh M$ given by $m\mapsto(m)_{n\ge 0}$. Upon tensoring with $\wh A$, we have a map $\wh A\otimes_A M\to\wh A\otimes_A\wh M$ which, on elementary tensors is given by 
\begin{equation*}
    (a_n)_{n\ge 0}\otimes_A m\mapsto(a_n)\otimes_A(m)_{n\ge 0},
\end{equation*}
where we are denoting the elements of $\wh A$ by Cauchy sequences. 

Now, consider the map $\wh A\otimes_A\wh M\to\wh M$ given by 
\begin{equation*}
    (a_n)_{n\ge 0}\otimes_A(m_n)_{n\ge 0}\mapsto (a_nm_n)_{n\ge 0}.
\end{equation*}
Composing this with the previous maps, we obtain a map $\phi_M:\wh A\otimes_A M\to\wh M$ given by 
\begin{equation*}
    (a_n)_{n\ge 0}\otimes_A m\mapsto (a_nm)_{n\ge 0}.
\end{equation*}
It is not hard to verify that this map is indeed $\wh A$-linear between $\wh A$-modules. Note that this map is valid for all filtered modules $M$ over a filtered ring $A$. So is the following theorem.

\begin{theorem}
    If $M$ is finitely generated, then $\phi_M$ is surjective. Further, if $A$ is noetherian, then $\phi_M$ is an isomorphism.
\end{theorem}
\begin{proof}
    If $M$ and $N$ are two $A$-modules, then it is not hard to verify that the following diagram commutes: 
    \begin{equation*}
        \xymatrix {
            (\wh A\otimes_A M)\oplus(\wh A\otimes_A N)\ar[r]^-{\sim}\ar[d]_{\phi_M\oplus\phi_N} & \wh A\otimes_A (M\oplus N)\ar[d]^{\phi_{M\oplus N}}\\
            \wh M\oplus\wh N\ar[r]_-\sim & \wh{M\oplus N}
        }
    \end{equation*}
    where the horizontal map on the bottom is given by $(m_i)_{i\ge 0}\oplus(n_i)_{i\ge 0}\mapsto(m_i\oplus n_i)_{i\ge 0}$. Now, note that $\phi_A: \wh A\otimes_A A\to\wh A$ is obviously an isomorphism. Thus, inductively, $\phi_F$ is an isomorphism whenever $F$ is a finite dimensional free $A$-module.

    Now, if $M$ is finitely generated, then, there is a finite dimensional free module $F$ and an exact sequence $N\to F\to M\to 0$. This fits into a commutative diagram, 
    \begin{equation*}
        \xymatrix {
            N\ar[d]_{\phi_N}\ar[r] & F\ar[d]_{\phi_F}^{\sim}\ar[r] & M\ar[d]_{\phi_M}\ar[r] & 0\ar@{=}[d]\\
            \wh N\ar[r] & \wh F\ar[r] & \wh M\ar[r] & 0
        }
    \end{equation*}
    with exact rows. Thus, $\phi_M$ is a surjection. Now, if $A$ is noetherian, then $N$ is finitely generated, since it is a submodule of $F$, which is a noetherian $A$-module. Due to the Four Lemma, $\phi_M$ must be an injection whence an isomorphism. This completes the proof.
\end{proof}

\section{\texorpdfstring{$\fraka$}{a}-adic filtration}

Let $A$ be a fitered ring with filtration $(A_n)_{n\ge 0}$ and $M$ a filtered $A$-module with filtration $(M_n)_{n\ge 0}$. We shall show that $\wh M$ has the structure of a $\wh A$-module. Indeed, for $(x_n)_{n\ge 0}\in\wh M$ and $(a_n)_{n\ge 0}\in\wh A$, define 
\begin{equation*}
    (a_n)_{n\ge 0}\cdot(x_n)_{n\ge 0} = (a_nx_n)_{n\ge 0}.
\end{equation*}
To see that this is well defined, suppose $(a_n)_{n\ge 0}\sim(a_n')_{n\ge0}$ and $(x_n)_{n\ge 0}\sim(x_n')_{n\ge 0}$. Then, 
\begin{equation*}
    a_nx_n - a_n'x_n' = a_n(x_n - x_n') + (a_n - a_n')x_n'.
\end{equation*}
Consider a basic open set $A_m$ containing $0$. For sufficiently large $n$, $x_n - x_n'\in M_m$ and $a_n - a_n'\in A_m$. The conlusion now follows.

Next, we examine the functoriality of completion. If $f: M\to N$ is a homomorphism of filtered $A$-modules, then there is an induced map $\wh f: \wh M\to\wh N$ of filtered $\wh A$-modules given by 
\begin{equation*}
    f((x_n)_{n\ge 0}) = (f(x_n))_{n\ge0}.
\end{equation*}
This map is obviously $\wh A$-linear. It is also not hard to see that 
\begin{equation*}
    \wh{g\circ f} = \wh g\circ\wh f \qquad\text{ and }\qquad\wh{\id_M} = \id_{\wh M}
\end{equation*}
whence completion is a functor from the category of filtered $A$-modules to the category of $\wh A$-modules.

\begin{proposition}
    Let $A$ be a noetherian ring and $0\longrightarrow M'\stackrel{f}{\longrightarrow} M\stackrel{g}{\longrightarrow} M''\longrightarrow 0$ be a short exact sequence of finitely generated $A$-modules. Then, 
    \begin{equation*}
        0\longrightarrow\wh{M'}\stackrel{\wh f}{\longrightarrow}\wh{M}\stackrel{\wh g}{\longrightarrow}\wh{M''}\longrightarrow 0
    \end{equation*}
    is a short exact sequence of $\wh A$-modules where the completions are $\fraka$-adic for some ideal $\fraka\unlhd A$.
\end{proposition}
\begin{proof}
    We may treat $M'$ as a submodule of $M$. The induced filtration is just $(M'\cap\fraka^n M)_{n\ge0}$. Due to \thref{lem:artin-rees}, this is a stable $\fraka$-filtration of $M'$ whence, is equivalent to the filtration $(\fraka^n M')_{n\ge 0}$. Thus, the completions are also isomorphic. Next, the induced filtration on $M''$ is obviously $(\fraka^n M'')_{n\ge 0}$. The conclusion now follows from the exactness of completion.
\end{proof}

\begin{corollary}
    In particular, $\wh A$ is a flat $A$-module when $A$ is noetherian.
\end{corollary}

\begin{proposition}\thlabel{prop:a-adic-completion}
    Let $A$ be a noethering, $\frakb\unlhd A$ an ideal in $A$, and $\wh A$ the $\fraka$-adic completion for an ideal $\fraka\unlhd A$. Then, 
    \begin{enumerate}[label=(\alph*)]
        \item $\wh\frakb = \frakb\wh A$. 
        \item $\wh{\frakb^n} = \wh{\frakb}^n$. 
        \item $\frakb^n/\frakb^{n + 1}\cong\wh{\frakb}^n/\wh{\frakb}^{n + 1}$ for all $n\ge 0$. 
        \item $\wh{\frakb}$ is contained in the Jacobson radical of $\wh A$.
    \end{enumerate}
\end{proposition}
\begin{proof}
\begin{enumerate}[label=(\alph*)]
    \item Consider the injection $\frakb\into A$. Upon tensoring with $\wh A$, we obtain an injection $\frakb\otimes_A\wh{A}\into\wh{A}$. The image of $\frakb\otimes_A\wh{A}$ under this map is given by $\frakb\wh{A}$, which is also the extension of the ideal $\frakb$ under canonical map $A\to\wh A$.

    \item Follows (a), since 
    \begin{equation*}
        \wh{\frakb^n} = \underbrace{\frakb^n\wh A = (\frakb\wh A)^n}_{\text{since we are extending an ideal}} = (\wh b)^n.
    \end{equation*}

    \item Recall that $A/\frakb^n\cong\wh A/\wh{\frakb^n}$. Now, apply the third isomorphism theorem. 
    
    \item For any $a\in\wh\fraka$, note that $1 - a$ is a unit since it has an inverse $1 + a + a^2 + \cdots$. This converges since $a^n\in\fraka^n$ and the topology on $\wh A$ is given by the fundamental system $\wh A\supseteq\wh\fraka\supseteq\wh\fraka^2\supseteq\cdots$. Ths conclusion now follows.\qedhere
\end{enumerate}
\end{proof}

\begin{corollary}
    If $(A,\frakm,k)$ is a noetherian local ring, then $\wh R$ is a local ring with unique maximal ideal, $\wh{\frakm}$.
\end{corollary}

\begin{theorem}[Krull's Intersection Theorem]\thlabel{thm:krull-intersection}
    Let $A$ be a noethering, $M$ a finitely generated $A$-module and $\fraka\unlhd A$ an ideal. Let $\wh M$ be the $\fraka$-adic completion of $M$. Then, the kernel of the canonical map $M\to\wh M$ consists of precisely those elements of $M$ that are annihilated by some element of $1 + \fraka$. That is, 
    \begin{equation*}
        \bigcap_{n = 0}^\infty\fraka^n M = \left\{x\in M\mid\exists a\in\fraka,~(1 + a)x = 0\right\}.
    \end{equation*}
\end{theorem}
\begin{proof}
    Note that the kernel of the map is precisely equal to $\bigcap_{n = 0}^\infty\fraka^n M$. First, if $x\in M$ is annihilated by $1 + a$ for some $a\in\fraka$, then 
    \begin{equation*}
        x = -ax = a^2x = -a^3x = \cdots\in\bigcap_{n = 0}^\infty\fraka^n M.
    \end{equation*}
    Conversely, let $N = \bigcap_{n = 0}^\infty\fraka^n M$. Then, due to \thref{lem:artin-rees}, there is a positive integer $n$ such that 
    \begin{equation*}
        \fraka^k N = \fraka^{k}(N\cap\fraka^n M) = N\cap\fraka^{n + k}M = N
    \end{equation*}
    for all $k\ge 0$. Choosing $k = 1$ and applying \thref{corr:stronger-nakayama}, the desired conclusion follows.
\end{proof}

\begin{corollary}
    If $A$ is a noetherian domain and $\fraka\lhd A$ is a proper ideal, then $\bigcap_{n = 0}^\infty\fraka^n = (0)$.
\end{corollary}
\begin{proof}
    Every element of $1 + \fraka$ is nonzero and thus cannot annihilate any other nonzero element.
\end{proof}

\subsection{Associated Graded Stuff}

\begin{definition}[Associated Graded Ring]
    Let $A$ be a filtered ring with filtration $(A_n)_{n\ge0}$. Define 
    \begin{equation*}
        G_n(A) := A_n/A_{n + 1}\quad\text{ and }\quad G(A) := \bigoplus_{n\ge 0}G_n(A).
    \end{equation*}
    This has a natural multiplication structure given by $(a + A_{n + 1})(b + A_{m + 1}) = ab + A_{m + n + 1}$, where $a\in A_n$ and $b\in A_m$. This gives $G(A)$ the structure of a graded ring and is known as the \emph{associated graded ring} of $A$.
\end{definition}

To see that the multiplication is well defined, suppose $a'\in A_n$ and $b'\in A_m$ such that $a + A_{n + 1} = a' + A_{n + 1}$ and $b + A_{m + 1} = b' + A_{m + 1}$. Then, 
\begin{equation*}
    ab - a'b' = (a - a')b + a'(b - b')\in A_{m + n + 1}.
\end{equation*}

\begin{remark}
    If $A$ has the $\fraka$-adic filtration for an ideal $\fraka\unlhd A$, then we denote $G(A)$ by $G_\fraka(A)$ to be explicit.
\end{remark}

\begin{definition}[Associated Graded Module]
    Let $A$ be a filtered ring with filtration $(A_n)_{n\ge 0}$ and $M$ a filtered $A$-module with filtration $(M_n)_{n\ge 0}$. Define 
    \begin{equation*}
        G_n(M) := M_n/M_{n + 1}\quad\text{ and }\quad G(M) := \bigoplus_{n\ge 0}G_n(M).
    \end{equation*}
    This has a natural $G(A)$-module structure given by 
    \begin{equation*}
        (a + A_{m + 1})(x + M_{n + 1}) = ax + M_{m + n + 1}
    \end{equation*}
    for $a\in A_m$ and $x\in M_n$. This is called the \emph{associated graded module} of $M$.
\end{definition}

To see that the multiplication is well defined, suppose $a'\in A_m$ and $x'\in M_n$ such that $a + A_{m + 1} = a' + A_{m + 1}$ and $x + M_{n + 1} = x' + M_{n + 1}$. Then, 
\begin{equation*}
    ax - a'x' = (a - a')x + (x - x')a'\in M_{m + n + 1}.
\end{equation*}

\begin{definition}[Functoriality of $G$]
    Let $A$ be a filtered ring with filtration $(A_n)_{n\ge 0}$ and $M, N$ filtered $A$-modules with filtrations $(M_n)_{n\ge 0}$ and $(N_n)_{n\ge 0}$ respectively. Let $f: M\to N$ be a homomorphism of filtered $A$-modules. 

    Define $G(f):G(M)\to G(N)$ on homogeneous elements by 
    \begin{equation*}
        G(f)(x + M_{n + 1}) = f(x) + N_{n + 1}.
    \end{equation*}
    This is a homomorphism of graded $G(A)$-modules. Further, it is functorial, which is not hard to verify.
\end{definition}


\begin{theorem}\thlabel{thm:properties-of-G}
    Let $A$ be a noethering and $\fraka\unlhd A$. Then, 
    \begin{enumerate}[label=(\alph*)]
        \item $G_\fraka(A)$ is a noethering. 
        \item $G_\fraka(A)$ and $G_{\wh\fraka}(\wh A)$ are isomorphic as graded rings. 
        \item if $M$ is a finitely generated $A$-module, and $(M_n)_{n\ge 0}$ is a stable $\fraka$-filtration of $M$, then $G(M)$ is a finitely generated $G_\fraka(A)$-module.
    \end{enumerate}
\end{theorem}
\begin{proof}
\begin{enumerate}[label=(\alph*)]
    \item Let $\fraka = (x_1,\dots,x_s)$ as an $A$-module and let $\overline x_i$ denote the image of $x_i$ under the projection $A\onto A/\fraka$. It is obvious that $G_\fraka(A)\cong A/\fraka[\overline x_1,\dots,\overline x_s]$. Therefore, $G_\fraka(A)$ is a nothering.

    \item Follows from \thref{prop:a-adic-completion}.

    \item There is an $n_0\ge 0$ such that $M_{n_0 + r} = \fraka^r M_{n_0}$ for all $r\ge 0$. We contend that $G(M)$ is generated by $\bigoplus_{n = 0}^{n_0} G_n(M)$ as a $G_\fraka(A)$-module. Indeed, consider some homogeneous element $\overline x\in M_{n_0 + r}/M_{n_0 + r + 1}$, and $x\in M_{n_0 + r}$ such that $\overline x = x + M_{n_0 + r + 1}$. Then, there is some $y\in M_{n_0}$ and $a\in\fraka^r$ such that $ay = x$. It now follows that $\overline a\overline y = \overline x$ where $\overline y\in G_{n_0}(M)$ and $\overline a$ is the image of $a$ in $\fraka^r/\fraka^{r + 1}$.

    Finally, note that each $G_n(M)$ is a finitely generated $A/\fraka$-module for $n\le n_0$. Whence, $G(M)$ is a finitely generated $G_\fraka(A)$-module.\qedhere
\end{enumerate} 
\end{proof}

\begin{lemma}
    Let $A$ be a filtered ring and $M,N$ filtered $A$-modules. Let $\wh\phi$ and $G(\phi)$ denote the induced maps between the associated graded modules and completed modules respectively. Then, 
    \begin{enumerate}[label=(\alph*)]
        \item if $G(\phi)$ is injective, then $\wh\phi$ is injective.
        \item if $G(\phi)$ is surjective, then $\wh\phi$ is surjective.
    \end{enumerate}
\end{lemma}
\begin{proof}
    The map $\phi$ induces maps $\alpha_n: M/M_n\to N/N_n$, which is not hard to see from the universal property of the quotient. This gives us a commutative diagram 
    \begin{equation*}
        \xymatrix {
            0\ar[r] & M_n/M_{n + 1}\ar[d]_{G_n(\phi)}\ar[r] & M/M_{n + 1}\ar[d]_{\alpha_{n + 1}}\ar[r] & M/M_{n}\ar[d]_{\alpha_n}\ar[r] & 0\\
            0\ar[r] & N_n/N_{n + 1}\ar[r] & N/N_{n + 1}\ar[r] & N/N_{n}\ar[r] & 0
        }.
    \end{equation*}
    Due to the Snake Lemma, we have the following exact seqence 
    \begin{equation*}
        0\rightarrow\ker G_n(\phi)\rightarrow\ker\alpha_{n + 1}\rightarrow\ker\alpha_n\rightarrow\coker G_n(\phi)\rightarrow\coker\alpha_{n + 1}\rightarrow\coker\alpha_n\rightarrow 0.
    \end{equation*}
    Suppose $G(\phi)$ is injective. Then, each $G_n(\phi)$ is injective, whence we can inductively argue that $\ker\alpha_n = 0$ for every $n\ge 0$ since the base case $\ker\alpha_0 = 0$ is trivial.

    Consequently, $\alpha: \{M/M_n\}\to\{N/N_n\}$ is an injective map of surjective systems. Consequently, under the inverse limit, it induces an injective map $\wh\phi:\wh M\to\wh N$. Similarly, one can handle the case when $G(\phi)$ is surjective. This completes the proof.
\end{proof}

\begin{lemma}\thlabel{lem:complicated-lemma}
    Let $\fraka\unlhd A$ such that $A$ is complete in the $\fraka$-topology and $M$ an $A$-module with $(M_n)_{n\ge 0}$ an $\fraka$-filtration of $M$ in which $M$ is Hausdorff. If $G(M)$ is a finitely generated $G(A)$-module, then $M$ is a finitely generated $A$-module.
\end{lemma}
\begin{proof}
    Suppose $G(M)$ is generated by the homogeneous elements $y_i$ for $1\le i\le s$ with homogeneous degree of $y_i$ being $n(i)\ge 0$. There is a corresponding $x_i\in M_{n(i)}$ whose image in $G_{n(i)}(M)$ is $y_i$. Let $F^{(i)}$ denote the $A$-module $A$ with $\fraka$-filtration given by $F^{(i)}_k = \fraka^{n(i) + k}$. Finally, let $F = \bigoplus_{i = 1}^s F^{(i)}$. Let $\phi^{(i)}: F^{(i)}\to M$ denote the map sending $1\in F^{(i)}$ to $x_i\in M$. This induces a homomorphism of filtered $A$-modules $\phi: F\to M$. This map in turn, induces a $G(A)$-module homomorphism $G(\phi): G(F)\to G(M)$.

    According to the way we had chosen the $x_i$'s, the map $G(\phi)$ is surjective and thus, $\wh\phi:\wh F\to\wh M$ is surjective. Let $\alpha: F\to\wh F$ and $\beta: M\to\wh M$ denote the canonical maps. The following diagram commutes.
    \begin{equation*}
        \xymatrix {
            F\ar[r]^{\phi}\ar[d]_\alpha & M\ar[d]^\beta\\
            \wh F\ar[r]_{\wh\phi} & \wh M
        }
    \end{equation*}
    Since $A$ is complete in the $\fraka$-adic topology we see that $A\cong\wh A$ and since $F$ is a free $A$-module, the map $\alpha$ must be an isomorphism. Further, since $M$ is Hausdorff in its chosen filtration, the map $\beta$ is an injection. Since the map $\wh\phi\circ\alpha$ is a surjection, it must be the case that $\phi$ is a surjection, consequently, $M$ is finitely generated.
\end{proof}

\begin{corollary}\thlabel{corr:complicated-lemma}
    With the hypotheses of \thref{lem:complicated-lemma}, if $G(M)$ is a noetherian $G(A)$-module, then $M$ is a noetherian $A$-module.
\end{corollary}
\begin{proof}
    Let $M'\subseteq M$ be a submodule. We shall show that $M'$ is finitely generated. If $(M_n)_{n\ge 0}$ is the filtration of $M$, then the induced filtration $(M'\cap M_n)_{n\ge 0}$ is also an $\fraka$-filtration, which we denote by $(M_n')_{n\ge 0}$. Note that the inclusion $M_n'\into M_n$ induces an injective homomorphism $M_n'/M_{n + 1}'\into M_n/M_{n + 1}$. Thus, the inclusion map $M'\into M$ which is also a map of filtered modules, induces an injective map $G(M')\into G(M)$. Since $G(M)$ is noetherian, $G(M')$ must be a finitely generated $G_\fraka(A)$-module. Finally, we also have 
    \begin{equation*}
        \{0\}\subseteq\bigcap_{n = 0}^\infty M_n'\subseteq\bigcap_{n = 0}^\infty M_n = \{0\}.
    \end{equation*}
    Now, we can complete using \thref{lem:complicated-lemma}.
\end{proof}


\begin{theorem}
    If $A$ is a noethering and $\fraka\unlhd A$, then the $\fraka$-adic completion $\wh A$ of $A$ is a noethering.
\end{theorem}
\begin{proof}
    Due to \thref{thm:properties-of-G}, $G_\fraka(A)\cong G_{\wh\fraka}(\wh A)$ and $G_{\fraka}(A)$ is a noethering. Apply \thref{corr:complicated-lemma} to the complete ring $\wh A$ and take $M = \wh A$ with the filtration $(\wh\fraka^n)_{n\ge 0}$. Note that this filtration induces a Hausdorff topology since $\wh a$ is contained in the Jacobson radical of $\wh A$. This completes the proof.
\end{proof}

\hrulefill 

\begin{theorem}
    Let $(A,\frakm, k)$ be a noetherian local ring and $M, N$ be finitely generated $A$-modules. Let $\wh{(\cdot)}$ denote the $\frakm$-adic completion. If $\wh M\cong\wh N$, then $M\cong N$.
\end{theorem}
\begin{proof}
    
\end{proof}