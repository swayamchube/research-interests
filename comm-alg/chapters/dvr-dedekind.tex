\section{General Valuations and Valuation Rings}

\begin{definition}[Valuation]
    A \emph{valuation} on a field $K$ is a map $v: K\to\Gamma\cup\{\infty\}$ where $\Gamma$ is an ordered abelian group such that for all $x,y\in K$,
    \begin{enumerate}
        \item $v(xy) = v(x) + v(y)$, that is, the restriction $v: K^\times\to\Gamma$ is a group homomorphism,
        \item $v(x + y)\ge\min\{v(x), v(y)\}$.
    \end{enumerate}
    The set 
    \begin{equation*}
        A = \{x\in K^\times\mid v(x)\ge 0\}
    \end{equation*}
    is called the \emph{valuation ring} of $K$ with respect to the valuation $v$. Simply stating ``$A$ is a valuation ring'' means $A$ is a valuation ring of $K = Q(A)$.
\end{definition}

That the set $A$ forms a ring follows from the fact that it is closed under addition, multiplication and subtraction.

\begin{proposition}
    Let $A$ be an integral domain and $K = Q(A)$, its field of fractions. Then, $A$ is a \emph{valuation ring} of $K$ iff for every $x\in K\backslash\{0\}$, we have $x\in A$ or $x^{-1}\in A$.
\end{proposition}
\begin{proof}
    The forward direction from the fact that $0 = v(1) = v(xx^{-1}) = v(x) + v(x^{-1})$. Conversely, let $\Gamma = K^\times/A^\times$ and $\pi: K^\times\onto\Gamma$ the natural projection. Define an order on $\Gamma$ as follows 
    \begin{itemize}
        \item Every element in $G$ is of the form $\pi(x)$ for $x\in K^\times$. According to the given hypothesis, $x\in A$ or $x^{-1}\in A$. In the former case, let $\pi(x)\ge 1_\Gamma$ and in the latter, $\pi(x) < 1_\Gamma$.
        \item To see that this is well defined, suppose $x,y\in K$ with $x/y\in A^\times$, then if $x\in A$ then $y = xu\in A$ where $u\in A^\times$, on the other hand, if $x^{-1}\in A$, then $y^{-1} = ux^{-1}\in A$ where $u\in A^\times$.
        \item This extends to a total order on $\Gamma$ by $\pi(x)\ge\pi(y)$ if and only if $\pi(xy^{-1})\ge 1_\Gamma$, that is, $xy^{-1}\in A$.
    \end{itemize}

    We now contend that $\pi$ is a valuation with valuation ring $A$. Since $\pi$ is a homomorphism, it suffices to check $\pi(x + y)\ge\min\{\pi(x),\pi(y)\}$. Indeed, suppose $\pi(x)\ge\pi(y)$, which is equivalent to stating $x/y\in A$. Then, $1 + x/y\in A$, consequently 
    \begin{equation*}
        \pi(x + y) = \pi(y(1 + x/y)) = \pi(y)\pi(1 + x/y)\ge\pi(y).
    \end{equation*}
    This completes the proof.
\end{proof}

\begin{proposition}
    Let $A$ be a valuation ring. Then 
    \begin{enumerate}[label=(\alph*)]
        \item $A$ is a local ring. 
        \item $A$ is normal.
    \end{enumerate}
\end{proposition}
\begin{proof}
\begin{enumerate}[label=(\alph*)]
    \item We shall show that the nonunits in $A$ form an ideal. Let $\frakm$ be the set of nonunits in $A$ and choose $x\in\frakm\backslash\{0\}$, $b\in A$. Then, $bx\ne 0$ since $x$ is not a zero divisor. We contend that $bx$ is a nonunit. For if not, then $b(bx)^{-1}$ would be an inverse of $x$.

    Next, let $x,y\in\frakm\backslash\{0\}$. According to the given condition, either $x/y$ or $y/x$ are in $A$. Without loss of generality, suppose $x/y\in A$. Then $x + y = y(1 + x/y)\in\frakm$ from the conclusion of the previous paragraph. Thus $\frakm$ is an ideal and $A$ is local.

    \item Indeed, let $\alpha\in K$ be integral over $A$. If $\alpha\in A$, there is nothing to prove. If not, then it satisifes an equation of the form 
    \begin{equation*}
        \alpha^n + b_{n - 1}\alpha^{n - 1} + \cdots + b_1\alpha + b_0
    \end{equation*}
    Upon multiplying by $\alpha^{-(n - 1)}$, we can represent $\alpha$ as a sum of elements in $A$, consequently, is an element of $A$, a contradiction.
\end{enumerate}
\end{proof}

\begin{proposition}
    Let $A$ be a domain. Then $A$ is a valuation ring if and only if the ideals in $A$ are totally ordered.
\end{proposition}
\begin{proof}
    $(\implies)$ Suppose not. Then, there are two distinct ideals $\fraka,\frakb$ with $\fraka\not\subseteq\frakb$ and $\frakb\not\subseteq\fraka$ whence we can pick $a\in\fraka\backslash\frakb$ and $b\in\frakb\backslash\fraka$. Since either $a/b\in A$ or $b/a\in A$, we must have $a|b$ or $b|a$. Without loss of generality, suppose $b|a$. Then, $a\in(b)\subseteq\frakb$, a contradiction. 

    $(\impliedby)$ Let $x = a/b\in K$. Consider the ideals $(a)$ and $(b)$ in $A$. Since the ideals of $A$ are totally ordered, either $(a)\subseteq (b)$ or $(b)\subseteq (a)$, and thus, either $x\in A$ or $x^{-1}\in A$. This completes the proof.
\end{proof}

\begin{definition}[B\'ezout Ring]
    A ring is said to be a \emph{B\'ezout ring} if every finitely generated ideal is principal.
\end{definition}

\begin{proposition}
    A ring is a valuation ring if and only if it is a local B\'ezout domain.
\end{proposition}
\begin{proof}
    Let $A$ be a valuation ring and $\fraka = (a_1,\ldots,a_n) = (a_1) + \cdots + (a_n)$. Since ideals in a valuation ring are totally ordered, there is an index $i$ such that $(a_j)\subseteq(a_i)$ for $1\le j\le n$, consequently, $\fraka = (a_i)$.

    Conversely, let $A$ be a local B\'ezout Domain and $x = a/b\in K = Q(A)$. If either $a$ or $b$ is a unit, then either $x$ or $x^{-1}\in A$. Then, there is $c\in A$ such that $(c) = (a,b)$ whence there are $a',b'\in A$ such that $a = ca'$ and $b = cb'$. Let $u\in A$ be such that $(u) = (a',b')$. Then, $(cu') = (c)$ whence $u$ is a unit. If neither $a'$ or $b'$ is a unit, then $(1) = (a',b') = (a') + (b')\subseteq\frakm$, a contradiction. Thus, either $a|b$ or $b|a$ which completes the proof.
\end{proof}

\section{Discrete Valuation Rings}

\begin{definition}[Discrete Valuation Ring]
    A valuation $v: K\to\Gamma\cup\{\infty\}$ is said to be a \emph{discrete valuation} when $\Gamma = \Z$ and $v$ is surjective. An integral domain $A$ is said to be a \emph{discrete valuation ring} if there is a discrete valuation $v$ on the field of fractions of $A$ such that $A$ is the corresponding valuation ring.
\end{definition}

First, since $A$ is a valuation ring of its field of fractions, say $K$, it is local and normal, i.e. integrally closed in $K$. Further, the maximal ideal $\frakm$ in $A$ is the set of all $x\in A$ with \underline{positive} valuations.

\begin{proposition}\thlabel{prop:dvr-is-local-pid}
    Let $A$ be a DVR. Then, $A$ is a local PID.
\end{proposition}
\begin{proof}
    Let $\frakm_k = \{x\in A\mid v(x)\ge k\}$. We first show that $\frakm_k$ is an ideal. Indeed, for all $x,y\in\frakm_k$, 
    \begin{equation*}
        v(x - y)\ge\min\{v(x), v(-y)\} = \min\{v(x),v(y)\}\ge k
    \end{equation*}
    and $v(xy) = v(x) + v(y)\ge k$.

    Next, we show that every non-zero ideal $\fraka$ in $A$ is one of the $\frakm_i$'s. Due to the well ordering of the naturals, there is an $x\in\fraka$ with $\displaystyle k = v(x) = \min_{a\in\fraka}v(a)$. Then, by the choice of $k$, $\fraka\subseteq\frakm_k$. Now, let $y\in\frakm_k$. Since $v$ is surjective, there is an element $z\in A$ with $v(z) = v(y) - v(x)$. Whence $xz\in\fraka$ and $v(xz) = v(y)$. Since $(xz) = (y)$, we must have $y\in\fraka$.

    Notice that these ideals form a descending chain 
    \begin{equation*}
        \frakm = \frakm_1\supseteq\frakm_2\supseteq\cdots.
    \end{equation*}

    Choose some $a\in A$ with $v(a) = 1$, which exists due to the surjectivity of $v$. Then, $\frakm = (a)$ and consequently, $\frakm_k = (a^k) = \frakm^k$. From this, we may conclude that $\frakm$ is the unique non-zero prime ideal in $A$ and every other ideal is a power of $\frakm$ and also principal. Thus $A$ is a local PID.
\end{proof}

\begin{theorem}\thlabel{thm:dvr-equivalence}
    Let $A$ be a noetherian local domain of Krull dimension $1$, $\frakm$ its maximal ideal and $k = A/\frakm$ its residue field. Then the following are equivalent: 
    \begin{enumerate}[label=(\alph*)]
        \item $A$ is a discrete valuation ring.
        \item $A$ is normal.
        \item $\frakm$ is principal. 
        \item $\dim_k(\frakm/\frakm^2) = 1$.
        \item Every non-zero ideal is a power of $\frakm$. 
        \item There is $x\in A$ such that every nonzero ideal is of the form $(x^k)$ for $k\ge 0$. 
    \end{enumerate}
\end{theorem}
\begin{proof}
    $(a)\implies(b)$ is obvious. 

    $(b)\implies(c)$. Let $a\in\frakm$. Since the ring is noetherian, $(a)$ has a primary decomposition, but since the Krull dimension is $1$, the only non-zero prime ideal is $\frakm$, we see that $\sqrt{(a)} = \frakm$. Since we are in a noethering, there is a positive integer $n$ such that $\frakm^n\subseteq (a)$ but $\frakm^{n - 1}\subsetneq(a)$. Let $b\in\frakm^{n - 1}\backslash(a)$ and $x = a/b$, $y = x^{-1} = b/a$ in $K = Q(A)$, the field of fractions. 

    First, since $b\notin(a)$, $y\notin A$ and therefore, is not integral over $A$. Since $\frakm$ is a finitely generated $A$-module, it cannot be an $A[y]$-module lest $y$ be integral over $A$ due to \thref{thm:equivalence-integral-extension}. Hence, $y\frakm\subsetneq\frakm$.

    Now consider $y\frakm$. For any $z\in\frakm$, $yz = bz/a\in A$ since $bz\in\frakm^n\subseteq(a)$. Thus, $y\frakm\subseteq A$. Since this is an ideal and is not contained in $\frakm$, we must have $y\frakm = A$, whence $\frakm = Ax = (x)$ and is principal. 

    $(c)\implies(d)$. Let $\frakm = (a)$ for some $a\in A$. Then, $\frakm/\frakm^2 = (\overline a)$ where $\overline a$ is the image of $a$. Thus, $\dim_k(\frakm/\frakm^2)\le 1$. Now, note that $\frakm\ne\frakm^2$, lest due to \thref{lem:nakayama}, we have $\frakm = 0$. Thus, $\dim_k(\frakm/\frakm^2)\ge 1$ and the conclusion follows.

    $(d)\implies(e)$. Let $\fraka$ be a proper non-zero ideal in $A$. Then, $\sqrt{\fraka} = \frakm$ as we have argued earlier and thus, there is a least positive integer $n$ such that $\frakm^n\subseteq\fraka$. Now, $A/\frakm^n$ is an artinian local ring with maximal ideal $\overline\frakm = \frakm/\frakm^2$. Consequently, 
    \begin{equation*}
        \dim_k(\overline\frakm/\overline\frakm^2) = \dim_k(\frakm/\frakm^2) = 1
    \end{equation*}
    whence, due to \texttt{<insert reference>}, every ideal in $A/\frakm^n$ is principal, in particular, $\overline\fraka$ is principal. \todo{Complete This Argument}

    $(e)\implies(f)$. Due to \thref{lem:nakayama}, $\frakm\supsetneq\frakm^2$, hence there is $x\in\frakm\backslash\frakm^2$. According to our hypothesis, $(x) = \frakm^n$ for some positive integer $n$. Due to our choice of $x$, we must have $n = 1$, whence $\frakm = (x)$. The conclusion now follows. 

    $(f)\implies(a)$. We shall explicitly create a valuation. First, note that we have $\frakm = (x)$ due to maximality and due to Nakayama's Lemma, $\frakm^k\ne\frakm^{k + 1}$ for if not, then $\frakm^k = 0$ whereby, $\frakm = 0$, upon taking radicals, a contradiction. 

    For each $a\in A$, $(a) = (x^k)$ for a unique $k$, since $(x^n)\supsetneq(x^{n + 1})$. Define $v(a) = k$ and extend it to $K = Q(A)$ by defining $v(a/b) = v(a) - v(b)$. This is obviously a well defined valuation and $v(a/b)\ge 0$ if and only if $(a) = (x^n)$ and $(b) = (x^m)$ for $n\ge m$, whence $a\in (b)$ and $a/b\in A$. Thus $A$ is the valuation ring of $K$ with respect to $v$. This completes the proof.
\end{proof}

\begin{proposition}
    $A$ is a DVR if and only if $A$ is a local PID which is not a field.
\end{proposition}
\begin{proof}
    If $A$ is a local PID which is not a field, then it is a noetherian local domain of Krull dimension $1$ with a principal maximal ideal. From \thref{thm:dvr-equivalence}, we have that $A$ is a DVR. Putting this together with \thref{prop:dvr-is-local-pid}, we have the desired conclusion.
\end{proof}

\begin{proposition}
    Let $A$ be a valuation ring that is not a field. Then $A$ is a DVR if and only if $A$ is noetherian.
\end{proposition}
\begin{proof}
    It suffices to show the converse. Since $A$ is noetherian, every ideal is finitely generated and thus principal. Hence, $A$ is a DVR.
\end{proof}

\section{Dedekind Domains}

\begin{theorem}\thlabel{thm:dedekind-domain-equivalence}
    Let $A$ be a noetherian domain of Krull dimension $1$. Then, the following are equivalent 
    \begin{enumerate}[label=(\alph*)]
        \item $A$ is integrally closed.
        \item Every primary ideal in $A$ is a prime power in a unique way.
        \item Every local ring $A_\frakp$ is a discrete valuation ring.
    \end{enumerate}
\end{theorem}
\begin{proof}
\end{proof}

\begin{definition}
    A ring satisfying the equivalent conditions of \thref{thm:dedekind-domain-equivalence}, is said to be a \emph{Dedekind domain}.
\end{definition}

\begin{theorem}
    In a Dedekind domain, every non-zero ideal has a unique factorization as a product of prime\footnote{Which in this case, are maximal.} ideals.
\end{theorem}
\begin{proof}
    From \thref{lem:noether-dim1-ideal-pp}, every ideal in a noetherian domain of Krull dimension $1$ has a unique factorization as a product of prime ideals. Then, from \thref{thm:dedekind-domain-equivalence} and \thref{thm:chinese-remainder}, the conclusion follows.
\end{proof}

\begin{proposition}
    Let $A$ be a Dedekind domain and $\fraka\subseteq A$ a nonzero ideal. Then, $A/\fraka$ is a principal ring.
\end{proposition}
\begin{proof}
    The ideal $\fraka$ has a prime factorization $\fraka = \frakp_1^{n_1}\cdots\frakp_s^{n_s}$ with $A/\fraka\cong\bigoplus_{i = 1}^s A/\frakp_i^{n_i}$. We shall show that each factor $A/\frakp_i^{n_i}$ is a principal ring, by showing that for every prime ideal $\frakp$, the ring $\overline A = A/\frakp^n$ is principal for every positive integer $n$.

    First, note that $\overline A$ must be artinian and local as we have argued in the previous chapters. Hence, due to \thref{lem:artin-local-pir}, it suffices to show that the maximal ideal in $\overline A$ is principal. Note that the maximal ideal in $\overline A$ is given by $p/\frakp^n$. If $n = 1$, then $A/\frakp^n$ is a field and there is nothing to prove. Now, suppose $n\ge 2$. Let $\overline p$ denote the maximal ideal $\frakp/\frakp^n$ in $A$. Then, $\overline\frakp^2 = \frakp^2/\frakp^n$, which may not be equal to $\overline{\frakp}$ due to \thref{lem:nakayama}.

    Choose some $\overline a\in\overline\frakp\backslash\overline\frakp^2$. We contend that $\overline\frakp = (\overline a)$. Let $a\in A$ be an element mapping to $\overline a$ under the projectio $A\twoheadrightarrow A/\frakp^n$. Then, $(a)\supseteq\frakp^n$, consequently, $\sqrt{(a)} = \frakp$ is maximal and thus $(a)$ is $\frakp$-primary, whence a power of $\frakp$. Since we chose $\overline{a}$ in $\overline\frakp\backslash\overline{\frakp}^2$, we must have $(a) = \frakp$ which completes the proof.
\end{proof}

\begin{corollary}
    Every ideal in a Dedekind domain is generated by at most two elements.
\end{corollary}
\begin{proof}
    \todo{Easy write up. Stop being lazy}
\end{proof}

\begin{theorem}
    The ring of integers $\mathcal{O}_K$ in an \underline{algebraic number field}\footnote{An algebraic number field is a finite field extension of $\Q$} $K\supseteq\Q$ is a Dedekind domain.
\end{theorem}
\begin{proof}
    \todo{Read the section on valuations and then add this}
\end{proof}

\begin{proposition}
    Let $A$ be a Dedekind domain and $\fraka,\frakb,\frakc\subseteq A$ be ideals. Then, 
    \begin{enumerate}[label=(\alph*)]
        \item $\fraka\cap(\frakb + \frakc) = \fraka\cap\frakb + \fraka\cap\frakc$ and
        \item $\fraka + \frakb\cap\frakc = (\fraka + \frakb)\cap(\fraka + \frakc)$.
    \end{enumerate}
\end{proposition}
\begin{proof}
    
\end{proof}

\section{Fractional Ideals}


\begin{definition}
    Let $A$ be an integral domain. A \emph{fractional ideal} of $A$ is a nonzero $A$-submodule $M$ of $K = Q(A)$, the field of fractions such that there is $d\in A$ with $dM\subseteq A$.
\end{definition}

The ideals contained in $A$ are now called ``integral ideals''. Obviously, every integral ideal is fractions. Let $M$ and $N$ be $A$-submodules of $K$. Define the modified colon operator as 
\begin{equation*}
    \langle M: N\rangle = \{x\in K\mid xN\subseteq M\}.
\end{equation*}

Similarly, one defines the sum and product of $A$-submodles of $K$ as 
\begin{align*}
    \sum_{i\in I}M_i &= \left\{\sum_{\text{finite}}m_i\mid m_i\in M_i\right\}\\
    MN &= \left\{\sum_{\text{finite}}x_iy_i\mid x_i\in M,~y_i\in N\right\}
\end{align*}

\begin{proposition}
    Let $M, N, P$ be $A$-submodule of $K = Q(A)$. Then, 
    \begin{equation*}
        M(N + P) = MN + MP
    \end{equation*}
\end{proposition}

\begin{proposition}
    Let $M$ and $N$ be $A$-submodules of $K$ and $S\subseteq A$ a multiplicative subset. Then 
    \begin{enumerate}[label=(\alph*)]
        \item $S^{-1}(MN) = (S^{-1}M)(S^{-1}N)$
        \item $S^{-1}\langle M:N\rangle\subseteq\langle S^{-1}M : S^{-1}N\rangle$. Equality holds when $N$ is finitely generated as an $A$-module.
    \end{enumerate}
\end{proposition}
\begin{proof}
    $(a)$. Let $x = (\sum_i m_in_i)/s\in S^{-1}(MN)$. Then, 
    \begin{equation*}
        x = \sum_{i}(m_i/s)(n_i/1)\in (S^{-1}M)(S^{-1}N).
    \end{equation*}
    On the other hand, if $x\in (S^{-1}M)(S^{-1}N)$, then $x = \sum_i(m_i/s_i)(n_i/t_i) = \sum_i (m_in_i/s_it_i)$. Let $s = \prod s_i$ and $t = \prod t_i$. Then, it is not hard to see that $x = m'n'/st\in S^{-1}(MN)$ and the conclusion follows. 

    $(b)$.
\end{proof}

\begin{proposition}
    Let $A$ be a noetherian domain. Then an $A$-submodule $M$ of $K = Q(A)$ is a fractional ideal if and only if $M$ is a finitely generated $A$-module.
\end{proposition}
\begin{proof}
    It is not hard to see that every finitely generated $A$-submodule $M$ of $K$ is fractional, for if it is generated by $x_1/y_1,\ldots,x_n/y_n$, then choosing $y = \prod_{i = 1}^n y_i$, we have $yM\subseteq A$.

    On the other hand, if $A$ is noetherian and $M$ a fractional ideal, then there is some $d\in A$ such that $dM\subseteq A$ and is an ideal, say $\fraka\subseteq A$. Thus $M = d^{-1}\fraka$ and is a finitely generated $A$-module.
\end{proof}

\begin{definition}
    Let $A$ be an integral domain. An $A$-submodule $M$ of $K = Q(A)$ is said to be \emph{invertible} if there is an $A$-submodule $N$ of $K$ with $MN = A$. 
\end{definition}

For an $A$-submodule $M$ of $K$, define
\begin{equation*}
    \langle A:M\rangle = \{x\in K\mid xM\subseteq A\}.
\end{equation*}
It is not hard to see that $\langle A:M\rangle$ is an $A$-submodule of $K$.

\begin{proposition}
    Let $A$ be an integral domain and $M$ an invertible ideal of $A$. Then, $M^{-1} = \langle A:M\rangle$ and $M$ is finitely generated.
\end{proposition}
\begin{proof}
    Let $N$ denote the inverse of $M$. Then 
    \begin{equation*}
        N\subseteq\langle A:M\rangle = \langle A:M\rangle MN\subseteq AN = N.
    \end{equation*}
    Since $M\langle A:M\rangle = A$, there exist, for $1\le i\le n$, $x_i\in M$ and $y_i\in\langle A:M\rangle$ such that $\sum_{i}x_iy_i = 1$. Hence, for any $x\in M$, we have 
    \begin{equation*}
        x = \sum_{i = 1}^n xx_iy_i = \sum_{i = 1}^n(y_ix)x_i.
    \end{equation*}
    Since each $y_i\in\langle A:M\rangle$, we have $y_ix\in A$ for $1\le i\le n$, thus $M$ is generated by $x_1,\ldots,x_n$, whence finitely generated.
\end{proof}

\begin{proposition}\thlabel{prop:inv-fractional-local}
    Let $M$ be a fractional ideal of an integral domain $A$. Then, the following are equivalent: 
    \begin{enumerate}[label=(\alph*)]
        \item $M$ is invertible. 
        \item $M$ is finitely generated and for each $\frakp\in\spec A$, $M_\frakp$ is invertible. 
        \item $M$ is finitely generated and for each $\frakm\in\mspec A$, $M_\frakm$ is invertible. 
    \end{enumerate}
\end{proposition}
\begin{proof}
    $(a)\implies(b)$ First, since $M$ is invertible, it is finitely generated as an $A$-module. We have 
    \begin{equation*}
        A_\frakp = (M\langle A:M\rangle)_\frakp = M_\frakp\langle A_\frakp:M_\frakp\rangle
    \end{equation*}
    whence $M_\frakp$ is invertible.

    $(b)\implies(c)$ Obvious. 

    $(c)\implies(a)$. Let $\fraka = M\langle A:M\rangle$. This is an integral ideal in $A$. Let $\iota:\fraka\into A$ denote the inclusion and let $\frakm\subseteq A$ be a maximal ideal. Then, 
    \begin{equation*}
        \fraka_\frakm = M_\frakm\langle A_\frakm:M_\frakm\rangle = A_\frakm.
    \end{equation*}
    Thus $\iota_\frakm$ is surjective for all maximal ideals $\frakm$ and due to \thref{prop:inj-surj-local}, $\fraka = A$.
\end{proof}

\begin{proposition}\thlabel{prop:dvr-fractional-ideal}
    Let $A$ be a local domain. Then $A$ is a DVR iff every non-zero fractional ideal of $A$ is invertible.
\end{proposition}
\begin{proof}
    $(\implies)$ Let $M$ be a fractional ideal of $A$ and $\frakm = (x)$. Then, there is $y\in A$ such that $yM\subseteq A$. Let $s > 0$ be chosen such that $(y) = (x^s)$. Then, $x^sM = yM\subseteq A$ is an ordinary ideal and is therefore equal to $(x^r)$ for some non-negative integer $r$. Then, $M = (x^{r - s})$ is principal and thus invertible. 

    $(\impliedby)$ First, every integral ideal is fractional and according to the hypothesis, finitely generated. Thus $A$ is noetherian. We shall now show that every nonzero proper integral ideal is a power of $\frakm$. Suppose not. Let $\Sigma$ be the set of all nonzero proper integral ideals in $A$ which are not powers of $\frakm$. Let $\fraka\in\Sigma$ be a maximal element\footnote{We do not need to invoke Zorn for this since the ring is Noetherian.}. Then, $\fraka\subsetneq\frakm$. But since $\frakm$ is invertible, we have 
    \begin{equation*}
        \frakm^{-1}\fraka\subseteq\frakm^{-1}\frakm = A
    \end{equation*}
    and thus $\frakm^{-1}\fraka$ is a proper integral ideal which contains $\fraka$, since $1\in\frakm^{-1}$. We contend that this containment is proper. For if not, then 
    \begin{equation*}
        \frakm^{-1}\fraka = \fraka\implies\fraka = \frakm\fraka
    \end{equation*}
    and due to \thref{lem:nakayama}, $\fraka = 0$. Thus, $\fraka\subsetneq\frakm^{-1}\fraka$ and due to the maximality of $\fraka$, there is a positive integer $k$ such that $\frakm^{-1}\fraka = \frakm^k$ whence $\fraka = \frakm^{k + 1}$, a contradiction. This completes the proof.
\end{proof}

\begin{proposition}
    Let $A$ be an integral domain. Then $A$ is a Dedekind domain iff every non-zero fractional ideal of $A$ is invertible.
\end{proposition}
\begin{proof}
    $(\implies)$ Let $M$ be a fractional ideal in $A$. Since $A$ is also noetherian, $M$ is finitely generated. Let $\frakp\in\spec A$. Then $M_\frakp$ is a fractional ideal in the DVR $A_\frakp$ whence invertible. We are done due to \thref{prop:inv-fractional-local}.

    $(\impliedby)$ \todo{convince yourself that the localization of a fractional ideal is a fractional ideal}
\end{proof}

\begin{corollary}
    If $A$ is a Dedekind domain, the non-zero fractional ideals of $A$ form a group with respect to multiplication.
\end{corollary}