\section{Rings of Fractions}

Define the relation $\sim_S$ on $A\times S$ by $(a,s)\sim_S(a',s')$ if there is $t\in S$ such that $t(s'a - sa') = 0$. That this is an equivalence relation is easy to verify. We shall use $a/s$ to denote the equivalence class $[(a,s)]$ in $A\times S/\sim_S$.

Consider the operations: 
\begin{equation*}
    \frac{a}{s} + \frac{a'}{s'} = \frac{s'a + sa'}{ss'}\qquad\frac{a}{s}\cdot\frac{a'}{s'} = \frac{aa'}{ss'}
\end{equation*}

It is not hard to see that these are well defined and endow $A\times S/\sim_S$ with a ring structure. We denote this ring by $S^{-1}A$ and is called the \textit{ring of fractions} of $A$ by $S$. 

There is a natural ring homomorphism $\varphi: A\to S^{-1}A$ given by $\varphi(x) = x/1$.
When $A$ is an integral domain and $S = A\backslash\{0\}$, $S^{-1}A$ is precisely the field of fractions.
Recall that if $\mathfrak p$ is a prime ideal in $A$, then $S = A\backslash\mathfrak p$ is a multiplicatively closed subset of $A$. We denote the ring $S^{-1}A$ by $A_{\mathfrak p}$.

\begin{theorem}
    The ring $A_{\mathfrak p}$ is local.
\end{theorem}
\begin{proof}
    Let $S = A\backslash\mathfrak p$ and define
    \begin{equation*}
        \mathfrak m = \left\{\frac{a}{s}~\bigg\vert~a\in\mathfrak p,~s\in S\right\}
    \end{equation*}
    It is not hard to see that $\mathfrak m$ is an ideal in $A_{\mathfrak p}$. We contend that $\mathfrak m$ is the ideal of non-units in $A_{\mathfrak p}$. Indeed, if $a/s\in\mathfrak m$ is a unit, then there is $b/t\in A_{\mathfrak p}$ such that $(ab)/(st) = 1$, consequently, there is $w\in S$ such that $w(ab - st) = 0$, whence $wst\in\mathfrak p$, a contradiction. 

    On the other hand, if $a/s\notin\mathfrak m$, then $a/s$ is a unit since $(a/s)\cdot(s/a) = 1$. Now, since the collection of all non-units forms an ideal, the ring must be local due to \thref{prop:non-unit-ideal-local}.
\end{proof}

Similarly, when we let $S = \{a^n\}_{n\ge 0}$ for some $a\in A$, we denote $S^{-1}A$ by $A_a$.

There is a degenerate case, when we allow $0\in S$, notice that the ring $S^{-1}A$ is the zero ring, since for all $a/s\in S^{-1}A$, we have $0(as) = 0$, therefore, $a/s = 0/s$.

\subsection{Universal Property}

Fix a multiplicative subset $S\subseteq A$. Let $\mathscr C$ denote the category with objects as pairs $(\phi, B)$ where $\phi: A\to B$ is a ring homomorphism such that $\phi(s)$ is a unit in $B$ for all $s\in S$. A morphism in this category is a map $f:(\phi, B)\to(\psi, C)$ making the following diagram commute.
\begin{equation*}
\xymatrix{
    A\ar[r]^{\psi}\ar[d]_\phi & C\\
    B\ar[ru]_f
}
\end{equation*}

The ring of fractions is an initial object in this category. Therefore, we have the following universal property. We shall verify in the ``proof'' that our construction of the field of fractions does satisfy this property and is therefore an initial object in $\mathscr C$.

\begin{proposition}
    Let $f: A\to B$ be a ring homomorphism such that $f(s)$ is a unit in $B$ for all $s\in S$. Then there is a unique ring homomorphism $g: S^{-1}A\to B$ making the following diagram commute 
    \begin{equation*}
    \xymatrix{
        A\ar[r]^f\ar[d]_\varphi & B\\
        S^{-1}A\ar@{.>}[ru]_-{\exists!g}
    }
    \end{equation*}
\end{proposition}
\begin{proof}
    Define the map $g: S^{-1}A\to B$ by $g(a/s) = g(a)g(s)^{-1}$. To see that this map is well defined, note that if $a/s = a'/s'$, then there is $t\in S$ such that $t(s'a - sa') = 0$, consequently, $g(t)(g(s')g(a) - g(s)g(a')) = 0$. As a result, $g(a)g(s)^{-1} = g(a')g(s')^{-1}$. From this, it follows immediately that $g$ is a ring homomorphism making the diagram commute.

    As for uniqueness, note that for all $a/s\in S^{-1}A$,
    \begin{equation*}
        g(a/s) = g(a/1)g(1/s) = g(a/1)g(s/1)^{-1} = f(a)f(s)^{-1}
    \end{equation*}
    which is fixed by the choice of $f$. This completes the proof.
\end{proof}

\section{Modules of Fractions}

Let $M$ be an $A$-module and $S\subseteq A$ be a multiplicatively closed subset. Define the relation $\sim_S$ on $M\times S$ by $(m,s)\sim_S(m',s')$ if and only if there is $t\in S$ such that $t(s'm - sm') = 0$. That this is an equivalence relation is easy to verify. We shall use $m/s$ to denote the equivalence class $[(m,s)]$ in $M\times S/\sim_S$.

It is not hard to see that $S^{-1}M$ forms an $A$-module. Further, it also has the structure of an $S^{-1}A$ module under the action 
\begin{equation*}
    \frac{a}{s}\cdot\frac{m}{t} = \frac{a\cdot m}{st}
\end{equation*}

Let $f: M\to N$ be an $A$-module homomorphism. Consider the map $S^{-1}f: S^{-1}M\to S^{-1}N$ given by 
\begin{equation*}
    S^{-1}f\left(\frac{m}{s}\right) = \frac{f(m)}{s}
\end{equation*}

We must first show that this is well defined. Indeed, if $m/s = m'/s'$, then there is $t\in S$ such that $t(s'm - sm') = 0$, consequently, $t(s'f(m) - sf(m')) = 0$, as a result, $f(m)/s = f(m')/s'$ in $S^{-1}M$. 

We now contend that $S^{-1}f$ is an $S^{-1}A$ module homomorphism. Indeed, we have 
\begin{equation*}
    S^{-1}f\left(\frac{m}{s} + \frac{a}{t}\frac{m'}{s'}\right) = S^{-1}f\left(\frac{ts' m + as m'}{sts'}\right) = \frac{f(ts'm + asm')}{sts'} = \frac{ts'f(m) + asf(m')}{sts'} = \frac{f(m)}{s} + \frac{f(m')}{s'}
\end{equation*}

Finally, let $f: M\to N$ and $g: N\to P$ be $A$-module homomorphisms. Then, 
\begin{equation*}
    S^{-1}(g\circ f)\left(\frac{m}{s}\right) = \frac{g(f(m))}{s}\qquad S^{-1}g\left(S^{-1}f\left(\frac{m}{s}\right)\right) = S^{-1}g\left(\frac{f(m)}{s}\right) = \frac{g(f(m))}{s}
\end{equation*}

\begin{theorem}\thlabel{thm:exactness-of-localization}
    $S^{-1}:A-\catMod\to S^{-1}A-\catMod$ is an exact functor.
\end{theorem}
\begin{proof}
    Let $M'\stackrel{f}{\longrightarrow}M\stackrel{g}{\longrightarrow}M''$ be an exact sequence. Then, for any $m'/s'\in S^{-1}M'$, we have 
    \begin{equation*}
        S^{-1}g\left(S^{-1}f\left(\frac{m'}{s'}\right)\right) = S^{-1}g\left(\frac{f(m')}{s'}\right) = \frac{g(f(m'))}{s'} = 0
    \end{equation*}
    As a result, $\im(S^{-1}f)\subseteq\ker(S^{-1}g)$. On the other hand, for $m/s\in\ker S^{-1}g$, we have $g(m)/s = 0$, consequently, there is $t\in S$ such that $tg(m) = 0$, equivalently, $g(tm) = 0$, whence, there is $m'\in M'$ such that $f(m') = tm$. Then, we have 
    \begin{equation*}
        f\left(\frac{m'}{st}\right) = \frac{f(m')}{st} = \frac{tm}{st} = \frac{m}{s}
    \end{equation*}
    whence, $\ker(S^{-1}g)\subseteq\im(S^{-1}f)$. This completes the proof.
\end{proof}

\begin{proposition}
    Let $N,P,\{M_i\}_{i\in I}$ be submodules of an $A$-module $M$. Then, for a multiplicatively closed $S\subseteq M$,
    \begin{enumerate}
        \item $S^{-1}(N\cap P) = S^{-1}N\cap S^{-1}P$
        \item $\displaystyle S^{-1}\left(\sum_{i\in I}M_i\right) = \sum_{i\in I}S^{-1}M_i$
    \end{enumerate}
\end{proposition}
\begin{proof}
    
\end{proof}