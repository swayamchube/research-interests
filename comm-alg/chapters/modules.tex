\section{Introduction}
Throughout this section, $R$ denotes a general ring which need not be commutative.

\begin{definition}[Module]
    A left $R$-module is an abelian group $(M,+)$ along with a ring action, that is, a ring homomorphism $\mu: R\to\End(M)$.
\end{definition}

Henceforth, unless specified otherwise, an \textit{$R$-module} refers to a \textit{left $R$-module}. Trivially note that $R$ is an $R$-module, so is any ideal in $R$ and so is every quotient ring $R/I$ where $I$ is an ideal in $R$. When $R$ is a field, an $R$-module is the same as a vector space.

Every abelian group $G$ trivially forms a $\Z$-module. Using this and the forthcoming \textit{Structure Theorem for Finitely Generated Modules over a PID}, we obtain the \textit{Structure Theorem for Finitely Generated Abelian Groups}.

\begin{definition}[Submodule]
    Let $M$ be an $R$-module. An $R$-submodule of $M$ is a subgroup $N$ of $M$ which is closed under the action of $R$.
\end{definition}

\begin{proposition}[Submodule Criteria]
    Let $M$ be an $R$-module. Then $\emptyset\subsetneq N\subseteq M$ is a submodule if and only if for all $x,y\in N$ and $r\in R$, $x + ry\in N$.
\end{proposition}
\begin{proof}
    Straightforward definition pushing.
\end{proof}

\begin{definition}[Module Homomorphism]
    Let $M, N$ be $R$-modules. A \textit{module homomorphism} is a group homomorphism $\phi: M\to N$ such that for all $x\in M$ and $r\in R$, $\phi(rx) = r\phi(x)$.
\end{definition}

In other words, a module homomorphism is simply an $R$-linear map. 

\begin{proposition}[Homomorphism Criteria]
    Let $M, N$ be $R$-modules. Then $\phi: M\to N$ is an $R$-module homomorphism if and only if for all $x,y\in M$ and $r\in R$, $\phi(x + ry) = \phi(x) + r\phi(y)$.
\end{proposition}
\begin{proof}
    Straightforward definition pushing.
\end{proof}

It is not hard to see, using the above proposition and the submodule criteria that the image of an $R$-module under a homomorphism is a submodule.


\begin{definition}[Kernel, Cokernel]
    Let $\phi:M\to N$ be an $R$-module homomorphism. We define 
    \begin{equation*}
        \ker\phi = \{x\in M\mid\phi(x) = 0\}\qquad\coker\phi = N/\phi(M)
    \end{equation*}
\end{definition}

For an $R$-module $M$, define the annihilator of $M$ in $R$ as 
\begin{equation*}
    \Ann_R(M) = \{r\in R\mid rx = 0~\forall x\in M\}
\end{equation*}
It is trivial to check that $\Ann(M)$ is a left ideal in $R$, and if $R$ were commutative, it would be an ideal.

\begin{proposition}
    If $I$ is an ideal contained in $\Ann_A(M)$, then $M$ is naturally an $A/I$-module.
\end{proposition}
\begin{proof}
    Define the action $(a + I)\cdot m = a\cdot m$. It is easy to check that this action is well defined. Further, 
    \begin{equation*}
        (a + I)\cdot((b + I)\cdot m) = (a + I)\cdot(bm) = (ab)\cdot m = ((a + I)(b + I))\cdot m
    \end{equation*}
    This completes the proof.
\end{proof}

In particular, if $I = \mathfrak m$ for some maximal ideal $\mathfrak m$, then $M$ forms a vector space over $A/\mathfrak m$.

\section{Free Modules}
Throughout this section, $R$ denotes a general ring which need not be commutative. The content of this section is taken from \cite{blyth}. 

We define the free module using a universal property and then provide a construction for it. This should establish uniqueness.

\begin{definition}[Universal Property of Free Modules]
    Let $S$ be a non-empty set. A \textit{free module on $S$} is an $R$-module $F$ together with a mapping $f: S\to F$ such that for every $R$-module $M$ and every set map $g: S\to M$, there is a unique $R$-module homomorphism $h: F\to M$ such that the following diagram commutes: 
    \begin{equation*}
    \xymatrix{
        S\ar[r]^g\ar[d]_f & M\\
        F\ar@{.>}[ru]_{\exists! h}
    }
    \end{equation*}
\end{definition}

Let $F$ be the set of all set functions $\phi: S\to R$ which takes nonzero values at finitely many elements of $S$. This has the structure of an $R$-module. Define the set map $f: S\to F$ by 
\begin{equation*}
    f(s)(t) = 
    \begin{cases}
        1 & s = t\\
        0 & \text{otherwise}
    \end{cases}
\end{equation*}

We contend that $(F,f)$ is a free module on $S$. Indeed, let $g: S\to M$ be a set map where $M$ is an $R$-module. Define the linear map $h: F\to M$ by 
\begin{equation*}
    h(f(s)) = g(s)
\end{equation*}

Since every element in $F$ can uniquely be written as a linear combination of elements in $\{f(s)\}_{s\in S}$, we have successfully defined a module homomorphism $h: F\to M$ such that $g = h\circ f$. The uniqueness of this map is quite obvious. Hence, $(F,f)$ is a free module on $S$.

\begin{definition}[Basis]
    Let $M$ be an $R$-module. Then $S\subseteq M$ is said to be a \textit{basis} if it is linearly independent and generates $M$.
\end{definition}

It is important to note that not every minimal generating set is a basis. Take for example the $\Z$-module $\Z$. Notice that $\{2,3\}$ is a minimal generating set but is not a basis for it is not linearly independent.

\section{Finitely Generated Modules}

\begin{definition}[Finitely Generated Module]
    An $R$-module $M$ is said to be finitely generated if there is a finite subset $S$ of $M$ which generates $M$. That is, there is no proper submodule $N$ of $M$ containing $S$.
\end{definition}

\begin{proposition}
    An $R$-module $M$ is finitely generated if $M$ is isomorphic to a quotient of $R^{\oplus n}$ for some positive integer $n$.
\end{proposition}
\begin{proof}
    We shall only prove the forward direction since the converse is trivial to prove. Suppose $M$ is finitely generated. Then, it is generated by a finite subset $S = \{x_1,\ldots,x_m\}$. Define the $R$-module homomorphism $\phi:R^{\oplus n}\to M$ by $(r_1,\ldots,r_n)\mapsto r_1x_1 + \cdots + r_nx_n$. From the first isomorphism theorem, we have $M\cong R^{\oplus n}/\ker\phi$.
\end{proof}

\begin{proposition}\thlabel{prop:CH-type}
    Let $M$ be a finitely generated $A$-module and $\mathfrak a$ an ideal of $A$. Let $\phi\in\End(M)$ such that $\phi(M)\subseteq\mathfrak aM$. Then, there are $a_0,\ldots,a_{n - 1}\in\mathfrak a$ such that 
    \begin{equation*}
        \phi^n + a_{n - 1}\phi^{n - 1} + \cdots + a_0 = 0
    \end{equation*}
    as an element of $\End(M)$, where $a_k$ is treated as the homomorphism $x\mapsto a_kx$ in $\End(M)$.
\end{proposition}
\begin{proof}
    Let $\{x_1,\ldots,x_n\}$ be a generating set for $M$. Then, for all $1\le i\le n$, there are coefficients $\{a_{i1},\ldots,a_{in}\}$ in $\mathfrak a$ such that 
    \begin{equation*}
        \phi(x_i) = \sum_{j = 1}^n a_{ij}x_j
    \end{equation*}
    We may rewrite this as 
    \begin{equation*}
        \sum_{j = 1}^n(\phi\delta_{ij} - a_{ij})x_j = 0
    \end{equation*}
    Let $B$ denote the matrix $(\phi\delta_{ij} - a_{ij})_{1\le i,j\le n}$. Then, multiplying by $\operatorname{adj}(B)$, we see that $\det(B)(x_j) = 0$ for all $1\le j\le n$ where $\det(B)$ is viewed as an element in $\End(M)$ and thus, is the zero map in $\End(M)$. It is not hard to see that $\det(B)$ is in the required form.
\end{proof}

\begin{lemma}[Nakayama]\thlabel{lem:nakayama}
    Let $M$ be a finitely generated module and $\mathfrak a\subseteq\mathfrak R$ be an ideal such that $M = \mathfrak aM$. Then, $M = 0$. 
\end{lemma}
\begin{proof}
    Let $\phi = \mathbf{id}$ be the identity homomorphism in $\End(M)$. Using \thref{prop:CH-type}, there are coefficients $a_0,\ldots,a_{n - 1}\in\mathfrak a$ satisfying the statement of the proposition. As a result, $x = 1 + a_{n - 1} + \ldots + a_0$ is the zero endomorphism. But since $a_{n - 1} + \ldots + a_0\in\mathfrak a\subseteq\mathfrak R$, $x$ is a unit and hence, $M = 0$.
\end{proof}

\begin{corollary}
    Let $M$ be a finitely generated $A$-module, $N$ a submodule of $M$ and $\mathfrak a\subseteq\mathfrak R$ an ideal. If $M = \mathfrak aM + N$ then $M = N$.
\end{corollary}
\begin{proof}
    We have $M/N = \mathfrak aM/N$, consequently, $M/N = 0$ and $M = N$ due to \thref{lem:nakayama}.
\end{proof}

\begin{lemma}
    Let $(A,\mathfrak m)$ be local and $k = A/\mathfrak m$. Let $M$ be a finitely generated $A$-module. Let $\{\overline x_1,\ldots,\overline x_n\}$ be elements in $M/\mathfrak m$ that form a basis for $M/\mathfrak m$ as a $k$-vector space. Then, $\{x_1,\ldots,x_n\}$ generates $M$.
\end{lemma}
\begin{proof}
    Let $N$ be the submodule generated by $\{x_1,\ldots,x_n\}$. Then, the composition $N\hookrightarrow M\twoheadrightarrow M/\mathfrak mM$ is surjective, consequently, $M = N + \mathfrak mM$ whence, it follows that $M = N$.
\end{proof}

\subsection*{Over a PID}
Throughout this section, let $R$ denote a principal ideal domain. 

\section{\texorpdfstring{$\Hom$}{} Modules and Functors}

For $R$-modules $M,N$, we denote the set of all $R$-module homomorphisms from $M$ to $N$ by $\Hom_R(M,N)$. When the choice of the ring $R$ is clear from the context, we shall denote this set by $\Hom(M,N)$.

\begin{proposition}
    Let $M,N$ be $A$-modules. Then $\Hom(M,N)$ has the structure of an $A$-module.
\end{proposition}
\begin{proof}
    It is obvious that $\Hom(M,N)$ has the structure of an abelian group. Define the natural action by $(af)(x) = af(x)$. It is not hard to see that this action is well defined.
\end{proof}

\begin{proposition}
    Let $\{M_\lambda\}_{\lambda\in\Lambda}$ be a collection of $A$-modules. Then, for any $A$-module $N$, we have a natural isomorphism
    \begin{equation*}
        \Hom_A\left(\bigoplus_{\lambda\in\Lambda}M_\lambda, N\right) = \prod_{\lambda\in\Lambda}\Hom_A(M_\lambda,N)
    \end{equation*}
\end{proposition}
\begin{proof}
    Since the direct sum is the product in $A-\mathbf{Mod}$, the conclusion follows from the universal property.
\end{proof}

\begin{theorem}
    Let $\phi: M\to N$ be an $A$-module homomorphism. Then, for every $R$-module $P$, there is an induced $A$-module homomorphism $\overline\phi:\Hom(N,P)\to\Hom(M,P)$ and an induced $A$-module homomorphism $\widetilde\phi:\Hom(P,M)\to\Hom(P,N)$. 
    
    Equivalently phrased, $\Hom(-,P)$ is a contravariant functor while $\Hom(P,-)$ is a covariant functor.
\end{theorem}
\begin{proof}
    We shall prove only the first half of the assertion since the second half follows from a similar proof. Define $\overline\phi$ using the following commutative diagram: 
    \begin{equation*}
    \xymatrix {
        M\ar[r]^{\phi}\ar@{.>}[rd]_{f\circ\phi} & N\ar[d]^f\\
        & P
    }
    \end{equation*}
    To see that this is indeed an $R$-module homomorphism, we need only verify that for all $f,g\in\Hom(N,P)$ and all $r\in R$, $(f + rg)\circ\phi = f\circ\phi + rg\circ\phi$ which is trivial to check.
\end{proof}

\section{Exact Sequences}

\begin{definition}
    A sequence of module homomorphisms 
    \begin{equation*}
        M\stackrel{f}{\longrightarrow} N\stackrel{g}{\longrightarrow}P
    \end{equation*}
    is said to be exact at $N$ if $\im f = \ker g$. A short exact sequence is a sequence of module homomorphisms: 
    \begin{equation*}
        0\longrightarrow M\stackrel{f}{\longrightarrow} N\stackrel{g}{\longrightarrow} P\longrightarrow 0
    \end{equation*}
    which is exact at $M$, $N$ and $P$.
\end{definition}

It is not hard to see that the sequence in the definition is short exact if and only if $f$ is injective, $g$ is surjective and $\im f = \ker g$.

\subsection{Diagram Chasing Poster Children}


\section{Tensor Product}

\begin{definition}[Bilinear Map]
    Let $M, N, P$ be $A$-modules. A map $T: M\times N\to P$ is said to be bilinear if for each $x\in M$, the mapping $T_x: N\to P$ given by $y\mapsto T(x,y)$ is $A$-linear and for each $y\in N$, the mapping $T_y: M\to P$ given by $x\mapsto T(x,y)$ is $A$-linear.
\end{definition}

Fix two $A$-modules $M$ and $N$. Let $\mathscr C$ denote the category of bilinear maps $T: M\times N\to P$ where $P$ is any $A$-module. A morphism between two bilinear maps $f: M\times N\to P_1$ and $g: M\times N\to P_2$ in this category is a module homomorphism $\phi: P_1\to P_2$ such that the following diagram commutes: 
\begin{equation*}
\xymatrix {
    M\times N\ar[r]^-f\ar[d]_g & P_1\ar@{.>}[ld]^\phi\\
    P_2
}
\end{equation*}

A universal object in $\mathscr C$ is called the tensor product of $M$ and $N$ and is denoted by $M\otimes N$. In other words, the tensor product is an initial object in the category $\mathscr C$.

\begin{definition}[Universal Property of the Tensor Product]
    Let $M,N,P$ be $A$-modules and $T: M\times N\to P$ be a bilinear map. Then, there is a unique $A$-module homomorphism $\phi: M\otimes N\to P$ such that the following diagram commutes: 
    \begin{equation*}
    \xymatrix {
        M\times N\ar[r]^-T\ar[d]_\varphi & P\\
        M\otimes N\ar@{.>}[ru]_{\exists!\phi}
    }
    \end{equation*}
\end{definition}

Of course, having the universal property would imply that the tensor product, if it exists, is unique upto a unique isomorphism. We shall now construct a tensor product of $M$ and $N$.

\subsection*{Constructing the Tensor Product}

Let $F$ be the free $A$-module on $M\times N$. Let us denote the basis elements of $F$ by $e_{(x,y)}$ where $x\in M$ and $y\in N$. Now, for all $x,x_1,x_2\in M$, $y,y_1,y_2\in N$ and $a\in A$, let $D$ denote the submodule generated by elements of the form: 
\begin{align*}
    &e_{(x_1 + x_2, y)} - e_{(x_1,y)} - e_{(x_2,y)}\\
    &e_{(x,y_1 + y_2)} - e_{(x,y_1)} - e_{(x,y_2)}\\
    &e_{(ax,y)} - ae_{(x,y)}\\
    &e_{(x,ay)} - ae_{(x,y)}
\end{align*}

Let $G = F/D$ and let $\varphi: M\times N\to G$ be the composition of the following maps: 
\begin{equation*}
    M\times N\hookrightarrow F\twoheadrightarrow G
\end{equation*}

Let $T: M\times N\to P$ be a bilinear map. Consider the following commutative diagram: 

\begin{equation*}
\xymatrix {
    M\times N\ar[r]^-T\ar@{^{(}->}_{\iota}[d] & P\\
    F\ar[r]_{\pi}\ar@{.>}[ru]|{\exists! f} & G\ar@{.>}[u]_{\exists!\phi}
}
\end{equation*}

To show that existence of $\phi$, we must show that $D\subseteq\ker f$, since we can then finish using the universal property of the kernel. But this is trivial to check and follows from the fact that $T$ is a bilinear map and completes the construction.

\begin{mdframed}
    Similarly, we define the tensor product for a finite sequence of $A$-modules $\{M_i\}_{i = 1}^n$. That is, given a multilinear map $T:\prod\limits_{i = 1}^n M_i\to P$, there is a unique $A$-module homomorphism $\phi$ such that the following diagram commutes: 
    \begin{equation*}
    \xymatrix{
        M_1\times\cdots\times M_n\ar[r]^-T\ar[d]_\varphi & P\\
        M_1\otimes\cdots\otimes M_n\ar@{.>}[ru]_-{\exists!\phi}
    }
    \end{equation*}
\end{mdframed}

\subsection*{Properties of Tensor Product}

Given two modules $M$ and $N$ with the canonical map $\varphi: M\times N\to M\otimes N$, we denote by $m\otimes n$, the element $\varphi(m,n)$ in $M\otimes N$.

\begin{proposition}
    Let $M, N, P$ be $A$-modules. Then, 
    \begin{enumerate}[label=(\alph*)]
    \item $M\otimes N\cong N\otimes M$ 
    \item $(M\otimes N)\otimes P\cong M\otimes(N\otimes P)\cong M\otimes N\otimes P$ 
    \item $M\oplus N\otimes P\cong(M\otimes P)\oplus(N\otimes P)$ 
    \item $A\otimes M\cong M$
    \end{enumerate}
\end{proposition}
\begin{proof}
\begin{enumerate}[label=(\alph*)]
\item First, we shall show that there are well defined homomorphisms $M\otimes N\to N\otimes M$ and $N\otimes M\to M\otimes N$ mapping $m\otimes n\mapsto n\otimes m$ and $n\otimes m\mapsto m\otimes n$ respectively. This is best done using the universal property. Let $T: M\times N\to N\times M$ be the isomorphism $m\times n\mapsto n\times m$. Consider now the following commutative diagram: 
\begin{equation*}
\xymatrix{
    M\times N\ar[d]_\varphi\ar[r]^-T & N\times M\ar[d]^{\varphi'}\\
    M\otimes N & N\otimes M
}
\end{equation*}

Since both $\varphi'$ and $T$ are bilinear, so is $\varphi\circ T$, consequently, there is a unique induced homomorphism $f: M\otimes N\to N\otimes M$ making the diagram commute, consequently, $f(m\otimes n) = \varphi'(T(m\times n)) = n\otimes m$.

Similarly, there is a homomorphism $g: N\otimes M\to M\otimes N$ such that $g(n\otimes m) = m\otimes m$. It is not hard to see that $g\circ f = \mathbf{id}_{M\otimes N}$ and $f\circ g = \mathbf{id}_{N\otimes M}$, consequently, they are isomorphisms.

\item 

\item

\item Consider the map $T: A\times M\to M$ given by $(a,m)\mapsto am$. It is not hard to see that this map is bilinear, consequently, there is a map $f: A\otimes M\to M$ such that the following diagram commutes: 
\begin{equation*}
\xymatrix {
    A\times M\ar[r]^-T\ar[d]_\varphi & M\\
    A\otimes M\ar@{.>}[ru]_-{f}
}
\end{equation*}
Note that $f(a\otimes m) = am$ by definition. Consider the map $g: M\to A\otimes M$ given by $g(m) = 1\otimes m$. It is not hard to see that $g$ is a well defined module homomorphism. Further, since $f\circ g$ and $g\circ f$ are the identity homomorphisms, they both must be isomorphisms.
\end{enumerate}
\end{proof}

\begin{example}
    Show that $\Z/m\Z\otimes\Z/n\Z\cong\Z/\gcd(m,n)\Z$ for all $m,n\in\N$. In particular, if $m$ and $n$ are coprime, then $\Z/m\Z\otimes\Z/n\Z = 0$.
\end{example}
\begin{proof}
    Consider the module homomorphism $T:\Z\to\Z/m\Z\otimes\Z/n\Z$. 
\end{proof}

Let $f: M\to M'$ and $g: N\to N'$ be $A$-module homomorphisms. Then, the map $\Phi: M\times N\to M'\otimes N'$ given by $\Phi(m,n) = f(m)\otimes g(n)$. It is not hard to see that $\Phi$ is bilinear. Consequently, it induces a map $f\otimes g: M\otimes N\to M'\otimes N'$ such that 
\begin{equation*}
    (f\otimes g)(x\otimes y) = f(x)\otimes g(y)
\end{equation*}

Further, if $f': M'\to M''$ and $g': N'\to N''$ are $A$-module homomorphisms, then we have another map $f'\otimes g': M'\otimes N'\to M''\otimes N''$ such that 
\begin{equation*}
    (f'\otimes g')(x\otimes y) = f'(x)\otimes g'(y)
\end{equation*}

Now, it is not hard to see that $(f'\circ f')\otimes(g'\circ g)$ and $(f'\otimes g')\circ(f\otimes g)$ agree on the elementary tensors, therefore, agree on all of $M\otimes N$.

\section{Right Exactness}

\begin{proposition}\thlabel{prop:hom-tensor-adjunction}
    Let $M,N,P$ be $A$-modules. Then, there is a natural isomorphism:
    \begin{equation*}
        \Hom_A(M,\Hom_A(N, P))\cong\Hom_A(M\otimes_A N, P)
    \end{equation*}
\end{proposition}
\begin{proof}
    Consider the map 
    \begin{equation*}
        \theta: \Hom_A(M\otimes_A N, P)\longrightarrow\Hom_A(M,\Hom_A(N, P))
    \end{equation*}
    given by $\theta(\alpha)(m)(n) = \alpha(m\otimes n)$. Now, pick some $\eta\in\Hom_A(M,\Hom_A(N,P))$. Define the map $\zeta: M\times N\to P$ given by $\zeta(m, n) = \eta(m)(n)$. Obviously, $\zeta$ is bilinear and induces a map $\delta: M\otimes_A N\to P$ such that $\delta(m\otimes n) = \eta(m)(n)$. Call the map sending $\eta\mapsto\delta$ as $\beta$ where 
    \begin{equation*}
        \beta: \Hom_A(M,\Hom_A(N, P))\to\Hom_A(M\otimes_A N, P)
    \end{equation*}
    and $\beta(\eta)(m\otimes n) = \eta(m)(n)$.

    We contend that $\theta$ and $\beta$ are inverses to one another. Indeed, 
    \begin{equation*}
        ((\beta\circ\theta)(\alpha))(m\otimes n) = \theta(\alpha)(m)(n) = \alpha(m\otimes n)
    \end{equation*}
    and 
    \begin{equation*}
        ((\theta\circ\beta)(\eta))(m)(n) = \beta(\eta)(m\otimes n) = \eta(m)(n)
    \end{equation*}
    whence the conclusion follows.
\end{proof}

In particular, we see that the functor $-\otimes_A N$ is the left adjoint of the functor $\Hom_A(N,-)$, consequently, $\Hom_A(N,-)$ is the right adjoint of $-\otimes_A N$.

\begin{theorem}
    The functor $-\otimes_A N$ is right exact. That is, given a exact sequence
    \begin{equation*}
        M'\stackrel{f}{\longrightarrow}M\stackrel{g}{\longrightarrow}M''\longrightarrow 0
    \end{equation*}
    the sequence 
    \begin{equation*}
        M'\otimes_A N\stackrel{f\otimes 1}{\longrightarrow}M\otimes_A N\stackrel{g\otimes 1}{\longrightarrow}M''\otimes_A N\longrightarrow 0
    \end{equation*}
\end{theorem}
\begin{proof}
    Since the given sequence is exact, so is 
    \begin{equation*}
        \Hom_A(M'',\Hom_A(N,P))\stackrel{\overline g}{\longrightarrow}\Hom_A(M,\Hom_A(N,P))\stackrel{\overline f}{\longrightarrow}\Hom_A(M',\Hom_A(N,P))\longrightarrow 0
    \end{equation*}
    but from \thref{prop:hom-tensor-adjunction}, so is
    \begin{equation*}
        \Hom_A(M''\otimes_A N, P)\longrightarrow\Hom_A(M\otimes_A N, P)\longrightarrow\Hom_A(M'\otimes_A N, P)\longrightarrow 0
    \end{equation*}
    Since the above sequence is exact for all modules $P$, we have the desired conclusion.
\end{proof}

The tensor product is not left exact. Conider the sequence of $\Z$-modules
\begin{equation*}
    0\hookrightarrow\Z\stackrel{f}{\longrightarrow}\Z
\end{equation*}
where $f(m) = 2m$. Upon tensoring with $\Z/2\Z$, we get the sequence 
\begin{equation*}
    0\longrightarrow\Z\otimes_{\Z}\Z/2\Z\stackrel{f\otimes1}{\longrightarrow}\Z\otimes_{\Z}\Z/2\Z
\end{equation*}

Note that 
\begin{equation*}
    (f\otimes 1)(m\otimes\overline n) = 2m\otimes\overline{n} = m\otimes(2\overline n) = m\otimes 0 = 0
\end{equation*}

Therefore, the sequence cannot be exact.

\section{Flat Modules}

\begin{definition}[Flat Module]
\end{definition}

\begin{theorem}
    Let $N$ be a $A$-module. Then, the following are equivalent 
    \begin{enumerate}[label=(\alph*)]
        \item $N$ is flat 
        \item If $0\rightarrow M'\rightarrow M\rightarrow M''\rightarrow 0$ is an exact sequence of $A$-modules, then the tensored sequence 
        \begin{equation*}
        0\longrightarrow M'\otimes_A N\stackrel{f\otimes 1}{\longrightarrow} M\otimes_A N\stackrel{g\otimes 1}{\longrightarrow} M''\otimes_A N\longrightarrow 0
        \end{equation*}
        is exact.
        \item If $f: M'\to M$ is injective, then $f\otimes 1: M'\otimes N\to M\otimes N$ is injective 
        \item If $f: M'\to M$ is injective and $M,M'$ are finitely generated, then $f\otimes_A 1: M'\otimes_A N\to M\otimes_A N$ is injective.
    \end{enumerate}
\end{theorem}
\begin{proof}
\hfill 
\begin{enumerate}[label=(\alph*)]
\item 
\end{enumerate}
\end{proof}

\section{Projective Modules}

\begin{theorem}\thlabel{thm:pojective-module-equivalence}
    For an $A$-module $P$, the following are equivalent: 
    \begin{enumerate}[label=(\alph*)]
    \item Every map $f: P\to M''$ can be lifted to $\widetilde{f}: P\to M$ in the following commutative diagram: 
    \begin{equation*}
    \xymatrix{
        & P\ar[ld]_{\widetilde f}\ar[d]^f & \\
        M\ar[r]_g & M''\ar@{->>}[r] & 0
    }
    \end{equation*}
    \item Every short exact sequence $0\rightarrow M'\rightarrow M\rightarrow P\rightarrow 0$ splits 
    \item There is a module $M$ such that $P\oplus M$ is free 
    \item The functor $\Hom_A(P,-)$ is exact.
    \end{enumerate}
\end{theorem}
\begin{proof}
\hfill
\begin{description}
\item[$(a)\Longrightarrow(b)$:] Taking $M'' = P$ and $f = \mathbf{id}_P$, we have the desired conclusion. 
\item[$(b)\Longrightarrow(c)$:] Let $F$ denote the free module on the set $P$. Then, the map $\Phi: F\to P$ given by $\Phi(e_x) = x$ for all $x\in P$ is a surjective $A$-module homomorphism. We have the following short exact sequence: 
\begin{equation*}
    0\rightarrow\ker\Phi\stackrel{\iota}{\longrightarrow}F\stackrel{\Phi}{\longrightarrow}P\rightarrow 0
\end{equation*}
This is known to split and thus, $F = \psi(P)\oplus\ker\Phi$ where $\psi: P\to F$ is the section.

\item[$(c)\Longrightarrow(d)$:] Let $M'\rightarrow M\rightarrow M''$ be an exact sequence of modules and $K$ be an $A$-module such that $P\oplus K = F\cong A^\Lambda$. Then, the induced sequence 
\begin{equation*}
    \prod_{\lambda\in\Lambda}M'\rightarrow\prod_{\lambda\in\Lambda}M\rightarrow\prod_{\lambda\in\Lambda}M''
\end{equation*}
is exact. We have seen that there is a natural isomorphism $\Hom_A(A,M)\stackrel{\sim}{\longrightarrow}M$, consequently, there is a natural isomorphism 
\begin{equation*}
    \Hom_A(A^{\oplus\Lambda}, M)\stackrel{\sim}{\longrightarrow}\prod_{\lambda\in\Lambda}M
\end{equation*}
whence it follows that the sequence 
\begin{equation*}
    \Hom_A(A^{\oplus\Lambda}A,M')\rightarrow\Hom_A(A^{\oplus\Lambda}A,M)\rightarrow\Hom_A(A^{\oplus\Lambda},M'')
\end{equation*}
But since $\Hom_A(A^{\oplus\Lambda},M)\cong\Hom_A(P,M)\oplus\Hom_A(K,M)$, we have the desired conclusion.

\item[$(d)\Longrightarrow(a)$:] Trivial.
\end{description}
\end{proof}

\begin{definition}[Projective Module]
    An $A$-module $P$ satisfying any one of the four equivalent conditions of \thref{thm:pojective-module-equivalence} is said to be a \textit{projective $A$-module}.
\end{definition}

In particular, from \thref{thm:pojective-module-equivalence}(c), we see that every free module is projective.

\begin{lemma}
    A finitely generated projective module $P$ over a local ring $(A,\mathfrak m)$ is free.
\end{lemma}
\begin{proof}
    Let $\{\overline x_1,\ldots,\overline x_n\}$ be a basis for $M/\mathfrak mM$ as a $k$-vector space where $k = A/\mathfrak m$. As we have seen earlier, $\{x_1,\ldots,x_n\}$ generates $M$. Let $F$ be the free module with basis $\{e_1,\ldots,e_n\}$ and $\Phi: F\to M$ be the module homomorphism given by $\Phi(e_i) = x_i$ and $K = \ker\Phi$. Since $M$ is projective, there is a module homomorphism $\psi: M\to F$ satisfying $\Phi\circ\psi = \mathbf{id}_M$ and $F = K\oplus\psi(M)$.

    We contend that $K = \mathfrak mK$. Indeed, let $x\in K$, then $x = \sum r_ie_i$ for a unique choice $\{r_1,\ldots,r_n\}$. Then, $\sum r_ix_i = 0$, consequently, $r_i\in\mathfrak m$ for all $i$. Since $F = K\oplus\psi(M)$, we may write $e_i = u_i + v_i$ for some $u_i\in K$ and $v_i\in\psi(M)$. As a result, 
    \begin{equation*}
        x - \sum r_iu_i = \sum r_iv_i\in\ker\Phi\cap\psi(M) = \{0\}
    \end{equation*}
    and the conclusion follows.

    Finally due to \thref{lem:nakayama}, we must have that $K = 0$ whence $M$ is free.
\end{proof}