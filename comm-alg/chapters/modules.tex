\section{Introduction}
Throughout this section, $R$ denotes a general ring which need not be commutative.

\begin{definition}[Module]
    A left $R$-module is an abelian group $(M,+)$ along with a ring action, that is, a ring homomorphism $\mu: R\to\End(M)$. Similarly, a right $R$-module is an abelian group $(M,+)$ along with a ring homomorphism $\mu: R^\text{op}\to\End(M)$ where $R^\text{op}$ is the opposite ring.
\end{definition}

Henceforth, unless specified otherwise, an \textit{$R$-module} refers to a \textit{left $R$-module}. Trivially note that $R$ is an $R$-module, so is any ideal in $R$ and so is every quotient ring $R/I$ where $I$ is an ideal in $R$. When $R$ is a field, an $R$-module is the same as a vector space.

Every abelian group $G$ trivially forms a $\Z$-module. Using this and the forthcoming \textit{Structure Theorem for Finitely Generated Modules over a PID}, we obtain the \textit{Structure Theorem for Finitely Generated Abelian Groups}.

There is also the notion of a bimodule: 
\begin{definition}
    For 
\end{definition}

\begin{definition}[Submodule]
    Let $M$ be an $R$-module. An $R$-submodule of $M$ is a subgroup $N$ of $M$ which is closed under the action of $R$.
\end{definition}

\begin{proposition}[Submodule Criteria]
    Let $M$ be an $R$-module. Then $\emptyset\subsetneq N\subseteq M$ is a submodule if and only if for all $x,y\in N$ and $r\in R$, $x + ry\in N$.
\end{proposition}
\begin{proof}
    Straightforward definition pushing.
\end{proof}

\begin{definition}[Module Homomorphism]
    Let $M, N$ be $R$-modules. A \textit{module homomorphism} is a group homomorphism $\phi: M\to N$ such that for all $x\in M$ and $r\in R$, $\phi(rx) = r\phi(x)$.
\end{definition}

In other words, a module homomorphism is simply an $R$-linear map. 

\begin{proposition}[Homomorphism Criteria]
    Let $M, N$ be $R$-modules. Then $\phi: M\to N$ is an $R$-module homomorphism if and only if for all $x,y\in M$ and $r\in R$, $\phi(x + ry) = \phi(x) + r\phi(y)$.
\end{proposition}
\begin{proof}
    Straightforward definition pushing.
\end{proof}

It is not hard to see, using the above proposition and the submodule criteria that the image of an $R$-module under a homomorphism is a submodule.


\begin{definition}[Kernel, Cokernel]
    Let $\phi:M\to N$ be an $R$-module homomorphism. We define 
    \begin{equation*}
        \ker\phi = \{x\in M\mid\phi(x) = 0\}\qquad\coker\phi = N/\phi(M)
    \end{equation*}
\end{definition}

For an $R$-module $M$, define the annihilator of $M$ in $R$ as 
\begin{equation*}
    \Ann_R(M) = \{r\in R\mid rx = 0~\forall x\in M\}
\end{equation*}
It is trivial to check that $\Ann_R(M)$ is a left ideal in $R$, and if $R$ were commutative, it would be an ideal. When $\Ann_A(M) = 0$, $M$ is said to be a \textit{faithful} $A$-module.

\begin{proposition}
    If $I$ is an ideal contained in $\Ann_A(M)$, then $M$ is naturally an $A/I$-module.
\end{proposition}
\begin{proof}
    Define the action $(a + I)\cdot m = a\cdot m$. It is easy to check that this action is well defined. Further, 
    \begin{equation*}
        (a + I)\cdot((b + I)\cdot m) = (a + I)\cdot(bm) = (ab)\cdot m = ((a + I)(b + I))\cdot m
    \end{equation*}
    This completes the proof.
\end{proof}

\begin{proposition}
    $N$ is an $A$-submodule of $M$ if and only if it is an $A/\Ann_A(M)$ submodule of $M$.
\end{proposition}
\begin{proof}
    Straightforward.
\end{proof}

In particular, if $I = \mathfrak m$ for some maximal ideal $\mathfrak m$, then $M$ forms a vector space over $A/\mathfrak m$.

\section{Free Modules}
Throughout this section, $R$ denotes a general ring which need not be commutative.

We define the free module using a universal property and then provide a construction for it. This should establish uniqueness.

\begin{definition}[Universal Property of Free Modules]
    Let $S$ be a non-empty set. A \textit{free module on $S$} is an $R$-module $F$ together with a mapping $f: S\to F$ such that for every $R$-module $M$ and every set map $g: S\to M$, there is a unique $R$-module homomorphism $h: F\to M$ such that the following diagram commutes: 
    \begin{equation*}
    \xymatrix{
        S\ar[r]^g\ar[d]_f & M\\
        F\ar@{.>}[ru]_{\exists! h}
    }
    \end{equation*}
\end{definition}

Let $F$ be the set of all set functions $\phi: S\to R$ which takes nonzero values at finitely many elements of $S$. This has the structure of an $R$-module. Define the set map $f: S\to F$ by 
\begin{equation*}
    f(s)(t) = 
    \begin{cases}
        1 & s = t\\
        0 & \text{otherwise}
    \end{cases}
\end{equation*}

We contend that $(F,f)$ is a free module on $S$. Indeed, let $g: S\to M$ be a set map where $M$ is an $R$-module. Define the linear map $h: F\to M$ by 
\begin{equation*}
    h(f(s)) = g(s)
\end{equation*}

Since every element in $F$ can uniquely be written as a linear combination of elements in $\{f(s)\}_{s\in S}$, we have successfully defined a module homomorphism $h: F\to M$ such that $g = h\circ f$. The uniqueness of this map is quite obvious. Hence, $(F,f)$ is a free module on $S$.

\begin{definition}[Basis]
    Let $M$ be an $R$-module. Then $S\subseteq M$ is said to be a \textit{basis} if it is linearly independent and generates $M$.
\end{definition}

It is important to note that not every minimal generating set is a basis. Take for example the $\Z$-module $\Z$. Notice that $\{2,3\}$ is a minimal generating set but is not a basis for it is not linearly independent.

\subsection{Over a PID}

Throughout this (sub)section, let $R$ denote a PID.

\begin{theorem}
    Let $F$ be a free $R$-module. If $M\le F$ is a submodule, then $M$ is free and $\dim M\le\dim F$.
\end{theorem}
\begin{proof}
    Let $F$ have basis $\{x_i\}_{i\in I}$ and $\pi_i: F\to R$ denote the natural projection. Using the Well Ordering Theorem, we may suppose that $(I,\leqq)$ has a well order, in particular, we may suppose that $I$ is a segment of ordinals. For each $i\in I$, denote
    \begin{equation*}
        M_i := M\cap\left\langle\{x_j\mid j\leqq i\}\right\rangle.
    \end{equation*}
    We shall show, using transfinite induction that each $M_\alpha$ is free module with rank less than or equal to $|\alpha|$. Consider the base case, $\alpha = 1$. Now, $M_1$ is a submodule of $Rx_1$ whence is isomorphic to an ideal of $R$, which is either the zero ideal of principal, i.e. free of rank $1$.

    We now move to the induction step. First, we consider the case when $i$ is a successor ordinal and due to the induction hypothesis, $M_{i - 1}$ is a free $R$-module of rank less than or equal to $|i - 1|$. Let 
    \begin{equation*}
        \fraka = \{\pi_i(x)\mid x\in M_i\}.
    \end{equation*}
    Note that $\fraka$ is an $R$-submodule of $R$ whence an ideal of $R$. Thus, either $\fraka = 0$ or $\fraka = (a)$ for some $a\in R\backslash\{0\}$. In the former case, $M_i = M_{i - 1}$ which completes the induction step. In the latter case, there is some $w\in M_i$ such that $\pi_i(w) = a$. 

    Now, consider any element $x\in M_i$. Then, $x$ can be written as a linear combination 
    \begin{equation*}
        x = \sum_{j < i}a_j x_j + bx_i.
    \end{equation*}
    According to our choice of $x$, $b\in (a)$ whence, there is a suitable $r\in R$ suchthat $x - rw\in M_{i - 1}$. That is, $M_i = M_{i - 1} + (w)$. We contend that this sum is direct. To see this, suppose $cw\in M_{i - 1}$ for some $c\in R$ and suppose $w = \sum_{j\leqq i}\alpha_j x_j$. Then, 
    \begin{equation*}
        \sum_{j\leqq i}c\alpha_j x_j = \sum_{j < i}\beta_j\alpha_j,
    \end{equation*}
    for some $\beta_j\in R$, implying that $cx_i = 0$, i.e. $c = 0$. Hence, $M_i = M_{i - 1}\oplus(w)$.

    Next, we deal with the case of limit ordinals. Suppose $\lambda$ is a limit ordinal in $I$. Then, we have a function $f:\lambda\to M$ such that for each $i < \lambda$, the module $M_i$ is free, generated by $\{f(j)\mid j\leqq i\}$. Whence, $M_\lambda$ is generated by $\{f(j)\mid j < \lambda\}$ and is obviously free. This completes the inductive step and thus the proof of the theorem.
\end{proof}

\section{Finitely Generated Modules}

\begin{definition}[Finitely Generated Module]
    An $R$-module $M$ is said to be finitely generated if there is a finite subset $S$ of $M$ which generates $M$. That is, there is no proper submodule $N$ of $M$ containing $S$.
\end{definition}

A submodule of a finitely generated module need not be finitely generated, let $A = \Z[x_1,x_2,\ldots]$ and consider $A$ as an $A$-module. The ideal $(x_1,x_2,\ldots)$ is not finitely generated.

\begin{proposition}
    An $R$-module $M$ is finitely generated if and only if $M$ is isomorphic to a quotient of $R^{\oplus n}$ for some positive integer $n$.
\end{proposition}
\begin{proof}
    We shall only prove the forward direction since the converse is trivial to prove. Suppose $M$ is finitely generated. Then, it is generated by a finite subset $S = \{x_1,\ldots,x_m\}$. Define the $R$-module homomorphism $\phi:R^{\oplus n}\to M$ by $(r_1,\ldots,r_n)\mapsto r_1x_1 + \cdots + r_nx_n$. From the first isomorphism theorem, we have $M\cong R^{\oplus n}/\ker\phi$.
\end{proof}

\begin{proposition}\thlabel{prop:CH-type}
    Let $M$ be a finitely generated $A$-module and $\mathfrak a$ an ideal of $A$. Let $\phi\in\End(M)$ such that $\phi(M)\subseteq\mathfrak aM$. Then, there are $a_0,\ldots,a_{n - 1}\in\mathfrak a$ such that 
    \begin{equation*}
        \phi^n + a_{n - 1}\phi^{n - 1} + \cdots + a_0 = 0
    \end{equation*}
    as an element of $\End(M)$, where $a_k$ is treated as the homomorphism $x\mapsto a_kx$ in $\End(M)$.
\end{proposition}
\begin{proof}
    Let $\{x_1,\ldots,x_n\}$ be a generating set for $M$. Then, for all $1\le i\le n$, there are coefficients $\{a_{i1},\ldots,a_{in}\}$ in $\mathfrak a$ such that 
    \begin{equation*}
        \phi(x_i) = \sum_{j = 1}^n a_{ij}x_j
    \end{equation*}
    We may rewrite this as 
    \begin{equation*}
        \sum_{j = 1}^n(\phi\delta_{ij} - a_{ij})x_j = 0
    \end{equation*}
    Let $B$ denote the matrix $(\phi\delta_{ij} - a_{ij})_{1\le i,j\le n}$. Then, multiplying by $\operatorname{adj}(B)$, we see that $\det(B)(x_j) = 0$ for all $1\le j\le n$ where $\det(B)$ is viewed as an element in $\End(M)$ and thus, is the zero map in $\End(M)$. It is not hard to see that $\det(B)$ is in the required form.
\end{proof}

\begin{corollary}\thlabel{corr:stronger-nakayama}
    Let $M$ be a finitely generated $A$-module and $\fraka$ an ideal of $A$ such that $\fraka M\subseteq M$, then there is $a\in\fraka$ such that $(1 + a)M = 0$.
\end{corollary}
\begin{proof}
    Substitute $\phi = \id$ in the above proposition.
\end{proof}

\begin{lemma}[Nakayama]\thlabel{lem:nakayama}
    Let $M$ be a finitely generated module and $\mathfrak a\subseteq\mathfrak R$ be an ideal such that $M = \mathfrak aM$. Then, $M = 0$. 
\end{lemma}
\begin{proof}
    Let $\phi = \mathbf{id}$ be the identity homomorphism in $\End(M)$. Using \thref{prop:CH-type}, there are coefficients $a_0,\ldots,a_{n - 1}\in\mathfrak a$ satisfying the statement of the proposition. As a result, $x = 1 + a_{n - 1} + \ldots + a_0$ is the zero endomorphism. But since $a_{n - 1} + \ldots + a_0\in\mathfrak a\subseteq\mathfrak R$, $x$ is a unit and hence, $M = 0$.
\end{proof}

\begin{corollary}
    Let $M$ be a finitely generated $A$-module, $N$ a submodule of $M$ and $\mathfrak a\subseteq\mathfrak R$ an ideal. If $M = \mathfrak aM + N$ then $M = N$.
\end{corollary}
\begin{proof}
    We have $M/N = \mathfrak aM/N$, consequently, $M/N = 0$ and $M = N$ due to \thref{lem:nakayama}.
\end{proof}

A surprising application of Nakayama is the following. 
\begin{proposition}
    Let $M$ be a finitely generated $A$-module and $\varphi: M\to M$ a surjective homomorphism. Then, $\varphi$ is an isomorphism.
\end{proposition}
\begin{proof}
    Notice that $M$ also has the structure of an $A[x]$-module where the action of $x$ is given by 
    \begin{equation*}
        x\cdot m = \varphi(m).
    \end{equation*}
    Further, note that $M$ is still finitely generated as an $A[x]$ module. Due to the surjectivity of $\varphi$, we have $(x)M = M$. Due to \thref{corr:stronger-nakayama}, there is some $f(x)\in A[x]$ such that $(1 + xf(x))M = 0$, that is, for each $m\in M$, $(1 + xf(x))m = 0$. If $m\in\ker\varphi$, then it is immediate that $0 = m + f(\varphi)(\varphi(m)) = m$ and thus $\varphi$ is injective. This completes the proof.
\end{proof}

\begin{lemma}
    Let $(A,\mathfrak m)$ be local and $k = A/\mathfrak m$. Let $M$ be a finitely generated $A$-module. Let $\{\overline x_1,\ldots,\overline x_n\}$ be elements in $M/\mathfrak mM$ that form a basis for $M/\mathfrak mM$ as a $k$-vector space. Then, $\{x_1,\ldots,x_n\}$ generates $M$.
\end{lemma}
\begin{proof}
    Let $N$ be the submodule generated by $\{x_1,\ldots,x_n\}$. Then, the composition $N\hookrightarrow M\twoheadrightarrow M/\mathfrak mM$ is surjective, consequently, $M = N + \mathfrak mM$ whence, it follows that $M = N$.
\end{proof}

\begin{proposition}
    Let $\fraka\unlhd A$ be a finitely generated ideal. If $\fraka$ is idempotent, then it is principal.
\end{proposition}
\begin{proof}
    We have $\fraka^2 = \fraka$ and viewing $\fraka$ as an $A$-module, due to \thref{corr:stronger-nakayama}, there is some $a\in\fraka$ such that $(1 - a)\fraka = 0$. In particular, $\fraka\subseteq(a)$, and thus, $\fraka = (a)$.
\end{proof}

\section{\texorpdfstring{$\Hom$}{} Modules and Functors}

For $R$-modules $M,N$, we denote the set of all $R$-module homomorphisms from $M$ to $N$ by $\Hom_R(M,N)$. When the choice of the ring $R$ is clear from the context, we shall denote this set by $\Hom(M,N)$.

\begin{proposition}
    Let $M,N$ be $A$-modules. Then $\Hom(M,N)$ has the structure of an $A$-module.
\end{proposition}
\begin{proof}
    It is obvious that $\Hom(M,N)$ has the structure of an abelian group. Define the natural action by $(af)(x) = af(x)$. It is not hard to see that this action is well defined.
\end{proof}

\begin{proposition}
    Let $\{M_\lambda\}_{\lambda\in\Lambda}$ be a collection of $A$-modules. Then, for any $A$-module $N$, we have a natural isomorphism
    \begin{equation*}
        \Hom_A\left(\bigoplus_{\lambda\in\Lambda}M_\lambda, N\right)\cong\prod_{\lambda\in\Lambda}\Hom_A(M_\lambda,N)
    \end{equation*}
\end{proposition}
\begin{proof}
    Since the direct sum is the coproduct in $A-\mathbf{Mod}$, the conclusion follows from the universal property.
\end{proof}

\begin{proposition}
    If $M$ is a finitely generated $A$-module and $\{N_i\}_{i\in I}$ is a collection of $A$-modules, then there is a natural isomorphism
    \begin{equation*}
        \Hom_A\left(M, \bigoplus_{i\in I}N_i\right)\cong\bigoplus_{i\in I}\Hom_A(M, N_i)
    \end{equation*}
\end{proposition}
\begin{proof}
    Consider the map 
    \begin{equation*}
        \Phi:\Hom_A\left(M, \bigoplus_{i\in I}N_i\right)\to\bigoplus_{i\in I}\Hom_A(M, N_i)
    \end{equation*}
    given by $\Phi(f) = (\pi_i\circ f)_{i\in I}$. That this map is well-defined follows from the fact that $M$ is a finitely generated $A$-module. The rest is a routine verification of injectivity and surjectivity.
\end{proof}

\begin{theorem}
    Let $\phi: M\to N$ be an $A$-module homomorphism. Then, for every $R$-module $P$, there is an induced $A$-module homomorphism $\overline\phi:\Hom(N,P)\to\Hom(M,P)$ and an induced $A$-module homomorphism $\widetilde\phi:\Hom(P,M)\to\Hom(P,N)$. 
    
    Equivalently phrased, $\Hom(-,P)$ is a contravariant functor while $\Hom(P,-)$ is a covariant functor.
\end{theorem}
\begin{proof}
    We shall prove only the first half of the assertion since the second half follows from a similar proof. Define $\overline\phi$ using the following commutative diagram: 
    \begin{equation*}
    \xymatrix {
        M\ar[r]^{\phi}\ar@{.>}[rd]_{f\circ\phi} & N\ar[d]^f\\
        & P
    }
    \end{equation*}
    To see that this is indeed an $R$-module homomorphism, we need only verify that for all $f,g\in\Hom(N,P)$ and all $r\in R$, $(f + rg)\circ\phi = f\circ\phi + rg\circ\phi$ which is trivial to check.
\end{proof}

\begin{theorem}
    $\Hom(M,-)$ is a \underline{left exact} functor.
\end{theorem}
\begin{proof}
    Let $0\rightarrow N'\stackrel f\rightarrow N\stackrel g\rightarrow N''$ be an exact sequence. First, we shall show that $\overline f$ is injective. Indeed, let $u\in\ker\overline f$. Then, $f\circ u$ is the zero morphism. But since $f$ is injective, we must have that $u$ is the zero morphism.

    Next, we shall show that $\im\overline f = \ker\overline g$. Obviously, $\overline g\circ\overline f = 0$ and thus it suffices to show $\ker\overline g\subseteq\ker\overline f$. Let $u\in\ker\overline g$. That is, $g\circ u = 0$. Then, we may define $v: M\to N'$ by $v(m) = f^{-1}(u(m))$, which is well defined since $f$ is injective. It is not hard to see that $v$ is a module homomorphism, implying the desired conclusion.
\end{proof}

\section{Exact Sequences}

\begin{definition}
    A sequence of module homomorphisms 
    \begin{equation*}
        M\stackrel{f}{\longrightarrow} N\stackrel{g}{\longrightarrow}P
    \end{equation*}
    is said to be exact at $N$ if $\im f = \ker g$. A short exact sequence is a sequence of module homomorphisms: 
    \begin{equation*}
        0\longrightarrow M\stackrel{f}{\longrightarrow} N\stackrel{g}{\longrightarrow} P\longrightarrow 0
    \end{equation*}
    which is exact at $M$, $N$ and $P$.
\end{definition}

It is not hard to see that the sequence in the definition is short exact if and only if $f$ is injective, $g$ is surjective and $\im f = \ker g$.

\subsection{Diagram Chasing Poster Children}
Throughout this (sub)section, $A,B,C$ are $R$-modules where $R$ is a commutative ring.

\begin{lemma}[Splitting Lemma]
    Let $0\longrightarrow A\stackrel{\iota}{\longrightarrow}B\stackrel{\pi}{\longrightarrow}C\longrightarrow 0$ be a short exact sequence. Then the following are equivalent.
    \begin{enumerate}[label=(\alph*)]
        \item There is $\varphi: C\to B$ such that $\pi\circ\varphi = \mathbf{id}_C$ 
        \item There is $\psi: B\to A$ such that $\psi\circ\iota = \mathbf{id}_A$ 
        \item There is an isomorphism $\Phi: B\to A\oplus C$ making the following diagram commute.
        \begin{equation*}
            \xymatrix {
                0\ar[r] & A\ar[d]_{\mathbf{id}_A}\ar[r]^\iota & B\ar[r]^\pi\ar@{.>}[d]|-{\Phi} & C\ar[d]^{\mathbf{id}_C}\ar[r] & 0\\
                0\ar[r] & A\ar[r] & A\oplus C\ar[r] & C\ar[r] & 0\\
            }
        \end{equation*}
    \end{enumerate}
\end{lemma}
\begin{proof}
$(a)\implies(b)$. Define $\psi(b) = \iota^{-1}(b - \varphi(\pi(b)))$. That this map is well defined follows from $\im\iota = \ker\pi$ and that it is a homomorphism is trivial. It is not hard to see that $\psi\circ\iota = \mathbf{id}_A$.

$(b)\Longrightarrow(c)$. Define the map $\Phi: B\to A\oplus C$ by $\Phi(b) = (\psi(b), \pi(b - \iota\circ\psi(b)))$. It is trivial to check that this is an $R$-module homomorphism. From the Short Five Lemma, it now follows that $\Phi$ is an isomorphism.

$(c)\Longrightarrow(a)$. Trivial.
\end{proof}


\section{Tensor Product}

\begin{definition}[Bilinear Map]
    Let $M, N, P$ be $A$-modules. A map $T: M\times N\to P$ is said to be bilinear if for each $x\in M$, the mapping $T_x: N\to P$ given by $y\mapsto T(x,y)$ is $A$-linear and for each $y\in N$, the mapping $T_y: M\to P$ given by $x\mapsto T(x,y)$ is $A$-linear.
\end{definition}

Fix two $A$-modules $M$ and $N$. Let $\mathscr C$ denote the category of bilinear maps $T: M\times N\to P$ where $P$ is any $A$-module. A morphism between two bilinear maps $f: M\times N\to P_1$ and $g: M\times N\to P_2$ in this category is a module homomorphism $\phi: P_1\to P_2$ such that the following diagram commutes: 
\begin{equation*}
\xymatrix {
    M\times N\ar[r]^-f\ar[d]_g & P_1\ar@{.>}[ld]^\phi\\
    P_2
}
\end{equation*}

A universal object in $\mathscr C$ is called the tensor product of $M$ and $N$ and is denoted by $M\otimes N$. In other words, the tensor product is an initial object in the category $\mathscr C$.

\begin{definition}[Universal Property of the Tensor Product]
    Let $M,N,P$ be $A$-modules and $T: M\times N\to P$ be a bilinear map. Then, there is a unique $A$-module homomorphism $\phi: M\otimes N\to P$ such that the following diagram commutes: 
    \begin{equation*}
    \xymatrix {
        M\times N\ar[r]^-T\ar[d]_\varphi & P\\
        M\otimes_A N\ar@{.>}[ru]_{\exists!\phi}
    }
    \end{equation*}
\end{definition}

Of course, having the universal property would imply that the tensor product, if it exists, is unique upto a unique isomorphism. We shall now construct a tensor product of $M$ and $N$.

\subsection*{Constructing the Tensor Product}

Let $F$ be the free $A$-module on $M\times N$. Let us denote the basis elements of $F$ by $e_{(x,y)}$ where $x\in M$ and $y\in N$. Now, for all $x,x_1,x_2\in M$, $y,y_1,y_2\in N$ and $a\in A$, let $D$ denote the submodule generated by elements of the form: 
\begin{align*}
    &e_{(x_1 + x_2, y)} - e_{(x_1,y)} - e_{(x_2,y)}\\
    &e_{(x,y_1 + y_2)} - e_{(x,y_1)} - e_{(x,y_2)}\\
    &e_{(ax,y)} - ae_{(x,y)}\\
    &e_{(x,ay)} - ae_{(x,y)}
\end{align*}

Let $G = F/D$ and let $\varphi: M\times N\to G$ be the composition of the following maps: 
\begin{equation*}
    M\times N\hookrightarrow F\twoheadrightarrow G
\end{equation*}

Let $T: M\times N\to P$ be a bilinear map. Consider the following commutative diagram: 

\begin{equation*}
\xymatrix {
    M\times N\ar[r]^-T\ar@{^{(}->}_{\iota}[d] & P\\
    F\ar[r]_{\pi}\ar@{.>}[ru]|{\exists! f} & G\ar@{.>}[u]_{\exists!\phi}
}
\end{equation*}

To show that existence of $\phi$, we must show that $D\subseteq\ker f$, since we can then finish using the universal property of the kernel. But this is trivial to check and follows from the fact that $T$ is a bilinear map and completes the construction.

\begin{mdframed}
    Similarly, we define the tensor product for a finite sequence of $A$-modules $\{M_i\}_{i = 1}^n$. That is, given a multilinear map $T:\prod\limits_{i = 1}^n M_i\to P$, there is a unique $A$-module homomorphism $\phi$ such that the following diagram commutes: 
    \begin{equation*}
    \xymatrix{
        M_1\times\cdots\times M_n\ar[r]^-T\ar[d]_\varphi & P\\
        M_1\otimes\cdots\otimes M_n\ar@{.>}[ru]_-{\exists!\phi}
    }
    \end{equation*}
\end{mdframed}

\begin{proposition}
    Let $F$ and $G$ be free $A$-modules with basis given by $\{f_i\}_{i\in I}$ and $\{g_j\}_{j\in J}$ respectively. Then, $F\otimes_A G$ is a free $A$-module with basis $\{f_i\otimes g_j\}_{i\in I,~j\in J}$.
\end{proposition}
\begin{proof}
    It is not hard to see that the set $\{f_i\otimes g_j\}_{i\in I,~j\in J}$ is generating for $F\otimes_A G$. Therefore, it suffices to show that this set is linearly independent. Suppose not, then there is a finite linear combination 
    \begin{equation*}
        \sum_{i\in I,~j\in J}a_{ij}f_i\otimes g_j = 0
    \end{equation*}
    Pick some $i_0\in I$ and $j_0\in J$. Let $\phi: F\times G\to A$ be the bilinear map such that 
    \begin{equation*}
        \phi(f_i,g_j) = 
        \begin{cases}
            1 & i = i_0\text{ and } j = j_0\\
            0 & \text{otherwise}
        \end{cases}
    \end{equation*}
    This induces an $A$-module homomorphism $\varphi: F\otimes G\to A$ such that 
    \begin{equation*}
        \varphi(f_i\otimes g_j) = 
        \begin{cases}
            1 & i = i_0\text{ and } j = j_0\\
            0 & \text{otherwise}
        \end{cases}
    \end{equation*}
    whence, it follows that $a_{i_0j_0} = 0$ and the collection $\{f_i\otimes g_j\}_{i\in I,~j\in J}$ is linearly independent.
\end{proof}

\subsection{Properties of Tensor Product}

Given two modules $M$ and $N$ with the canonical map $\varphi: M\times N\to M\otimes N$, we denote by $m\otimes n$, the element $\varphi(m,n)$ in $M\otimes N$.

\begin{proposition}
    Let $M, N, P$ be $A$-modules and $\{M_i\}_{i\in I}$ a collection of $A$-modules. Then, 
    \begin{enumerate}[label=(\alph*)]
    \item $M\otimes_A N\cong N\otimes_A M$ 
    \item $(M\otimes_A N)\otimes_A P\cong M\otimes_A(N\otimes_A P)\cong M\otimes_A N\otimes_A P$ 
    \item $\left(\bigoplus_{i\in I}M_i\right)\otimes_A N\cong\bigoplus_{i\in I}(M_i\otimes_A N)$
    \item $A\otimes_A M\cong M$
    \end{enumerate}
\end{proposition}
\begin{proof}
\begin{enumerate}[label=(\alph*)]
\item First, we shall show that there are well defined homomorphisms $M\otimes N\to N\otimes M$ and $N\otimes M\to M\otimes N$ mapping $m\otimes n\mapsto n\otimes m$ and $n\otimes m\mapsto m\otimes n$ respectively. This is best done using the universal property. Let $T: M\times N\to N\times M$ be the isomorphism $m\times n\mapsto n\times m$. Consider now the following commutative diagram: 
\begin{equation*}
\xymatrix{
    M\times N\ar[d]_\varphi\ar[r]^-T & N\times M\ar[d]^{\varphi'}\\
    M\otimes N & N\otimes M
}
\end{equation*}

Since both $\varphi'$ and $T$ are bilinear, so is $\varphi\circ T$, consequently, there is a unique induced homomorphism $f: M\otimes N\to N\otimes M$ making the diagram commute, consequently, $f(m\otimes n) = \varphi'(T(m\times n)) = n\otimes m$.

Similarly, there is a homomorphism $g: N\otimes M\to M\otimes N$ such that $g(n\otimes m) = m\otimes m$. It is not hard to see that $g\circ f = \mathbf{id}_{M\otimes N}$ and $f\circ g = \mathbf{id}_{N\otimes M}$, consequently, they are isomorphisms.

\item We shall show $(M\otimes_A N)\otimes_A P\cong M\otimes_A N\otimes_A P$ since the proof of the other isomorphism follows analogously. Fix some $z\in P$ and consider the map $f_z: M\times N\to M\otimes_A N\otimes_A P$ given by $(x,y)\mapsto x\otimes y\otimes z$. This is an $A$-linear map and thus induces a map $g_z: M\otimes_A N\to M\otimes_A N\otimes_A P$ given by $g_z(x\otimes y) = x\otimes y\otimes z$. The map $G: (M\otimes_A N)\times P\to M\otimes_A N\otimes_A P$ given by $G(x\otimes y, z) = g_z(x\otimes y) = x\otimes y\otimes z$ is a well defined $A$-linear map which induces a map $h: (M\otimes_A N)\otimes_A P\to M\otimes_A N\otimes_A P$ given by $(x\otimes y)\otimes z\mapsto x\otimes y\otimes z$.

On the other hand, the map $F: M\times N\times P\to (M\otimes_A N)\otimes_A P$ given by $(x,y,z)\mapsto x\otimes y\otimes z$ is $A$-linear and thus induces a map $f: M\otimes_A N\otimes_A P\to (M\otimes_A N)\otimes_A P$ given by $x\otimes y\otimes z\mapsto(x\otimes y)\otimes z$. Since the maps $f$ and $h$ are inverses to one another for elementary tensors, they are inverses to one another over their respective domains, whereby both are isomorphisms.

\item Define the map $f:\left(\bigoplus_{i\in I}M_i\right)\times N\to\bigoplus(M_i\otimes_A N)$ by $f((m_i)\otimes n) = (m_i\otimes n)$, which is a bilinear map. This induces a map $\phi:\left(\bigoplus_{i\in I}M_i\right)\otimes_A N\to\bigoplus_{i\in I}(M_i\otimes_A N)$ such that $f((m_i)\otimes n) = (m_i\otimes n)$. 

Now, consider the map $f_i: M_i\times N\to M\otimes N$ given by $f_i(m_i, n) = \iota_i(m_i)\otimes n$. This induces a map $g_i: M_i\otimes_A N\to M\otimes N$ such that $g_i(m_i\otimes n) = \iota_i(m_i)\otimes n$. We may now define a map $\psi:\bigoplus_{i\in I} (M_i\otimes_A N)\to\left(\bigoplus_{i\in I}M_i\right)\otimes_A N$ given by 
\begin{equation*}
    \psi((m_i\otimes n_i)) = \sum g_i(m_i\otimes n_i)
\end{equation*}
Obviously the sum on the right is a finite sum. Further, since each each $g_i$ is well defined, so is $\psi$. 

Lastly, we shall show that $\phi$ and $\psi$ are inverses to one another. Indeed, 
\begin{equation*}
    \psi\circ\phi((m_i)\otimes n) = \psi((m_i\otimes n)) = \sum \iota_i(m_i)\otimes n = (m_i)\otimes n
\end{equation*}
and 
\begin{equation*}
    \phi\circ\psi((m_i\otimes n_i)) = \sum\phi(g_i(m_i\otimes n_i)) = (m_i\otimes n_i)
\end{equation*}

\item Consider the map $T: A\times M\to M$ given by $(a,m)\mapsto am$. It is not hard to see that this map is bilinear, consequently, there is a map $f: A\otimes M\to M$ such that the following diagram commutes: 
\begin{equation*}
\xymatrix {
    A\times M\ar[r]^-T\ar[d]_\varphi & M\\
    A\otimes M\ar@{.>}[ru]_-{f}
}
\end{equation*}
Note that $f(a\otimes m) = am$ by definition. Consider the map $g: M\to A\otimes M$ given by $g(m) = 1\otimes m$. It is not hard to see that $g$ is a well defined module homomorphism. Further, since $f\circ g$ and $g\circ f$ are the identity homomorphisms, they both must be isomorphisms.
\end{enumerate}
\end{proof}

\begin{example}
    Show that $\Z/m\Z\otimes\Z/n\Z\cong\Z/\gcd(m,n)\Z$ for all $m,n\in\N$. In particular, if $m$ and $n$ are coprime, then $\Z/m\Z\otimes\Z/n\Z = 0$.
\end{example}
\begin{proof}
    Consider the module homomorphism $T:\Z\to\Z/m\Z\otimes\Z/n\Z$. 
\end{proof}

Let $f: M\to M'$ and $g: N\to N'$ be $A$-module homomorphisms. Then, the map $\Phi: M\times N\to M'\otimes N'$ given by $\Phi(m,n) = f(m)\otimes g(n)$. It is not hard to see that $\Phi$ is bilinear. Consequently, it induces a map $f\otimes g: M\otimes N\to M'\otimes N'$ such that 
\begin{equation*}
    (f\otimes g)(x\otimes y) = f(x)\otimes g(y)
\end{equation*}

Further, if $f': M'\to M''$ and $g': N'\to N''$ are $A$-module homomorphisms, then we have another map $f'\otimes g': M'\otimes N'\to M''\otimes N''$ such that 
\begin{equation*}
    (f'\otimes g')(x\otimes y) = f'(x)\otimes g'(y)
\end{equation*}

Now, it is not hard to see that $(f'\circ f')\otimes(g'\circ g)$ and $(f'\otimes g')\circ(f\otimes g)$ agree on the elementary tensors, therefore, agree on all of $M\otimes N$.

\subsection{Restriction and Extension of Scalars}

Let $\phi: A\to B$ be a homomorphism of rings. We shall 
\begin{itemize}
    \item convert an $B$-module into an $A$-module. This is known as \textit{restriction of scalars}.
    \item construct from an $A$-module a $B$-module. This is known as \textit{extension of scalars}.
\end{itemize}

The first is rather easy to do. Begin with an $B$-module $M$ and define the action of $A$ by $a\cdot m = \phi(a)\cdot m$. That this is a valid ring action is easy to verify. As for the second, note that the homomorphism $\phi$ gives $B$ the structure of an $A$-module whereby, we may consider the tensor product of $A$-modules $B\otimes_A M$. Now, for $b,b'\in B$, define 
\begin{equation*}
    b'\cdot(b\otimes m) = bb'\otimes m
\end{equation*}
It is not hard to see that this is a ring, whereby, $B\otimes_A M$ is also a $B$-module.

\section{Right Exactness}

\begin{proposition}\thlabel{prop:hom-tensor-adjunction}
    Let $M,N,P$ be $A$-modules. Then, there is a natural isomorphism:
    \begin{equation*}
        \Hom_A(M,\Hom_A(N, P))\cong\Hom_A(M\otimes_A N, P)
    \end{equation*}
\end{proposition}
\begin{proof}
    Consider the map 
    \begin{equation*}
        \theta: \Hom_A(M\otimes_A N, P)\longrightarrow\Hom_A(M,\Hom_A(N, P))
    \end{equation*}
    given by $\theta(\alpha)(m)(n) = \alpha(m\otimes n)$. Now, pick some $\eta\in\Hom_A(M,\Hom_A(N,P))$. Define the map $\zeta: M\times N\to P$ given by $\zeta(m, n) = \eta(m)(n)$. Obviously, $\zeta$ is bilinear and induces a map $\delta: M\otimes_A N\to P$ such that $\delta(m\otimes n) = \eta(m)(n)$. Call the map sending $\eta\mapsto\delta$ as $\beta$ where 
    \begin{equation*}
        \beta: \Hom_A(M,\Hom_A(N, P))\to\Hom_A(M\otimes_A N, P)
    \end{equation*}
    and $\beta(\eta)(m\otimes n) = \eta(m)(n)$.

    We contend that $\theta$ and $\beta$ are inverses to one another. Indeed, 
    \begin{equation*}
        ((\beta\circ\theta)(\alpha))(m\otimes n) = \theta(\alpha)(m)(n) = \alpha(m\otimes n)
    \end{equation*}
    and 
    \begin{equation*}
        ((\theta\circ\beta)(\eta))(m)(n) = \beta(\eta)(m\otimes n) = \eta(m)(n)
    \end{equation*}
    whence the conclusion follows.
\end{proof}

In particular, we see that the functor $-\otimes_A N$ is the left adjoint of the functor $\Hom_A(N,-)$, consequently, $\Hom_A(N,-)$ is the right adjoint of $-\otimes_A N$.

\begin{theorem}
    The functor $-\otimes_A N$ is right exact. That is, given a exact sequence
    \begin{equation*}
        M'\stackrel{f}{\longrightarrow}M\stackrel{g}{\longrightarrow}M''\longrightarrow 0
    \end{equation*}
    the sequence 
    \begin{equation*}
        M'\otimes_A N\stackrel{f\otimes 1}{\longrightarrow}M\otimes_A N\stackrel{g\otimes 1}{\longrightarrow}M''\otimes_A N\longrightarrow 0
    \end{equation*}
\end{theorem}
\begin{proof}
    Since the given sequence is exact, so is 
    \begin{equation*}
        \Hom_A(M'',\Hom_A(N,P))\stackrel{\overline g}{\longrightarrow}\Hom_A(M,\Hom_A(N,P))\stackrel{\overline f}{\longrightarrow}\Hom_A(M',\Hom_A(N,P))\longrightarrow 0
    \end{equation*}
    but from \thref{prop:hom-tensor-adjunction}, so is
    \begin{equation*}
        \Hom_A(M''\otimes_A N, P)\longrightarrow\Hom_A(M\otimes_A N, P)\longrightarrow\Hom_A(M'\otimes_A N, P)\longrightarrow 0
    \end{equation*}
    Since the above sequence is exact for all modules $P$, we have the desired conclusion.
\end{proof}

The tensor product is not left exact. Conider the sequence of $\Z$-modules
\begin{equation*}
    0\hookrightarrow\Z\stackrel{f}{\longrightarrow}\Z
\end{equation*}
where $f(m) = 2m$. Upon tensoring with $\Z/2\Z$, we get the sequence 
\begin{equation*}
    0\longrightarrow\Z\otimes_{\Z}\Z/2\Z\stackrel{f\otimes1}{\longrightarrow}\Z\otimes_{\Z}\Z/2\Z
\end{equation*}

Note that 
\begin{equation*}
    (f\otimes 1)(m\otimes\overline n) = 2m\otimes\overline{n} = m\otimes(2\overline n) = m\otimes 0 = 0
\end{equation*}

Therefore, the sequence cannot be exact.

\begin{theorem}
    There is a natural isomorphism 
    \begin{equation*}
        (M\otimes_A B)\otimes_B(M'\otimes_A B)\cong (M\otimes_A M')\otimes_A B.
    \end{equation*}
\end{theorem}
\begin{proof}
    Note that the functor $(-\otimes_A B)\otimes_B(M'\otimes_A B)$ is right exact, since it is a composition of tensor products. First, note that the isomorphism is obvious when $M$ is a free module. Now suppose $M$ were arbitrary. Then, we have an exact sequence 
    \begin{equation*}
        \bigoplus_{j\in J}A\longrightarrow\bigoplus_{j\in J}A\longrightarrow M\longrightarrow 0.
    \end{equation*}
    Denote by $F' =\bigoplus_{j\in J} A$ and $F = \bigoplus_{i\in I} A$. Let
    \begin{equation*}
        \theta_M: (M\otimes_A B)\otimes_B(M'\otimes_A B)\to (M\otimes_A M')\otimes_A B.
    \end{equation*}
    We have shown above that $\theta_M$ is an isomorphism whenever $M$ is a free module. In particular, we have a commutative diagram with exact rows.
    \begin{equation*}
        \xymatrix {
            (F'\otimes_A B)\otimes_B(M'\otimes_A B)\ar[r]\ar[d]_{\cong} & (F\otimes_A B)\otimes_B(M'\otimes_A B)\ar[r]\ar[d]_{\cong} & (M\otimes_A B)\otimes_B(M'\otimes_A B)\ar[r]\ar[d]_{\theta_M} & 0\ar@{=}[d]\\
            (F'\otimes_A M')\otimes_A B\ar[r] & (F\otimes_A M')\otimes_A B\ar[r] & (M\otimes_A M')\otimes_A B\ar[r] & 0
        }
    \end{equation*}
    Conclude using the five lemma.
\end{proof}

\begin{theorem}
    Let $\phi: A\to B$ be a ring homomorphism. Let $M$ be an $A$-module and $N$ a $B$-module. Note that $N$ is also an $A$-module owing to the restriction of scalars. Then, there is a natural isomorphism 
    \begin{equation*}
        (M\otimes_A B)\otimes_B N\stackrel{\sim}{\longrightarrow} M\otimes_A N
    \end{equation*}
    of $B$-modules.
\end{theorem}
\begin{proof}
    Easy proof using universal properties.
\end{proof}

More generally, the following is true. 

\begin{theorem}
    Let $M$ be an $A$-module, $P$ a $B$-module and $N$ an $(A, B)$-bimodule. Then, $M\otimes_A N$ is naturally a $B$-module, $N\otimes_B P$ an $A$-module and there is an natural isomorphism of $(A,B)$-bimodules: 
    \begin{equation*}
        (M\otimes_A N)\otimes_B P\cong M\otimes_A (N\otimes_B P).
    \end{equation*}
\end{theorem}
\begin{proof}
    
\end{proof}

As an application, we have the following.
\begin{proposition}\thlabel{prop:tensor-over-local-zero}
    Let $A$ be a local ring and $M, N$ finitely generated $A$-modules. Then, $M\otimes_A N = 0$ if and only if $M = 0$ or $N = 0$.
\end{proposition}
\begin{proof}
    Let $k$ denote the residue field of $A$. Then, 
    \begin{gather*}
        (M\otimes_A k)\otimes_k(k\otimes_A N)\cong((M\otimes_A k)\otimes_k k)\otimes_A N\cong (M\otimes_A(k\otimes_k k))\otimes_A N\cong(M\otimes_A k)\otimes_A N\\
        \cong (k\otimes_A M)\otimes_A N\cong k\otimes_A(M\otimes_A N) = 0.
    \end{gather*}
    But a tensor product of vector spaces is $0$ if and only if one of the two vector spaces is $0$. Hence, either $M\otimes_A k = 0$ or $N\otimes_A k = 0$, whence, it follows from \thref{lem:nakayama}, that $M = 0$ or $N = 0$.
\end{proof}

\section{Flat Modules}

\begin{definition}[Flat Module]
    An $A$-module $M$ is said to be flat if the functor $-\otimes_A N$ is exact.
\end{definition}

We know that $-\otimes_AN$ is right exact, hence, it suffices to show that the functor is left exact.

\begin{theorem}
    Let $N$ be a $A$-module. Then, the following are equivalent 
    \begin{enumerate}[label=(\alph*)]
        \item $N$ is flat 
        \item If $0\rightarrow M'\rightarrow M\rightarrow M''\rightarrow 0$ is an exact sequence of $A$-modules, then the tensored sequence 
        \begin{equation*}
        0\longrightarrow M'\otimes_A N\stackrel{f\otimes 1}{\longrightarrow} M\otimes_A N\stackrel{g\otimes 1}{\longrightarrow} M''\otimes_A N\longrightarrow 0
        \end{equation*}
        is exact.
        \item If $f: M'\to M$ is injective, then $f\otimes 1: M'\otimes N\to M\otimes N$ is injective 
        \item If $f: M'\to M$ is injective and $M,M'$ are finitely generated, then $f\otimes_A 1: M'\otimes_A N\to M\otimes_A N$ is injective.
    \end{enumerate}
\end{theorem}
\begin{proof}
\hfill 
\begin{description}
\item[$(a)\Longleftrightarrow(b)$:] Is well known.
\item[$(b)\Longrightarrow(c)$:] Immediate from considering the short exact sequence $0\rightarrow M'\rightarrow M\rightarrow M/M'\rightarrow0$.
\item[$(c)\Longrightarrow(b)$:] Since $-\otimes_A N$ is known to be right exact as well. 
\item[] \textcolor{red}{TODO: Complete this later}
\end{description}
\end{proof}

\begin{proposition}
    Let $\{M_i\}_{i\in I}$ be a collection of $A$-modules. Then, $M = \bigoplus\limits_{i\in I}M_i$ is flat if and only if $M_i$ is flat for each $i\in I$.
\end{proposition}
\begin{proof}
    From the fact that 
    \begin{equation*}
        M\otimes_A N\cong\bigoplus_{i\in I}(M_i\otimes_A N)
    \end{equation*}
    and the isomorphism is natural.
\end{proof}

\begin{corollary}
    Free modules are flat.
\end{corollary}
\begin{proof}
    Obviously, $A$ is a flat $A$-module, therefore, $\bigoplus_{\lambda\in\Lambda}A$ is free for every indexing set $\Lambda$.
\end{proof}

\begin{proposition}
    Let $B$ be an $A$-algebra and $M$ a flat $A$-module. Then, $M\otimes_A B$ is a flat $B$-module.
\end{proposition}
\begin{proof}
    Follows from the natural isomorphism $(M\otimes_A B)\otimes_B N\cong M\otimes_A N$.
\end{proof}

\begin{lemma}\thlabel{lem:flat-and-tor}
    The following are equivalent.
    \begin{enumerate}[label=(\alph*)]
        \item $M$ is flat. 
        \item $\Tor_i^A(N, M) = 0$ for all $i > 0$ and all $A$-modules $N$. Equivalently, $\Tor_i^A(M, N) = 0$ for all $i > 0$ and all $A$-modules $N$.
        \item $\Tor^A_1(N, M) = 0$ for all $A$-modules $N$. Equivalently, $\Tor_1^A(M, N) = 0$ for all $A$-modules $N$.
        \item $\Tor^A_1(N, M) = 0$ for all finitely generated $A$-modules $N$. Equivalently, $\Tor^A_1(M, N) = 0$ for all finitely generated $A$-modules $N$.
    \end{enumerate}

    The equivalent statements follow from the balancing property of $\Tor$.
\end{lemma}
\begin{proof}
    $(a)\implies(b)$ is immediate from the definition of $\Tor$ while $(b)\implies(c)$ and $(c)\implies(d)$ is something a third grader could figure out. It remains to show that $(c)\implies(a)$. Let $N$ and $N'$ be finitely generated $A$-modules and $f: N'\to N$ an injective homomorphism. Then, there is a short exact sequence 
    \begin{equation*}
        0\into N'\stackrel{f}{\longrightarrow}N\longrightarrow \coker f\onto 0.
    \end{equation*}
    This sequence gives rise to a $\Tor$ long exact sequence 
    \begin{equation*}
        \Tor^A_1(N', M)\to\Tor^A_1(N, M)\to\Tor^A_1(\coker f, M)\to N'\otimes_A M\to N\otimes_A M\to \coker f\otimes_A M\to 0.
    \end{equation*}
    Since $N$ and $N'$ are finitely generated, so is $\coker f$, whence we have a an exact sequence 
    \begin{equation*}
        0\to N'\otimes_A M\to N\otimes_A M\to \coker f\otimes_A M\to 0.
    \end{equation*}
    In particular, this means that $N'\otimes_A M\to N\otimes_A M$ is injective and $M$ is flat.
\end{proof}

\begin{lemma}
    If $0\to M'\to M\to M''\to 0$ be a short exact sequence of $A$-modules with $M''$ flat, then $M'$ is flat if and only if $M$ is flat.
\end{lemma}
\begin{proof}
    Again, using the $\Tor$ long exact sequence, we have 
    \begin{equation*}
        \cdots\to\Tor_n^A(N', M)\to\Tor_n^A(N, M)\to 0\to\Tor_{n - 1}^A(N', M)\to\Tor_{n - 1}^A(N, M)\to 0\to\cdots.
    \end{equation*}
    The conclusion is now obvious.
\end{proof}

\begin{lemma}
    $M$ is flat if and only if $\Tor_1^A(N, M) = 0$ for every cyclic $A$-module $N$, equivalently, $\Tor^A_1(M, N) = 0$ for every cyclic $A$-module $N$.
\end{lemma}
\begin{proof}
    The forward direction is clear. We shall prove the converse. Let $N$ be a finitely generated $A$-module, say generated by $\{x_1,\dots,x_n\}$. Let $N_i$ denote the submodule generated by $\{x_1,\dots, x_i\}$ for $1\le i\le n$. We have a short exact sequence 
    \begin{equation*}
        0\longrightarrow N_i\longrightarrow N_{i + 1}\longrightarrow N_{i + 1}/N_i\longrightarrow 0.
    \end{equation*}
    Note that $N_1$ is cyclic and thus, $\Tor^A_1(N_1, M) = 0$. We shall inductively show that $\Tor^A_1(N_i, M) = 0$. The induction step follows from the $\Tor$ long exact sequence, since 
    \begin{equation*}
        \xymatrix {
            \Tor^A_1(N_i, M)\ar[r]\ar@{=}[d] & \Tor^A_1(N_{i + 1}, M)\ar[r] & \Tor^A_1(N_{i + 1}/N_i, M)\ar@{=}[d]\\
            0\ar[r] & ?\ar[r] & 0
        }
    \end{equation*}
    and thus $\Tor^A_1(N_{i + 1}, M) = 0$. In particular,
    \begin{equation*}
        \Tor^A_1(N, M) = \Tor^A_1(N_n, M) = 0.
    \end{equation*}
    This completes the proof.
\end{proof}
\begin{corollary}
    $M$ is flat if and only if $\Tor^A_1(A/\fraka, M) = 0$ for every ideal $\fraka\unlhd A$. Equivalently, $\Tor^A_1(M,A/\fraka) = 0$ for every ideal $\fraka\unlhd A$.
\end{corollary}
\begin{proof}
    Every cyclic $A$-module is isomorphic to $A/\fraka$ as $A$-modules for some ideal $\fraka\unlhd A$.
\end{proof}

\begin{lemma}
    $M$ is flat if and only if $\Tor^A_1(A/\fraka, M) = 0$ for every finitely generated ideal $\fraka\unlhd A$. Equivalently, $\Tor^A_1(M,A/\fraka) = 0$ for every finitely genrated ideal $\fraka\unlhd A$.
\end{lemma}
\begin{proof}
    We shall show that for any $A$-module $M$, $\Tor^A_1(A/\fraka, M) = 0$ for every finitely generated ideal $\fraka$ is equivalent to $\Tor^A_1(A/\fraka, M) = 0$ for every ideal $\fraka$.

    To see this, note that $\Tor^A_1(A/\fraka, M) = 0$ if and only if the sequence 
    \begin{equation*}
        0\longrightarrow\fraka\otimes_A M\longrightarrow A\otimes_A M
    \end{equation*}
    is exact, where this equivalence follows from the $\Tor$ long exact sequence. \todo{complete this}.
\end{proof}

\begin{proposition}\thlabel{prop:flatness-bourbaki}
    Suppose $M''$ is a flat $A$-module and
    \begin{equation*}
        0\longrightarrow M'\stackrel{f}{\longrightarrow} M\stackrel{g}{\longrightarrow}M''\longrightarrow 0
    \end{equation*}
    is a short exact sequence. If $N$ is any $A$-module, then 
    \begin{equation*}
        0\longrightarrow M'\otimes_A N\stackrel{f\otimes 1}{\longrightarrow} M\otimes_A N\stackrel{g\otimes 1}{\longrightarrow} M''\otimes_A N\longrightarrow 0
    \end{equation*}
    is exact.
\end{proposition}
\begin{proof}[Proof 1]
    The short proof is to consider the tail of the $\Tor$ long exact sequence.
    \begin{equation*}
        \cdots\to\Tor^A_1(M'', N)\to M'\otimes_A N\to M\otimes_A N\to M''\otimes_A N\to 0.
    \end{equation*}
    Since $M''$ is flat, $\Tor^A_1(M'', N) = 0$ and the conclusion follows.
\end{proof}
\begin{proof}[Proof 2]
    This is a beautiful proof using the Snake Lemma. Let $F$ be a free module that surjects onto $N$ with kernel $N'$. Then, we have a short exact sequence 
    \begin{equation*}
        0\longrightarrow N'\longrightarrow F\longrightarrow N\longrightarrow 0.
    \end{equation*}
    We can now construct the following commutative diagram 
    \begin{equation*}
    \xymatrix {
        & M'\otimes_A N'\ar[r]\ar[d] & M\otimes_A N'\ar[r]\ar[d] & M''\otimes_A N'\ar[r]\ar[d] & 0\\
        0\ar[r] & M'\otimes_A F\ar[r] & M\otimes_A F\ar[r] & M''\otimes_A F\ar[r] & 0
    }
    \end{equation*}
    The snake lemma gives an exact sequence 
\end{proof}

\begin{lemma}\thlabel{prop:fp-flat-local-free}
    A finitely presented flat module over a local ring is free.
\end{lemma}
\begin{proof}
    Let $M$ be a finitely presented (and therefeore, finitely generated) flat module over a local ring $(A,\frakm, k)$. Let $x_1,\dots,x_n\in M$ be such that $\overline x_1,\dots,\overline x_n\in M/\frakm M$ form a basis as a $k$-vector space. As we have seen earlier, $x_1,\dots,x_n$ then generate $M$ as an $A$-module. Let $F = A^{\oplus n}$ and $\varphi: F\onto M$ denote the surjection that maps the $i$-th basis element of $F$ to $x_i$. Then, $\ker\varphi$ is finitely generated due to \thref{prop:ker-fg-fp}. We have a short exact sequence 
    \begin{equation*}
        0\longrightarrow\ker\varphi\longrightarrow F\stackrel{\varphi}{\longrightarrow}M\longrightarrow 0.
    \end{equation*}
    Tensoring with $k$ and invoking \thref{prop:flatness-bourbaki}, we have 
    \begin{equation*}
        0\longrightarrow\ker\varphi\otimes_A k\longrightarrow F\otimes_A k\longrightarrow M\otimes_A k\longrightarrow 0
    \end{equation*}
    is exact. But note that $F\otimes_A k\longrightarrow M\otimes_A k$ is a surjection of vector spaces of the same dimension. Therefore, an isomorphism of $k$-vector spaces, consequently, also an isomorphism of $A$-modules. In particular, this means that $\ker\varphi\otimes_A k = 0$. Finally, due to \thref{lem:nakayama}, $\ker\varphi = 0$ and $F\cong M$. This completes the proof.
\end{proof}

\begin{definition}
    A ring $A$ is said to be \emph{von Neumann regular} or \emph{absolutely flat} if every $A$-module is flat.
\end{definition}

\begin{theorem}
    The following are equivalent
    \begin{enumerate}[label=(\alph*)]
        \item $A$ is absolutely flat. 
        \item Every principal ideal in $A$ is idempotent. 
        \item Every finitely generated ideal in $A$ is a direct summand of $A$ as $A$-modules.
    \end{enumerate}
\end{theorem}
\begin{proof}
    
\end{proof}

\begin{proposition}
    A local ring is absolutely flat if and only if it is a field.
\end{proposition}
\begin{proof}
    Let $(A,\frakm, k)$ be local and absolutely flat. Then, there is a short exact sequence 
    \begin{equation*}
        0\to\frakm\to A\to A/\frakm\to 0.
    \end{equation*}
    Since $A/\frakm$ is a flat $A$-module, we may tensor to obtain the short exact sequence 
    \begin{equation*}
        0\to\frakm\otimes_A A/\frakm\to A\otimes_A A/\frakm\to A/\frakm\otimes_A A/\frakm\to 0.
    \end{equation*}
    Note that $A/\frakm\otimes_A A/\frakm = A/\frakm$ as $A$-modules and hence, 
    \begin{equation*}
        0\to m\otimes_A A/\frakm\to A/\frakm\to A/\frakm\to 0
    \end{equation*}
    is exact whence $m\otimes_A A/\frakm = 0$, consequently, $m = 0$ due to \thref{lem:nakayama}.
\end{proof}


\section{Projective Modules}

\begin{theorem}\thlabel{thm:pojective-module-equivalence}
    For an $A$-module $P$, the following are equivalent: 
    \begin{enumerate}[label=(\alph*)]
    \item Every map $f: P\to M''$ can be lifted to $\widetilde{f}: P\to M$ in the following commutative diagram: 
    \begin{equation*}
    \xymatrix{
        & P\ar[ld]_{\widetilde f}\ar[d]^f & \\
        M\ar[r]_g & M''\ar@{->>}[r] & 0
    }
    \end{equation*}
    \item Every short exact sequence $0\rightarrow M'\rightarrow M\rightarrow P\rightarrow 0$ splits 
    \item There is a module $M$ such that $P\oplus M$ is free 
    \item The functor $\Hom_A(P,-)$ is exact.
    \end{enumerate}
\end{theorem}
\begin{proof}
\hfill
\begin{description}
\item[$(a)\Longrightarrow(b)$:] Taking $M'' = P$ and $f = \mathbf{id}_P$, we have the desired conclusion. 
\item[$(b)\Longrightarrow(c)$:] Let $F$ denote the free module on the set $P$. Then, the map $\Phi: F\to P$ given by $\Phi(e_x) = x$ for all $x\in P$ is a surjective $A$-module homomorphism. We have the following short exact sequence: 
\begin{equation*}
    0\rightarrow\ker\Phi\stackrel{\iota}{\longrightarrow}F\stackrel{\Phi}{\longrightarrow}P\rightarrow 0
\end{equation*}
This is known to split and thus, $F = \psi(P)\oplus\ker\Phi$ where $\psi: P\to F$ is the section.

\item[$(c)\Longrightarrow(d)$:] Let $M'\rightarrow M\rightarrow M''$ be an exact sequence of modules and $K$ be an $A$-module such that $P\oplus K = F\cong A^\Lambda$. Then, the induced sequence 
\begin{equation*}
    \prod_{\lambda\in\Lambda}M'\rightarrow\prod_{\lambda\in\Lambda}M\rightarrow\prod_{\lambda\in\Lambda}M''
\end{equation*}
is exact. We have seen that there is a natural isomorphism $\Hom_A(A,M)\stackrel{\sim}{\longrightarrow}M$, consequently, there is a natural isomorphism 
\begin{equation*}
    \Hom_A(A^{\oplus\Lambda}, M)\stackrel{\sim}{\longrightarrow}\prod_{\lambda\in\Lambda}M
\end{equation*}
whence it follows that the sequence 
\begin{equation*}
    \Hom_A(A^{\oplus\Lambda}A,M')\rightarrow\Hom_A(A^{\oplus\Lambda}A,M)\rightarrow\Hom_A(A^{\oplus\Lambda},M'')
\end{equation*}
But since $\Hom_A(A^{\oplus\Lambda},M)\cong\Hom_A(P,M)\oplus\Hom_A(K,M)$, we have the desired conclusion.

\item[$(d)\Longrightarrow(a)$:] Trivial.
\end{description}
\end{proof}

\begin{definition}[Projective Module]
    An $A$-module $P$ satisfying any one of the four equivalent conditions of \thref{thm:pojective-module-equivalence} is said to be a \textit{projective $A$-module}.
\end{definition}

In particular, from \thref{thm:pojective-module-equivalence}(c), we see that every free module is projective.

\begin{lemma}
    A finitely generated projective module $P$ over a local ring $(A,\mathfrak m)$ is free.
\end{lemma}
\begin{proof}
    Let $\{\overline x_1,\ldots,\overline x_n\}$ be a basis for $M/\mathfrak mM$ as a $k$-vector space where $k = A/\mathfrak m$. As we have seen earlier, $\{x_1,\ldots,x_n\}$ generates $M$. Let $F$ be the free module with basis $\{e_1,\ldots,e_n\}$ and $\Phi: F\to M$ be the module homomorphism given by $\Phi(e_i) = x_i$ and $K = \ker\Phi$. Since $M$ is projective, there is a module homomorphism $\psi: M\to F$ satisfying $\Phi\circ\psi = \mathbf{id}_M$ and $F = K\oplus\psi(M)$.

    We contend that $K = \mathfrak mK$. Indeed, let $x\in K$, then $x = \sum r_ie_i$ for a unique choice $\{r_1,\ldots,r_n\}$. Then, $\sum r_ix_i = 0$, consequently, $r_i\in\mathfrak m$ for all $i$. Since $F = K\oplus\psi(M)$, we may write $e_i = u_i + v_i$ for some $u_i\in K$ and $v_i\in\psi(M)$. As a result, 
    \begin{equation*}
        x - \sum r_iu_i = \sum r_iv_i\in\ker\Phi\cap\psi(M) = \{0\}
    \end{equation*}
    and the conclusion follows.

    Finally due to \thref{lem:nakayama}, we must have that $K = 0$ whence $M$ is free.
\end{proof}

\begin{proposition}
    Projective modules are flat.
\end{proposition}
\begin{proof}
    Follows from the fact that free modules are flat and projective modules are direct summands of free modules.
\end{proof}

\begin{theorem}
    Let $I$ denote the unit interval and $A = C(I)$, the ring of real valued continuous functions on $I$. Let $M$ denote the $A$-module of continuous functions that vanish in a neighborhood of zero. Then, $M$ is a projective $A$-module.
\end{theorem}
\begin{proof}
    Consider the functions $r_i: I\to\R$ given by 
    \begin{equation*}
        r_i(x) = 
        \begin{cases}
            0 & 0\le x\le\frac{1}{2^i}\\
            2^ix - 1 & \frac{1}{2^{i}}\le x\le\frac{1}{2^{i - 1}}\\
            1 & x\ge\frac{1}{2^{i - 1}}
        \end{cases}.
    \end{equation*}
    For any $f\in M$, there is a sufficiently large $i$ such that $r_i f = f$. Define the map $\Phi: M\to\bigoplus_{i = 1}^\infty A$ by 
    \begin{equation*}
        \Phi(f) = \left((1 - r_{i - 1})f\right)_{i\in\N}.
    \end{equation*}
    For sufficiently large $i$, $(1 - r_{i - 1})f = 0$. Now, define $\Psi:\bigoplus_{i = 1}^\infty A\to M$ by 
    \begin{equation*}
        \Psi((a_i)_{i\in\N}) = \sum_{i = 1}^\infty r_ia_i.
    \end{equation*}
    For $f\in M$, we have 
    \begin{equation*}
        \Psi\circ\Phi(f) = \sum_{i = 1}^\infty r_i(1 - r_{i - 1})f = \sum_{i = 1}^\infty (r_i - r_{i - 1})f = f.
    \end{equation*}
    The last equality follows from the fact the sum was essentially finite. In particular, this means that $M$ is a direct summand of a free module which completes the roof.
\end{proof}

\begin{remark}
    It is also true that $M$ is not a free $A$-module. I do not know of an elementary proof yet.
\end{remark}

\section{Injective Modules}

\begin{theorem}[Baer's Criterion]\thlabel{thm:baer}
    Let $Q$ be an $A$-module. Then $Q$ is injective if and only if for every ideal $\fraka$ of $A$, every $A$-module homomorphism $f:\fraka\to Q$ can be extended to an $A$-module homomorphism $\wt f: A\to Q$.
\end{theorem}
\begin{proof}
    $(\implies)$ Trivial.

    $(\impliedby)$ Let $M\subseteq N$ be $A$-modules. It suffices to show that every $A$-module homomorphism $f: M\to Q$ can be extended to an $A$-module homomorphism $f: N\to Q$. We shall first show that given $x\in N\backslash M$, the map $f$ can be extended to a map $f': M + (x)\to Q$. Indeed, let $\fraka = (M : x)$. Consider the map $g:\fraka\to Q$ given by $g(a) = f(ax)$. This is obviously an $A$-module homomorphism and according to the hypothesis, can be extended to an $A$-module homomorphism $\wt g: A\to Q$. Using this, we may define 
    \begin{equation*}
        f'(m + ax) = f(m) + \wt g(x)~\forall a\in A.
    \end{equation*}
    It is straightforward to check that this is an $A$-module homomorphism which extends $f$.

    Now, let $(\Sigma,\leqq)$ denote the poset of maps $\phi: M'\to Q$ where $M\le M'\le N$ are $A$-modules with the relation $\phi\leqq\psi$ if $\psi$ is an extension of $\phi$. It is not hard to argue that every chain in $\Sigma$ has an upper bound. Thus, due to Zorn, there is a maximal element $\wt f: M'\to Q$ for some $M\le M'\le N$. If $M'\ne N$, then by choosing some $x\in N\backslash M'$, we may extend the map $\wt f$ to a map from $M' + (x)$ to $Q$, a contradiction. This completes the proof.
\end{proof}

\begin{proposition}
    Let $R$ be a PID. Then, $M$ is an injective $R$-module if and only if it is divisible.
\end{proposition}
\begin{proof}
    Suppose $M$ is injective. Let $a\in A\backslash\{0\}$ and $x\in M$. Then, the map $f:(a)\to M$ which maps $a\mapsto x$ can be extended to a map from $A$ to $M$. If $f(1) = y$, then $ay = x$ whence $M$ is divisible.

    Conversely, if $M$ is divisible, then given any map $f: (a)\to M$, if $f(a) = x$, then there is $y\in M$ such that $ay = M$. Now, the map $\wt f: A\to M$ given by $f(1) = y$ extends $f$ whereby $M$ is injective. This completes the proof.
\end{proof}

\section{Algebras}

\begin{definition}
    An \textit{$A$-algebra} is a ring homomorphism $\phi: A\to B$. This endows $B$ with the structure of an $A$-module. The algebra is said to be of \textit{finite type} if $B$ is finitely generated as an $A$-module. A homomorphism between algebras $(\phi_1,B_1)$ and $(\phi_2,B_2)$ is a map $\varphi: B_1\to B_2$ making the following diagram commute.
    \begin{equation*}
        \xymatrix{
            A\ar[r]^{\phi_1}\ar[d]_{\phi_2} & B_2\\
            B_1\ar[ru]_{\varphi}
        }
    \end{equation*}
    This gives rise to a locally small category $A-\catAlg$ with morphisms as defined above.
\end{definition}

An $A$-algebra $B$ is said to be \textbf{finite} if it is finitely generated as an $A$-module. On the other hand, it is said to be \textbf{finitely generated} or of \textbf{finite type} if it is the homomorphic image of a polynomial ring $A[x_1,\ldots,x_n]$ for some positive integer $n$.

\begin{proposition}
    If $C$ is a finite $B$-algebra and $B$ is a finite $A$-algebra, then $C$ is a finite $A$-algebra.
\end{proposition}

\begin{proposition}
    If $C$ is a $B$-algebra of finite type and $B$ is an $A$-algebra of finite type, then $C$ is an $A$-algebra of finite type.
\end{proposition}
\begin{proof}
    There is a surjective ring homomorphism $\varphi: A[x_1,\ldots,x_n]\to B$ and a surjective ring homomorphism $\psi: B[y_1,\ldots,y_m]\to C$. It is not hard to see that there is a surjective ring homomorphism $\Phi: A[x_1,\ldots,x_n,y_1,\ldots,y_m]\to C$ thereby completing the proof.
\end{proof}

\subsection{Tensor Product of Algebras}

Consider the two $A$-algebras $f: A\to B$ and $f: A\to C$. Then, the map 
\begin{equation*}
    \mu: B\times C\times B\times C\to B\otimes_A C
\end{equation*}
given by $\mu(b,c,b',c') = bb'\otimes cc'$ is $A$-multilinear, whereby it induces a map 
\begin{equation*}
    \mu': B\otimes_A C\otimes_A B\otimes_A C\to B\otimes_A C 
\end{equation*}
given by $\mu'(b\otimes c\otimes b'\otimes c') = bb'\otimes cc'$. Let $D = B\otimes_A C$. Then, we have $\mu': D\otimes_A D\to D$ given by $\mu'(b\otimes c,b'\otimes c') = bb'\otimes cc'$.

Let $\varphi: D\times D\to D\otimes_A D$ be the natural map. Then, the composition $\cdot = \mu'\circ\varphi: D\times D\to D$ is given by 
\begin{equation*}
    (b\otimes c)\cdot(b'\otimes c') = bb'\otimes cc'
\end{equation*}

We contend that $(D\otimes_A D,+,\cdot,0\otimes 0, 1\otimes 1)$ is a ring. To do this, we need only verify that multiplication distributes over addition. Indeed, 
\begin{align*}
    (b\otimes c)\cdot(b'\otimes c' + b''\otimes c'') &= \mu'\left((b\otimes c)\otimes (b'\otimes c' + b''\otimes c'')\right)\\
    &= \mu'((b\otimes c\otimes b'\otimes c') + (b\otimes c\otimes b''\otimes c''))\\
    &= bb'\otimes cc' + bb''\otimes cc''
\end{align*}

Let $i: B\to B\otimes_A C$ be the map $b\mapsto b\otimes 1$ and $j: C\to B\otimes_A C$ be the map $c\mapsto 1\otimes c$. Then the square in the following diagram commutes.
\begin{equation*}
    \xymatrix {
        A\ar[r]^f\ar[d]_g & B\ar[d]^-i\ar@/^1.5pc/[rdd]^p\\
        C\ar@/_1.5pc/[drr]_q\ar[r]_-{j} & B\otimes_A C\ar@{.>}[rd]_{\exists!r}\\
        & & T
    }
\end{equation*} 

Let $T$ be an $A$-algebra with $A$-algebra morphisms $p: B\to T$ and $q: C\to T$. We contend that there is a unique $A$-algebra morphism $r: B\otimes_A C\to T$ making the above diagram commute. 

To construct the map $r$, consider the multilinear map $\Phi: B\times C\to T$ given by $\Phi(b, c) = p(b)q(c)$. This induces a map $r: B\otimes_A C\to T$ given by 
\begin{equation*}
    r(b\otimes c) = p(b)q(c).
\end{equation*}
This map obviously makes the diagram commute. It remains to show that $r$ is indeed an $A$-algebra morphism, for which, it suffices to show that it is a ring homomorphism. Indeed, 
\begin{equation*}
    r((b\otimes c)(b'\otimes c')) = r(bb'\otimes cc') = p(bb')q(cc') = p(b)q(c)p(b')q(c') = r(b\otimes c)r(b'\otimes c').
\end{equation*}

\textbf{Hence, the tensor tensor product is a \emph{coproduct} in the category of $A$-algebras.}


\section{Structure Theorem for Modules over a PID}
Throughout this section, let $R$ be a PID.

\begin{lemma}
    A finitely generated torsion free $R$-module is free.
\end{lemma}
\begin{proof}
    Let $M$ be a finitely generated torsion free $R$-module. Let $\{x_1,\dots,x_m\}$ be a set of generators. Pick a maximal linearly independent subset $\{v_1,\dots,v_n\}$ of $\{x_1,\dots, x_m\}$. Then, for each $x_i$, there is a linear combination 
    \begin{equation*}
        a_ix_i + b_1v_1 + \dots + b_nv_n = 0
    \end{equation*}
    with $a_i\ne 0$ due to the maximality of $\{v_1,\dots,v_n\}$. Therefore, $a_ix_i\in (v_1,\dots,v_n) = (v_1)\oplus\dots\oplus(v_n)$. Let $a = a_1\cdots a_m$. Then, the map $\phi: M\to (v_1,\dots, v_n)$ given by $\phi(m) = am$ is an injective map. Thus, $M$ is isomorphic to a submodule of $(v_1,\dots,v_n)$. But note that 
    \begin{equation*}
        (v_1,\dots,v_n) = (v_1)\oplus\dots\oplus(v_n)
    \end{equation*}
    is a free module and hence so is $M$.
\end{proof}

The statement is no longer true upon dropping the finitely generated condition, for example, $\Q$ as a $\Z$-module is not free but is torsion free.

\begin{definition}
    Let $E$ be an $R$-module. For $x\in E$, an element $r\in R$ such that $\Ann_R(x) = (r)$ is said to be a \textit{period} of $x$. An element $c\in R$ is said to be an \textit{exponent} for $E$ (resp. for $x$) if $cE = 0$ (resp. $cx = 0$). The elements $x_1,\ldots,x_n\in E$ are said to be \textit{independent} if 
    \begin{equation*}
        (x_i)\cap(x_1,\ldots,\widehat{x_i},\ldots,x_n) = 0
    \end{equation*}
    In this case, $(x_1,\ldots,x_n) = (x_1)\oplus\cdots\oplus(x_n)$.
\end{definition}

\begin{remark}
    In order to show that $x_1,\ldots,x_n$ are independent, it suffices to show that given any linear combination $a_1x_1 + \cdots + a_nx_n = 0$, we must have $a_ix_i = 0$ for all $1\le i\le n$. Further note that the notion of independence is not the same as that of linear independence. That is, we may have an independent set which is not linearly independent, for each element in the set may be torsion.
\end{remark}

The following lemma essentially states that it is possible to lift an independent set in a quotient module to the original module. 

\begin{lemma}[Lifting Lemma]\thlabel{lem:lifting-lemma-pid}
    Let $E$ be a torsion module with exponent $p^r$ for some prime $p\in R$ and $x_1\in E$ be an element of period $p^r$. Let $\overline E = E/(x_1)$ and $\overline y_1,\ldots,\overline y_m$ be independent elements of $\overline E$. Then for each $1\le i\le m$, there is a representative $y_i\in E$ of $\overline y_i$ such that the period of $y_i$ is same as the period of $\overline y_i$. Further, $x_1,y_1,\ldots,y_m$ are independent.
\end{lemma}
\begin{proof}
    Let $\overline y\in\overline E$, then, $\Ann(\overline y)\supseteq\Ann(\overline E)\supseteq(p^r)$ whereby, $\Ann(y) = (p^n)$ for some $n\le r$. Thus, $p^n y \in (x_1)$ whence there is $p^sc\in R$ with $p\nmid c$ such that $p^ny = p^sc x_1$. Now, $p^scx_1$ has period $p^{r - s}$ and thus $y$ has period $p^{n + r - s}$. This immediately implies that $n + r - s\le s$ and equivalently $n\le s$. Consider now the element $z = y - p^{s - n}cx_1$. This is a representative for $\overline y$ and its period is $p^n$. This shows that we may lift the $y_i$'s to $E$.

    Finally, we must show that the liftings are independent. Indeed, suppose 
    \begin{equation*}
        ax_1 + a_1y_1 + \cdots + a_my_m = 0
    \end{equation*}
    then moving to $\overline{E}$, we have $a_1\overline y_1 + \cdots + a_m\overline y_m = 0$ but since $\overline y_1,\ldots,\overline y_m$ are independent, $a_i\overline y_i = 0$ for each $1\le i\le m$. Now, if $p^{r_i}$ is the period of $\overline y_i$ (we have argued earlier that this must be a power of $p$) and consequently, $p^{r_i}\mid a_i$. This immediately implies that $a_iy_i = 0$ and thus $ax_1 = 0$, which completes the proof.
\end{proof}

Let $E$ be a finitely generated torsion module. For a prime $p\in R$, define
\begin{equation*}
    E[p] = \{x\in E\mid\exists n\in\N,~p^nx = 0\}
\end{equation*}
That this is a submodule is easy to verify. Further, it is finitely generated since it is the submodule of a finitely generated module over a PID. 

Let $\Ann(E) = (\alpha)$ where $\alpha = up_1^{t_1}\cdots p_r^{t_r}$ where $u\in R^\times$.

\begin{lemma}
    \begin{equation*}
        E\cong\bigoplus_{i = 1}^r E[p_i]
    \end{equation*}
\end{lemma}
\begin{proof}
    Let $q_i = \alpha/p_i^{t_i}$. Then, $(q_1,\dots,q_r) = 1$ and hence, there are $\gamma_1,\dots,\gamma_r\in R$ such that $\gamma_1q_1 + \dots + \gamma_rq_r = 1$. For any $x\in E$, we have 
    \begin{equation*}
        x = \gamma_1q_1 x + \dots + \gamma_rq_rx
    \end{equation*}
    where $\gamma_iq_ix\in E[p_i]$ for each $i$. Thus, $E = \sum_{i = 1}^rE[p_i]$.

    We shall now show that this sum is direct. Indeed, suppose $x_i\in E[p_i]$ such that $x_1 + \dots + x_r = 0$. Multiplying this equation by $q_i$, we have $q_ix_i = 0$, consequently, $\Ann(x_i)\supseteq(p_i^{t_i}, q_i) = (1)$, that is, $x_i = 0$ for each $i$. This completes the proof.
\end{proof}

Since $E[p]$ is finitely generated, we may let $E = E[p]$ henceforth. Since $E[p]$ is finitely generated, take a generating set $\{x_1,\ldots,x_n\}$. Since $(p^m)\subseteq\Ann(x_i)$ for some $m\in\N$, we must have $\Ann(x_i) = (p^{n_i})$ for some $n_i$. As a result, 
\begin{equation*}
    \Ann(E)\supseteq\bigcap_{i = 1}^r\Ann(x_i)\ne 0
\end{equation*}
whence $\Ann(E) = (p^{n})$ for some positive integer $n$. We shall now show that $E$ has a decomposition. Let $\mathfrak M(E)$ denote the minimum cardinality of a generating set of $E$. Obviously this exists since $E$ has at least one generating set.

Let $x_1\in E$ be an element in a generating set with cardinality $\mathfrak M(E)$ such that $\Ann(x_1)$ divides the annihilator ideal of every other element in the aforementioned generating set. This can be done because the generating set has finite cardinality. 

Let $\overline E = E/(x_1)$. Obviously, $\frakM(\overline E) < \frakM(E)$ whereby, there is a decomposition $\overline E\cong (\overline y_1)\oplus\cdots\oplus(\overline y_m)$ with $(\overline y_i)\cong R/(p^{r_i})$. Due to the \thref{lem:lifting-lemma-pid}, there are corresponding elements $y_1,\ldots,y_m\in E$ such that the period of $y_i$ is that of $\overline y_i$, and $x_1,y_1,\ldots,y_m$ are independent. This shows that the following short exact sequence spilts: 
\begin{equation*}
    0\rightarrow(x_1)\rightarrow E\rightarrow\overline E\rightarrow 0
\end{equation*}
whence $E\cong(x_1)\oplus(y_1)\oplus\cdots\oplus(y_m)$. This completes the proof of the existence of a decomposition.

\subsection{The Jordan Canonical Form}

Let $k$ be an algebraically closed field 

\section{Finitely Presented Modules}

I need to place this section somewhere nice.

\begin{definition}[Finitely Presented]
    An $A$-module $M$ is said to be finitely presented if there are positive integers $m$ and $n$ and an exact sequence $A^m\rightarrow A^n\rightarrow M\rightarrow 0$.
\end{definition}

Obviously, every finitely presented module is finitely generated. Further, if $A$ is a \textit{noethering}, then an $A$-module is finitely generated if and only if it is finitely presented.

\begin{proposition}
    If $M$ is finitely presented, then for every $A$-module $N$ and multiplicative subset $S$ of $A$, 
    \begin{equation*}
        S^{-1}\Hom_A(M,N)\cong\Hom_{S^{-1}A}(S^{-1}M,S^{-1}N)
    \end{equation*}
\end{proposition}
\begin{proof}
    There is a natural map $T:S^{-1}\Hom_A(M,N)\to\Hom_{S^{-1}A}(S^{-1}M,S^{-1}N)$ given by $(\phi/s)(m/t) = \phi(m)/st$. We shall show that this map is an isomorphism when $M$ is finitely presented. To do so, we must first show that this is an isomorphism when $M = A^n$ for some positive integer $n$. Indeed, since localization commutes with direct sums, we have 
    \begin{equation*}
        S^{-1}\Hom_A(A^{\oplus n}, N)\cong S^{-1}\prod\Hom_A(A,N)\cong\prod\Hom_{S^{-1}A}(S^{-1}A, S^{-1}N)\cong\Hom_{S^{-1}A}(S^{-1}A^{\oplus n}, N).
    \end{equation*}
    Since $M$ is finitely presented, we have an exact sequence $A^m\rightarrow A^n\rightarrow M\rightarrow 0$ for some positive integers $m$ and $n$. We have a commutative diagram. 
    \begin{equation*}
        \xymatrix{
            0\ar[r]\ar@{=}[d] & S^{-1}\Hom_A(M,N)\ar[r]\ar[d] & S^{-1}\Hom_A(A^n, N)\ar[r]\ar[d]_\sim & S^{-1}\Hom_A(A^m,N)\ar[d]_\sim\\
            0\ar[r] & \Hom_{S^{-1}A}(S^{-1}M,S^{-1}N)\ar[r] & \Hom_{S^{-1}A}(S^{-1}A^n, S^{-1}N)\ar[r] & \Hom_{S^{-1}A}(S^{-1}A^m,S^{-1}N)\\
        }
    \end{equation*}
    and the conclusion follows from the five lemma (just add another column of zeros to the left).
\end{proof}

\begin{proposition}\thlabel{prop:ker-fg-fp}
    Let $N$ be finitely generated and $M$ a finitely presented $A$-module. If $f: N\onto M$ is a surjection, then $\ker f$ is finitely generated.
\end{proposition}
\begin{proof}
    Let $A^m\to A^n\to M\to 0$ be an exact sequence. Then, there is a commutative diagram
    \begin{equation*}
        \xymatrix {
            & A^m\ar[r]\ar@{.>}[d]_{\exists h} & A^n\ar[r]\ar@{.>}[d]_{\exists g} & M\ar[r]\ar@{=}[d]^{\id_M} & 0\\
            0\ar[r] & \ker f\ar[r] & N\ar[r]_f & M\ar[r] & 0
        }
    \end{equation*}
    with exact rows. Since $A^m$ and $A^n$ are projective $A$-modules, the map $\id_M$ can be lifted to maps $g: A^n\to N$ and $h: A^m\to\ker f$. Due to the Snake Lemma, there is an exact sequence 
    \begin{equation*}
        0 = \ker\id_M\to\coker h\to\coker g\to\coker\id_M = 0
    \end{equation*}
    whence $\coker h\cong\coker g$. But since $N$ is finitely generated, so is $\coker g$ and hence so is $\coker h$. Finally, we have an exact sequence 
    \begin{equation*}
        A^m\to\ker f\to\coker h
    \end{equation*}
    where $A^m$ and $\coker h$ are finitely generated. Thus, $\ker f$ is finitely generated.\todo{cite 2 out of 3 lemma}
\end{proof}