\section{Chain Conditions}

A totally ordered sequence $\{x_n\}_{n = 1}^\infty$ in the poset $(\Sigma,\leqq)$ is said to be \textit{stationary} if there is an index $n$ such that $x_n = x_{n + 1} = \cdots$.

\begin{definition}
    An $A$-module $M$ is said to be \textit{noetherian} or equivalently said to satisfy the \textit{ascending chain condition} if every chain in the poset of submodules of $M$ ordered by $\subseteq$ is stationary.

    Similarly, $M$ is said to be \textit{artinian} equivalently said to satisfy the \textit{descending chain condition} if every chain in the poset of submodules of $M$ ordered by $\supseteq$ is stationary.
\end{definition}

A ring $A$ is said to be noetherian (resp. artinian) if it is noetherian (resp. artinian) as an $A$-module.

\begin{proposition}
    Let $(\Sigma,\leqq)$ be a poset. Then, the following are equivalent: 
    \begin{enumerate}[label=(\alph*)]
        \item Every chain in $\Sigma$ is stationary.
        \item Every subset of $\Sigma$ has a maximal element.
    \end{enumerate}
\end{proposition}
The proof is omitted owing to its triviality.

\begin{lemma}
    An $A$-module $M$ is noetherian if and only if every submodule is finitely generated.
\end{lemma}
\begin{proof}
    
\end{proof}
\begin{corollary}
    A ring $A$ is noetherian if and only if every ideal is finitely generated.
\end{corollary}
\begin{corollary}
    Every submoule of a noetherian $A$-module is noetherian.
\end{corollary}

\begin{proposition}
    $M$ is a noetherian (resp. artinian) $A$-module if and only if it is a noetherian (resp. artinian) $A/\Ann_A(M)$-module.
\end{proposition}
\begin{proof}
    Since the poset of $A/\Ann_A(M)$-submodules of $M$ is the same as the poset of $A$-submodules of $M$, the conclusion follows.
\end{proof}

\begin{lemma}[2/3-lemma]
    Consider the short exact sequence $0\rightarrow M'\rightarrow M\rightarrow M''\rightarrow 0$. Then $M$ is noetherian (resp. artinian) if and only if both $M'$ and $M''$ are noetherian (resp. artinian).
\end{lemma}
\begin{proof}
    
\end{proof}
\begin{corollary}
    Let $\{M_i\}_{i = 1}^n$ be $A$-modules. Then, $\displaystyle\bigoplus_{i = 1}^n M_i$ is noetherian (resp. artinian) if and only if each $M_i$ is noetherian (resp. artinian).
\end{corollary}
\begin{proof}
    The forward direction is obvious. For the converse, induct on $n$ using the short exact sequence: 
    \begin{equation*}
        0\longrightarrow M_n\longrightarrow\bigoplus_{i = 1}^n M_i\longrightarrow\bigoplus_{i = 1}^{n - 1}M_i\longrightarrow 0
    \end{equation*}
\end{proof}

\begin{proposition}
    If $A$ is a noethering (resp. artinian ring), then so is $A/\fraka$ for any ideal $\fraka$ in $A$.
\end{proposition}
\begin{proof}
    $A/\fraka$ is a noetherian (resp. artinian) $A$-module and thus a noetherian (resp. artinian) $A/\fraka$-module.
\end{proof}

\begin{proposition}
    Let $M$ be an $A$-module and $\phi\in\End_A(M)$. 
    \begin{enumerate}[label=(\alph*)]
        \item If $M$ is noetherian and $\phi$ is surjective, then $\phi$ is injective.
        \item If $M$ is artinian and $\phi$ is injective, then $\phi$ is surjective.
    \end{enumerate}
\end{proposition}
\begin{proof}
\begin{enumerate}[label=(\alph*)]
\item Consider the ascending chain of submodules
\begin{equation*}
    \ker\phi\subseteq\ker\phi^2\subseteq\cdots
\end{equation*}
Since $M$ is noetherian, there is an index $n$ such that $\ker\phi^n = \ker\phi^{n + 1}$. Let $x\in\ker\phi^n$. Due to the surjectivity of $\phi$, there is $y\in M$ such that $\phi(y) = x$, whence $\phi^{n + 1}(y) = 0$ and $y\in\ker\phi^{n + 1} = \ker\phi^n$. Therefore, $\ker\phi^n = 0$ and $\phi$ is injective.

\item Consider the descending chain of submodules
\begin{equation*}
    \im\phi\supseteq\im\phi^2\supseteq\cdots
\end{equation*}
Since $M$ is artinian, there is an index $n$ such that $\im\phi^n = \im\phi^{n + 1}$. Then, for every $x\in M$, there is $y\in M$ such that $\phi^n(x) = \phi^{n + 1}(y)$, whence $x = \phi(y)$, this establishes surjectivity.
\end{enumerate}
\end{proof}

\begin{lemma}
    Supose there is a sequence of maximal ideals $\frakm_1,\ldots,\frakm_n$ in $A$ such that $(0) = \frakm_1\cdots\frakm_n$. Then, $A$ is a noethering if and only if it is artinian.
\end{lemma}
\begin{proof}
    Suppose $A$ is a noethering. Consider the $A/\frakm_i$ module $\frakm_1\cdots\frakm_{i - 1}/\frakm_1\cdots\frakm_i$
\end{proof}

\section{Noetherian Rings}

Recall that $A$ is a noetherian ring if it is a noetherian $A$-module.

\begin{lemma}
    If $A$ is Noetherian and $\phi: A\to B$ is a surjective ring homomorphism, then $B$ is also Noetherian.
\end{lemma}
\begin{proof}
    Since $B\cong A/\ker\phi$, the conclusion follows.
\end{proof}

\begin{proposition}
    If $A$ is a noethering and $S\subseteq A$ is a multiplicative subset, then $S^{-1}A$ is a noethering.
\end{proposition}
\begin{proof}
    Recall that every ideal in $S^{-1}A$ is finitely generated. Let $I\subseteq S^{-1}A$ be an ideal then there is $\fraka\subseteq A$ an ideal such that $S^{-1}\fraka = I$. Since $A$ is noetherian, $\fraka$ is generated by a finite set $\{x_1,\ldots,x_n\}$, whereby $I$ is generated by the set $\{x_1/1,\ldots,x_n/1\}$. This completes the proof.
\end{proof}

But recall, as we have seen earlier, that being a noethering is not a local property, a counterexample to which is an infinite product of fields.

\begin{theorem}[Hilbert Basis Theorem]
    If $A$ is Noetherian, then so is $A[x]$.
\end{theorem}
Note that the converse is also true since $A\cong A[x]/(x)$. The following proof is due to Sarges.
\begin{proof}
    We shall show that every ideal in $A[x]$ is finitely generated. Suppose not and let $I\subseteq A[x]$ be an ideal that is not finitely generated. Choose $f_1\in I$ with minimum degree. Now, inductively, choose $f_{k + 1}\in I\backslash(f_1,\ldots,f_k)$ with the minimum degree. Obviously, this process goes on indefinitely, since we have assumed $I$ to not be finitely generated. We now have 
    \begin{align*}
        f_1 &= a_1x^{d_1} + \text{lower degree terms}\\
        f_2 &= a_2x^{d_2} + \text{lower degree terms}\\
        &\vdots\\
        f_n &= a_nx^{d_n} + \text{lower degree terms}\\
        &\vdots
    \end{align*}
    with $d_1\le d_2\le\cdots$. We also have the following ascending chain of ideals in $A$, 
    \begin{equation*}
        (a_1)\subseteq(a_1,a_2)\subseteq\cdots
    \end{equation*}
    Therefore, there is $n\in\N$ such that $(a_1,\ldots,a_n) = (a_1,\ldots,a_n,a_{n + 1})$. Consequently, we may write $a_{n + 1}$ as a linear combination of $a_1,\ldots,a_n$, say 
    \begin{equation*}
        a_{n + 1} = b_1a_1 + \cdots + b_na_n
    \end{equation*}
    for some $b_1,\ldots,b_n\in A$. Let 
    \begin{equation*}
        g = f_{n + 1} - (b_1x^{d_{n + 1} - d_1}f_1 + \cdots + b_nx^{d_{n + 1} - d_n}f_n)
    \end{equation*}
    It is not hard to argue that $g\in I\backslash(f_1,\ldots,f_n)$, but $\deg g\le\deg f_{n + 1}$, a contradiction. This completes the proof.
\end{proof}

An analogous theorem, with an analogous proof is true wherein $A[x]$ is replaced by $A\llbracket x\rrbracket$.

\begin{corollary}
    For a field $k$, the polynomial ring $k[x_1,\ldots,x_n]$ in finitely many indeterminates is noetherian.
\end{corollary}

\begin{corollary}
    If $A$ is a noethering, then every $A$-algebra of finite type is a noethering.
\end{corollary}

If $A\subseteq B$ is a ring extension with both $A$ and $B$ noetherian, it is not necessary that $B$ is an $A$-algebra of finite type. Indeed, consider $\overline\Q/\Q$ an extension of fields.

On the other hand, even if $B$ is an $A$-algebra of finite type and noetherian, it is not necessary for $A$ to be noetherian. Indeed, consider the ring inclusion 
\begin{equation*}
    k[xy,xy^2,\ldots]\subsetneq k[x,y]
\end{equation*}
The former is not noetherian owing to the chain of ideals 
\begin{equation*}
    (xy)\subsetneq(xy,xy^2)\subsetneq\cdots
\end{equation*}
while the latter obviously is noetherian.

\begin{proposition}
    Let $M$ be a noetherian $A$-module. Then, $A/\Ann_A(M)$ is a noethering.
\end{proposition}
\begin{proof}
    Since $M$ is noetherian, it is finitely generated. Let $\{m_1,\ldots,m_n\}$ be a set of generators. Then, $\displaystyle\Ann_A(M) = \bigcap_{i = 1}^n\Ann_M(m_i)$. Consider the map $\phi: A\to M^n$ given by $\phi(a) = (am_1,\ldots,am_n)$. Note that $\ker\phi = \Ann_A(M)$. Thus, we have a short exact sequence 
    \begin{equation*}
        0\longrightarrow A/\Ann_A(M)\longrightarrow A\longrightarrow\phi(A)\longrightarrow 0.
    \end{equation*}
    Consequently, $A/\Ann_A(M)$ is a noetherian $A$-module and thus a noetherian $A/\Ann_A(M)$-module, whence a noethering.
\end{proof}

\begin{lemma}[Artin-Tate Lemma]\thlabel{lem:artin-tate}
    Let $A\subseteq B\subseteq C$ be rings with $A$ noetherian, and $C$ an $A$-algebra of finite type. If either 
    \begin{enumerate}[label=(\alph*)]
        \item $C$ is a finite $B$-algebra\footnote{Recall that this is the same as being finitely generated as a $B$-module}, or 
        \item $C$ is integral over $B$,
    \end{enumerate}
    then $B$ is an $A$-algebra of finite type.
\end{lemma}
\begin{proof}
    Note that $(a)\iff(b)$ due to \thref{thm:equivalence-integral-extension}. We shall show that $(a)$ implies the desired conclusion. Since $C$ is an $A$-algebra of finite type, say it is generated by $\{x_1,\ldots,x_n\}$ as an $A$-algebra. Similarly, since it is a finite $B$-algebra, it is finitely generated as a $B$-module, say by $\{y_1,\ldots,y_m\}$. Therefore, there are coefficients $b_{ij}$ and $b_{ijk}$ in $B$ such that 
    \begin{align*}
        x_i &= \sum_{j = 1}^m b_{ij}y_j\\
        y_iy_j &= \sum_{k = 1}^m b_{ijk}y_k.
    \end{align*}
    Let $B_0 = A[\{b_{ij}\}\cup\{b_{ijk}\}]\subseteq B$. Since $A$ is noetherian, and $B_0$ is an $A$-algebra of finite type, it is a noethering. 
    
    Now, since $C$ is a finite type $A$-algebra, every element of $C$ is a polynomial in the $x_i$'s with coefficients in $A$. Using the first set of relations, it is a polynomial in the $y_i$'s with coefficients in $B_0$. Using the second set of relations, it is a linear combination of the $y_i$'s with coefficients in $B_0$, whereby $C$ is a finite $B_0$-algebra.

    Since $C$ is a finitely generated $B_0$-module it is noetherian and thus $B$, being a $B_0$-submodule, is a finitely generated $B_0$-module and consequently, a $B_0$-algebra of finite type. Thus, $B$ is an $A$-algebra of finite type.
\end{proof}

\begin{lemma}[Cohen]
    $A$ is a noethering if and only if every prime ideal in $A$ is finitely generated.
\end{lemma}
\begin{proof}
    We shall prove the converse. Let $\Sigma$ be the poset of proper ideals that are not finitely generated, which we suppose is nonempty. If $\mathscr C$ is a chain in $\Sigma$, then $I = \bigcup_{\fraka\in\mathscr C}\fraka$ may not be finitely generated for if it were, then there is a set of generators $\{r_1,\ldots,r_n\}$ and thus there would exist $\fraka\in\mathscr C$ containing $\{r_1,\ldots,r_n\}$ whereby equal to $I$, contradiction. Hence, $I$ is an upper bound for $\mathscr C$ and due to Zorn's Lemma, there is a maximal element $\frakp\in\Sigma$.

    Since $\frakp$ may not be prime, there are $x,y\notin\frakp$ such that $xy\in\frakp$. Consider $\frakp + (x)$. This strictly contains $\frakp$ and therefore, is finitely generated. The generators of $\frakp + (x)$ are of the form $p_i + a_ix$ for $1\le i\le n$ for some positive integer $n$. 

    Consider the ideal $(\frakp: x)$. This contains $\frakp + (y)$ which strictly contains $\frakp$ an thus, is finitely generated. Say $(\frakp: x) = (x_1,\ldots,x_m)$ for some positive integer $m$. Let $\fraka = (p_1,\ldots,p_n,xx_1,\ldots,xx_m)$. We contend that $\fraka = \frakp$.

    Obviously, $\fraka\subseteq\frakp$. On the other hand, for any $p\in\frakp$, there is a representation 
    \begin{equation*}
        p + x = b_1p_1 + \cdots + b_np_n + cx
    \end{equation*}
    for some $b_1,\ldots,b_n,c\in A$, consequently, $p\in\fraka$. Thus, $\fraka = \frakp$, which is a contradiction to the choice of $\frakp$. Hence, $\Sigma$ is empty and $A$ is a noethering.
\end{proof}

\subsection{Primary Decomposition}

\begin{definition}[Irreducible]
    An ideal $\fraka\subseteq A$ is said to be \textit{irreducible} if for all ideals $\frakb,\frakc\subseteq A$,
    \begin{equation*}
        \fraka = \frakb\cap\frakc\Longrightarrow \fraka = \frakb\text{ or }\fraka = \frakc
    \end{equation*}
\end{definition}

\begin{lemma}
    In a noethering, every ideal can be expressed as a finite intersection of irreducible ideals.
\end{lemma}
\begin{proof}
    Let $\Sigma$ be the poset of ideals that cannot be expressed as a finite intersection of irreducible ideals in $A$. Suppose $\Sigma$ is nonempty, then every chain in $\Sigma$ is finite (owing to noetherian-ness) whence has an upper bound, thus $\Sigma$ has a maximal element (Zorn's Lemma), say $\fraka$. Note that $\fraka$ cannot be irreducible, therefore, there are ideals $\frakb,\frakc$ properly containing $\fraka$ such that $\fraka = \frakb\cap\frakc$. Due to the maximality of $\fraka$, both $\frakb$ and $\frakc$ can be expressed as a finite intersection of irreducible ideals in $A$, as a result, so can $\fraka$, a contradiction. Thus $\Sigma$ must be empty and the proof is complete.
\end{proof}

\begin{lemma}
    Every irreducible ideal in a noethering is primary.
\end{lemma}
\begin{proof}
    Let $\frakq\subseteq A$ be an irreducible ideal. We shall show that $(0)$ is primary in $A/\frakq$, which is equivalent to $\frakq$ being primary. Let $x,y\in A/\frakq$ such that $xy = 0$. If $x\ne 0$, then consider the chain 
    \begin{equation*}
        \Ann(y)\subseteq\Ann(y^2)\subseteq\cdots
    \end{equation*}
    Since $A/\frakq$ is a noethering, there is a positive integer $n$ such that $\Ann(y^n) = \Ann(y^{n + 1})$. We contend that $(x)\cap(y^n) = 0$. Indeed, if $z\in(x)\cap(y^n)$, then there are $u,v\in A/\frakq$ such that $z = ux = vy^n$. Then,
    \begin{equation*}
        vy^{n + 1} = zy = uxy = 0
    \end{equation*}
    whence $v\in\Ann(y^{n + 1}) = \Ann(y^n)$, whereby $z = 0$. But since $(0)$ is irreducible and $x\ne 0$, we must have $y^n = 0$ and $(0)$ is primary. This completes the proof.
\end{proof}

\begin{corollary}
    A noethering has finitely many minimal prime ideals.
\end{corollary}
\begin{proof}
    Since $A$ is noetherian, the ideal $(0)$ has a primary decomposition and the minimal primes belonging to $(0)$ are precisely the minimal primes in $A$ and thus are finite.
\end{proof}


\section{Artinian Rings}

Recall that $A$ is artinian if it is an artinian module over itself.

\begin{proposition}
    Let $A$ be an artinian ring. Then $A$ has finitely many maximal ideals.
\end{proposition}
\begin{proof}
    Suppose not. Then, we have a sequence $\{\frakm_i\}_{i = 1}^\infty$ of pairwise distinct maximal ideals. Consider the sequence of ideals $\{\frakm_1\cdots\frakm_n\}_{n = 1}^\infty$. We contend that the inclusion $\frakm_1\cdots\frakm_{n - 1}\supseteq\frakm_1\cdots\frakm_n$ is strict. Indeed, for all $1\le i\le n - 1$, pick $x_i\in\frakm_i\backslash\frakm_n$. Then, $x_1\cdots x_{n - 1}\notin\frakm_n$, since $A\backslash\frakm_n$ is a multiplicatively closed subset. Thus, $x_1\cdots x_{n - 1}\in\frakm_1\cdots\frakm_{n - 1}\backslash\frakm_1\cdots\frakm_n$. This is a contradiction to $A$ being artinian.
\end{proof}

\begin{proposition}
    Let $A$ be an artinian ring. Then every prime ideal in $A$ is maximal.
\end{proposition}
\begin{proof}
    Let $\frakp$ be a prime ideal in $A$. Then $A' = A/\frakp$ is an Artinian integral domain. We shall show that this is a field, for which it suffices to show that every element is invertible. Choose $x'\in A'$ and let $\phi: A'\to A'$ be the $A'$-module homomorphism that maps $a\mapsto x'a$. Since $A'$ is an integral domain, this map is injective and since $A'$ is artinian, it is also an isomorphism. Consequently, there is some $y'\in A'$ such that $x'y' = 1$ and the conclusion follows.
\end{proof}

\begin{corollary}
    Let $A$ be an artinian ring. Then $\frakN(A) = \frakR(A)$.
\end{corollary}

\begin{lemma}
    Let $A$ be an artinian ring. Then $\frakN(A)$ is nilpotent.
\end{lemma}
\begin{proof}
    We shall denote $\frakN(A)$ by $\frakN$ for the sake of brevity. Consider the decreasing chain 
    \begin{equation*}
        \frakN\supseteq\frakN^2\supseteq\cdots
    \end{equation*}
    Then there is an index $n$ such that $\fraka = \frakN^n = \frakN^{n + 1} = \cdots$. Suppose for the sake of contradiction that $\fraka\ne 0$. Let $\Sigma$ be the set of ideals $\frakb$ such that $\fraka\frakb\ne 0$. Obviously $\Sigma$ is empty since it contains $\fraka$. Since $A$ is artinian, $\Sigma$ has a minimal element $\frakc$\footnote{We have not invoked Zorn to conclude this.}. 

    We contend that $\frakc$ is principal. Indeed, there is an element $x\in\frakc$ such that $x\fraka\ne 0$. Thus, $(x)\fraka\ne 0$. Owing to the minimality of $\frakc$, we must have $\frakc = (x)$. 

    Consider now the ideal $(x)\fraka$. This is a subset of $(x)$ and 
    \begin{equation*}
        ((x)\fraka)\fraka^k = (x)\fraka^{k + 1} = (x)\fraka\ne0
    \end{equation*}
    whence $(x)\fraka\in\Sigma$ and again, owing to the minimality of $(x) = \frakc$, we have $(x)\fraka = (x)$. Hence, there is some $y\in\fraka$ such that $xy = x$. We now have 
    \begin{equation*}
        x = xy = xy^2 = \cdots
    \end{equation*}
    Since $y\in\fraka\subseteq\frakN$, it is nilpotent, whence $x = 0$, a contradiction. Thus $\fraka = 0$ and this completes the proof.
\end{proof}

\begin{theorem}
    $A$ is artinian if and only if it is a noethering with krull dimension zero.
\end{theorem}
\begin{proof}
\end{proof}