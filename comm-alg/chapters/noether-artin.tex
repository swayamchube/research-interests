\section{Chain Conditions}

\begin{proposition}
    Let $M$ be an $A$-module and $\phi\in\End_A(M)$. 
    \begin{enumerate}[label=(\alph*)]
        \item If $M$ is noetherian and $\phi$ is surjective, then $\phi$ is injective.
        \item If $M$ is artinian and $\phi$ is injective, then $\phi$ is surjective.
    \end{enumerate}
\end{proposition}
\begin{proof}
\begin{enumerate}[label=(\alph*)]
\item Consider the ascending chain of submodules
\begin{equation*}
    \ker\phi\subseteq\ker\phi^2\subseteq\cdots
\end{equation*}
Since $M$ is noetherian, there is an index $n$ such that $\ker\phi^n = \ker\phi^{n + 1}$. Let $x\in\ker\phi^n$. Due to the surjectivity of $\phi$, there is $y\in M$ such that $\phi(y) = x$, whence $\phi^{n + 1}(y) = 0$ and $y\in\ker\phi^{n + 1} = \ker\phi^n$. Therefore, $\ker\phi^n = 0$ and $\phi$ is injective.

\item Consider the descending chain of submodules
\begin{equation*}
    \im\phi\supseteq\im\phi^2\supseteq\cdots
\end{equation*}
Since $M$ is artinian, there is an index $n$ such that $\im\phi^n = \im\phi^{n + 1}$. Then, for every $x\in M$, there is $y\in M$ such that $\phi^n(x) = \phi^{n + 1}(y)$, whence $x = \phi(y)$, this establishes surjectivity.
\end{enumerate}
\end{proof}

\section{Noetherian Rings}

\begin{lemma}
    If $A$ is Noetherian and $\phi: A\to B$ is a surjective ring homomorphism, then $B$ is also Noetherian.
\end{lemma}

\begin{theorem}[Hilbert Basis Theorem]
    If $A$ is Noetherian, then so is $A[x]$.
\end{theorem}
Note that the converse is also true since $A\cong A[x]/(x)$. The following proof is due to Sarges.
\begin{proof}
    We shall show that every ideal in $A[x]$ is finitely generated. Suppose not and let $I\subseteq A[x]$ be an ideal that is not finitely generated. Choose $f_1\in I$ with minimum degree. Now, inductively, choose $f_{k + 1}\in I\backslash(f_1,\ldots,f_k)$ with the minimum degree. Obviously, this process goes on indefinitely, since we have assumed $I$ to not be finitely generated. We now have 
    \begin{align*}
        f_1 &= a_1x^{d_1} + \text{lower degree terms}\\
        f_2 &= a_2x^{d_2} + \text{lower degree terms}\\
        &\vdots\\
        f_n &= a_nx^{d_n} + \text{lower degree terms}\\
        &\vdots
    \end{align*}
    with $d_1\le d_2\le\cdots$. We also have the following ascending chain of ideals in $A$, 
    \begin{equation*}
        (a_1)\subseteq(a_1,a_2)\subseteq\cdots
    \end{equation*}
    Therefore, there is $n\in\N$ such that $(a_1,\ldots,a_n) = (a_1,\ldots,a_n,a_{n + 1})$. Consequently, we may write $a_{n + 1}$ as a linear combination of $a_1,\ldots,a_n$, say 
    \begin{equation*}
        a_{n + 1} = b_1a_1 + \cdots + b_na_n
    \end{equation*}
    for some $b_1,\ldots,b_n\in A$. Let 
    \begin{equation*}
        g = f_{n + 1} - (b_1x^{d_{n + 1} - d_1}f_1 + \cdots + b_nx^{d_{n + 1} - d_n}f_n)
    \end{equation*}
    It is not hard to argue that $g\in I\backslash(f_1,\ldots,f_n)$, but $\deg g\le\deg f_{n + 1}$, a contradiction. This completes the proof.
\end{proof}

An analogous theorem, with an analogous proof is true wherein $A[x]$ is replaced by $A\llbracket x\rrbracket$.

\subsection{Primary Decomposition}

\begin{definition}[Irreducible]
    An ideal $\fraka\subseteq A$ is said to be \textit{irreducible} if for all ideals $\frakb,\frakc\subseteq A$,
    \begin{equation*}
        \fraka = \frakb\cap\frakc\Longrightarrow \fraka = \frakb\text{ or }\fraka = \frakc
    \end{equation*}
\end{definition}

\begin{lemma}
    In a noethering, every ideal can be expressed as a finite intersection of irreducible ideals.
\end{lemma}
\begin{proof}
    Let $\Sigma$ be the poset of ideals that cannot be expressed as a finite intersection of irreducible ideals in $A$. Suppose $\Sigma$ is nonempty, then every chain in $\Sigma$ is finite (owing to noetherian-ness) whence has an upper bound, thus $\Sigma$ has a maximal element (Zorn's Lemma), say $\fraka$. Note that $\fraka$ cannot be irreducible, therefore, there are ideals $\frakb,\frakc$ properly containing $\fraka$ such that $\fraka = \frakb\cap\frakc$. Due to the maximality of $\fraka$, both $\frakb$ and $\frakc$ can be expressed as a finite intersection of irreducible ideals in $A$, as a result, so can $\fraka$, a contradiction. Thus $\Sigma$ must be empty and the proof is complete.
\end{proof}

\begin{lemma}
    Every irreducible ideal in a noethering is primary.
\end{lemma}
\begin{proof}
    Let $\frakq\subseteq A$ be an irreducible ideal. We shall show that $(0)$ is primary in $A/\frakq$, which is equivalent to $\frakq$ being primary. Let $x,y\in A/\frakq$ such that $xy = 0$. If $x\ne 0$, then consider the chain 
    \begin{equation*}
        \Ann(y)\subseteq\Ann(y^2)\subseteq\cdots
    \end{equation*}
    Since $A/\frakq$ is a noethering, there is a positive integer $n$ such that $\Ann(y^n) = \Ann(y^{n + 1})$. We contend that $(x)\cap(y^n) = 0$. Indeed, if $z\in(x)\cap(y^n)$, then there are $u,v\in A/\frakq$ such that $z = ux = vy^n$. Then,
    \begin{equation*}
        vy^{n + 1} = zy = uxy = 0
    \end{equation*}
    whence $v\in\Ann(y^{n + 1}) = \Ann(y^n)$, whereby $z = 0$. But since $(0)$ is irreducible and $x\ne 0$, we must have $y^n = 0$ and $(0)$ is primary. This completes the proof.
\end{proof}

\section{Artinian Rings}
