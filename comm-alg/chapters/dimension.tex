\section{Hilbert Polynomial}

Throughout this section, let $A = \bigoplus_{n = 0}^\infty A_n$ be a noetherian graded ring and $\lambda$ a $\Z$-valued additive function on the category of $A$-modules.

\begin{definition}[Euler-Poincar\'e Series]
    Associate with each $A$-module $M$, the \emph{Euler-Poincar\'e} series, 
    \begin{equation*}
        P_\lambda(M, t) = \sum_{n = 0}^\infty\lambda(M_n)t^n\in\Z\llbracket t\rrbracket.
    \end{equation*}
    We often write $P(M, t)$ for $P_\lambda(M, t)$ when the additive function is clear.
\end{definition}

\begin{theorem}[Hilbert-Serre]
    $P(M, t)$ is a rational function in $t$ of the form 
    \begin{equation*}
        P(M, t) = \frac{f(t)}{\prod_{i = 1}^n (1 - t^{n_i})}
    \end{equation*}
    where $f(t)\in\Z[t]$.
\end{theorem}
\begin{proof}
    Suppose $A$ is generated over $A_0$ by $s$ homomgeneous elements, $x_1,\dots,x_s$. We shall induct on $s$. Let $k_s$ denote the degree of $x_s$. Then, for all $n\ge 0$, we have an exact sequence 
    \begin{equation*}
        0\rightarrow K_n\rightarrow M_n\xrightarrow{x_s} M_{n + k_s}\rightarrow L_{n + k_s}\rightarrow 0.
    \end{equation*}
    Thus, 
    \begin{equation*}
        \lambda(K_n) - \lambda(M_n) + \lambda(M_{n + k_s}) - \lambda(L_{n + k_s}).
    \end{equation*}
    Multiplying by $t^{n + k_s}$ and adding, we obtain 
    \begin{align*}
        0 &= t^{k_s}\sum_{n = 0}^\infty\lambda(K_n)t^n - t^{k_s}\sum_{n = 0}^\infty\lambda(M_n)t^n + \sum_{n = 0}^\infty\lambda(M_{n + k_s})t^{n + k_s} - \sum_{n = 0}^\infty\lambda(L_{n + k_s})t^{n + k_s}\\
        &= t^{k_s}P(K_n , t) - t^{k_s}P(M, t) + P(M, t) - P(L, t) + g(t)
    \end{align*}
    where $g(t)\in\Z[t]$. Note that $K = \bigoplus_{n = 0}^\infty K_n$ and $L = \bigoplus_{n = 0}^\infty L_n$ is a graded $A' = A_0[x_1,\dots,x_{s - 1}]$. Invoking the induction hypothesis, we are done.
\end{proof}

\begin{proposition}\thlabel{prop:q-stable-filtration-polynomial}
    Let $(A,\frakm)$ be a noetherian local ring with $\frakq$ an $\frakm$-primary ideal, $M$ a finitely generated $A$-module and $(M_n)_{n\ge 0}$ a $\frakq$-stable filtration. Then, 
    \begin{enumerate}[label=(\alph*)]
        \item $M/M_n$ is of finite length for every $n\ge 0$.
        \item for all sufficiently large $n$, this length is a polynomial $g(n)$ of degree $\le s$ where $s$ is the least number of generators of $\frakq$.
        \item the degree and leading coefficient of $g(n)$ is independent of the chosen filtration.
    \end{enumerate}
\end{proposition}
\begin{proof}
\begin{enumerate}[label=(\alph*)]
    \item Note that $M_i/M_{i + 1}$ is naturally an $A/\frakq$-module, which has finite length, since $A/\frakq$ is Artin local. Using the additivity of length, we have 
    \begin{equation*}
        l(M/M_n) = \sum_{i = 1}^n (M_{i - 1}/M_i),
    \end{equation*}
    which is finite.

    \item 

    \item Let $(\wt M_n)_{n\ge 0}$ be another $\frakq$-stable filtration of $M$. Then, these two are equivalent filtrations, that is, there is $n_0 > 0$ such that $M_{n + n_0}\subseteq\wt M_n$ and $\wt M_{n + n_0}\subseteq M_n$ for all $n\ge0$. Thus, $\wt g(n + n_0)\ge g(n)$ and $g(n + n_0)\ge\wt g(n)$. Consequently, $\lim_{n\to\infty} g(n)/\wt g(n) = 1$. This completes the proof.\qedhere
\end{enumerate}
\end{proof}

\begin{definition}[Hilbert-Samuel Polynomial]
    With the notation of \thref{prop:q-stable-filtration-polynomial}, the polynomial $g(n)$ corresponding to the filtration $(\frakq^n)_{n\ge 0}$ is denoted by $\chi_\frakq^M(n)$ and is called the \emph{Hilbert-Samuel polynomial}. For sufficiently large $n$, we have 
    \begin{equation*}
        \chi_\frakq^M(n) = l(M/\frakq^n M).
    \end{equation*}
    If $M = A$, we write $\chi_\frakq(n)$ for $\chi_\frakq^A(n)$ and call it the \emph{characteristic polynomial} of the $\frakm$-primary ideal $\frakq$.
\end{definition}

\begin{corollary}
    Let $(A,\frakm)$ be a noetherian local ring and $\frakq$ an $\frakm$-primary ideal in $A$. Then, the length $l(A/\frakq^n)$ is a polynomial $\chi_\frakq(n)$ of degree $\le s$ for sufficiently large $n$, where $s$ is the least number of generators of $\frakq$.
\end{corollary}

\begin{proposition}\thlabel{prop:deg-equal-primary-maximal}
    Let $(A,\frakm)$ be a noetherian local ring and $\frakq$ an $\frakm$-primary ideal in $A$. Then, 
    \begin{equation*}
        \deg\chi_\frakq(n) = \deg\chi_\frakm(n).
    \end{equation*}
\end{proposition}
\begin{proof}
    There is a positive integer $r > 0$ such that $\frakm^r\subseteq\frakq\subseteq\frakm$. Then, for sufficiently large $n$, we have 
    \begin{equation*}
        \chi_\frakm(n)\le\chi_\frakq(n)\le\chi_\frakm(rn).
    \end{equation*}
    Since the $\chi_\frakm$ and $\chi_\frakq$ are polynomials for sufficiently large $n$, it must be th case that $\deg\chi_\frakm = \deg\chi_\frakq$. This completes the proof.
\end{proof}

\begin{definition}
    With the notation of \thref{prop:deg-equal-primary-maximal}, let $d(A)$ denote the degree of $\chi_\frakm$.
\end{definition}

\section{Noetherian Local Rings}

Throughout this section, let $(A,\frakm)$ be a noetherian local ring with maximal ideal $\frakm$. Denote by $\delta(A)$, the least number of generators of an $\frakm$-primary ideal of $A$. From the last section, we already know that $\delta(A)\ge d(A)$.

\begin{lemma}
    Let $M$ be a finitely-generated $A$-module, $x\in A$ not a zero-divisor in $M$, and $M' = M/xM$. Then, 
    \begin{equation*}
        \deg\chi_\frakq^{M'}\le\chi_\frakq^{M} - 1.
    \end{equation*}
\end{lemma}
\begin{proof}
    Let $N = xM$, and $N_n = N\cap\frakq^n M$. Then, due to \thref{lem:artin-rees}, $(N_n)_{n\ge 0}$ is a $\frakq$-stable filtration of $N$. \todo{complete this}
\end{proof}

\begin{corollary}
    If $x$ is not a zero-divisor in $A$, then $d(A/(x))\le d(A) - 1$.
\end{corollary}

\begin{proposition}
    $d(A)\ge\dim A$.
\end{proposition}
\begin{proof}
    We shall induct on $d(A)$. If $d(A) = 0$, then $l(A/\frakm^n)$ is eventually constant. Therefore, $\frakm^n = \frakm^{n + 1}$ for sufficiently large $n$, whence $\frakm = 0$, i.e. $A$ is artinian and $\dim A = 0$.

    Suppose now that $d = 0$ and let $\frakp_0\subsetneq\frakp_1\subsetneq\dots\subsetneq\frakp_r$ be any chain of prime ideals in $A$. If $r = 0$, then there is nothing more to prove. If $r > 0$, then let $x\in\frakp_1\backslash\frakp_0$ and let $A' = A/\frakp_0$ with $x'$ denoting the image of $x$ in $A'$.

    Let $\frakm'$ denote the image of $\frakm$ under the surjection $A\onto A'$. Then, $(A',\frakm')$ is a local ring with an induced surjection $A/\frakm^n\onto A'/\frakm'^n$ through the following commutative diagram. 
    \begin{equation*}
        \xymatrix {
            A\ar@{->>}[r]\ar@{->>}[d] & A'\ar@{->>}[r] & A'/\frakm'^n\\
            A/\frakm^n\ar@{.>>}[rru]_{\exists}
        }
    \end{equation*}
    Consequently, $l(A/\frakm^n)\ge l(A'/\frakm'^n)$. Therefore, $d(A)\ge d(A')$, due to the standard polynomial argument. 

    Now, $d(A'/(x'))\le d(A') - 1\le d(A) - 1$ and hence, the induction hypothesis applies to $A'/(x')$, that is, $d(A'/(x'))\ge\dim A'/(x')$. Note that the image of the strictly ascending chain $\frakp_1\subsetneq\dots\subsetneq\frakp_r$ in $A'/(x')$ is a strictly ascending chain of prime ideals of length $r - 1$ whence 
    \begin{equation*}
        r - 1 \le\dim A'/(x')\le d(A'/(x'))\le d(A) - 1\implies r\le d(A)
    \end{equation*}
    which completes the proof.
\end{proof}

\begin{corollary}
    If $A$ is a noetherian local ring, then $\dim A$ is finite.
\end{corollary}

\begin{corollary}
    A prime ideal in a noethering has finte height. Therefore, the set of primes in a noethering satisfies the descending chain condition.
\end{corollary}

\begin{proposition}
    Let $\dim A = d$. Then, there exists an $\frakm$-primary ideal in $A$ generated by $d$ elements. Therefore, $\dim A\ge\delta(A)$.
\end{proposition}
\begin{proof}
    We shall inductively construct a sequence $x_1,\dots,x_d$ of elements in $A$ such that the ideal $(x_1,\dots,x_i)$ has height at least $i$. First, if $d = 0$, then $A$ is an artinian local ring and the conclusion follows since $\frakm$ is nilpotent.

    Suppose now that $d\ge 1$. There are finitely many minimal primes in $A$. Pick an $x_1$ that lies in $\frakm$ but not in any of the minimal primes. It follows from the choice of $x_1$ that any prime ideal containing $x_1$ must have height at least $1$. Let the sequence $x_1,\dots,x_i$ have been constructed. Thus, any prime ideal containing $(x_1,\dots,x_i)$ has height at least $i$. Let $\{\frakp_1,\dots,\frakp_r\}$ be the set of height $i$ prime ideals containing $(x_1,\dots,x_i)$. Note that these would be minimal over $x_1,\dots,x_i$ and hence, are finitely many. Usig prime avoidance, pick an $x_{i + 1}\in\frakm\backslash\bigcup_{j = 1}^r\frakp_r$. It is not hard to argue that any prime containing $(x_1,\dots,x_{i + 1})$ must have height at least $i + 1$.

    Finally, let $\fraka = (x_1,\dots,x_d)$. Then, any prime containing $\fraka$ must have height at least $d$ and hence, must be maximal. As a result, there is a unique minimal prime belonging to $\fraka$, namely $\frakm$, whence $\fraka$ is $\frakm$-primary. This completes the proof.
\end{proof}

Putting everything together, we have the following theorem:

\begin{theorem}[Dimension Theorem]
    For any notherian local ring $A$, the following integers are equal: 
    \begin{enumerate}[label=(\alph*)]
        \item $\dim A$. 
        \item the degree of the characteristic polynomial $\chi_\frakm$.
        \item the least number of generators of an $\frakm$-primary ideal of $A$.
    \end{enumerate}
\end{theorem}

\begin{corollary}
    $\dim A\le\dim_k(\frakm/\frakm^2)$ where $k = A/\frakm$ is the residue field.
\end{corollary}
\begin{proof}
    There are $x_1,\dots,x_s\in\frakm$ generating it such that their images in $\frakm/\frakm^2$ form a $k$-basis. Thus, 
    \begin{equation*}
        \dim A = \delta(A)\le s = \dim_k(\frakm/\frakm^2).\qedhere
    \end{equation*}
\end{proof}

\begin{corollary}[Krull's Hauptidealsatz]\thlabel{corr:stronger-hauptidealsatz}
    Let $A$ be a noethering and $\frakp\unlhd A$ a prime ideal. Then, the following are equivalent: 
    \begin{enumerate}[label=(\alph*)]
        \item $\hght(\frakp)\le n$. 
        \item There is an ideal $\fraka\unlhd A$, generated by $n$ elements, such that $\frakp$ is a minimal prime belonging to $\fraka$.
    \end{enumerate}
\end{corollary}
\begin{proof}
    $\impliedby$ Let $\fraka = (x_1,\dots,x_n)$. In $A_\frakp$, the ideal $\fraka A_\frakp$ is $\frakp A_\frakp$-primary, since its radical is $\frakp A_\frakp$, which is a maximal ideal. 

    $\implies$ We have $\dim A_\frakp\le n$ and thus, there is a $\frakp A_\frakp$-primary ideal $\frakb$ of $A_\frakp$ generated by $n$ elements. Let $\fraka$ denote the contraction of $\frakb$ in $A$. Then, $\fraka$ is generated by $n$ elements and is contained in $\frakp$. Moreover, $\frakp$ must be the minimal prime containing $\fraka$ due to the order preserving bijection between the primes contained in $\frakp$ and $\spec(A_\frakp)$. This completes the proof.\qedhere
\end{proof}

As an application, we have the following attractive result, taken from Hartshorne's Algebraic Geometry.

\begin{proposition}
    Let $A$ be a noetherian domain. Then, $A$ is a UFD if and only if every height $1$ prime in $A$ is principal.
\end{proposition}
\begin{proof}
    The forward direction is trivial and does not require the noetherian hypothesis. Conversely, note that every noetherian domain is a factorization domain and hence, it suffices to show that all irreducibles in $A$ are prime. Let $f\in A$ be an irreducible and $\frakp\in\spec(A)$ be a minimal prime containing $f$. Due to \thref{corr:stronger-hauptidealsatz}, $\hght(\frakp)\le 1$ and hence, is equal to $1$, whence, is principal, say $\frakp = (x)$. Then, $f = xy$ for some $y\in A$. Since $f$ is irreducible, we must have $(f) = \frakp$, consequently, $f$ is prime. This completes the proof.
\end{proof}

\section{Dimension Theory of Polynomial Algebras}

This section has been taken from \cite{serre-local-algebra}.

\begin{lemma}\thlabel{lem:lying-over-polynomial}
    Let $B = A[x]$, $\frakp\subseteq A$ a prime ideal and $\frakq\subseteq\frakq'$ two prime ideals in $B$ lying over $A$ such that $\frakq\subsetneq\frakq'$. Then, $\frakq = \frakp B$.
\end{lemma}
\begin{proof}
    Note that $B/\frakp B\cong (A/\frakp)[x]$, and both the primes $\frakq$ and $\frakq'$ must contain $\frakp B$. Therefore, upon quotienting out by $\frakp B$, we have reduced to the case of $A$ being an integral domain and $\frakp = (0)$.

    Now, localize at $S = A\backslash\{0\}$ to reduce to the case of $A$ being a field and $\frakp = (0)$. Note that $\frakq'\cap S = \emptyset$ therefore, they extend to prime ideals in $S^{-1}B\cong (S^{-1}A)[x]$. But this is obvious, since any non-zero prime ideal in $A[x]$ is maximal, owing to it being a PID. This completes the proof.
\end{proof}

\begin{theorem}
    If $B = A[x]$, then 
    \begin{equation*}
        \dim A + 1\le \dim B\le 2\dim A + 1.
    \end{equation*}
\end{theorem}

\begin{lemma}\thlabel{lem:minimal-remains-minimal-polynomial}
    Let $B = A[x]$ and $\fraka\unlhd A$. Let $\frakp\unlhd A$ be a minimal prime ideal containing $\fraka$. Then, $\frakp B$ is a minimal prime ideal containing $\fraka B$ in $B$.
\end{lemma}
\begin{proof}
    Suppose $\frakp B$ were not minimal among the primes containing $\fraka B$. Then, there is some prime $\frakq$ with $\fraka B\subseteq\frakq\subsetneq\frakp B$. Note that $\frakq\cap A$ is a prime ideal in $A$ containing $\fraka$ and is contained in $\frakp$, therefore, $\frakq\cap A = \frakp$. Consequently, due to \thref{lem:lying-over-polynomial}, $\frakq = \frakp B$, a contradiction.
\end{proof}

\begin{proposition}\thlabel{prop:height-polynomial-extension}
    Let $A$ be a noethering and $\frakp\unlhd A$ a prime ideal. If $B = A[x]$, then $\hght(\frakp) = \hght(\frakp B)$.
\end{proposition}
\begin{proof}
    Let $n = \hght(\frakp)$. Then, there is a strictly ascending chain $\frakp_0\subsetneq\frakp_1\subsetneq\dots\subsetneq\frakp_n$ of prime ideals in $A$. Then, $\frakp_0 B\subseteq\frakp_1B\subsetneq\dots\subsetneq\frakp_nB$ is a strictly ascending chain of prime ideals in $B$. Hence, $\hght(\frakp B)\ge n$.

    Conversely, there is an ideal $\fraka$ generated by $n$ elements, contained in $\frakp$ such that $\frakp$ is minimal among the primes containing $\fraka$. Then, $\fraka B\subseteq\frakp B$, $\fraka B$ is generated by $n$ elements and due to \thref{lem:minimal-remains-minimal-polynomial} and \thref{corr:stronger-hauptidealsatz}, $\hght(\frakp B)\le n$.
\end{proof}

\begin{theorem}
    Let $A$ be a noethering. Then, 
    \begin{equation*}
        \dim(A[x_1,\dots,x_n]) = \dim A + n.
    \end{equation*}
\end{theorem}
\begin{proof}
    It suffices to prove the theorem for $n = 1$. Let $B = A[x]$. We know that $\dim(B)\ge\dim A + 1$. We shall show that $\dim(B)\le\dim A + 1$. Let $\frakq_0\subsetneq\dots\subsetneq\frakq_r$ be a strictly ascending chain of prime ideals in $B$. Define $\frakp_i = \frakq_i\cap A$ for $0\le i\le r$. If all the $\frakp_i$'s are distinct, then $r\le\dim A$.

    Suppose now that the $\frakp_i$'s are not distinct. Let $j$ be the maximum index such that $\frakp_j = \frakp_{j + 1}$. Then, $\frakq_j = \frakp_j B$. Due to \thref{prop:height-polynomial-extension}, $\hght(\frakp_j) = \hght(\frakp_j B) = \hght(\frakq_j)$. Now, note that $\frakp_{j + 1}\subsetneq\dots\subsetneq\frakp_{r}$. Hence, 
    \begin{equation*}
        \dim A\ge r - (j + 1) + \hght(\frakp_j) = r - (j + 1) + \hght(\frakq_j)\ge r - 1 \implies r\le\dim A + 1.
    \end{equation*}
    This completes the proof.
\end{proof}

\begin{corollary}
    Let $k$ be a field. Then, $\dim(k[x_1,\dots,x_n]) = n$.
\end{corollary}

\section{Dimension of a Variety}

\section{Dimension Theory of Affine \texorpdfstring{$k$}{k}-Algebras}

\begin{theorem}
    Let $A$ be an affine $k$-domain. Then, $\dim(A) = \trdeg_k(Q(A))$.
\end{theorem}
\begin{proof}
    Follows from the Going Up Theorem and Noether's Normalization Lemma.
\end{proof}

\begin{theorem}
    Let $A$ be an affine $k$-domain and $\frakp\in\spec(A)$. Then, 
    \begin{equation*}
        \dim(A) = \dim(A/\frakp) + \hght(\frakp)
    \end{equation*}
\end{theorem}
\begin{proof}
    Using Noether's Normalization Lemma, there is a polynomial algebra $B = k[y_1,\dots,y_n]\subseteq A$ such that $A/B$ is an integral extension. Hence, $\dim(A) = \dim(B)$. Let $\frakq = \frakp\cap B$. Due to the Going Down Theorem, $\hght(\frakq) = \hght(\frakp)$. Further, $A/\frakp$ is integral over $B/\frakq$ whence it suffices to prove the theorem for polynoimal algebras $A = k[x_1,\dots,x_n]$. We shall do so by induction on $\hght(\frakp)$.\\

    \noindent\textbf{Claim.} If $\frakp\subseteq A = k[y_1,\dots,y_n]$ is a height $1$ prime, then $\dim(A/\frakp) = \dim(A) - 1$.

    Let $a\in\frakp$ be a non-zero element. This admits a unique factorization in terms of irreducibles $a = f_1\cdots f_r$. Hence, there is an $f_i\in\frakp$. Since $(f_i)$ is a non-zero prime ideal, we must have $\frakp = (f_i)$. Since $f_i$ is non-zero, it contains at least one monomial. Suppose, without loss of generality that $y_n$ occurs in this monomial.
    \begin{equation*}
        f_i(y_1,\dots,y_n) = \sum_{j = 0}^d g_j(y_1,\dots,y_{n - 1})y_n^j
    \end{equation*}
    where $g_j\in k[y_1,\dots,y_{n - 1}]$ with at least one of the $g_j$'s being non-zero. 

    Note that $\overline y_1,\dots,\overline y_{n - 1}\in A/\frakp$ are algebraically independent. This is easy to see by examining the degree of $y_n$. But, we also see that $\overline y_n\in Q(A/\frakp)$ is algebraic over $k[\overline y_1,\dots,\overline y_{n - 1}]$ and hence, $\trdeg_k(Q(A/\frakp)) = n - 1$, whence, $\dim(A/\frakp) = n - 1 = \dim(A) - 1$. $\square$

    From the Claim, we see that the theorem is true for all height $1$ primes. We shall now induct on $\hght(p)$. Let $r = \hght(\frakp)$. Then, there is a chain $(0)\subsetneq\frakp_1\subsetneq\dots\subsetneq\frakp_r = \frakp$. Set $B = A/\frakp_1$. Then, $\dim(B) = \dim(A) - 1$ and $\hght(\frakp/\frakp_1) = r - 1$ and the induction hypothesis applies to obtain 
    \begin{equation*}
        \dim(A) - 1 = \dim(B) = \dim(A/\frakp) + \hght(\frakp/\frakp_1) = \dim(A/\frakp) + \hght(\frakp) - 1.
    \end{equation*}
    This completes the proof.
\end{proof}

\begin{proposition}
    Let $A$ be an affine $k$-domain with $\dim A = d$. Then, every saturated, maximal chain $(0) = \frakp_0\subsetneq\frakp_1\subsetneq\dots\subsetneq\frakp_n$ has length $d$.
\end{proposition}
\begin{proof}
    We shall prove this by induction on $d$. The base case with $d = 0$ is trivial. Now, let $B = A/\frakp_1$. Then, $\dim B = d - 1$ and $(0) = \overline\frakp_1\subsetneq\dots\subsetneq\overline\frakp_n$ is a saturated, maximal chain and hence, has length $d - 1$. The conclusion follows.
\end{proof}

\begin{theorem}
    Let $A$ be an affine $k$-algebra. Then, given any two prime ideals $\frakp\subseteq\frakq$ in $A$, every saturated chain of prime ideals from $\frakp$ to $\frakq$ has the same length.
\end{theorem}
\begin{proof}
    Let $\frakp = \frakp_0\subsetneq\dots\subsetneq\frakp_r = \frakq$ be a saturated chain. There are surjections 
    \begin{equation*}
        A/\frakp = A/\frakp_0\onto A/\frakp_1\onto\dots\onto A/\frakp_r = A/\frakq.
    \end{equation*}
    And, $\dim(A/\frakp_{i + 1}) = \dim(A/\frakp_i) - 1$. Consequently, $\dim(A/\frakq) = \dim(A/\frakp) - r$. This means 
    \begin{equation*}
        r = \dim(A/\frakp) - \dim(A/\frakq) = \hght(\frakq) - \hght(\frakp).
    \end{equation*}
    This completes the proof.
\end{proof}

\begin{corollary}
    Let $A$ be an affine $k$-domain with $\dim A = d$. Then, $\hght(\frakm) = d$ for every maximal ideal $\frakm\subsetneq A$.
\end{corollary}

\section{Dimension Theory of Power Series Algebras}

\begin{lemma}
    A maximal ideal in $A\llbracket x\rrbracket$ is of the form $(\frakm, x)$ where $\frakm$ is a maximal ideal in $A$.
\end{lemma}
\begin{proof}
\end{proof}

\begin{theorem}
    Let $A$ be a noethering. Then, $\dim A\llbracket x\rrbracket = \dim A + 1$.
\end{theorem}
\begin{proof}
    Let $\frakp_0\subsetneq\dots\subsetneq\frakp_n$ be a chain of primes in $A$. Then, 
    \begin{equation*}
        \frakp_0\llbracket x\rrbracket\subsetneq\dots\subsetneq\frakp_n\llbracket x\rrbracket\subseteq\frakp_n\llbracket x\rrbracket + (x)
    \end{equation*}
    is a chain of prime ideals in $A\llbracket x\rrbracket$. Hence, $\dim A\llbracket x\rrbracket\ge\dim A + 1$.

    Conversely, let $\frakM$ be a maximal ideal in $A\llbracket x\rrbracket$. Then, $\frakM = (\frakm, x)$ where $\frakm$ is a maximal ideal in $A$. Let $n = \hght\frakm$, then there are elements $a_1,\dots,a_n\in A$ such that $\frakm$ is minimal over $(a_1,\dots, a_n)$. Consequently, $\frakm + (x)$ is minimal over $(a_1,\dots,a_n, x)$ in $A\llbracket x\rrbracket$. Hence, $\hght\frakM\le n + 1\le\dim A + 1$. The conclusion follows.
\end{proof}

\begin{remark}
    The above result also follows from the fact that completions preserve dimension but that requires a significant amount of machinery.
\end{remark}