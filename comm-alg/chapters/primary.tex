\section{Primary Decomposition of Ideals}

A primary ideal is a generalization of the ideals $p^n\Z$ in $\Z$, as is evident from the following definition.

\begin{definition}[Primary Ideals]
    An ideal $\frakq\subseteq A$ is said to be \textit{primary} if for every ordered pair $x,y\in A$,
    \begin{equation*}
        xy\in\frakq\implies x\in\frakq\text{ or }y^n\in\frakq\text{ for some } n > 0
    \end{equation*}
\end{definition}


From the definition, we see that every prime ideal is primary. It is not hard to see that
\begin{itemize}
\item $\frakq$ is primary if and only if every zero divisor in $A/\frakq$ is nilpotent. 
\item $\frakq$ is primary if and only if $(0)$ is primary in $A/\frakq$.
\end{itemize}


\begin{proposition}
    If $\frakq$ is primary, then $\sqrt{\frakq}$ is prime. Further, $\sqrt{\frakq}$ is the smallest prime ideal containing $\frakq$.
\end{proposition}
\begin{proof}
    Suppose $xy\in\sqrt{\frakq}$, then there is $n > 0$ such that $x^ny^n\in\frakq$, consequently, there is an $m > 0$ such that $x^n\in\frakq$ or $y^{mn}\in\frakq$, therefore, $x\in\sqrt{\frakq}$ or $y\in\sqrt{\frakq}$, whence $\sqrt{\frakq}$ is prime. The second assertion is trivial.
\end{proof}

If $\frakq$ is a primary ideal, then $\frakp = \sqrt{\frakq}$ is called the \textit{associated prime ideal} of $\frakq$ and $\frakq$ is said to be \textit{$\frakp$-primary}.

Consider the ring $A = k[x,y]$ and the ideal $\frakq = (x,y^2)$. The quotient ring $A/\frakq$ is isomorphic to $k[y]/(y^2)$ where every zero divisor is nilpotent consequently, $\frakq$ is primary. The radical ideal $\frakp = \sqrt{\frakq} = (x,y)$ is a prime ideal such that $\frakp^2\subsetneq\frakq\subsetneq\frakp$, therefore, $\frakq$ is not a prime power.

On the other hand, consider the ring $A = k[x,y,z]/(xy - z^2)$ and the prime ideal $\frakp = (\overline x,\overline z)\subseteq A$. We contend that $\frakp^2\subseteq A$ is not primary. Indeed, note that $\overline x\overline y = \overline z^2\in\frakp^2$ but $\overline x\notin\frakp^2$ and $\overline y\notin\frakp^2$, and the conclusion follows.

\begin{proposition}
    If $\sqrt{\fraka}$ is maximal, then $\fraka$ is primary.
\end{proposition}
\begin{proof}
    Let $\frakm = \sqrt{\fraka}$ and $\phi: A\to A/\fraka$ denote the natural map. Then, $\phi(\sqrt{\fraka})$ is the maximal ideal in $A/\fraka$ and is also the nilradical of $A/\fraka$, consequently, $A/\fraka$ is local and every non-unit is nilpotent. Hence, $\fraka$ is primary.
\end{proof}

\begin{proposition}
    Let $\phi: A\to B$ be a ring homomorphism. If $\frakq\unlhd B$ is a primary ideal in $B$, then $\frakq^c$ is a primary ideal in $A$.
\end{proposition}
\begin{proof}
    There is an injection $A/\frakq^c\into B/\frakq$. If $(0)$ is primary in $B/\frakq$ then it is primary in $A/\frakq^c$.
\end{proof}

\begin{lemma}\thlabel{lem:intersection-p-primary}
    If $\{\frakq_i\}_{i = 1}^n$ are $\frakp$-primary, then so is $\frakq=\bigcap_{i = 1}^n\frakq_i$.
\end{lemma}
\begin{proof}
    Obviously, 
    \begin{equation*}
        \sqrt{\frakq} = \bigcap_{i = 1}^n\sqrt{\frakq_i} = \frakp
    \end{equation*}
    Let $xy\in\frakq$. If $y\in\frakp$, then we are done, since $\frakp = \sqrt{\frakq}$. Else, $y^n\notin\frakq_i$ for every positive integer $n$, since $\frakp = \sqrt{\frakq_i}$ whereby $x\in\frakq_i$ for each $1\le i\le n$ and the conclusion follows.
\end{proof}

\begin{lemma}
    Let $\frakq$ be a $\frakp$-primary ideal and $x\in A$. Then 
    \begin{enumerate}[label=(\alph*)]
        \item if $x\in\frakq$, then $(\frakq: x) = (1)$.
        \item if $x\notin\frakq$, then $(\frakq: x)$ is $\frakp$-primary.
        \item if $x\notin\frakp$, then $(\frakq: x) = \frakq$.
    \end{enumerate}
\end{lemma}
\begin{proof}
\begin{enumerate}[label=(\alph*)]
    \item Trivial.
    \item If $y\in(\frakq:x)$, then $xy\in\frakq$, therefore, $y\in\frakp$. Thus, we have $\frakq\subseteq(\frakq:x)\subseteq\frakp$. Taking radicals, $\frakp\subseteq\sqrt{(\frakq:x)}\subseteq\frakp$, whereby $\sqrt{(\frakq: x)} = \frakp$.

    On the other hand, if $yz\in(\frakq: x)$, then $xyz\in\frakq$. If $z\in\frakp$, then we are done. Else, $xy\in\frakq$ and $y\in(q:x)$ whence $(q: x)$ is $\frakp$-primary.
    \item If $y\in(q: x)$, then $yx\in\frakq$. Since $x\notin\frakp$, we must have $y\in\frakq$. This completes the proof.
\end{enumerate}
\end{proof}

\begin{definition}[Primary Decomposition]
    A \textit{primary decomposition} of an ideal $\fraka\subseteq A$ is an expression of $\fraka$ as a \textit{finite} intersection of primary ideals.
    \begin{equation*}
        \fraka = \bigcap_{i = 1}^n\frakq_i
    \end{equation*}
    The ideal $\fraka$ is said to be \textit{decomposable} if it has a primary decomposition. Moreover, if for all $1\le i\le n$, $\sqrt{\frakq_i}$ are distinct and 
    \begin{equation*}
        \bigcap_{j\ne i}\frakq_j\not\subseteq\frakq_i
    \end{equation*}
    then the primary decomposition is said to be \textit{minimal}.
\end{definition}

Using \thref{lem:intersection-p-primary}, it is not hard to see that every decomposable ideal has a minimal decomposition.

\begin{theorem}[First Uniqueness Theorem]\thlabel{thm:first-uniqueness-theorem}
    Let $\fraka\subseteq A$ be a decomposable ideal and 
    \begin{equation*}
        \fraka = \bigcap_{i = 1}^n\frakq_i
    \end{equation*}
    be a minimal primary decomposition with $\frakp_i = \sqrt{\frakq_i}$. Then, the $\frakp_i$'s are precisely the prime ideals the occur in the set $\{\sqrt{(\fraka: x)}\mid x\in A\}$.
\end{theorem}
\begin{proof}
    First, note that 
    \begin{equation*}
        \sqrt{(\fraka: x)} = \sqrt{\bigcap_{i = 1}^n(\frakq_i: x)} = \bigcap_{i = 1}^n\sqrt{(\frakq_i: x)} = \bigcap_{x\notin q_i}\frakp_i
    \end{equation*}
    Using \thref{prop:prime-containment}, $\sqrt{(\fraka : x)} = \frakp_j$ for some index $j$. 

    Conversely, for every $1\le j\le n$, there is $x_j\in\bigcap_{i\ne j}\frakq_i\backslash\frakq_j$. This obviously exists since the decomposition is minimal. It now follows from \thref{prop:prime-containment} and the decomposition of $\sqrt{(\fraka: x)}$ we derived above that $\sqrt{(\fraka : x)} = \frakp_j$.
\end{proof}

\begin{proposition}\thlabel{prop:min-primes-belonging}
    Let $\fraka$ be a decomposable ideal. Then any prime ideal $\frakp\supseteq\fraka$ contains a \textit{minimal} prime ideal belonging to $\fraka$, and thus the minimal prime ideals belonging to $\fraka$ are precisely the minimal prime ideals in the set of all prime ideals containing $\fraka$.
\end{proposition}
\begin{proof}
    Let $\frakp$ be a minimal prime ideal containing $\fraka$. Consider a minimal primary decomposition of $\fraka$ given by 
    \begin{equation*}
        \frakp\supseteq\fraka = \bigcap_{i = 1}^n\frakq_i.
    \end{equation*}
    Let $\frakp_i = \sqrt{\frakq_i}$, then 
    \begin{equation*}
        \frakp\supseteq\sqrt{\fraka} = \bigcap_{i = 1}^n\frakp_i
    \end{equation*}
    and due to \thref{prop:prime-containment}, there is an index $j$ such that $\frakp\supseteq\frakp_j$ whence $\frakp_j = \frakp$. Thus, every minimal prime ideal containing $\fraka$ belongs to $\fraka$.
\end{proof}

\begin{proposition}
    Let $S$ be a multiiplcatively closed subset of $A$ and $\frakq$ be a $\frakp$-primary ideal. 
    \begin{enumerate}[label=(\alph*)]
        \item If $S\cap\frakp\ne\emptyset$, then $S^{-1}\frakq = S^{-1}A$. 
        \item If $S\cap\frakp = \emptyset$, then $S^{-1}\frakq$ is $S^{-1}\frakp$-primary and its contraction in $A$ is $\frakq$.
    \end{enumerate}
\end{proposition}
\begin{proof}
    $(a)$ is trivial. $(b):$ Recall that we have 
    \begin{equation*}
        \frakq^{ec} = \bigcup_{s\in S}(\frakq: s) = \bigcup_{s\in S}\frakq
    \end{equation*}
    where the last equality follows from the fact that $S\cap\frakq = \emptyset$. It remains to show that $S^{-1}\frakq$ is primary. Indeed, let $x/s\cdot y/t\in S^{-1}\frakq$. Then, there is $z\in\frakq$ and $w,u\in S$ such that $w(xyu - stz) = 0$. But since $wu\notin\frakq$, we must have $xy\in\frakq$, whereby $x\in\frakq$ or $y^n\in\frakq$ for some positive integer $n$, implying that either $x/s\in S^{-1}\frakq$ or $y^n/t^n\in S^{-1}\frakq$. This completes the proof.
\end{proof}

\begin{definition}[Isolated Set of Associated Primes]
    A set $\Sigma$ of prime ideals associated with $\fraka$ is said to be \emph{isolated} if it satisfies the following condition: 
    \begin{quote}
        if $\frakp'$ is a prime ideal belonging to $\fraka$ with $\frakp'\subseteq\frakp$ for some $\frakp\in\Sigma$, then $\frakp'\in\Sigma$
    \end{quote}
\end{definition}

\begin{theorem}[Second Uniqueness Theorem]\thlabel{thm:second-uniqueness-theorem}
    Let $\fraka$ be a decomposable ideal with a primary decomposition $\fraka = \bigcap_{i = 1}^n\frakq_i$. Let $\sqrt{\frakq_i} = \frakp_i$. Suppose $\Sigma = \{\frakp_{i_1},\ldots,\frakp_{i_m}\}$ is an isolated set of associated primes of $\fraka$, then $\bigcap_{j = 1}^m\frakq_{i_j}$ is independent of the chosen decomposition.
\end{theorem}
\begin{proof}
    Let $S = A\backslash\bigcup_{j = 1}^m\frakp_{i_j}$. Then, $\frakp_k\cap S = \emptyset$ if and only if $\frakp_k\subseteq\bigcap_{j = 1}^m\frakp_j$ whence due to \thref{prop:prime-containment}, there is a prime $\frakp_{i_t}$ containing $\frakp_k$ and equivalently, $\frakp_k\in\Sigma$. 

    Whence, upon localizing with $S$, we have 
    \begin{equation*}
        S^{-1}\fraka = \bigcap_{i = 1}^n S^{-1}\frakq_i = \bigcap_{j = 1}^m S^{-1}\frakq_{i_j}
    \end{equation*}
    Contracting both sides, we have 
    \begin{equation*}
        \fraka^{ec} = \left(\bigcap_{j = 1}^m S^{-1}\frakq_{i_j}\right) = \bigcap_{j = 1}^m\frakq_{i_j}^{ec} = \bigcap_{j = 1}^m\frakq_{i_j}
    \end{equation*}
    and the conclusion follows.
\end{proof}

\begin{corollary}
    In particular, the primary ideals which correspond to the minimal primes associated to $\fraka$ are uniquely determined.
\end{corollary}

\begin{proposition}
    Let $X$ be an infinite compact Hausdorff space. Then, $(0)$ is not decomposable in $C(X)$, the ring of continuous functions on $X$.
\end{proposition}
\begin{proof}
    Suppose $(0) = \bigcap_{i = 1}^n\frakq_i$. Recall that the maximal ideals in $X$ are in bijection with the points of $X$. Denote the maximal ideal corresponding to a point $x\in X$ by $\frakm_x$. 

    For each $\frakq_i$, choose a maximal ideal $\frakm_{x_i}$ containing it. Choose some $x\in X\backslash\{x_1,\dots,x_n\}$. Choose an open set $V$ containing $\{x_1,\dots,x_n\}$ and an open set $U$ containing $x$ such that $U\cap V = \emptyset$. 

    Using Urysohn's Lemma, choose continuous functions $f,g: X\to[0,1]$ such that $f(x) = 1$ and $\Supp(f)\subseteq U$ and $g(x_i) = 1$ for every $i$ and $\Supp(g)\subseteq V$. By our choice of $g$, note that $g^m\notin\frakq_i$ for every $1\le i\le n$ and every positive integer $m$. Since $fg = 0$, we must have $f\in\frakq_i$ for every $1\le i\le n$, implying that $f = 0$, a contradiction. This completes the proof.
\end{proof}

\begin{definition}[Symbolic Power]
    Let $\frakp\in\Spec A$. The \emph{$n$-th symbolic power of $\frakp$} is defined to be the contraction of the ideal $\frakp^nA_\frakp$ in $A$, denoted $\frakp^{(n)}$.
\end{definition}

Being the contraction of a primary ideal in $A_\frakp$, the symbolic power is always a primary ideal. Moreover, $\sqrt{\frakp^{(n)}} = \frakp$ whence, it is $\frakp$-primary.

\begin{proposition}
    With notation as above, 
    \begin{enumerate}[label=(\alph*)]
        \item $\frakp^{(n)}$ is a $\frakp$-primary ideal. 
        \item if $\frakp^n$ has a primary decomposition, then $\frakp^{(n)}$ is its $\frakp$-primary component.
    \end{enumerate}
\end{proposition}
\begin{proof}
    $(a)$ follows from the fact that the contraction of primary ideals is primary. 

    $(b)$ Note that $\frakp^n$ obviously would have a $\frakp$-primary component and that would be given by the contraction of $S^{-1}\frakp^n$ where $S = A\backslash\frakp$. The conclusion follows.
\end{proof}

\section{Associated Primes of Modules}

\begin{definition}
    Let $a\in A$ and $M$ and $A$-module. The homomorphism $a_M: M\to M$ given by $x\mapsto ax$ for all $x\in M$ is called the \emph{principal homomorphism}. We say that $a_M$ is \emph{locally nilpotent} if for each $x\in M$, there is an integer $n\in\N$ such that $a^n x = 0$.
\end{definition}

\begin{remark}
    If $M$ is finitely generated, then $a_M$ is locally nilpotent if and only if it is nilpotent.
\end{remark}

\begin{proposition}
    Let $a\in A$ and $M$ an $A$-module. Then $a_M$ is locally nilpotent if and only if $a\in\frakp$ for each $\frakp\in\Supp_A(M)$, that is, $a\in\bigcap\limits_{\frakp\in\Supp_A(M)}\frakp$.
\end{proposition}
\begin{proof}
    Suppose $a_M$ is locally nilpotent and $\frakp\in\Supp(M)$. Then, there is some $x\in M$ such that $x/1\ne 0$ in $M_\frakp$, that is, $\fraka = \Ann_A(x)\subseteq\frakp$. Since $a_M$ is locally nilpotent, there is a positive integer $n$ such that $a^n\in\fraka$ whence $a\in\frakp$. 

    Conversely, suppose $a_M$ is not locally nilpotent whence there is some $x\in M$ such that $a^nx\ne0$ for all $n\in\N$. Let $\frakp$ be a prime ideal not intersecting $\{1,a,a^2,\dots\}$ and containing $\Ann_A(x)$\footnote{That we can do this is an easy application of Zorn's Lemma}. Then, $x/1\ne 0$ in $M_\frakp$ whence $M_\frakp\ne 0$ and $\frakp\in\Supp(M)$, but $a\notin\frakp$. This completes the proof.
\end{proof}

\begin{definition}[Associated Primes]
    For an $A$-module $M$, a prime $\frakp\in\spec(A)$ is said to be \emph{associated} with $M$ if there is $x\in M$ such that $\frakp = \Ann_A(x)$. The set of all associated primes of a module $M$ is denoted by $\Ass(M)$.

    Equivalently, a prime $\frakp$ is an associated prime of $M$ if there is an injection of $A$-modules, $A/\frakp\into M$.
\end{definition}

\begin{proposition}
    If the poset 
    \begin{equation*}
        \Sigma = \{\Ann_A(x)\mid x\in M\backslash\{0\}\}
    \end{equation*}
    has a maximal element, then it is prime.
\end{proposition}
\begin{proof}
    Let $\frakp$ be a maximal element of $\Sigma$ under inclusion. Let $a,b\in A$ with $ab\in\frakp$. If either $a$ or $b$ is zero, then, trivially, $a\in\frakp$ or $b\in\frakp$. Suppose now that both $a,b$ are nonzero. Let $x\in M$ be such that $\frakp = \Ann_A(x)$ and suppose $b\notin\frakp$. Then, $\frakp\subseteq\Ann_A(bx)\ne(1)$ and due to maximality, we must have $\frakp = \Ann_A(bx)$, and thus $a\in\frakp$. This completes the proof.
\end{proof}

\begin{corollary}
    Modules over noetherings have associated primes.
\end{corollary}

\begin{lemma}
    Let $A$ be a noethering and $M$ an $A$-module with $a\in A$. Then, $a_M$ is injective if and only if $a$ does not lie in any of the associated primes of $M$.
\end{lemma}
\begin{proof}
    If $a_M$ is injective, then $a$ is not in the annihilator of any nonzero element, therefore, not an element of any associated prime. On the other hand, suppose $a_M$ is not injective. Then, there is some nonzero $x\in M$ such that $a\in\Ann_A(x)$. Consider the poset of all proper annihilators containing $\Ann_A(x)$. Since $A$ is a noethering, this has a maximal element, say $\frakp$. Note that $\frakp$ is also maximal in the poset of all proper annihilators whence is prime and hence $a$ is contained in an associated prime. This completes the proof.
\end{proof}

\begin{lemma}
    Let $A$ be a noethering and $M$ and $A$-module. Then, every $\frakp\in\Supp(M)$ contains an associated prime.
\end{lemma}
\begin{proof}
    If $\frakp\in\Supp(M)$, then there is some $x\in M$ such that $(Ax)_\frakp\ne0$, consequently, $(Ax)_\frakp$ has an associated prime, say $\frakq$. First, we contend that $\frakq\subseteq\frakp$. Suppose not, then there is some $a\in\frakq\backslash\frakp$. Since $\frakq$ is an associated prime, there is some $0\ne y/s\in (Ax)_\frakp$ such that $\frakq = \Ann_{A_\frakp}(y/s)$. In particular, $by/s = 0$. But $b/1$ is invertible in $A_\frakp$ whence $y/s = 0$, a contradiction. Thus $\frakq\subseteq\frakp$.

    Next, we shall show that $\frakq$ is an associated prime of $M$. Since $A$ is a noethering, $\frakq$ is finitely generated, say by $b_1,\dots,b_n$. Then, $b_iy/s = 0$ for each $i$, consequently, there is some $s_i\notin\frakp$ such that $s_ib_iy = 0$. Let $t = s_1\cdots s_n\notin\frakp$. We contend that $\frakq = \Ann_A(ty)$. Obviously, $\frakq\subseteq\Ann_A(ty)$. On the other hand, if $b\in\Ann_A(ty)$, then $bty = 0$ whereby $by/s = 0$ and $b\in\frakq$. This completes the proof.
\end{proof}


\begin{corollary}
    Let $A$ be a noethering and $M$ an $A$-module with $a\in A$. The following are equivalent: 
    \begin{enumerate}[label=(\alph*)]
        \item $a_M$ is locally nilpotent. 
        \item for each $\frakp\in\Ass(M)$, $a\in\frakp$.
        \item for each $\frakp\in\Supp(M)$, $a\in\frakp$.
    \end{enumerate}
\end{corollary}
\begin{proof}
    $(a)\implies(b)$ is immediate from the definition while $(c)\implies(a)$ has been proven. Both these implications do not require the noethering hypothesis. The implication $(b)\implies(c)$ has been proven above and requires the noethering hypothesis.
\end{proof}

\begin{lemma}
    Let $N$ be a submodule of $M$. Then, 
    \begin{equation*}
        \Ass(N)\subseteq\Ass(M)\subseteq\Ass(N)\cup\Ass(M/N).
    \end{equation*}
\end{lemma}
\begin{proof}
    It is obvious that $\Ass(N)\subseteq\Ass(M)$. Now, let $\frakp\in\Ass(M)$. Then, there is some $x\in M$ such that $\frakp = \Ann_A(x)$. If $x\in N$, then we are done. If not, then consider $Ax\cap N$. If $Ax\cap N = 0$, then, $Ax$ is isomorphic to the image of $Ax$ under the projection $M/N$. Therefore, $\frakp$ is an associated prime of some submodule of $M/N$. On the other hand, if $Ax\cap N\ne 0$, then there is some $y = ax\in N$ for some $a\in A$. 

    Obviously $\frakp$ annihilates $y$. If $b\in A$ annihilates $y$, then $bax = 0$ whence $ba\in\frakp$. But since $y\ne 0$, $a\notin\frakp$ and thus $b\in\frakp$. This completes the proof.
\end{proof}

\begin{lemma}
    Let $S\subseteq A$ be a multiplicative subset and $N$ an $S^{-1}A$-module. Then, 
    \begin{equation*}
        \Ass_{S^{-1} A}(N) = S^{-1}\Ass_A(N)\backslash\{S^{-1}A\},
    \end{equation*}
    where 
    \begin{equation*}
        S^{-1}\Ass_A(N) := \{S^{-1}\frakp\mid\frakp\in\Ass_A(N)\}.
    \end{equation*}
\end{lemma}
\begin{proof}
\end{proof}

\begin{lemma}
    Let $A$ be a noethering, $M$ a non-zero $A$-module and $S$ a multiplicative subset of $A$. Then, 
    \begin{equation*}
        \Ass_{S^{-1}A}(S^{-1}M) = S^{-1}\Ass_A(M)\backslash\{S^{-1}A\}.
    \end{equation*}
\end{lemma}
\begin{proof}
\end{proof}

\begin{corollary}
    Let $A$ be a noethering. Then, 
    \begin{equation*}
        \frakp\in\Ass_A(M)\iff \frakp A_\frakp\in\Ass_{A_\frakp}(M_\frakp).
    \end{equation*}
\end{corollary}
\begin{proof}
    Take $S = A\backslash\frakp$ and the conclusion follows from the above lemma.
\end{proof}

\begin{proposition}
    Let $A$ be a noethering and $M\ne 0$ a finitely generated (and hence, noetherian) $A$-module. Then, there is a filtration 
    \begin{equation*}
        M = M_0\supsetneq M_1\supsetneq\dots\supsetneq M_n = 0
    \end{equation*}
    such that $M_i/M_{i + 1}\cong A/\frakp_i$ for some $\frakp_i\in\Spec(A)$.
\end{proposition}
\begin{proof}
    Since $A$ is a noethering, $\Ass_A(M)$ is non-empty. Pick a prime $\frakp_0\in\Ass_A(M)$. Then, there is an injection $A/\frakp_0\into M$. Then, there is a submodule $N_0$ of $M$ that is isomorphic to $A/\frakp_0$. If $M = N_0$, then we are done. If not, then consider $M/N_0$. This also has an associated prime $\frakp_1$ and hence, there is a submodle $N_1$ of $M$ containing $N_0$ such that $M/N_1\cong A/\frakp_1$. Continuing this way, we obtain a sequence (the finiteness of this sequence requires $M$ to be noetherian):
    \begin{equation*}
        N_0\subsetneq N_1\subsetneq\dots\subsetneq N_n = M
    \end{equation*}
    where $M/N_i\cong A/\frakp_i$ for some $\frakp_i\in\Spec(A)$. This completes the proof.
\end{proof}

\begin{lemma}
    Let $A$ be a noethering. Then, the set of all zero divisors on $M$ is given by 
    \begin{equation*}
        \bigcup_{\frakp\in\Ass_A(M)}\frakp.
    \end{equation*}
\end{lemma}
\begin{proof}
    If $a\in A$ is a zero divisor on $M$, then, the set 
    \begin{equation*}
        \{\fraka\unlhd A\mid a\in\fraka\text{ and }\fraka = \Ann_A(x)\text{ for some }x\in M\}
    \end{equation*}
    admits a maximal element (due to noetherian-ness), say $\frakp$. This is an associated prime and contains $a$. The converse is trivial.
\end{proof}

\begin{lemma}
    Let $A$ be a noethering and $M$ a finitely generated (equivalently, noetherian) $A$-module. Then, $\Ass_A(M)$ is finite.
\end{lemma}
\begin{proof}
    As we have seen earlier, $M$ admits a filtration 
    \begin{equation*}
        M = M_0\supsetneq M_1\supsetneq\dots\supsetneq M_n = 0
    \end{equation*}
    where $M_i/M_{i + 1}\cong A/\frakp_i$ for some $\frakp_i\in\Spec(A)$. We have short exact sequences 
    \begin{equation*}
        0\longrightarrow M_{i + 1}\longrightarrow M_i\longrightarrow M_i/M_{i + 1}\longrightarrow 0.
    \end{equation*}
    Then, 
    \begin{equation*}
        \Ass_A(M_i)\subseteq\Ass_A(M_{i + 1})\cup\Ass_A(M_i/M_{i + 1}) = \Ass_A(M_{i + 1})\cup\{\frakp_i\}.
    \end{equation*}
    Inductively, we see that 
    \begin{equation*}
        \Ass_A(M)\subseteq\{\frakp_0,\dots,\frakp_{n - 1}\}.\qedhere
    \end{equation*}
\end{proof}

\begin{lemma}
    Let $A$ be any ring. Then,
    \begin{equation*}
        \Ass_A(M)\subseteq\Supp_A(M).
    \end{equation*}
\end{lemma}
\begin{proof}
    Let $\frakp\in\Ass_A(M)$. Then, there is an injection $A/\frakp\into M$. Localizing at $\frakp$, we have an injection $Q(A/\frakp)\into M_\frakp$. Thus, $M_\frakp\ne 0$ and $\frakp\in\Supp_A(M)$.
\end{proof}

\begin{lemma}
    Let $A$ be a noethering. The minimal elements of $\Ass_A(M)$ and $\Supp_A(M)$ are the same.
\end{lemma}
\begin{proof}
    Let $\frakp\in\Ass_A(M)$ be minimal. We have seen that $\frakp\in\Supp_A(M)$. Suppose $\frakq\in\Supp_A(M)$ with $\frakq\subseteq\frakp$. Note that 
    \begin{equation*}
        \Ass_{A_\frakq}(M_\frakq) = \left(\Ass_A(M)\right)_{\frakq}\backslash\{A_\frakq\} = \emptyset.
    \end{equation*}
    Therefore, $M_\frakq = 0$ and $\frakq\notin\Supp_A(M)$. This shows that the minimal primes of $\Ass_A(M)$ are a subset of the minimal primes of $\Supp_A(M)$.

    Conversely, suppose $\frakp\in\Supp_A(M)$ is minimal. Then, 
    \begin{equation*}
        \emptyset\ne\Ass_{A_\frakp}(M_\frakp) = \left(\Ass_A(M)\right)_\frakp\backslash\{A_\frakp\},
    \end{equation*}
    where the first ``equality'' follows from the fact that $M_\frakp\ne 0$. Hence, there is a prime ideal $\frakq\subseteq\frakp$ that is an associated prime of $M$ and hence, also lies in the support of $M$. It follows that $\frakq = \frakp$ whence $\frakp\in\Ass_A(M)$. This completes the proof.
\end{proof}


\section{Primary Decomposition of Modules}

\begin{definition}
    Let $M$ be an $A$-module. A submodule $Q$ of $M$ is said to be \emph{primary} if $Q\ne M$ and for each $a\in A$, the homomorphism $a_{M/Q}$ is either injective or nilpotent.

    Equivalently, the above definition implies that if $a_{M/Q}$ is a zero-divisor, then it is nilpotent.
\end{definition}


\begin{proposition}
    Let $Q$ be a primary submodule of $M$. Then, 
    \begin{equation*}
        \frakp := \{a\in A\mid a_{M/Q}\text{ is nilpotent}\}
    \end{equation*}
    is a prime ideal.
\end{proposition}
\begin{proof}
    Let $ab\in\frakp$, that is, $(ab)_{M/Q}$ is nilpotent. If $a\notin\frakp$, then $a_{M/Q}$ is injective and thus $b_{M/Q}$ is nilpotent, i.e. $b\in\frakp$. This completes the proof.
\end{proof}
