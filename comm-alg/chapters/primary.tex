A primary ideal is a generalization of the ideals $p^n\Z$ in $\Z$, as is evident from the following definition.

\begin{definition}[Primary Ideals]
    An ideal $\frakq\subseteq A$ is said to be \textit{primary} if 
    \begin{equation*}
        xy\in\frakq\Longrightarrow x\in\frakq\text{ or }y^n\in\frakq\text{ for some } n > 0
    \end{equation*}
\end{definition}


From the definition, we see that every prime ideal is primary. It is not hard to see that
\begin{itemize}
\item $\frakq$ is primary if and only if every zero divisor in $A/\frakq$ is nilpotent. 
\item $\frakq$ is primary if and only if $(0)$ is primary in $A/\frakq$.
\end{itemize}


\begin{proposition}
    If $\frakq$ is primary, then $\sqrt{\frakq}$ is prime. Further, $\sqrt{\frakq}$ is the smallest prime ideal containing $\frakq$.
\end{proposition}
\begin{proof}
    Suppose $xy\in\sqrt{\frakq}$, then there is $n > 0$ such that $x^ny^n\in\frakq$, consequently, there is an $m > 0$ such that $x^n\in\frakq$ or $y^{mn}\in\frakq$, therefore, $x\in\sqrt{\frakq}$ or $y\in\sqrt{\frakq}$, whence $\sqrt{\frakq}$ is prime. The second assertion is trivial.
\end{proof}

If $\frakq$ is a primary ideal, then $\frakp = \sqrt{\frakq}$ is called the \textit{associated prime ideal} of $\frakq$ and $\frakq$ is said to be \textit{$\frakp$-primary}.

Consider the ring $A = k[x,y]$ and the ideal $\frakq = (x,y^2)$. The quotient ring $A/\frakq$ is isomorphic to $k[y]/(y^2)$ where every zero divisor is nilpotent consequently, $\frakq$ is primary. The radical ideal $\frakp = \sqrt{\frakq} = (x,y)$ is a prime ideal such that $\frakp^2\subsetneq\frakq\subsetneq\frakp$, therefore, $\frakq$ is not a prime power.

On the other hand, consider the ring $A = k[x,y,z]/(xy - z^2)$ and the prime ideal $\frakp = (\overline x,\overline z)\subseteq A$. We contend that $\frakp^2\subseteq A$ is not primary. Indeed, note that $\overline x\overline y = \overline z^2\in\frakp^2$ but $\overline x\notin\frakp^2$ and $\overline y\notin\frakp^2$, and the conclusion follows.

\begin{proposition}
    If $\sqrt{\fraka}$ is maximal, then $\fraka$ is primary.
\end{proposition}
\begin{proof}
    Let $\frakm = \sqrt{\fraka}$ and $\phi: A\to A/\fraka$ denote the natural map. Then, $\phi(\sqrt{\fraka})$ is the maximal ideal in $A/\fraka$ and is also the nilradical of $A/\fraka$, consequently, $A/\fraka$ is local and every non-unit is nilpotent. Hence, $\fraka$ is primary.
\end{proof}

\begin{lemma}\thlabel{lem:intersection-p-primary}
    If $\{\frakq_i\}_{i = 1}^n$ are $\frakp$-primary, then so is $\frakq=\bigcap_{i = 1}^n\frakq_i$.
\end{lemma}
\begin{proof}
    Obviously, 
    \begin{equation*}
        \sqrt{\frakq} = \bigcap_{i = 1}^n\sqrt{\frakq_i} = \frakp
    \end{equation*}
    Let $xy\in\frakq$. If $y\in\frakp$, then we are done, since $\frakp = \sqrt{\frakq}$. Else, $y^n\notin\frakq_i$ for every positive integer $n$, since $\frakp = \sqrt{\frakq_i}$ whereby $x\in\frakq_i$ for each $1\le i\le n$ and the conclusion follows.
\end{proof}

\begin{lemma}
    Let $\frakq$ be a $\frakp$-primary ideal and $x\in A$. Then 
    \begin{enumerate}[label=(\alph*)]
        \item if $x\in\frakq$, then $(\frakq: x) = (1)$.
        \item if $x\notin\frakq$, then $(\frakq: x)$ is $\frakp$-primary.
        \item if $x\notin\frakp$, then $(\frakq: x) = \frakq$.
    \end{enumerate}
\end{lemma}
\begin{proof}
\begin{enumerate}[label=(\alph*)]
    \item Trivial.
    \item If $y\in(\frakq:x)$, then $xy\in\frakq$, therefore, $y\in\frakp$. Thus, we have $\frakq\subseteq(\frakq:x)\subseteq\frakp$. Taking radicals, $\frakp\subseteq\sqrt{(\frakq:x)}\subseteq\frakp$, whereby $\sqrt{(\frakq: x)} = \frakp$.

    On the other hand, if $yz\in(\frakq: x)$, then $xyz\in\frakq$. If $z\in\frakp$, then we are done. Else, $xy\in\frakq$ and $y\in(q:x)$ whence $(q: x)$ is $\frakp$-primary.
    \item If $y\in(q: x)$, then $yx\in\frakq$. Since $x\notin\frakp$, we must have $y\in\frakq$. This completes the proof.
\end{enumerate}
\end{proof}

\begin{definition}[Primary Decomposition]
    A \textit{primary decomposition} of an ideal $\fraka\subseteq A$ is an expression of $\fraka$ as a \textit{finite} intersection of primary ideals.
    \begin{equation*}
        \fraka = \bigcap_{i = 1}^n\frakq_i
    \end{equation*}
    The ideal $\fraka$ is said to be \textit{decomposable} if it has a primary decomposition. Moreover, if for all $1\le i\le n$, $\sqrt{\frakq_i}$ are distinct and 
    \begin{equation*}
        \bigcap_{j\ne i}\frakq_j\not\subseteq\frakq_i
    \end{equation*}
    then the primary decomposition is said to be \textit{minimal}.
\end{definition}

Using \thref{lem:intersection-p-primary}, it is not hard to see that every decomposable ideal has a minimal decomposition.

\begin{theorem}[First Uniqueness Theorem]\thlabel{thm:first-uniqueness-theorem}
    Let $\fraka\subseteq A$ be a decomposable ideal and 
    \begin{equation*}
        \fraka = \bigcap_{i = 1}^n\frakq_i
    \end{equation*}
    be a minimal primary decomposition with $\frakp_i = \sqrt{\frakq_i}$. Then, the $\frakp_i$'s are precisely the prime ideals the occur in the set $\{\sqrt{(\fraka: x)}\mid x\in A\}$.
\end{theorem}
\begin{proof}
    
\end{proof}