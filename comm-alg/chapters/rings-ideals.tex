\section{Nilradical and Jacobson radical}

\begin{definition}[Multiplicatively Closed]
    A subset $S\subseteq A$ is said to be \textit{multiplicatively closed} if 
    \begin{enumerate}[label=(\alph*)]
        \item $1\in S$ 
        \item for all $x,y\in S$, $xy\in S$
    \end{enumerate}
\end{definition}

\begin{proposition}
    Let $S\subsetneq A\backslash\{0\}$ be a multiplicatively closed subset. Then, there is a prime ideal $\mathfrak p$ disjoint from $S$.
\end{proposition}

\section{Local Rings}
\begin{definition}
    A commutative ring $A$ is said to be local if it has a unique maximal ideal.
\end{definition}

\begin{proposition}\thlabel{prop:non-unit-ideal-local}
    $A$ is local if and only if the subset of non-units form an ideal.
\end{proposition}

Obviously, a field $k$ is a local ring. On the other hand, the polynomial ring $k[x]$ is not local, since both $x$ and $1 - x$ are non-units but their sum is a unit. 

We contend that the ring $A = k[x_1,x_2,\ldots]/(x_1,x_2,\ldots)^2$ is local. Indeed, let $\pi$ denote the canonical map $k[x_1,x_2,\ldots]\to A$ and $\mathfrak m$ be maximal in $A$. Then, $\pi^{-1}(\mathfrak m)$ is maximal in $k[x_1,x_2,\ldots]$ and contains $(x_1,x_2,\ldots)^2$, therefore, contains $(x_1,x_2,\ldots)$. But the latter is maximal and therefore, $\pi^{-1}(\mathfrak m) = (x_1,x_2,\ldots)$ whence the maximal ideal is unique. Thus $A$ is local..


\section{Operations on Ideals}
\section{The Zariski Topology}