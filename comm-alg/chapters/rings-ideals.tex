\begin{definition}[Krull Dimension]
    A sequence $\{\frakp_0,\ldots,\frakp_n\}$ of prime ideals in $A$ is said to be strictly ascending of length $n$ if $\frakp_0\subsetneq\cdots\subsetneq\frakp_n$. The \textit{Krull dimension} of $A$ is defined to be the supremum of the lengths of all strictly ascending sequences of prime ideals in $A$ and is denoted by $\dim A$.
\end{definition}

\begin{proposition}
    Let $A$ and $B$ be rings. Then, every prime ideal in $A\times B$ is of the form $\frakp\times B$ where $\frakp\subseteq A$ is a prime ideal or $A\times\frakq$ where $\frakq\subseteq B$ is a prime ideal.
\end{proposition}
\begin{proof}
    It is known that ideals in $A\times B$ are of the form $\fraka\times\frakb$ where $\fraka$ and $\frakb$ are ideals in $A$ and $B$ respectively. Consequently, the quotient 
    \begin{equation*}
        A\times B/\fraka\times\frakb\cong A/\fraka\times B/\frakb
    \end{equation*}
    For $\fraka\times\frakb$ we require $A/\fraka\times B/\frakb$ to be an integral domain. This is possible if and only if either $\fraka$ is a prime and $\frakb = B$ or $\fraka = A$ and $\frakb$ is a prime. This completes the proof.
\end{proof}

\section{Nilradical and Jacobson radical}

\begin{definition}[Multiplicatively Closed]
    A subset $S\subseteq A$ is said to be \textit{multiplicatively closed} if 
    \begin{enumerate}[label=(\alph*)]
        \item $1\in S$ 
        \item for all $x,y\in S$, $xy\in S$
    \end{enumerate}
\end{definition}

\begin{proposition}
    Let $S\subsetneq A\backslash\{0\}$ be a multiplicatively closed subset. Then, there is a prime ideal $\mathfrak p$ disjoint from $S$.
\end{proposition}

\section{Local Rings}
\begin{definition}
    A commutative ring $A$ is said to be local if it has a unique maximal ideal.
\end{definition}

\begin{proposition}\thlabel{prop:non-unit-ideal-local}
    $A$ is local if and only if the subset of non-units form an ideal.
\end{proposition}

Obviously, a field $k$ is a local ring. On the other hand, the polynomial ring $k[x]$ is not local, since both $x$ and $1 - x$ are non-units but their sum is a unit. 

We contend that the ring $A = k[x_1,x_2,\ldots]/(x_1,x_2,\ldots)^2$ is local. Indeed, let $\pi$ denote the canonical map $k[x_1,x_2,\ldots]\to A$ and $\mathfrak m$ be maximal in $A$. Then, $\pi^{-1}(\mathfrak m)$ is maximal in $k[x_1,x_2,\ldots]$ and contains $(x_1,x_2,\ldots)^2$, therefore, contains $(x_1,x_2,\ldots)$. But the latter is maximal and therefore, $\pi^{-1}(\mathfrak m) = (x_1,x_2,\ldots)$ whence the maximal ideal is unique. Thus $A$ is local.


\section{Operations on Ideals}

Obviously, the intersection $\fraka\cap\frakb$ of two ideals is an ideal. The sum of ideals is defined as the following collection
\begin{equation*}
    \sum_{i\in I}\fraka_i = \left\{\sum_{\text{finite }i\in I}a_i\bigg\vert~a_i\in\fraka_i\right\}
\end{equation*}
It is not hard to argue that the sum is the smallest ideal containing the ideals $\{\fraka_i\}_{i\in I}$. The product of two ideals is defined as 
\begin{equation*}
    \fraka\frakb = \left\{\sum_{\text{finite}}a_ib_i\bigg\vert~a_i\in\fraka,~b_i\in\frakb\right\}
\end{equation*}
Inductively, we may define powers of an ideal as $\fraka^n = \fraka\fraka^{n - 1}$ with the convention that $\fraka^0 = (1) = A$.

\begin{proposition}
    Let $\fraka,\frakb,\frakc\subseteq A$ be ideals. Then, 
    \begin{equation*}
        \fraka(\frakb + \frakc) = \fraka\frakb + \fraka\frakc
    \end{equation*}
\end{proposition}
\begin{proof}
    Obviously, $\fraka\frakb\subseteq\fraka(\frakb + \frakc)$ and $\fraka\frakc\subseteq\fraka(\frakb + \frakc)$ and thus, $\fraka\frakb + \fraka\frakc\subseteq\fraka(\frakb + \frakc)$. On the other hand, any element of $\fraka(\frakb + \frakc)$ is a finite sum of the form $\sum_{i} a_i(b_i + c_i)$ which can be rearranged as $\sum_{i} a_ib_i + \sum_{i}a_ic_i\in\fraka\frakb + \fraka\frakc$. This completes the proof.
\end{proof}

\begin{proposition}\thlabel{prop:prime-containment}
    \begin{enumerate}[label=(\alph*)]
    \item Let $\frakp_1,\ldots,\frakp_n$ be prime ideals and let $\fraka$ be an ideal contained in $\bigcup_{i = 1}^n\frakp_i$. Then $\fraka\subseteq\frakp_i$ for some $1\le i\le n$.
    \item Let $\fraka_1,\ldots,\fraka_n$ be ideals and let $\frakp$ be a prime ideal containing $\bigcap_{i = 1}^n\fraka_i$. Then $\fraka_i\subseteq\frakp$ for some $i$.
    \end{enumerate}
\end{proposition}

For ideals $\fraka,\frakb\subseteq A$, define their ideal quotient as 
\begin{equation*}
    (\fraka:\frakb) = \{x\in A\mid x\frakb\subseteq\fraka\}
\end{equation*}

\begin{proposition}
    Let $\fraka,\frakb,\frakc\subseteq A$ be ideals. Then 
    \begin{enumerate}
        \item $(\fraka:\frakb)\frakb\subseteq\fraka$
        \item $((\fraka:\frakb):\frakc) = (\fraka:\frakb\frakc)$
        \item $(\bigcap_{i\in I}\fraka_i:\frakb) = \bigcap_{i\in I}(\fraka_i:\frakb)$
    \end{enumerate}
\end{proposition}

\begin{proposition}
    If every prime ideal in $A$ is principal, then $A$ is a principal ring.
\end{proposition}
\begin{proof}
    Suppose not. Let $\Sigma$ be the poset of ideals in $A$ that are not principal, ordered by inclusion and $\{\fraka_i\}_{i\in I}$ be a chain in $\Sigma$. Let $\fraka = \bigcup_{i\in I}\fraka_i$. We claim that $\fraka$ is not principal, for if it were, then $\fraka = (a)$ for some $a\in A$. Then, $a\in\fraka_i$ for some $i\in I$ whence $\fraka_i = (a)$, a contradiction. Hence, every chain in $\Sigma$ has an upper bound, therefore, $\Sigma$ has a maximal element, say $\frakp$.

    We contend that $\frakp$ is a prime ideal. Suppose not, then there are $a,b\notin\frakp$ such that $ab\in\frakp$. \textcolor{red}{Add in later}
\end{proof}

\begin{proposition}
    Let $A$ be a UFD. Then $A$ is a PID if and only if $\dim A\le 1$.
\end{proposition}

\subsection{Radical Ideals}

\begin{definition}[Radical Ideal]
    For an ideal $\fraka\subseteq A$, we define its \textit{radical} as 
    \begin{equation*}
        \sqrt{\fraka} = \{x\in A\mid x^n\in\fraka\text{ for some }n\in\N\}
    \end{equation*}
    An ideal which is the radical of some ideal is called a \textit{radical ideal}.
\end{definition}

Obviously, $\fraka\subseteq\sqrt{\fraka}$. From our definition, it is not hard to see that the radical is the smallest radical ideal that contains a certain ideal. As a result, if $\fraka\subseteq\frakb$ are ideals, then $\sqrt{\fraka}\subseteq\sqrt{\frakb}$.

\begin{proposition}
    Let $\fraka,\frakb\subseteq A$ be ideals. Then, 
    \begin{enumerate}[label=(\roman*)]
        \item $\sqrt{\sqrt{\fraka}} = \sqrt{\fraka}$
        \item $\sqrt{\fraka\frakb} = \sqrt{\fraka\cap\frakb} = \sqrt{\fraka}\cap\sqrt{\frakb}$
        \item $\sqrt{\fraka^n} = \sqrt{\fraka}$ for every $n\in\N$
        \item $\sqrt{\fraka + \frakb} = \sqrt{\sqrt{\fraka} + \sqrt{\frakb}}$
    \end{enumerate}
\end{proposition}
\begin{proof}
\begin{enumerate}[label=(\roman*)]
    \item Trivial.
    \item Since $\fraka\frakb\subseteq\fraka\cap\frakb$, we must have $\sqrt{\fraka\frakb}\subseteq\sqrt{\fraka\cap\frakb}$. On the other hand, if $x\in\sqrt{\fraka\cap\frakb}$, there is a positive integer $n$ such that $x^n\in\fraka\cap\frakb$, therefore, $x^{2n}\in\fraka\frakb$, and $x\in\sqrt{\fraka\frakb}$. This establishes the first equality.

    As for the second inequality, if $x\in\sqrt{\fraka\cap\frakb}$, then there is a positive integer $n$ such that $x^n\in\fraka\cap\frakb$, therefore, $x\in\sqrt{\fraka}$ and $x\in\sqrt{\frakb}$. Conversely, if $x\in\sqrt{\fraka}\cap\sqrt{\frakb}$, then there are positive integers $m$ and $n$ such that $x^m\in\fraka$ and $x^n\in\frakb$, consequently, $x^{m + n}\in\fraka\cap\frakb$, and the conclusion follows. 

    \item Immediate from $(ii)$.
    \item Obviously, $\sqrt{\fraka + \frakb}\subseteq\sqrt{\sqrt{\fraka} + \sqrt{\frakb}}$. On the other hand, note that $\sqrt{\fraka + \frakb}$ is a radical ideal containing $\sqrt{\fraka}$ and $\sqrt{\frakb}$, therefore, contains $\sqrt{\fraka} + \sqrt{\frakb}$. Hence, $\sqrt{\fraka + \frakb}\supseteq\sqrt{\sqrt{\fraka} + \sqrt{\frakb}}$ and the conclusion follows.
\end{enumerate}
\end{proof}
\begin{corollary}
    Ideals $\fraka$ and $\frakb$ are comaximal if and only if $\sqrt{\fraka}$ and $\sqrt{\frakb}$ are comaximal.
\end{corollary}

For a prime ideal $\frakp$, note that $\sqrt{\frakp} = \frakp$ and due to $(iii)$, $\sqrt{\frakp^n} = \frakp$ for every positive integer $n$. 

\begin{proposition}
    Let $\fraka\subseteq A$ be an ideal with maximal radical. Then $A/\fraka$ is a local ring of dimension $0$.
\end{proposition}
\begin{proof}
    Let $\overline\frakm$ be a maximal ideal in $A/\fraka$. Since $\overline\frakm$ is prime, its preimage in $A$ is a prime ideal $\frakm$ containing $\fraka$, thus, it must contain $\sqrt{\fraka}$, which is maximal, whence $\frakm = \sqrt{\fraka}$. Consequently $\overline\frakm = \sqrt{\fraka}/\fraka$ and is uniquely determined.

    On the other hand, if $\overline\frakp$ is a prime ideal in $A/\fraka$, using a similar argument as above, one may conclude that $\overline\frakp$ is maximal and thus $\dim(A/\fraka) = 0$.
\end{proof}

\section{Extension and Contraction of Ideals}

\begin{definition}
    Let $\phi: A\to B$ be a ring homomorphism. If $\fraka\subseteq A$ is an ideal, then we define its extension $\fraka^e = \phi(\fraka)A$. If $\frakb\subseteq B$ is an ideal, then we define its contraction $\frakb^c = \phi^{-1}(\frakb)$.
\end{definition}

\begin{proposition}
    \begin{enumerate}[label=(\alph*)]
        \item $\fraka\subseteq\fraka^{ec}$ and $\frakb\supseteq\frakb^{ce}$ 
        \item $\frakb^c = \frakb^{cec}$ and $\fraka^{e} = \fraka^{ece}$
        \item If $C$ is the set of contracted ideals in $A$ and $E$ is the set of extended ideals in $B$, then $\fraka\mapsto\fraka^e$ is a bijection from $C$ to $E$.
    \end{enumerate}
\end{proposition}
\begin{proof}
\begin{enumerate}[label=(\alph*)]
    \item Trivial.
    \item We have $\fraka^{e}\subseteq(\fraka^{ec})^e$ and $\fraka^e\supseteq(\fraka^e)^{ce}$. Similarly, $\frakb^c\supseteq (\frakb^c)^{ec}$ and $\frakb^{c}\subseteq(\frakb^{c})^{ec}$ whence $\frakb^c = \frakb^{cec}$.
    \item Simply note that the maps $\fraka\mapsto\fraka^e$ and $\frakb\mapsto\frakb^c$ from $C$ to $E$ and $E$ to $C$ are inverses to one another.
\end{enumerate}
\end{proof}

\section{The Zariski Topology}

\begin{definition}[Prime Spectrum]
    For a commutative ring $A$, define 
    \begin{equation*}
        \spec A = \{\mathfrak p\mid\mathfrak p\text{ is a prime ideal in }A\}
    \end{equation*}
    This is called the \textit{prime spectrum} of the ring. Similarly, define 
    \begin{equation*}
        \mspec A = \{\mathfrak m\mid\mathfrak m\text{ is a maximal ideal in } A\}
    \end{equation*}
\end{definition}

For each $E\subseteq A$, define 
\begin{equation*}
    V(E) = \{\frakp\in\spec A\mid E\subseteq\frakp\}
\end{equation*}

\begin{proposition}
\begin{enumerate}[label=(\alph*)]
    \item If $\mathfrak a$ is the ideal generated by $E$, then $V(E) = V(\mathfrak a) = V(\sqrt{\fraka})$
    \item $V(0) = X$ and $V(1) = \emptyset$ 
    \item If $\{E_i\}_{i\in I}$ is a family of subsets of $A$, then 
    \begin{equation*}
        V\left(\bigcup_{i\in I}E_i\right) = \bigcap_{i\in I}V(E_i)
    \end{equation*}
\end{enumerate}
\end{proposition}

It is not hard to see that the collection 
\begin{equation*}
    \mathcal T = \{\spec A\backslash V(E)\mid E\subseteq A\}
\end{equation*}
is a topology on $\spec A$. This is known as the \textit{Zariski Topology}. In particular, $V(E)$ form closed subsets in $\spec A$ under the Zariski topology. 

\begin{proposition}
    For each $f\in A$, let $D(f) = \spec A\backslash V(f)$. Then, the collection $\{D(f)\}_{f\in A}$ forms a basis for the Zariski topology on $\spec A$.
\end{proposition}

\begin{proposition}
    Let $f: A\to B$ be a ring homomorphism. Then, the map $f_*:\spec B\to\spec A$ given by $f_*(\mathfrak q) = f^{-1}(\mathfrak p)$ is a continuous map. Further, if $g: B\to C$ is a ring homomorphism, then $(g\circ f)_* = f_*\circ g_*$.
\end{proposition}
\begin{proof}
    Let $\mathfrak a\subseteq A$ be an ideal. We shall show that $f_*^{-1}(V(\mathfrak a))$ is closed in $B$. Note that 
    \begin{align*}
        f_*^{-1}(V_A(\mathfrak a)) &= \{\frakp\mid \mathfrak a\subseteq f_*(\frakp)\}\\
        &= \{\frakp\in\spec B\mid\mathfrak a\subseteq f^{-1}(\frak p)\}\\
        &= V_B((f(\mathfrak a)))
    \end{align*}
    whence the conclusion follows.

    Next, for any $\frakp\in\spec C$, we have 
    \begin{equation*}
        (f_*\circ g_*)(\frakp) = f_*(g^{-1}(\frakp)) = f^{-1}(g^{-1}(\frakp)) = (g\circ f)^{-1}(\frakp)
    \end{equation*}
    This completes the proof.
\end{proof}

This shows that $\spec$ is a contravariant functor from $\catCRing$ to $\catTop$. 

\subsection{On the Topological Properties}
\begin{proposition}
    $\spec A$ is Hausdorff if and only if $\dim A = 0$.
\end{proposition}
\begin{proof} 
$(\Longrightarrow)$ We shall show that if $\spec A$ is $T_1$, then $\dim A = 0$. Indeed, if $\spec A$ is $T_1$, then $\{\frakp\}$ is a closed set for very prime ideal $\frakp$, therefore, there is an ideal $I\subseteq A$ such that $V(I) = \{\frakp\}$. As a result, $V(\frakp) = \{\frakp\}$ and $\frakp$ is maximal.

$(\Longleftarrow)$ Suppose $\dim A = 0$. Let $\frakp$ and $\mathfrak q$ be distinct ideals. We contend that there are $f\notin\frakp$ and $g\notin\frakq$ such that $fg$ is contained in every prime ideal in $A$, equivalently, $fg$ is contained in $\mathfrak N(A)$. Suppose not, that is, for every pair $f\notin\frakp$ and $g\notin\frakq$, there is a prime ideal $\frakp$ disjoint from $\{f,g\}$.

Let $X = A\backslash(\frakp\cap\frakq)$. Let $\Sigma$ be the collection of ideals $\mathfrak a$ contained in $\frakp\cap\frakq$ such that for every finite subset $F\subseteq X$, there is a prime ideal $\mathfrak P$ containing $\mathfrak a$ that is disjoint from $F$. It is not hard to see that $(0)\in\Sigma$ and that every ascending chain has an upper bound given by the union of all elements in the chain.

Let $J$ be a maximal element in $\Sigma$ whose existence is guaranteed due to Zorn's Lemma. We shall show that $J$ is prime. Indeed, let $xy\in J$ with $y\notin J$. Then, $J + (y)\notin\Sigma$, therefore, there is a finite subset $F_0\subseteq X$ such that for each prime ideal $\mathfrak P$ containing $J + (y)$, $\mathfrak P\cap F_0\ne\emptyset$.

Now, let $F\subseteq X$ be finite, then so is $F\cup F_0$, therefore, there is a prime ideal $I$ containing $J$ such that $I\cap(F\cup F_0) = \emptyset$, which implies that $y\notin I$, lest $J + (y)\subseteq I$. But since $xy\in J\subseteq I$, we must have that $x\in I$. This shows that $J + (x)\subseteq I$, therefore, $(J + (x))\cap F = \emptyset$ whence $J + (x)\in\Sigma$ and $x\in J$ due to the maximality. This shows that $J$ is prime.

Finally, we see that if there is a prime ideal $J$ contained in $\frakp\cap\frakq$, contradicting $\dim A = 0$. Thus, there is $f\notin\frakp$ and $g\notin\frakq$ such that $fg$ is contained in $\mathfrak N(A)$. Consider the basic open sets $D(f)$ and $D(g)$, which contain $\frakp$ and $\frakq$ respectively and their intersection $D(f)\cap D(g) = D(fg)$ is the empty set since $fg$ is contained in ever prime ideal, thus, $\spec A$ is Hausdorff.
\end{proof}

\begin{corollary}
    If $\spec A$ is $T_1$, then $\spec A$ is Hasudorff.
\end{corollary}

