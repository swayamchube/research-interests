\documentclass[oneside]{report}
\usepackage{../swayam}

\usepackage{stmaryrd}

\newcommand{\setItemNumber}[1]{\setcounter{enumi}{\numexpr#1-1\relax}}

\title{Commutative Algebra}
\author{Swayam Chube}
\date{\today}

\begin{document}
\maketitle

\begin{abstract}
    Throughout this report, unless mentioned otherwise, all rings are assumed to be commutative. The term \textit{noethering} is a portmanteau that is used in place of ``noetherian ring'' and is attributed to the accidental genius of \href{https://youtu.be/RrjJfyEF7Ak?t=1259}{Aryaman Maithani}.
\end{abstract}

\tableofcontents


% \part{Theory Building}
\chapter{Rings and Ideals}
\section{Nilradical and Jacobson radical}

\begin{definition}[Multiplicatively Closed]
    A subset $S\subseteq A$ is said to be \textit{multiplicatively closed} if 
    \begin{enumerate}[label=(\alph*)]
        \item $1\in S$ 
        \item for all $x,y\in S$, $xy\in S$
    \end{enumerate}
\end{definition}

\begin{proposition}
    Let $S\subsetneq A\backslash\{0\}$ be a multiplicatively closed subset. Then, there is a prime ideal $\mathfrak p$ disjoint from $S$.
\end{proposition}

\section{Local Rings}
\begin{definition}
    A commutative ring $A$ is said to be local if it has a unique maximal ideal.
\end{definition}

\begin{proposition}\thlabel{prop:non-unit-ideal-local}
    $A$ is local if and only if the subset of non-units form an ideal.
\end{proposition}

Obviously, a field $k$ is a local ring. On the other hand, the polynomial ring $k[x]$ is not local, since both $x$ and $1 - x$ are non-units but their sum is a unit. 

We contend that the ring $A = k[x_1,x_2,\ldots]/(x_1,x_2,\ldots)^2$ is local. Indeed, let $\pi$ denote the canonical map $k[x_1,x_2,\ldots]\to A$ and $\mathfrak m$ be maximal in $A$. Then, $\pi^{-1}(\mathfrak m)$ is maximal in $k[x_1,x_2,\ldots]$ and contains $(x_1,x_2,\ldots)^2$, therefore, contains $(x_1,x_2,\ldots)$. But the latter is maximal and therefore, $\pi^{-1}(\mathfrak m) = (x_1,x_2,\ldots)$ whence the maximal ideal is unique. Thus $A$ is local..


\section{Operations on Ideals}
\section{The Zariski Topology}

\chapter{Modules}
\section{Introduction}
Throughout this section, $R$ denotes a general ring which need not be commutative.

\begin{definition}[Module]
    A left $R$-module is an abelian group $(M,+)$ along with a ring action, that is, a ring homomorphism $\mu: R\to\End(M)$.
\end{definition}

Henceforth, unless specified otherwise, an \textit{$R$-module} refers to a \textit{left $R$-module}. Trivially note that $R$ is an $R$-module, so is any ideal in $R$ and so is every quotient ring $R/I$ where $I$ is an ideal in $R$. When $R$ is a field, an $R$-module is the same as a vector space.

Every abelian group $G$ trivially forms a $\Z$-module. Using this and the forthcoming \textit{Structure Theorem for Finitely Generated Modules over a PID}, we obtain the \textit{Structure Theorem for Finitely Generated Abelian Groups}.

\begin{definition}[Submodule]
    Let $M$ be an $R$-module. An $R$-submodule of $M$ is a subgroup $N$ of $M$ which is closed under the action of $R$.
\end{definition}

\begin{proposition}[Submodule Criteria]
    Let $M$ be an $R$-module. Then $\emptyset\subsetneq N\subseteq M$ is a submodule if and only if for all $x,y\in N$ and $r\in R$, $x + ry\in N$.
\end{proposition}
\begin{proof}
    Straightforward definition pushing.
\end{proof}

\begin{definition}[Module Homomorphism]
    Let $M, N$ be $R$-modules. A \textit{module homomorphism} is a group homomorphism $\phi: M\to N$ such that for all $x\in M$ and $r\in R$, $\phi(rx) = r\phi(x)$.
\end{definition}

In other words, a module homomorphism is simply an $R$-linear map. 

\begin{proposition}[Homomorphism Criteria]
    Let $M, N$ be $R$-modules. Then $\phi: M\to N$ is an $R$-module homomorphism if and only if for all $x,y\in M$ and $r\in R$, $\phi(x + ry) = \phi(x) + r\phi(y)$.
\end{proposition}
\begin{proof}
    Straightforward definition pushing.
\end{proof}

It is not hard to see, using the above proposition and the submodule criteria that the image of an $R$-module under a homomorphism is a submodule.

For $R$-modules $M,N$, we denote the set of all $R$-module homomorphisms from $M$ to $N$ by $\Hom_R(M,N)$. When the choice of the ring $R$ is clear from the context, we shall denote this set by $\Hom(M,N)$.

\begin{proposition}
    Let $M,N$ be $R$-modules. Then $\Hom(M,N)$ forms an $R$-module.
\end{proposition}
\begin{proof}
    It is obvious that $\Hom(M,N)$ has the structure of an abelian group. Define the natural action by $(rf)(x) = rf(x)$. It is not hard to see that this action is well defined.
\end{proof}

\begin{proposition}
    Let $\phi: M\to N$ be an $R$-module homomorphism. Then, for every $R$-module $P$, there is an induced $R$-module homomorphism $\overline\phi:\Hom(N,P)\to\Hom(M,P)$ and an induced $R$-module homomorphism $\widetilde\phi:\Hom(P,M)\to\Hom(P,N)$. 
    
    Equivalently phrased, $\Hom(-,P)$ is a contravariant functor while $\Hom(P,-)$ is a covariant functor.
\end{proposition}
\begin{proof}
    We shall prove only the first half of the assertion since the second half follows from a similar proof. Define $\overline\phi$ using the following commutative diagram: 
    \begin{equation*}
    \xymatrix {
        M\ar[r]^{\phi}\ar@{.>}[rd]_{f\circ\phi} & N\ar[d]^f\\
        & P
    }
    \end{equation*}
    To see that this is indeed an $R$-module homomorphism, we need only verify that for all $f,g\in\Hom(N,P)$ and all $r\in R$, $(f + rg)\circ\phi = f\circ\phi + rg\circ\phi$ which is trivial to check.
\end{proof}

\begin{definition}[Kernel, Cokernel]
    Let $\phi:M\to N$ be an $R$-module homomorphism. We define 
    \begin{equation*}
        \ker\phi = \{x\in M\mid\phi(x) = 0\}\qquad\coker\phi = N/\phi(M)
    \end{equation*}
\end{definition}

For an $R$-module $M$, define the annihilator of $M$ in $R$ as 
\begin{equation*}
    \Ann(M) = \{r\in R\mid rx = 0~\forall x\in M\}
\end{equation*}
It is trivial to check that $\Ann(M)$ is a left ideal in $R$, and if $R$ were commutative, it would be an ideal.

\section{Free Modules}
Throughout this section, $R$ denotes a general ring which need not be commutative. The content of this section is taken from \cite{blyth}. 

We define the free module using a universal property and then provide a construction for it. This should establish uniqueness.

\begin{definition}
    Let $S$ be a non-empty set. A \textit{free module on $S$} is an $R$-module $F$ together with a mapping $f: S\to F$ such that for every $R$-module $M$ and every set map $g: S\to M$, there is a unique $R$-module homomorphism $h: F\to M$ such that the following diagram commutes: 
    \begin{equation*}
    \xymatrix{
        S\ar[r]^g\ar[d]_f & M\\
        F\ar@{.>}[ru]_{\exists! h}
    }
    \end{equation*}
\end{definition}

\section{Finitely Generated Modules}

\begin{definition}[Finitely Generated Module]
    An $R$-module $M$ is said to be finitely generated if there is a finite subset $S$ of $M$ which generates $M$. That is, there is no proper submodule $N$ of $M$ containing $S$.
\end{definition}

\begin{proposition}
    An $R$-module $M$ is finitely generated if $M$ is isomorphic to a quotient of $R^{\oplus n}$ for some positive integer $n$.
\end{proposition}
\begin{proof}
    We shall only prove the forward direction since the converse is trivial to prove. Suppose $M$ is finitely generated. Then, it is generated by a finite subset $S = \{x_1,\ldots,x_m\}$. Define the $R$-module homomorphism $\phi:R^{\oplus n}\to M$ by $(r_1,\ldots,r_n)\mapsto r_1x_1 + \cdots + r_nx_n$. From the first isomorphism theorem, we have $M\cong R^{\oplus n}/\ker\phi$.
\end{proof}

\begin{proposition}\thlabel{prop:CH-type}
    Let $M$ be a finitely generated $A$-module and $\mathfrak a$ an ideal of $A$. Let $\phi\in\End(M)$ such that $\phi(M)\subseteq\mathfrak aM$. Then, there are $a_0,\ldots,a_{n - 1}\in\mathfrak a$ such that 
    \begin{equation*}
        \phi^n + a_{n - 1}\phi^{n - 1} + \cdots + a_0 = 0
    \end{equation*}
    as an element of $\End(M)$, where $a_k$ is treated as the homomorphism $x\mapsto a_kx$ in $\End(M)$.
\end{proposition}
\begin{proof}
    Let $\{x_1,\ldots,x_n\}$ be a generating set for $M$. Then, for all $1\le i\le n$, there are coefficients $\{a_{i1},\ldots,a_{in}\}$ in $\mathfrak a$ such that 
    \begin{equation*}
        \phi(x_i) = \sum_{j = 1}^n a_{ij}x_j
    \end{equation*}
    We may rewrite this as 
    \begin{equation*}
        \sum_{j = 1}^n(\phi\delta_{ij} - a_{ij})x_j = 0
    \end{equation*}
    Let $B$ denote the matrix $(\phi\delta_{ij} - a_{ij})_{1\le i,j\le n}$. Then, multiplying by $\operatorname{adj}(B)$, we see that $\det(B)(x_j) = 0$ for all $1\le j\le n$ where $\det(B)$ is viewed as an element in $\End(M)$ and thus, is the zero map in $\End(M)$. It is not hard to see that $\det(B)$ is in the required form.
\end{proof}

\begin{lemma}[Nakayama]
    Let $M$ be a finitely generated module and $\mathfrak a\subseteq\mathfrak R$ be an ideal such that $M = \mathfrak aM$. Then, $M = 0$. 
\end{lemma}
\begin{proof}
    Let $\phi = \mathbf{id}$ be the identity homomorphism in $\End(M)$. Using \thref{prop:CH-type}, there are coefficients $a_0,\ldots,a_{n - 1}\in\mathfrak a$ satisfying the statement of the proposition. As a result, $x = 1 + a_{n - 1} + \ldots + a_0$ is the zero endomorphism. But since $a_{n - 1} + \ldots + a_0\in\mathfrak a\subseteq\mathfrak R$, $x$ is a unit and hence, $M = 0$.
\end{proof}

\subsection*{Over a PID}
Throughout this section, let $R$ denote a principal ideal domain. 

\section{Exact Sequences}

\begin{definition}
    A sequence of module homomorphisms 
    \begin{equation*}
        M\stackrel{f}{\longrightarrow} N\stackrel{g}{\longrightarrow}P
    \end{equation*}
    is said to be exact at $N$ if $\im f = \ker g$. A short exact sequence is a sequence of module homomorphisms: 
    \begin{equation*}
        0\longrightarrow M\stackrel{f}{\longrightarrow} N\stackrel{g}{\longrightarrow} P\longrightarrow 0
    \end{equation*}
    which is exact at $M$, $N$ and $P$.
\end{definition}

It is not hard to see that the sequence in the definition is short exact if and only if $f$ is injective, $g$ is surjective and $\im f = \ker g$.

\begin{theorem}
    For all $R$-modules $X$, $\Hom(X,-)$ is a left exact functor. That is, $0\longrightarrow M\longrightarrow N\longrightarrow P$ is exact if and only if $0\longrightarrow\Hom(X,M)\longrightarrow\Hom(X,N)\longrightarrow\Hom(X,P)$ is exact.
\end{theorem}
\begin{proof}
    Consider the following commutative diagram where the row is exact.
    \begin{equation*}
    \xymatrix{
        & & X\ar[ld]_u\ar[d]^v& \\
        0\ar[r] & M\ar[r]^f & N\ar[r]^g & P
    }
    \end{equation*}
    Let $u\in\ker\overline f$. Since $f$ is injective, it is obvious that $u$ must be the trivail homomorphism. Next, we must show that $\im\overline f = \ker\overline g$. First, note that $\overline f\circ\overline g = \overline{f\circ g} = 0$ since $\Hom(X,-)$ is a covariant functor. Finally, suppose $v\in\ker\overline g$. Then, $g\circ v = 0$, consequently, $\im v\subseteq\im f$. Now, since $f$ is injective, $f^{-1}(\im v)$ is a submodule of $M$ and hence, the map $w: X\to M$ given by $x\mapsto f^{-1}(v(x))$ is well defined and $f\circ w = v$. 

    For the converse, simply note that $\Hom(R,M)$ is isomorphic to $M$.
\end{proof}

\subsection*{Diagram Chasing}

\section{Tensor Product}

Throughout this section, $R$ denotes a general ring which need not be commutative.

\begin{definition}[Bilinear Map]
    Let $M, N, P$ be $R$-modules. A map $T: M\times N\to P$ is said to be bilinear if for each $x\in M$, the mapping $T_x: N\to P$ given by $y\mapsto T(x,y)$ is $R$-linear and for each $y\in N$, the mapping $T_y: M\to P$ given by $x\mapsto T(x,y)$ is $R$-linear.
\end{definition}

Fix two $R$-modules $M$ and $N$. Let $\mathscr C$ denote the category of bilinear maps $T: M\times N\to P$ where $P$ is any $R$-module. A morphism between two bilinear maps $f: M\times N\to P_1$ and $g: M\times N\to P_2$ in this category is a module homomorphism $\phi: P_1\to P_2$ such that the following diagram commutes: 
\begin{equation*}
\xymatrix {
    M\times N\ar[r]^-f\ar[d]_g & P_1\ar@{.>}[ld]^\phi\\
    P_2
}
\end{equation*}

A universal object in $\mathscr C$ is called the tensor product of $M$ and $N$ and is denoted by $M\otimes N$. In other words, the tensor product is an initial object in the category $\mathscr C$.

\begin{definition}[Universal Property of the Tensor Product]
    Let $M,N,P$ be $R$-modules and $T: M\times N\to P$ be a bilinear map. Then, there is a unique $R$-module homomorphism $\phi: M\otimes N\to P$ such that the following diagram commutes: 
    \begin{equation*}
    \xymatrix {
        M\times N\ar[r]^-T\ar[d]_\varphi & P\\
        M\otimes N\ar@{.>}[ru]_{\exists!\phi}
    }
    \end{equation*}
\end{definition}

Of course, having the universal property would imply that the tensor product, if it exists, is unique upto a unique isomorphism. We shall now construct a tensor product of $M$ and $N$.

\subsection*{Constructing the Tensor Product}

Let $F$ be the free $R$-module on $M\times N$. Let us denote the basis elements of $F$ by $e_{(x,y)}$ where $x\in M$ and $y\in N$. Now, for all $x,x_1,x_2\in M$, $y,y_1,y_2\in N$ and $r\in R$, let $D$ denote the submodule generated by elements of the form: 
\begin{align*}
    &e_{(x_1 + x_2, y)} - e_{(x_1,y)} - e_{(x_2,y)}\\
    &e_{(x,y_1 + y_2)} - e_{(x,y_1)} - e_{(x,y_2)}\\
    &e_{(rx,y)} - re_{(x,y)}\\
    &e_{(x,ry)} - re_{(x,y)}
\end{align*}

Let $G = F/D$ and let $\varphi: M\times N\to G$ be the composition of the following maps: 
\begin{equation*}
    M\times N\hookrightarrow F\twoheadrightarrow G
\end{equation*}

Let $T: M\times N\to P$ be a bilinear map. Consider the following commutative diagram: 

\begin{equation*}
\xymatrix {
    M\times N\ar[r]^-T\ar@{^{(}->}_{\iota}[d] & P\\
    F\ar[r]_{\pi}\ar@{.>}[ru]|{\exists! f} & G\ar@{.>}[u]_{\exists!\phi}
}
\end{equation*}

To show that existence of $\phi$, we must show that $D\subseteq\ker f$, since we can then finish using the universal property of the kernel. But this is trivial to check and follows from the fact that $T$ is a bilinear map and completes the construction.

\begin{mdframed}
    Similarly, we define the tensor product for a finite sequence of $R$-modules $\{M_i\}_{i = 1}^n$. That is, given a multilinear map $T:\prod\limits_{i = 1}^n M_i\to P$, there is a unique $R$-module homomorphism $\phi$ such that the following diagram commutes: 
    \begin{equation*}
    \xymatrix{
        M_1\times\cdots\times M_n\ar[r]^-T\ar[d]_\varphi & P\\
        M_1\otimes\cdots\otimes M_n\ar@{.>}[ru]_-{\exists!\phi}
    }
    \end{equation*}
\end{mdframed}

\subsection*{Properties of Tensor Product}

Given two modules $M$ and $N$ with the canonical map $\varphi: M\times N\to M\otimes N$, we denote by $m\otimes n$, the element $\varphi(m,n)$ in $M\otimes N$.

\begin{proposition}
    Let $M, N, P$ be $A$-modules. Then, 
    \begin{enumerate}[label=(\alph*)]
    \item $M\otimes N\cong N\otimes M$ 
    \item $(M\otimes N)\otimes P\cong M\otimes(N\otimes P)\cong M\otimes N\otimes P$ 
    \item $M\oplus N\otimes P\cong(M\otimes P)\oplus(N\otimes P)$ 
    \item $A\otimes M\cong M$
    \end{enumerate}
    Further, in each case, the isomorphism is unique.
\end{proposition}
\begin{proof}
In each case, it suffices to show that both modules have the same universal property, which would imply a unique isomorphism between the two modules.
\begin{enumerate}[label=(\alph*)]
\item Consider the map $T: M\times N\to N\times M$ given by $(m,n)\mapsto(n,m)$. Let $\varphi: M\times N\to M\otimes N$ and $\varphi': N\times M\to N\otimes M$ be the canonical morphisms. Consider now the following commutative diagram: 
\begin{equation*}
\xymatrix{
    M\times N\ar[r]^-T_-\sim\ar[d]_\varphi & N\times M\ar[d]^{\varphi'}\\
    M\otimes N\ar@<-.5ex>@{.>}[r]_-{\phi} & N\otimes M\ar@<-.5ex>@{.>}[l]_-{\phi'}
}
\end{equation*}
Define the map $\phi(m\otimes n) = n\otimes m$ and $\phi'(n\otimes m) = m\otimes n$. It is not hard to see that $\phi$ and $\phi'$ make the diagram commute. Further, since $\varphi'\circ T$ is bilinear, $\phi$ is the unique morphism making the diagram commute and similarly for $\phi'$. Finally, since $\phi$ and $\phi'$ are  
\end{enumerate}
\end{proof}

\chapter{Localization}
\section{Rings of Fractions}

Define the relation $\sim_S$ on $A\times S$ by $(a,s)\sim_S(a',s')$ if there is $t\in S$ such that $t(s'a - sa') = 0$. That this is an equivalence relation is easy to verify. We shall use $a/s$ to denote the equivalence class $[(a,s)]$ in $A\times S/\sim_S$.

Consider the operations: 
\begin{equation*}
    \frac{a}{s} + \frac{a'}{s'} = \frac{s'a + sa'}{ss'}\qquad\frac{a}{s}\cdot\frac{a'}{s'} = \frac{aa'}{ss'}
\end{equation*}

It is not hard to see that these are well defined and endow $A\times S/\sim_S$ with a ring structure. We denote this ring by $S^{-1}A$ and is called the \textit{ring of fractions} of $A$ by $S$. 

There is a natural ring homomorphism $\varphi: A\to S^{-1}A$ given by $\varphi(x) = x/1$.
When $A$ is an integral domain and $S = A\backslash\{0\}$, $S^{-1}A$ is precisely the field of fractions.
Recall that if $\mathfrak p$ is a prime ideal in $A$, then $S = A\backslash\mathfrak p$ is a multiplicatively closed subset of $A$. We denote the ring $S^{-1}A$ by $A_{\mathfrak p}$.

\begin{theorem}
    The ring $A_{\mathfrak p}$ is local.
\end{theorem}
\begin{proof}
    Let $S = A\backslash\mathfrak p$ and define
    \begin{equation*}
        \mathfrak m = \left\{\frac{a}{s}~\bigg\vert~a\in\mathfrak p,~s\in S\right\}
    \end{equation*}
    It is not hard to see that $\mathfrak m$ is an ideal in $A_{\mathfrak p}$. We contend that $\mathfrak m$ is the ideal of non-units in $A_{\mathfrak p}$. Indeed, if $a/s\in\mathfrak m$ is a unit, then there is $b/t\in A_{\mathfrak p}$ such that $(ab)/(st) = 1$, consequently, there is $w\in S$ such that $w(ab - st) = 0$, whence $wst\in\mathfrak p$, a contradiction. 

    On the other hand, if $a/s\notin\mathfrak m$, then $a/s$ is a unit since $(a/s)\cdot(s/a) = 1$. Now, since the collection of all non-units forms an ideal, the ring must be local due to \thref{prop:non-unit-ideal-local}.
\end{proof}

\begin{proposition}
    Let $\mathfrak m$ be the unique maximal ideal of $A_{\mathfrak p}$. Then, $A_{\mathfrak p}/\mathfrak m\cong Q(A/\mathfrak p)$ where the latter is the field of fractions of $A/\mathfrak p$.
\end{proposition}
\begin{proof}
    \textcolor{red}{TODO: Add in later}
\end{proof}

Similarly, when we let $S = \{a^n\}_{n\ge 0}$ for some $a\in A$, we denote $S^{-1}A$ by $A_a$.

There is a degenerate case, when we allow $0\in S$, notice that the ring $S^{-1}A$ is the zero ring, since for all $a/s\in S^{-1}A$, we have $0(as) = 0$, therefore, $a/s = 0/s$.

\subsection{Universal Property}

Fix a multiplicative subset $S\subseteq A$. Let $\mathscr C$ denote the category with objects as pairs $(\phi, B)$ where $\phi: A\to B$ is a ring homomorphism such that $\phi(s)$ is a unit in $B$ for all $s\in S$. A morphism in this category is a map $f:(\phi, B)\to(\psi, C)$ making the following diagram commute.
\begin{equation*}
\xymatrix{
    A\ar[r]^{\psi}\ar[d]_\phi & C\\
    B\ar[ru]_f
}
\end{equation*}

The ring of fractions is an initial object in this category. Therefore, we have the following universal property. We shall verify in the ``proof'' that our construction of the field of fractions does satisfy this property and is therefore an initial object in $\mathscr C$.

\begin{proposition}
    Let $f: A\to B$ be a ring homomorphism such that $f(s)$ is a unit in $B$ for all $s\in S$. Then there is a unique ring homomorphism $g: S^{-1}A\to B$ making the following diagram commute 
    \begin{equation*}
    \xymatrix{
        A\ar[r]^f\ar[d]_\varphi & B\\
        S^{-1}A\ar@{.>}[ru]_-{\exists!g}
    }
    \end{equation*}
\end{proposition}
\begin{proof}
    Define the map $g: S^{-1}A\to B$ by $g(a/s) = g(a)g(s)^{-1}$. To see that this map is well defined, note that if $a/s = a'/s'$, then there is $t\in S$ such that $t(s'a - sa') = 0$, consequently, $g(t)(g(s')g(a) - g(s)g(a')) = 0$. As a result, $g(a)g(s)^{-1} = g(a')g(s')^{-1}$. From this, it follows immediately that $g$ is a ring homomorphism making the diagram commute.

    As for uniqueness, note that for all $a/s\in S^{-1}A$,
    \begin{equation*}
        g(a/s) = g(a/1)g(1/s) = g(a/1)g(s/1)^{-1} = f(a)f(s)^{-1}
    \end{equation*}
    which is fixed by the choice of $f$. This completes the proof.
\end{proof}

\section{Modules of Fractions}

Let $M$ be an $A$-module and $S\subseteq A$ be a multiplicatively closed subset. Define the relation $\sim_S$ on $M\times S$ by $(m,s)\sim_S(m',s')$ if and only if there is $t\in S$ such that $t(s'm - sm') = 0$. That this is an equivalence relation is easy to verify. We shall use $m/s$ to denote the equivalence class $[(m,s)]$ in $M\times S/\sim_S$.

As in the previous section, there is a natural $A$-module homomorphism $\varphi: M\to S^{-1}M$ given by $\varphi(m) = m/1$. This map is called the \textit{localization map}.

It is not hard to see that $S^{-1}M$ forms an $A$-module. Further, it also has the structure of an $S^{-1}A$ module under the action 
\begin{equation*}
    \frac{a}{s}\cdot\frac{m}{t} = \frac{a\cdot m}{st}
\end{equation*}

Let $f: M\to N$ be an $A$-module homomorphism. Consider the map $S^{-1}f: S^{-1}M\to S^{-1}N$ given by 
\begin{equation*}
    S^{-1}f\left(\frac{m}{s}\right) = \frac{f(m)}{s}
\end{equation*}

We must first show that this is well defined. Indeed, if $m/s = m'/s'$, then there is $t\in S$ such that $t(s'm - sm') = 0$, consequently, $t(s'f(m) - sf(m')) = 0$, as a result, $f(m)/s = f(m')/s'$ in $S^{-1}M$. 

We now contend that $S^{-1}f$ is an $S^{-1}A$ module homomorphism. Indeed, we have 
\begin{equation*}
    S^{-1}f\left(\frac{m}{s} + \frac{a}{t}\frac{m'}{s'}\right) = S^{-1}f\left(\frac{ts' m + as m'}{sts'}\right) = \frac{f(ts'm + asm')}{sts'} = \frac{ts'f(m) + asf(m')}{sts'} = \frac{f(m)}{s} + \frac{f(m')}{s'}
\end{equation*}

Finally, let $f: M\to N$ and $g: N\to P$ be $A$-module homomorphisms. Then, 
\begin{equation*}
    S^{-1}(g\circ f)\left(\frac{m}{s}\right) = \frac{g(f(m))}{s}\qquad S^{-1}g\left(S^{-1}f\left(\frac{m}{s}\right)\right) = S^{-1}g\left(\frac{f(m)}{s}\right) = \frac{g(f(m))}{s}
\end{equation*}

\begin{theorem}\thlabel{thm:exactness-of-localization}
    $S^{-1}:A-\catMod\to S^{-1}A-\catMod$ is an exact functor.
\end{theorem}
\begin{proof}
    Let $M'\stackrel{f}{\longrightarrow}M\stackrel{g}{\longrightarrow}M''$ be an exact sequence. Then, for any $m'/s'\in S^{-1}M'$, we have 
    \begin{equation*}
        S^{-1}g\left(S^{-1}f\left(\frac{m'}{s'}\right)\right) = S^{-1}g\left(\frac{f(m')}{s'}\right) = \frac{g(f(m'))}{s'} = 0
    \end{equation*}
    As a result, $\im(S^{-1}f)\subseteq\ker(S^{-1}g)$. On the other hand, for $m/s\in\ker S^{-1}g$, we have $g(m)/s = 0$, consequently, there is $t\in S$ such that $tg(m) = 0$, equivalently, $g(tm) = 0$, whence, there is $m'\in M'$ such that $f(m') = tm$. Then, we have 
    \begin{equation*}
        f\left(\frac{m'}{st}\right) = \frac{f(m')}{st} = \frac{tm}{st} = \frac{m}{s}
    \end{equation*}
    whence, $\ker(S^{-1}g)\subseteq\im(S^{-1}f)$. This completes the proof.
\end{proof}

\begin{proposition}
    Let $N,P,\{M_i\}_{i\in I}$ be submodules of an $A$-module $M$. Then, for a multiplicatively closed $S\subseteq M$,
    \begin{enumerate}[label=(\alph*)]
        \item $S^{-1}(N\cap P) = S^{-1}N\cap S^{-1}P$
        \item $\displaystyle S^{-1}\left(\sum_{i\in I}M_i\right) = \sum_{i\in I}S^{-1}M_i$
    \end{enumerate}
\end{proposition}
\begin{proof}
\begin{enumerate}[label=(\alph*)]
\item We have the exact sequences $0\rightarrow N\cap P\rightarrow N$ and $0\rightarrow N\cap P\rightarrow P$. Due to \thref{thm:exactness-of-localization}, the sequences $0\rightarrow S^{-1}(N\cap P)\rightarrow S^{-1}N$ and $0\rightarrow S^{-1}(N\cap P)\rightarrow S^{-1}N$ are exact, consequently, $S^{-1}(N\cap P)\subseteq S^{-1}N\cap S^{-1}P$.

On the other hand, if $n/s = p/t$ for some $n\in N$, $p\in P$ and $s,t\in S$, there is some $u\in S$ such that $u(tn - sp) = 0$, equivalently, $m = utn = usp\in N\cap P$. Thus, $m/(stu) = n/s = p/t$, and the conclusion follows.

\item Let $\overline M = \sum_{i\in I}M_i$. Then, there is the exact sequence $0\rightarrow M_i\rightarrow \overline M$. Then, due to \thref{thm:exactness-of-localization}, the sequence $0\rightarrow S^{-1}M_i\rightarrow S^{-1}\overline M$ is exact. Consequently, $\displaystyle\sum_{i\in I}S^{-1}M_i\subseteq S^{-1}\overline M$.

On the other hand, any element in $S^{-1}\overline M$ is of the form $(m_{i_1} + \cdots + m_{i_n})/s = m_{i_1}/s + \cdots + m_{i_n}/s$ for some $m_{i_n}\in M_{i_n}$ and $s\in S$. The conclusion follows.
\end{enumerate}
\end{proof}

\chapter{Primary Decomposition}
\section{Primary Decomposition of Ideals}

A primary ideal is a generalization of the ideals $p^n\Z$ in $\Z$, as is evident from the following definition.

\begin{definition}[Primary Ideals]
    An ideal $\frakq\subseteq A$ is said to be \textit{primary} if for every ordered pair $x,y\in A$,
    \begin{equation*}
        xy\in\frakq\implies x\in\frakq\text{ or }y^n\in\frakq\text{ for some } n > 0
    \end{equation*}
\end{definition}


From the definition, we see that every prime ideal is primary. It is not hard to see that
\begin{itemize}
\item $\frakq$ is primary if and only if every zero divisor in $A/\frakq$ is nilpotent. 
\item $\frakq$ is primary if and only if $(0)$ is primary in $A/\frakq$.
\end{itemize}


\begin{proposition}
    If $\frakq$ is primary, then $\sqrt{\frakq}$ is prime. Further, $\sqrt{\frakq}$ is the smallest prime ideal containing $\frakq$.
\end{proposition}
\begin{proof}
    Suppose $xy\in\sqrt{\frakq}$, then there is $n > 0$ such that $x^ny^n\in\frakq$, consequently, there is an $m > 0$ such that $x^n\in\frakq$ or $y^{mn}\in\frakq$, therefore, $x\in\sqrt{\frakq}$ or $y\in\sqrt{\frakq}$, whence $\sqrt{\frakq}$ is prime. The second assertion is trivial.
\end{proof}

If $\frakq$ is a primary ideal, then $\frakp = \sqrt{\frakq}$ is called the \textit{associated prime ideal} of $\frakq$ and $\frakq$ is said to be \textit{$\frakp$-primary}.

Consider the ring $A = k[x,y]$ and the ideal $\frakq = (x,y^2)$. The quotient ring $A/\frakq$ is isomorphic to $k[y]/(y^2)$ where every zero divisor is nilpotent consequently, $\frakq$ is primary. The radical ideal $\frakp = \sqrt{\frakq} = (x,y)$ is a prime ideal such that $\frakp^2\subsetneq\frakq\subsetneq\frakp$, therefore, $\frakq$ is not a prime power.

On the other hand, consider the ring $A = k[x,y,z]/(xy - z^2)$ and the prime ideal $\frakp = (\overline x,\overline z)\subseteq A$. We contend that $\frakp^2\subseteq A$ is not primary. Indeed, note that $\overline x\overline y = \overline z^2\in\frakp^2$ but $\overline x\notin\frakp^2$ and $\overline y\notin\frakp^2$, and the conclusion follows.

\begin{proposition}
    If $\sqrt{\fraka}$ is maximal, then $\fraka$ is primary.
\end{proposition}
\begin{proof}
    Let $\frakm = \sqrt{\fraka}$ and $\phi: A\to A/\fraka$ denote the natural map. Then, $\phi(\sqrt{\fraka})$ is the maximal ideal in $A/\fraka$ and is also the nilradical of $A/\fraka$, consequently, $A/\fraka$ is local and every non-unit is nilpotent. Hence, $\fraka$ is primary.
\end{proof}

\begin{proposition}
    Let $\phi: A\to B$ be a ring homomorphism. If $\frakq\unlhd B$ is a primary ideal in $B$, then $\frakq^c$ is a primary ideal in $A$.
\end{proposition}
\begin{proof}
    There is an injection $A/\frakq^c\into B/\frakq$. If $(0)$ is primary in $B/\frakq$ then it is primary in $A/\frakq^c$.
\end{proof}

\begin{lemma}\thlabel{lem:intersection-p-primary}
    If $\{\frakq_i\}_{i = 1}^n$ are $\frakp$-primary, then so is $\frakq=\bigcap_{i = 1}^n\frakq_i$.
\end{lemma}
\begin{proof}
    Obviously, 
    \begin{equation*}
        \sqrt{\frakq} = \bigcap_{i = 1}^n\sqrt{\frakq_i} = \frakp
    \end{equation*}
    Let $xy\in\frakq$. If $y\in\frakp$, then we are done, since $\frakp = \sqrt{\frakq}$. Else, $y^n\notin\frakq_i$ for every positive integer $n$, since $\frakp = \sqrt{\frakq_i}$ whereby $x\in\frakq_i$ for each $1\le i\le n$ and the conclusion follows.
\end{proof}

\begin{lemma}
    Let $\frakq$ be a $\frakp$-primary ideal and $x\in A$. Then 
    \begin{enumerate}[label=(\alph*)]
        \item if $x\in\frakq$, then $(\frakq: x) = (1)$.
        \item if $x\notin\frakq$, then $(\frakq: x)$ is $\frakp$-primary.
        \item if $x\notin\frakp$, then $(\frakq: x) = \frakq$.
    \end{enumerate}
\end{lemma}
\begin{proof}
\begin{enumerate}[label=(\alph*)]
    \item Trivial.
    \item If $y\in(\frakq:x)$, then $xy\in\frakq$, therefore, $y\in\frakp$. Thus, we have $\frakq\subseteq(\frakq:x)\subseteq\frakp$. Taking radicals, $\frakp\subseteq\sqrt{(\frakq:x)}\subseteq\frakp$, whereby $\sqrt{(\frakq: x)} = \frakp$.

    On the other hand, if $yz\in(\frakq: x)$, then $xyz\in\frakq$. If $z\in\frakp$, then we are done. Else, $xy\in\frakq$ and $y\in(q:x)$ whence $(q: x)$ is $\frakp$-primary.
    \item If $y\in(q: x)$, then $yx\in\frakq$. Since $x\notin\frakp$, we must have $y\in\frakq$. This completes the proof.
\end{enumerate}
\end{proof}

\begin{definition}[Primary Decomposition]
    A \textit{primary decomposition} of an ideal $\fraka\subseteq A$ is an expression of $\fraka$ as a \textit{finite} intersection of primary ideals.
    \begin{equation*}
        \fraka = \bigcap_{i = 1}^n\frakq_i
    \end{equation*}
    The ideal $\fraka$ is said to be \textit{decomposable} if it has a primary decomposition. Moreover, if for all $1\le i\le n$, $\sqrt{\frakq_i}$ are distinct and 
    \begin{equation*}
        \bigcap_{j\ne i}\frakq_j\not\subseteq\frakq_i
    \end{equation*}
    then the primary decomposition is said to be \textit{minimal}.
\end{definition}

Using \thref{lem:intersection-p-primary}, it is not hard to see that every decomposable ideal has a minimal decomposition.

\begin{theorem}[First Uniqueness Theorem]\thlabel{thm:first-uniqueness-theorem}
    Let $\fraka\subseteq A$ be a decomposable ideal and 
    \begin{equation*}
        \fraka = \bigcap_{i = 1}^n\frakq_i
    \end{equation*}
    be a minimal primary decomposition with $\frakp_i = \sqrt{\frakq_i}$. Then, the $\frakp_i$'s are precisely the prime ideals the occur in the set $\{\sqrt{(\fraka: x)}\mid x\in A\}$.
\end{theorem}
\begin{proof}
    First, note that 
    \begin{equation*}
        \sqrt{(\fraka: x)} = \sqrt{\bigcap_{i = 1}^n(\frakq_i: x)} = \bigcap_{i = 1}^n\sqrt{(\frakq_i: x)} = \bigcap_{x\notin q_i}\frakp_i
    \end{equation*}
    Using \thref{prop:prime-containment}, $\sqrt{(\fraka : x)} = \frakp_j$ for some index $j$. 

    Conversely, for every $1\le j\le n$, there is $x_j\in\bigcap_{i\ne j}\frakq_i\backslash\frakq_j$. This obviously exists since the decomposition is minimal. It now follows from \thref{prop:prime-containment} and the decomposition of $\sqrt{(\fraka: x)}$ we derived above that $\sqrt{(\fraka : x)} = \frakp_j$.
\end{proof}

\begin{proposition}\thlabel{prop:min-primes-belonging}
    Let $\fraka$ be a decomposable ideal. Then any prime ideal $\frakp\supseteq\fraka$ contains a \textit{minimal} prime ideal belonging to $\fraka$, and thus the minimal prime ideals belonging to $\fraka$ are precisely the minimal prime ideals in the set of all prime ideals containing $\fraka$.
\end{proposition}
\begin{proof}
    Let $\frakp$ be a minimal prime ideal containing $\fraka$. Consider a minimal primary decomposition of $\fraka$ given by 
    \begin{equation*}
        \frakp\supseteq\fraka = \bigcap_{i = 1}^n\frakq_i.
    \end{equation*}
    Let $\frakp_i = \sqrt{\frakq_i}$, then 
    \begin{equation*}
        \frakp\supseteq\sqrt{\fraka} = \bigcap_{i = 1}^n\frakp_i
    \end{equation*}
    and due to \thref{prop:prime-containment}, there is an index $j$ such that $\frakp\supseteq\frakp_j$ whence $\frakp_j = \frakp$. Thus, every minimal prime ideal containing $\fraka$ belongs to $\fraka$.
\end{proof}

\begin{proposition}
    Let $S$ be a multiiplcatively closed subset of $A$ and $\frakq$ be a $\frakp$-primary ideal. 
    \begin{enumerate}[label=(\alph*)]
        \item If $S\cap\frakp\ne\emptyset$, then $S^{-1}\frakq = S^{-1}A$. 
        \item If $S\cap\frakp = \emptyset$, then $S^{-1}\frakq$ is $S^{-1}\frakp$-primary and its contraction in $A$ is $\frakq$.
    \end{enumerate}
\end{proposition}
\begin{proof}
    $(a)$ is trivial. $(b):$ Recall that we have 
    \begin{equation*}
        \frakq^{ec} = \bigcup_{s\in S}(\frakq: s) = \bigcup_{s\in S}\frakq
    \end{equation*}
    where the last equality follows from the fact that $S\cap\frakq = \emptyset$. It remains to show that $S^{-1}\frakq$ is primary. Indeed, let $x/s\cdot y/t\in S^{-1}\frakq$. Then, there is $z\in\frakq$ and $w,u\in S$ such that $w(xyu - stz) = 0$. But since $wu\notin\frakq$, we must have $xy\in\frakq$, whereby $x\in\frakq$ or $y^n\in\frakq$ for some positive integer $n$, implying that either $x/s\in S^{-1}\frakq$ or $y^n/t^n\in S^{-1}\frakq$. This completes the proof.
\end{proof}

\begin{definition}[Isolated Set of Associated Primes]
    A set $\Sigma$ of prime ideals associated with $\fraka$ is said to be \emph{isolated} if it satisfies the following condition: 
    \begin{quote}
        if $\frakp'$ is a prime ideal belonging to $\fraka$ with $\frakp'\subseteq\frakp$ for some $\frakp\in\Sigma$, then $\frakp'\in\Sigma$
    \end{quote}
\end{definition}

\begin{theorem}[Second Uniqueness Theorem]\thlabel{thm:second-uniqueness-theorem}
    Let $\fraka$ be a decomposable ideal with a primary decomposition $\fraka = \bigcap_{i = 1}^n\frakq_i$. Let $\sqrt{\frakq_i} = \frakp_i$. Suppose $\Sigma = \{\frakp_{i_1},\ldots,\frakp_{i_m}\}$ is an isolated set of associated primes of $\fraka$, then $\bigcap_{j = 1}^m\frakq_{i_j}$ is independent of the chosen decomposition.
\end{theorem}
\begin{proof}
    Let $S = A\backslash\bigcup_{j = 1}^m\frakp_{i_j}$. Then, $\frakp_k\cap S = \emptyset$ if and only if $\frakp_k\subseteq\bigcap_{j = 1}^m\frakp_j$ whence due to \thref{prop:prime-containment}, there is a prime $\frakp_{i_t}$ containing $\frakp_k$ and equivalently, $\frakp_k\in\Sigma$. 

    Whence, upon localizing with $S$, we have 
    \begin{equation*}
        S^{-1}\fraka = \bigcap_{i = 1}^n S^{-1}\frakq_i = \bigcap_{j = 1}^m S^{-1}\frakq_{i_j}
    \end{equation*}
    Contracting both sides, we have 
    \begin{equation*}
        \fraka^{ec} = \left(\bigcap_{j = 1}^m S^{-1}\frakq_{i_j}\right) = \bigcap_{j = 1}^m\frakq_{i_j}^{ec} = \bigcap_{j = 1}^m\frakq_{i_j}
    \end{equation*}
    and the conclusion follows.
\end{proof}

\begin{corollary}
    In particular, the primary ideals which correspond to the minimal primes associated to $\fraka$ are uniquely determined.
\end{corollary}

\begin{proposition}
    Let $X$ be an infinite compact Hausdorff space. Then, $(0)$ is not decomposable in $C(X)$, the ring of continuous functions on $X$.
\end{proposition}
\begin{proof}
    Suppose $(0) = \bigcap_{i = 1}^n\frakq_i$. Recall that the maximal ideals in $X$ are in bijection with the points of $X$. Denote the maximal ideal corresponding to a point $x\in X$ by $\frakm_x$. 

    For each $\frakq_i$, choose a maximal ideal $\frakm_{x_i}$ containing it. Choose some $x\in X\backslash\{x_1,\dots,x_n\}$. Choose an open set $V$ containing $\{x_1,\dots,x_n\}$ and an open set $U$ containing $x$ such that $U\cap V = \emptyset$. 

    Using Urysohn's Lemma, choose continuous functions $f,g: X\to[0,1]$ such that $f(x) = 1$ and $\Supp(f)\subseteq U$ and $g(x_i) = 1$ for every $i$ and $\Supp(g)\subseteq V$. By our choice of $g$, note that $g^m\notin\frakq_i$ for every $1\le i\le n$ and every positive integer $m$. Since $fg = 0$, we must have $f\in\frakq_i$ for every $1\le i\le n$, implying that $f = 0$, a contradiction. This completes the proof.
\end{proof}

\begin{definition}[Symbolic Power]
    Let $\frakp\in\Spec A$. The \emph{$n$-th symbolic power of $\frakp$} is defined to be the contraction of the ideal $\frakp^nA_\frakp$ in $A$, denoted $\frakp^{(n)}$.
\end{definition}

Being the contraction of a primary ideal in $A_\frakp$, the symbolic power is always a primary ideal. Moreover, $\sqrt{\frakp^{(n)}} = \frakp$ whence, it is $\frakp$-primary.

\begin{proposition}
    With notation as above, 
    \begin{enumerate}[label=(\alph*)]
        \item $\frakp^{(n)}$ is a $\frakp$-primary ideal. 
        \item if $\frakp^n$ has a primary decomposition, then $\frakp^{(n)}$ is its $\frakp$-primary component.
    \end{enumerate}
\end{proposition}
\begin{proof}
    $(a)$ follows from the fact that the contraction of primary ideals is primary. 

    $(b)$ Note that $\frakp^n$ obviously would have a $\frakp$-primary component and that would be given by the contraction of $S^{-1}\frakp^n$ where $S = A\backslash\frakp$. The conclusion follows.
\end{proof}

\section{Associated Primes of Modules}

\begin{definition}
    Let $a\in A$ and $M$ and $A$-module. The homomorphism $a_M: M\to M$ given by $x\mapsto ax$ for all $x\in M$ is called the \emph{principal homomorphism}. We say that $a_M$ is \emph{locally nilpotent} if for each $x\in M$, there is an integer $n\in\N$ such that $a^n x = 0$.
\end{definition}

\begin{remark}
    If $M$ is finitely generated, then $a_M$ is locally nilpotent if and only if it is nilpotent.
\end{remark}

\begin{proposition}
    Let $a\in A$ and $M$ an $A$-module. Then $a_M$ is locally nilpotent if and only if $a\in\frakp$ for each $\frakp\in\Supp_A(M)$, that is, $a\in\bigcap\limits_{\frakp\in\Supp_A(M)}\frakp$.
\end{proposition}
\begin{proof}
    Suppose $a_M$ is locally nilpotent and $\frakp\in\Supp(M)$. Then, there is some $x\in M$ such that $x/1\ne 0$ in $M_\frakp$, that is, $\fraka = \Ann_A(x)\subseteq\frakp$. Since $a_M$ is locally nilpotent, there is a positive integer $n$ such that $a^n\in\fraka$ whence $a\in\frakp$. 

    Conversely, suppose $a_M$ is not locally nilpotent whence there is some $x\in M$ such that $a^nx\ne0$ for all $n\in\N$. Let $\frakp$ be a prime ideal not intersecting $\{1,a,a^2,\dots\}$ and containing $\Ann_A(x)$\footnote{That we can do this is an easy application of Zorn's Lemma}. Then, $x/1\ne 0$ in $M_\frakp$ whence $M_\frakp\ne 0$ and $\frakp\in\Supp(M)$, but $a\notin\frakp$. This completes the proof.
\end{proof}

\begin{definition}[Associated Primes]
    For an $A$-module $M$, a prime $\frakp\in\spec(A)$ is said to be \emph{associated} with $M$ if there is $x\in M$ such that $\frakp = \Ann_A(x)$. The set of all associated primes of a module $M$ is denoted by $\Ass(M)$.

    Equivalently, a prime $\frakp$ is an associated prime of $M$ if there is an injection of $A$-modules, $A/\frakp\into M$.
\end{definition}

\begin{proposition}
    If the poset 
    \begin{equation*}
        \Sigma = \{\Ann_A(x)\mid x\in M\backslash\{0\}\}
    \end{equation*}
    has a maximal element, then it is prime.
\end{proposition}
\begin{proof}
    Let $\frakp$ be a maximal element of $\Sigma$ under inclusion. Let $a,b\in A$ with $ab\in\frakp$. If either $a$ or $b$ is zero, then, trivially, $a\in\frakp$ or $b\in\frakp$. Suppose now that both $a,b$ are nonzero. Let $x\in M$ be such that $\frakp = \Ann_A(x)$ and suppose $b\notin\frakp$. Then, $\frakp\subseteq\Ann_A(bx)\ne(1)$ and due to maximality, we must have $\frakp = \Ann_A(bx)$, and thus $a\in\frakp$. This completes the proof.
\end{proof}

\begin{corollary}
    Modules over noetherings have associated primes.
\end{corollary}

\begin{lemma}
    Let $A$ be a noethering and $M$ an $A$-module with $a\in A$. Then, $a_M$ is injective if and only if $a$ does not lie in any of the associated primes of $M$.
\end{lemma}
\begin{proof}
    If $a_M$ is injective, then $a$ is not in the annihilator of any nonzero element, therefore, not an element of any associated prime. On the other hand, suppose $a_M$ is not injective. Then, there is some nonzero $x\in M$ such that $a\in\Ann_A(x)$. Consider the poset of all proper annihilators containing $\Ann_A(x)$. Since $A$ is a noethering, this has a maximal element, say $\frakp$. Note that $\frakp$ is also maximal in the poset of all proper annihilators whence is prime and hence $a$ is contained in an associated prime. This completes the proof.
\end{proof}

\begin{lemma}
    Let $A$ be a noethering and $M$ and $A$-module. Then, every $\frakp\in\Supp(M)$ contains an associated prime.
\end{lemma}
\begin{proof}
    If $\frakp\in\Supp(M)$, then there is some $x\in M$ such that $(Ax)_\frakp\ne0$, consequently, $(Ax)_\frakp$ has an associated prime, say $\frakq$. First, we contend that $\frakq\subseteq\frakp$. Suppose not, then there is some $a\in\frakq\backslash\frakp$. Since $\frakq$ is an associated prime, there is some $0\ne y/s\in (Ax)_\frakp$ such that $\frakq = \Ann_{A_\frakp}(y/s)$. In particular, $by/s = 0$. But $b/1$ is invertible in $A_\frakp$ whence $y/s = 0$, a contradiction. Thus $\frakq\subseteq\frakp$.

    Next, we shall show that $\frakq$ is an associated prime of $M$. Since $A$ is a noethering, $\frakq$ is finitely generated, say by $b_1,\dots,b_n$. Then, $b_iy/s = 0$ for each $i$, consequently, there is some $s_i\notin\frakp$ such that $s_ib_iy = 0$. Let $t = s_1\cdots s_n\notin\frakp$. We contend that $\frakq = \Ann_A(ty)$. Obviously, $\frakq\subseteq\Ann_A(ty)$. On the other hand, if $b\in\Ann_A(ty)$, then $bty = 0$ whereby $by/s = 0$ and $b\in\frakq$. This completes the proof.
\end{proof}


\begin{corollary}
    Let $A$ be a noethering and $M$ an $A$-module with $a\in A$. The following are equivalent: 
    \begin{enumerate}[label=(\alph*)]
        \item $a_M$ is locally nilpotent. 
        \item for each $\frakp\in\Ass(M)$, $a\in\frakp$.
        \item for each $\frakp\in\Supp(M)$, $a\in\frakp$.
    \end{enumerate}
\end{corollary}
\begin{proof}
    $(a)\implies(b)$ is immediate from the definition while $(c)\implies(a)$ has been proven. Both these implications do not require the noethering hypothesis. The implication $(b)\implies(c)$ has been proven above and requires the noethering hypothesis.
\end{proof}

\begin{lemma}
    Let $N$ be a submodule of $M$. Then, 
    \begin{equation*}
        \Ass(N)\subseteq\Ass(M)\subseteq\Ass(N)\cup\Ass(M/N).
    \end{equation*}
\end{lemma}
\begin{proof}
    It is obvious that $\Ass(N)\subseteq\Ass(M)$. Now, let $\frakp\in\Ass(M)$. Then, there is some $x\in M$ such that $\frakp = \Ann_A(x)$. If $x\in N$, then we are done. If not, then consider $Ax\cap N$. If $Ax\cap N = 0$, then, $Ax$ is isomorphic to the image of $Ax$ under the projection $M/N$. Therefore, $\frakp$ is an associated prime of some submodule of $M/N$. On the other hand, if $Ax\cap N\ne 0$, then there is some $y = ax\in N$ for some $a\in A$. 

    Obviously $\frakp$ annihilates $y$. If $b\in A$ annihilates $y$, then $bax = 0$ whence $ba\in\frakp$. But since $y\ne 0$, $a\notin\frakp$ and thus $b\in\frakp$. This completes the proof.
\end{proof}

\begin{lemma}
    Let $S\subseteq A$ be a multiplicative subset and $N$ an $S^{-1}A$-module. Then, 
    \begin{equation*}
        \Ass_{S^{-1} A}(N) = S^{-1}\Ass_A(N)\backslash\{S^{-1}A\},
    \end{equation*}
    where 
    \begin{equation*}
        S^{-1}\Ass_A(N) := \{S^{-1}\frakp\mid\frakp\in\Ass_A(N)\}.
    \end{equation*}
\end{lemma}
\begin{proof}
\end{proof}

\begin{lemma}
    Let $A$ be a noethering, $M$ a non-zero $A$-module and $S$ a multiplicative subset of $A$. Then, 
    \begin{equation*}
        \Ass_{S^{-1}A}(S^{-1}M) = S^{-1}\Ass_A(M)\backslash\{S^{-1}A\}.
    \end{equation*}
\end{lemma}
\begin{proof}
\end{proof}

\begin{corollary}
    Let $A$ be a noethering. Then, 
    \begin{equation*}
        \frakp\in\Ass_A(M)\iff \frakp A_\frakp\in\Ass_{A_\frakp}(M_\frakp).
    \end{equation*}
\end{corollary}
\begin{proof}
    Take $S = A\backslash\frakp$ and the conclusion follows from the above lemma.
\end{proof}

\begin{proposition}
    Let $A$ be a noethering and $M\ne 0$ a finitely generated (and hence, noetherian) $A$-module. Then, there is a filtration 
    \begin{equation*}
        M = M_0\supsetneq M_1\supsetneq\dots\supsetneq M_n = 0
    \end{equation*}
    such that $M_i/M_{i + 1}\cong A/\frakp_i$ for some $\frakp_i\in\Spec(A)$.
\end{proposition}
\begin{proof}
    Since $A$ is a noethering, $\Ass_A(M)$ is non-empty. Pick a prime $\frakp_0\in\Ass_A(M)$. Then, there is an injection $A/\frakp_0\into M$. Then, there is a submodule $N_0$ of $M$ that is isomorphic to $A/\frakp_0$. If $M = N_0$, then we are done. If not, then consider $M/N_0$. This also has an associated prime $\frakp_1$ and hence, there is a submodle $N_1$ of $M$ containing $N_0$ such that $M/N_1\cong A/\frakp_1$. Continuing this way, we obtain a sequence (the finiteness of this sequence requires $M$ to be noetherian):
    \begin{equation*}
        N_0\subsetneq N_1\subsetneq\dots\subsetneq N_n = M
    \end{equation*}
    where $M/N_i\cong A/\frakp_i$ for some $\frakp_i\in\Spec(A)$. This completes the proof.
\end{proof}

\begin{lemma}
    Let $A$ be a noethering. Then, the set of all zero divisors on $M$ is given by 
    \begin{equation*}
        \bigcup_{\frakp\in\Ass_A(M)}\frakp.
    \end{equation*}
\end{lemma}
\begin{proof}
    If $a\in A$ is a zero divisor on $M$, then, the set 
    \begin{equation*}
        \{\fraka\unlhd A\mid a\in\fraka\text{ and }\fraka = \Ann_A(x)\text{ for some }x\in M\}
    \end{equation*}
    admits a maximal element (due to noetherian-ness), say $\frakp$. This is an associated prime and contains $a$. The converse is trivial.
\end{proof}

\begin{lemma}
    Let $A$ be a noethering and $M$ a finitely generated (equivalently, noetherian) $A$-module. Then, $\Ass_A(M)$ is finite.
\end{lemma}
\begin{proof}
    As we have seen earlier, $M$ admits a filtration 
    \begin{equation*}
        M = M_0\supsetneq M_1\supsetneq\dots\supsetneq M_n = 0
    \end{equation*}
    where $M_i/M_{i + 1}\cong A/\frakp_i$ for some $\frakp_i\in\Spec(A)$. We have short exact sequences 
    \begin{equation*}
        0\longrightarrow M_{i + 1}\longrightarrow M_i\longrightarrow M_i/M_{i + 1}\longrightarrow 0.
    \end{equation*}
    Then, 
    \begin{equation*}
        \Ass_A(M_i)\subseteq\Ass_A(M_{i + 1})\cup\Ass_A(M_i/M_{i + 1}) = \Ass_A(M_{i + 1})\cup\{\frakp_i\}.
    \end{equation*}
    Inductively, we see that 
    \begin{equation*}
        \Ass_A(M)\subseteq\{\frakp_0,\dots,\frakp_{n - 1}\}.\qedhere
    \end{equation*}
\end{proof}

\begin{lemma}
    Let $A$ be any ring. Then,
    \begin{equation*}
        \Ass_A(M)\subseteq\Supp_A(M).
    \end{equation*}
\end{lemma}
\begin{proof}
    Let $\frakp\in\Ass_A(M)$. Then, there is an injection $A/\frakp\into M$. Localizing at $\frakp$, we have an injection $Q(A/\frakp)\into M_\frakp$. Thus, $M_\frakp\ne 0$ and $\frakp\in\Supp_A(M)$.
\end{proof}

\begin{lemma}
    Let $A$ be a noethering. The minimal elements of $\Ass_A(M)$ and $\Supp_A(M)$ are the same.
\end{lemma}
\begin{proof}
    Let $\frakp\in\Ass_A(M)$ be minimal. We have seen that $\frakp\in\Supp_A(M)$. Suppose $\frakq\in\Supp_A(M)$ with $\frakq\subseteq\frakp$. Note that 
    \begin{equation*}
        \Ass_{A_\frakq}(M_\frakq) = \left(\Ass_A(M)\right)_{\frakq}\backslash\{A_\frakq\} = \emptyset.
    \end{equation*}
    Therefore, $M_\frakq = 0$ and $\frakq\notin\Supp_A(M)$. This shows that the minimal primes of $\Ass_A(M)$ are a subset of the minimal primes of $\Supp_A(M)$.

    Conversely, suppose $\frakp\in\Supp_A(M)$ is minimal. Then, 
    \begin{equation*}
        \emptyset\ne\Ass_{A_\frakp}(M_\frakp) = \left(\Ass_A(M)\right)_\frakp\backslash\{A_\frakp\},
    \end{equation*}
    where the first ``equality'' follows from the fact that $M_\frakp\ne 0$. Hence, there is a prime ideal $\frakq\subseteq\frakp$ that is an associated prime of $M$ and hence, also lies in the support of $M$. It follows that $\frakq = \frakp$ whence $\frakp\in\Ass_A(M)$. This completes the proof.
\end{proof}


\section{Primary Decomposition of Modules}

\begin{definition}
    Let $M$ be an $A$-module. A submodule $Q$ of $M$ is said to be \emph{primary} if $Q\ne M$ and for each $a\in A$, the homomorphism $a_{M/Q}$ is either injective or nilpotent.

    Equivalently, the above definition implies that if $a_{M/Q}$ is a zero-divisor, then it is nilpotent.
\end{definition}


\begin{proposition}
    Let $Q$ be a primary submodule of $M$. Then, 
    \begin{equation*}
        \frakp := \{a\in A\mid a_{M/Q}\text{ is nilpotent}\}
    \end{equation*}
    is a prime ideal.
\end{proposition}
\begin{proof}
    Let $ab\in\frakp$, that is, $(ab)_{M/Q}$ is nilpotent. If $a\notin\frakp$, then $a_{M/Q}$ is injective and thus $b_{M/Q}$ is nilpotent, i.e. $b\in\frakp$. This completes the proof.
\end{proof}


\chapter{Integral Extensions}
\begin{definition}[Integral Extension]
    Let $A\subseteq B$ be a subring. Then, $\alpha\in B$ is said to be \textit{integral} over $A$ if it satisfies a monic polynomial in $A[x]$. The extension $A\hookrightarrow B$ is said to be integral if every element of $B$ is integral over $A$.

    Similarly, if $\fraka\subseteq A$ is an ideal, then $\alpha\in B$ is said to be \textit{integral} over $\fraka$ if it satisfies a monic polynomial in $A[x]$ with coefficients in $\fraka$.
\end{definition}

\begin{theorem}\thlabel{thm:equivalence-integral-extension}
    Let $A\subseteq B$ be a subring and $\alpha\in B$. Then, the following are equivalent: 
    \begin{enumerate}[label=(\alph*)]
        \item $\alpha$ is integral over $A$
        \item $A[\alpha]$ is a finitely generated $A$-module 
        \item $A[\alpha]$ is contained in a subring $C$ of $B$ such that $C$ is a finitely generated $A$-module 
        \item There is a faithful $A[\alpha]$-module $M$ which is finitely generated as an $A$-module.
    \end{enumerate}
\end{theorem}
\begin{proof}
\begin{description}
    \item[$(a)\Longrightarrow(b)$:] If $\alpha^n + a_{n - 1}\alpha^{n - 1} + \cdots + a_0 = 0$. Then, it is not hard to argue that $\{1,\alpha,\ldots,\alpha^{n - 1}\}$ generated $A[\alpha]$ over $A$.
    \item[$(b)\Longrightarrow(c)$:] Take $C = A[\alpha]$ 
    \item[$(c)\Longrightarrow(d)$:] $C$ is a faithful $A[\alpha]$ module which is a finitely generated $A$-module.
    \item[$(d)\Longrightarrow(a)$:] Let $\phi: M\to M$ be the map $m\mapsto\alpha\cdot m$. We have $\phi(M)\subseteq AM$, consequently, due to \thref{prop:CH-type} (since $\fraka = A$ is an ideal in $A$), there are $a_i\in A$ such that 
    \begin{equation*}
        (\alpha^n + a_{n - 1}\alpha^{n - 1} + \cdots + a_0)\cdot m = 0
    \end{equation*}
    for each $m\in M$. But since $M$ is a faithful $A[\alpha]$-module, we must have $\alpha^{n} + a_{n - 1}\alpha^{n - 1} + \cdots + a_0 = 0$, whereby $\alpha$ is integral over $A$.\qedhere
\end{description}
\end{proof}

\begin{corollary}
    If $B$ is a finite $A$-algebra, then $B/A$ is an integral extension. In particular, every element of $B$ is integral over $A$.
\end{corollary}

\begin{proposition}
    Let $G$ be a finite group of ring automorphisms of $A$ and let 
    \begin{equation*}
        A^G := \{a\in A\mid g\cdot a = a,~\forall g\in G\}.
    \end{equation*}
    Then, $A/A^G$ is an integral extension.
\end{proposition}
\begin{proof}
    It is easy to see that $A^G$ is a subring of $A$. For any $a\in A$, consider the monic polynomial 
    \begin{equation*}
        f(x) = \prod_{\sigma\in G}(x - \sigma(a)).
    \end{equation*}
    This is obviously a polynomial with coefficients in $A^G$ and has $x$ as a root. Thus, $x$ is integral over $A^G$.
\end{proof}

\begin{proposition}\thlabel{prop:integral-finite-type}
    Let $\{\alpha_i\}_{i = 1}^n$ be elements of $B$, each integral over $A$. Then the ring $A[\alpha_1,\ldots,\alpha_n]$ is a finitely generated $A$-module, equivalently, a finite $A$-algebra.
\end{proposition}
\begin{proof}
    Let $A_k$ denote the subring $A[\alpha_1,\dots,\alpha_k]$ for $k\ge 1$. We shall induct on $k$ with the convention $A_0 = A$. Obviously $A_0$ is a finite $A$-algebra. We have $A_{k + 1} = A_k[\alpha_{k + 1}]$ and thus is a finite $A_k$-algebra. But since $A_k$ is a finite $A$-algebra, so is $A_{k + 1}$, thereby completing the proof.
\end{proof}

\begin{corollary}
    The set $C$ of elements of $B$ which are integral over $A$ is a subring of $B$ containing $A$.
\end{corollary}
\begin{proof}
    Let $\alpha,\beta\in C$. Then, $A[\alpha,\beta]$ is a finite $A$-algebra. Now, $A\subseteq A[\alpha - \beta]\subseteq A[\alpha,\beta]$ and $A\subseteq A[\alpha\beta]\subseteq A[\alpha,\beta]$ whereby both $\alpha - \beta,\alpha\beta\in C$ and $C$ is a ring.
\end{proof}

The set $C$ as defined above is called the \textit{integral closure of $A$ in $B$}. If $C = A$, then $A$ is said to be \textit{integrally closed in $B$}.

\begin{theorem}\thlabel{thm:integral-extension-transitive}
    Let $A\subseteq B\subseteq C$ such that $B/A$ and $C/B$ are integral extensions. Then $C/A$ is an integral extension.
\end{theorem}
\begin{proof}
    Let $\alpha\in C$. Then, 
    \begin{equation*}
        \alpha^n + b_{n - 1}\alpha^{n - 1} + \cdots + b_0 = 0
    \end{equation*}
    for some $b_i\in B$. Then, $\alpha$ is integral over $B' = A[b_0,\ldots,b_{n - 1}]$, consequently, $B'[\alpha]$ is a finite $B'$-algebra. But since $B'$ is a finite $A$-algebra, $B'[\alpha]$ is a finite $A$-algebra and $\alpha$ is integral over $A$.
\end{proof}

\begin{corollary}
    Let $A\subseteq B$ and $C$ be the integral closure of $A$ in $B$. Then, $C$ is integrally closed in $B$.
\end{corollary}
\begin{proof}
    Let $\alpha\in B$ be integral over $C$. Then, $C[\alpha]$ is integral over $C$, whereby $C[\alpha] = C$.
\end{proof}

\begin{proposition}\thlabel{prop:int-ext-localization}
    Let $A\subseteq B$ be an integral extension. Then, 
    \begin{enumerate}[label=(\alph*)]
        \item if $\frakb\subseteq B$ is an ideal and $\pi: B\to B/\frakb$ is the canonical surjection, then $B/\frakb$ is integral over $\pi(A)$. In particular, due to the First Isomorphism Theorem, we see that $B/\frakb$ is integral over a copy of $A/\fraka$ where $\fraka = \frakb\cap A$.
        \item if $S\subseteq A$ is multiplicatively closed, then $S^{-1}B$ is integral over $S^{-1}A$.
    \end{enumerate}
\end{proposition}
\begin{proof}
\begin{enumerate}[label=(\alph*)]
    \item Let $\beta\in B/\frakb$, then there is some $\alpha\in B$ such that $\pi(\alpha) = \beta$. Then, there are $a_0,\ldots,a_{n - 1}\in A$ such that 
    \begin{equation*}
        \alpha^n + a_{n - 1}\alpha^{n - 1} + \cdots + a_0 = 0
    \end{equation*}
    whereby 
    \begin{equation*}
        \beta^n + \pi(a_{n - 1})\beta^{n - 1} + \cdots + \pi(a_0) = 0
    \end{equation*}
    and the conclusion follows. 
    \item Let $\alpha/s\in S^{-1}B$. Since $\alpha$ is integral over $A$, there are $a_0,\ldots,a_{n - 1}\in A$ such that 
    \begin{equation*}
        \alpha^n + a_{n - 1}\alpha^{n - 1} + \cdots + a_0 = 0
    \end{equation*}
    then 
    \begin{equation*}
        (\alpha/s)^n + (a_{n - 1}/s)(\alpha/s)^{n - 1} + \cdots + a_0/s^n = 0
    \end{equation*}
    which completes the proof.
\end{enumerate}
\end{proof}

\section{The Cohen-Seidenberg Theorems}

\subsection{Going Up Theorem}

\begin{proposition}\thlabel{prop:int-ext-field-field}
    Let $A\subseteq B$ be an integral extension of integral domains. Then $A$ is a field if and only if $B$ is a field.
\end{proposition}
\begin{proof}
    $\implies$ If $x\in B\backslash\{0\}$ is integral over $A$, then 
    \begin{equation*}
        x^n + a_{n - 1}x^{n - 1} + \dots + a_0 = 0
    \end{equation*}
    for some $a_i\in A$. Then, $x(x^{n - 1} + a_{n - 1}x^{n - 2} + \dots + a_1) = -a_0$, in particular, $x$ is a unit in $B$.

    $\impliedby$ Let $x\in A\backslash\{0\}$. Then, $x^{-1}\in B$ is integral over $A$ and satisfies an equation of the form 
    \begin{equation*}
        x^{-n} + a_{n - 1}x^{-(n - 1)} + \dots + a_0 = 0.
    \end{equation*}
    Multiplying this equation by $x^{n - 1}$, we have 
    \begin{equation*}
        x^{-1} = -(a_{n - 1} + a_{n - 2}x + \dots + a_0x^{n - 1})\in A,
    \end{equation*}
    whence $A$ is a field.
\end{proof}

\begin{proposition}\thlabel{prop:q-max-iff-p-max}
    Let $A\subseteq B$ be an integral extension, $\frakq\subseteq B$ a prime ideal and $\frakp = \frakq^c = \frakq\cap A$. Then $\frakq$ is maximal if and only if $\frakp$ is maximal.
\end{proposition}
\begin{proof}
    Due to \thref{prop:int-ext-localization}, $B/\frakq$ is integral over a copy of $A/\frakp$. The conclusion now follows from the above proposition.
\end{proof}

\begin{proposition}
    Let $A\subseteq B$ be an integral extension. Let $\frakq,\frakq'\subseteq B$ be prime ideals of $B$ such that $\frakq\subseteq\frakq'$. If $\frakq\cap A = \frakq'\cap A = \frakp$, then $\frakq = \frakq'$.
\end{proposition}
\begin{proof}
    Let $S = A\backslash\frakp$ and treat all rings and ideals as $A$-modules. Then, $S^{-1}A\subseteq S^{-1}B$ is an integral extension and since $\frakq\cap S = \frakq'\cap S = \emptyset$, the ideals $S^{-1}\frakq$ and $S^{-1}\frakq'$ are prime ideals in $B$ such that 
    \begin{equation*}
        S^{-1}\frakq\cap S^{-1}A = S^{-1}(\frakq\cap A) = S^{-1}\frakp = S^{-1}(\frakq'\cap A) = S^{-1}\frakq'\cap S^{-1}A
    \end{equation*}
    where all the above equalities follow from treating $\frakp,\frakq,\frakq',A$ as $A$-submodules of $B$, in particular, due to \thref{prop:localization-commutes-modules}.

    But note that $S^{-1}\frakp$ is maximal in $A$ whence $S^{-1}\frakq = S^{-1}\frakq'$ due to the previous proposition. But recall that under localization, the contraction after extension of prime ideals is the prime ideal itself, whereby the contraction of $S^{-1}\frakq$ is $\frakq$ whence $\frakq = \frakq'$.
\end{proof}

\begin{lemma}
    Let $A\subseteq B$ be rings, $B$ integral over $A$, and let $\frakp$ be a prime ideal of $A$. Then there is a prime ideal $\frakq$ of $B$ such that $\frakq\cap A = \frakp$.
\end{lemma}

\subsection{Going Down Theorem}

\begin{definition}
    An integral domain is said to be \textit{normal} if it is integrally closed in its field of fractions.
\end{definition}

For example, $\Z$ is integrally closed since the only algebraic integers in $\Q$ are the integers.

\begin{lemma}\thlabel{lem:localization-maps-closure-to-closure}
    Let $A\subseteq B$ be rings and $C$ the integral closure of $A$ in $B$. Let $S\subseteq A$ be multiplicatively closed. Then $S^{-1}C$ is the integral closure of $S^{-1}A$.
\end{lemma}
\begin{proof}
    Since $C$ is integral over $A$, we have that $S^{-1}C$ is integral over $S^{-1}A$. It remains to show that any element that is integral over $S^{-1}A$ is contained in $S^{-1}C$. Indeed, let $b/s\in S^{-1}B$ be an element in $S^{-1}A$ that is contained in the integral closure. Then, there are $a_i/s_i$ such that 
    \begin{equation*}
        (b/s)^n + a_{n - 1}/s_{n - 1}(b/s)^{n - 1} + \cdots + a_0/s_0 = 0
    \end{equation*}
    Let $t = s_1\cdots s_{n - 1}$ and multiply the equation throughout by $(st)^n$ to obtain 
    \begin{equation*}
        \frac{(bt)^n + b_{n - 1}(bt)^{n - 1} + \cdots + b_0}{1} = 0.
    \end{equation*}
    Thus, there is $u\in S$ such that 
    \begin{equation*}
        u\left[(bt)^n + b_{n - 1}(bt)^{n - 1} + \cdots + b_0\right] = 0
    \end{equation*}
    Again, multiply the equation by $u^{n - 1}$ to obtain 
    \begin{equation*}
        (ubt)^n + c_{n - 1}(ubt)^{n - 1} + \cdots + c_0 = 0,
    \end{equation*}
    consequently, $ubt$ is integral over $A$, therefore, lies in $C$. As a result, $b/s = (ubt)/(sut)\in S^{-1}C$. This completes the proof.
\end{proof}

\begin{lemma}
    Let $A$ be an integral domain and $S\subseteq A$ a multiplicatively closed subset. If $A$ is normal, then $S^{-1}A$ is normal.
\end{lemma}
\begin{proof}
    Let $K$ denote the field of fractions of $A$. Since $A$ is an integral domain, the natural map $A\to S^{-1}A$ is an inclusion. Moreover, the inclusion $A\to K$ maps every element of $A$ to a unit and thus induces an inclusion $S^{-1}A\to K$. We can now treat $A\subseteq S^{-1}A\subseteq K$. Since $K$ is a field, the field of fractions of $S^{-1}A$ must also be contained in $K$. Therefore, it suffices to show that $S^{-1}A$ is integrally closed in $K$. But from \thref{lem:localization-maps-closure-to-closure}, we see that $S^{-1}A$ is the integral closure of $S^{-1}A$ in $S^{-1}K = K$. This completes the proof.
\end{proof}

\begin{proposition}
    Let $A$ be an integral domain. Then, the following are equivalent: 
    \begin{enumerate}[label=(\alph*)]
        \item $A$ is normal 
        \item $A_\frakp$ is normal for all $\frakp\in\spec A$ 
        \item $A_\frakm$ is normal for all $\frakm\in\mspec A$
    \end{enumerate}
\end{proposition}
\begin{proof}
    $(a)\implies(b)$ follows from the previous lemma and $(b)\implies(c)$ is obvious. We shall show that $(c)\implies(a)$. Let $K$ be the field of fractions of $A$ and $C$ denote the integral closure of $A$ in $K$. Let $\iota:A\hookrightarrow C$ be the inclusion map. We shall show that $\iota$ is a surjection. Note that both $A$ and $C$ are integral domains and $C_\frakm$ is the integral closure of $A_\frakm$ in $K$ and therefore, in $Q(A_\frakm)$, consequently, $A_\frakm = C_\frakm$ due to $(c)$. As a result, $\iota_\frakm$ is a surjection for all maximal ideals $\frakm$ implying that $\iota$ is a surjection.
\end{proof}

\begin{lemma}
    Let $C$ be the integral closure of $A$ in $B$ and let $\fraka\subseteq A$ be an ideal. Then, the integral closure of $\fraka$ in $B$ is $\sqrt{\fraka^e}$ where the extension is taken through the inclusion $A\hookrightarrow C$.
\end{lemma}
\begin{proof}
    If $x\in C$ is integral over $\fraka$, then $x$ satisfies an equationo the form 
    \begin{equation*}
        x^r + a_{r - 1}x^{r - 1} + \dots + a_0
    \end{equation*}
    with $a_i\in\fraka$. Thus, $x^r\in\fraka^e$ whence $x\in\sqrt{\fraka^e}$.

    Conversely, suppose $x\in\sqrt{a^e}$, then there is a positive integer $n$ such that $x^n\in\fraka^e$. Then, $x^n = a_1x_1 + \dots + a_mx_m$ where each $a_i\in\fraka$ and $x_i\in C$. Let $M = A[x_1,\dots,x_m]$. Since each $x_i$ is integral over $A$, $M$ is a finitely generated $A$-module. Let $\phi: M\to M$ be the homomorphism given by $\phi(y) = x^ny$. Then, $\phi(M)\subseteq\fraka M$. Thus, $\phi$ satisfies and equation of the form 
    \begin{equation*}
        \phi^r + a_{r - 1}\phi^{r - 1} + \dots + a_0\id = 0
    \end{equation*}
    whre $a_i\in\fraka$. Thus, $x$ is integral over $\fraka$.
\end{proof}

\begin{proposition}
    Let $A\subseteq B$ be integral domains with $A$ integrally closed. Let $\alpha\in B$ be integral over an ideal $\fraka$ of $A$. Then $\alpha$, viewed as an element of $L := Q(B)\supseteq Q(A) =: K$ is algebraic over the field of fractions $K$ of $A$. Further, if the minimal polynomial of $\alpha$ over $K$ is given by $x^n + a_{n - 1}x^{n - 1} + \cdots + a_0$, then each $a_i$ is an element of $\sqrt{\fraka}$.
\end{proposition}
\begin{proof}
    Let $\alpha_1,\dots,\alpha_k$ be the distinct conjugates of $\alpha$ in $\overline K$, an algebraic closure of $K$ containing $L$. Then, each $\alpha_i$ is integral over $\fraka$. The irreducible polynomial of $\alpha$ over $K$ is given by 
    \begin{equation*}
        \prod_{i = 1}^k \left(x - \alpha_i\right)^{e}
    \end{equation*}
    for some exponent $e$. In particular, the coefficients of the non-leading terms are polynomials in th $\alpha_i$'s whence are integral over $\fraka$ and also lie in $A$, whence are elements of $\sqrt{\fraka}$. This completes the proof.
\end{proof}

\begin{theorem}[Going Down Theorem]
    Let $A\subseteq B$ be an integral extension of integral domains with $A$ integrally closed. Suppose $\frakp_1\supseteq\dots\supseteq\frakp_n$ are prime ideals in $A$ and correspondingly $\frakq_1\supseteq\dots\supseteq\frakq_m$ are prime ideals in $B$ with $m < n$ and $\frakq_i\cap A = \frakp_i$ for $1\le i\le m$, then there are prime ideals $\frakq_{m + 1}\supseteq\frakq_{n}$ with $\frakq_m\supseteq\frakq_{m + 1}$ such that $\frakq_i\cap A = \frakp_i$ for $m < i\le n$.
\end{theorem}
\begin{proof}
    We shall prove this in the case $m = 1$ and $n = 2$. This obviously suffices to prove the theorem in its full generality. Consider the composition of maps 
    \begin{equation*}
        A\longrightarrow B\longrightarrow B_{\frakq_1}
    \end{equation*}
    where the composition shall be denoted by $f: A\to B_{\frakq_1}$. It suffices to show that there is a prime in $B_{\frakq_1}$ contracting to $\frakp_2$. Due to \thref{thm:prime-is-contraction}, it suffices to show that $\frakp^{ec} = \frakp$ where the extension and contraction is taken with respect to $f$.

    Let $x\in\frakp_2 B_{\frakq_1}$. Then, $x = y/s$ for some $y\in B\frakp_2$ and $s\in S$. Note that $s$ is integral over the ideal $(1)$ in $A$ and thus its minimal polynomial over $K$ is of the form 
    \begin{equation*}
        t^r + a_{r - 1}t^{r - 1} + \dots + a_0
    \end{equation*}
    where $a_i\in A$. 
    
    Now, $y\in B$ and lies in $\frakp_2B\subseteq\sqrt{\frakp_2B}$ whence is integral over $\frakp_2$ and hence its minimal polynomial over $K$ is of the form 
    \begin{equation*}
        t^{r'} + b_{r' - 1}t^{r' - 1} + \dots + b_0
    \end{equation*}
    with $b_i\in\sqrt{\frakp_2} = \frakp_2$. Since $s = y/x$ in $Q(B)$, the minimal polynomials of $s$ and $y$ over $K$ must have the same degree, that is, $r = r'$ and $b_i = x^{r - i}a_i$ for $0\le i\le r - 1$. If $x\notin\frakp_2$, then $a_i\in\frakp_2$ for $0\le i\le r - 1$, which would imply $s^r\in\frakp_2B\subseteq\frakp_1B\subseteq\frakq_1$, i.e. $s\in\frakq_1$, which is absurd. Thus, $x\in\frakp_2$ whence $\frakp_2^{ec}\subseteq\frakp_2$, which completes the proof.
\end{proof}

\section{Noether's Normalization Lemma}

\begin{lemma}\thlabel{lem:towards-noether-normalization}
    Let $k$ be a field and $F\in k[X_1,\dots,X_n]$ a non-constant polynomial. Then there is a $k$-algebra automorphism 
    \begin{equation*}
        \varphi: k[X_1,\dots,X_n]\to k[X_1,\dots,X_n]
    \end{equation*}
    such that $\varphi(X_n) = X_n$ and 
    \begin{equation*}
        \varphi(F) = aX_n^d + f_{d - 1}X_n^{d - 1} + \cdots + f_1X_n + f_0
    \end{equation*}
    where $f_i\in k[X_1,\dots,X_{n - 1}]$ for $1\le i\le d - 1$.
\end{lemma}
\begin{proof}
    We shall pick an automorphism of the form $\varphi(X_i) = X_i + X_n^{t_i}$ for some positive integer $t_i$ for each $1\le i\le n - 1$. We shall choose these $t_i$'s at the end of the proof.

    First, note that for $1\le i\le n - 1$, $\varphi(X_i - X_n^{t_i}) = X_i$ whence $\varphi$ is a surjection. Since $k[X_1,\dots,X_n]$ is a noethering, $\varphi$ is an isomorphism.

    Let $\Lambda\subseteq\N^n$ be a finite subset such that 
    \begin{equation*}
        F = \sum_{\alpha\in\Lambda}a_\alpha X^\alpha
    \end{equation*}
    where $a_\alpha\in k^\times$ for each $\alpha\in\Lambda$. For each $\alpha\in\Lambda$, define $\omega(\alpha) = t_1\alpha_1 + \cdots + t_{n - 1}\alpha_{n - 1} + \alpha_n$.

    Choose a positive integer $N$ greater than 
    \begin{equation*}
        \max_{\alpha\in\Lambda}\max_{1\le i\le n}\alpha_i
    \end{equation*}
    and set $t_i = N^i$ for $1\le i\le n - 1$. It is not hard to see that all the $\omega(\alpha)$'s are distinct. 

    We have 
    \begin{equation*}
        \varphi(F) = \sum_{\alpha\in\Lambda}\left(a_\alpha\prod_{i = 1}^{n - 1}(X_i + X_n^{t_i})^{\alpha_i}\right)X_n^{\alpha_n}
    \end{equation*}
    and since the $\omega(\alpha)$'s are distinct, there is a unique term in the above expansion that contributes to the term with maximum exponent of $X_n$ whence the coefficient of this term is a constant in $K^\times$. This completes the proof.
\end{proof}

\begin{theorem}[Noether Normalization]\thlabel{thm:noether-normalization}
    Let $k$ be a field and $A$ a finitely generated $k$-algebra. Then, there are $z_1,\dots,z_m\in A$ such that 
    \begin{enumerate}[label=(\alph*)]
        \item $z_1,\dots,z_m$ are algebraically independent over $k$\footnote{$m = 0$ is permitted}. That is, the evaluation map 
        \begin{equation*}
            \mathbf{ev}: k[X_1,\dots,X_m]\onto k[z_1,\dots,z_m]
        \end{equation*}
        from the ring of polynomials in $m$ variables over $k$ is an isomorphism. 
        \item $A$ is integral over $k[z_1,\dots,z_m]$.
    \end{enumerate}
\end{theorem}
\begin{proof}
    We shall prove this statement by induction on the cardinality $n$ of the smallest generating set of $A$ as a $k$-algebra. The base case with $n = 0$ is trivial. Since $A$ is a finitely generated $k$-algebra, there is a surjective ring homomomrphism 
    \begin{equation*}
        \pi: k[X_1,\dots,X_n]\onto A.
    \end{equation*}
    Choose a non-constant polynomial $G\in\ker\pi$. Due to \thref{lem:towards-noether-normalization}, there is an automorphism $\varphi$ of $k[X_1,\dots,X_n]$ which sends $G$ to a polynomial $F$ of the form 
    \begin{equation*}
        aX_n^d + f_{d - 1}X_n^{d - 1} + \cdots + f_1X_n + f_0
    \end{equation*}
    where $a\in k^\times$. We now have the following sequence of ring homomorphisms
    \begin{equation*}
        k[X_1,\dots,X_n]\stackrel{\varphi^{-1}}{\longrightarrow}k[X_1,\dots,X_n]\stackrel{\pi}{\longrightarrow} A
    \end{equation*}
    with $F\in\ker(\pi\circ\varphi^{-1})$. Let $x_i = (\pi\circ\varphi^{-1})(X_i)$, then, $F(x_1,\dots,x_n) = 0$. That is, 
    \begin{equation*}
        x_n^d + a^{-1}f_{d - 1}(x_1,\dots,x_{n - 1})x_n^{d - 1} + \cdots + a^{-1}f_0(x_0,\dots,x_{n - 1}) = 0,
    \end{equation*}
    and thus, $x_n$ is algebraic over $B = k[x_1,\dots,x_{n - 1}]$. Due to the induction hypothesis, there are algebraically independent $z_1,\dots,z_m\in B$ such that $B$ is integral over $k[z_1,\dots,z_m]$. 

    We have shown that $x_n$ is integral over $B$ and thus $B\subseteq A$ is an integral extension whence $k[z_1,\dots,z_m]\subseteq A$ is an integral extension. This completes the proof.
\end{proof}

\subsection{Various Forms of the Nullstellensatz}

\begin{lemma}[Zariski's Lemma]
    Let $K/k$ be an extension of fields such that $K$ is a finitely generated $k$-algebra. Then, $K/k$ is a finite extension.
\end{lemma}
\begin{proof}
    According to \thref{thm:noether-normalization}, there are $z_1,\dots,z_m\in K$ such that $K$ is integral over $k[z_1,\dots,z_m]$, which is an integral domain whence a field due to \thref{prop:int-ext-field-field}. We note that $m$ may not be positive since a polynomial ring can never be a field. Hence, $K/k$ is algebraic and since $K$ is a finitely generated $k$-algebra, the extension $K/k$ must be finite.
\end{proof}

\begin{theorem}[Hilbert's Nullstellensatz, Weak Form 1]\thlabel{thm:weak-nullstellensatz-1}
    Let $k$ be an algebraically closed field. Then, any maximal ideal in $k[x_1,\dots,x_n]$ is of the form $(x_1 - a_1,\dots,x_n - a_n)$.
\end{theorem}
\begin{proof}
    It suffices to show the converse. Let $\frakm$ be a maximal ideal in $k[x_1,\dots,x_n]$. We now have a commutative diagram 
    \begin{equation*}
        \xymatrix {
            k\ar@{^(->}[r]^-{\iota}\ar[rd]_-{\varphi} & k[x_1,\dots,x_n]\ar@{->>}[d]^-{\pi}\\
            & k[x_1,\dots,x_n]/\frakm = K
        }
    \end{equation*}
    where $\varphi := \pi\circ\iota$.

    The map $\varphi$ gives $K$ the structure of a finitely generated $k$-algebra and thus $\varphi$ is surjective (since $k$ is algebraically closed). Let $\pi(x_i) = a_i$ for $1\le i\le n$. Due to the surjectivity of $\varphi$, for each $a_i$, there is some $a_i'\in k$ with $\varphi(a_i') = a_i$ whence $\pi(x_i - a_i') = 0$ and 
    \begin{equation*}
        (x_1 - a_1',\dots,x_n - a_n')\subseteq\frakm.
    \end{equation*}
    But since the former is a maximal ideal, we must have equality.
\end{proof}

\begin{theorem}[Hilbert's Nullstellensatz, Weak Form 2]\thlabel{thm:weak-nullstellensatz-2}
    Let $k$ be an algebraically closed field and $S\subseteq k^n$. Then, $I(S) = (1)$ if and only if $S = \emptyset$.
\end{theorem}
\begin{proof}
    $(\implies)$ If $S\ne\emptyset$, then let $a = (a_1,\dots,a_n)$ be a point in $S$. Then, $I(S)\subseteq\frakm_a = (x_1 - a_1,\dots,x_n - a_n)$.
    $(\impliedby)$ If $I(S)\ne(1)$, then it is contained in some maximal ideal $\frakm = \frakm_a$ for some $a\in k^n$, whence $a\in S$. This completes the proof.
\end{proof}

\begin{theorem}[Hilbert's Nullstellensatz, Strong Form]
    Let $\fraka\subseteq k[x_1,\dots,x_n]$ be an ideal. Then, 
    \begin{equation*}
        I(V(\fraka)) = \sqrt{\fraka}
    \end{equation*}
\end{theorem}
The following proof is due to Rabinowitsch.
\begin{proof}
    First, note that the inclusion $\sqrt{\fraka}\subseteq I(V(\fraka))$ is obvious for if $f\in\sqrt{\fraka}$, then there is a positive integer $r$ such that $f^r\in\fraka$ whence $f^r$ vanishes at all points in $V(\fraka)$ and thus $f^r\in I(V(\fraka))$.

    We shall now prove the other inclusion. Since $\fraka$ is finitely generated, let $f_1,\dots,f_m$ be a set of generators for $\fraka$ and let $f\in I(V(\fraka))$. Consider now the ring $B = k[x_0,x_1,\dots,x_n]$ which contains $A = k[x_1,\dots,x_n]$ as a subring and treat all polynomials as elements of $B$. The polynomials
    \begin{equation*}
        f_1,\dots,f_m, 1 - x_0f
    \end{equation*}
    do not have any common zeros. Let $\frakb\subseteq B$ denote the ideal generated by these polynomials. Due to \thref{thm:weak-nullstellensatz-2} and the fact that the polynomials have no common zeros, we must have $\frakb = B$. Consequently, there are polynomials $g_0,\dots,g_n\in k[x_1,\dots,x_n]$ such that 
    \begin{equation*}
        1 = g_0(1 - x_0f) + g_1f_1 + \dots + g_mf_m.
    \end{equation*}

    Consider now the evaluation map $\mathbf{ev}: B\to k(x_1,\dots,x_n)$ which maps $x_0\mapsto 1/f$ and $x_i\mapsto x_i$ for $1\le i\le n$. It is not hard to see that this is a ring homomorphism. Under this map, the above equality transforms to 
    \begin{equation*}
        1 = g_1(1/f,x_1,\dots,x_n)f_1(x_1,\dots,x_n) + \dots + g_m(1/f,x_1,\dots,x_n)f_m(x_1,\dots,x_m).
    \end{equation*}
    Since all the $g_i$'s and $f_i$'s are polynomials, we may clear out the denominators by multiplying with a suitable power of $f$, say $f^N$. Then, we have 
    \begin{equation*}
        f^N = h_1(x_1,\dots,x_n)f_1 + \dots + h_m(x_1,\dots,x_n)f_m
    \end{equation*}
    whereby $f^N\in\fraka$ for some positive integer $N$ and equivalently, $f\in\sqrt{\fraka}$. This completes the proof.
\end{proof}

\chapter{Noetherian and Artinian Rings and Modules}
\section{Chain Conditions}

A totally ordered sequence $\{x_n\}_{n = 1}^\infty$ in the poset $(\Sigma,\leqq)$ is said to be \textit{stationary} if there is an index $n$ such that $x_n = x_{n + 1} = \cdots$.

\begin{definition}
    An $A$-module $M$ is said to be \textit{noetherian} or equivalently said to satisfy the \textit{ascending chain condition} if every chain in the poset of submodules of $M$ ordered by $\subseteq$ is stationary.

    Similarly, $M$ is said to be \textit{artinian} equivalently said to satisfy the \textit{descending chain condition} if every chain in the poset of submodules of $M$ ordered by $\supseteq$ is stationary.
\end{definition}

A ring $A$ is said to be noetherian (resp. artinian) if it is noetherian (resp. artinian) as an $A$-module.

\begin{proposition}
    Let $(\Sigma,\leqq)$ be a poset. Then, the following are equivalent: 
    \begin{enumerate}[label=(\alph*)]
        \item Every chain in $\Sigma$ is stationary.
        \item Every subset of $\Sigma$ has a maximal element.
    \end{enumerate}
\end{proposition}
The proof is omitted owing to its triviality.

\begin{lemma}
    An $A$-module $M$ is noetherian if and only if every submodule is finitely generated.
\end{lemma}
\begin{proof}
    
\end{proof}
\begin{corollary}
    A ring $A$ is noetherian if and only if every ideal is finitely generated.
\end{corollary}
\begin{corollary}
    Every submoule of a noetherian $A$-module is noetherian.
\end{corollary}

\begin{proposition}
    $M$ is a noetherian (resp. artinian) $A$-module if and only if it is a noetherian (resp. artinian) $A/\Ann_A(M)$-module.
\end{proposition}
\begin{proof}
    Since the poset of $A/\Ann_A(M)$-submodules of $M$ is the same as the poset of $A$-submodules of $M$, the conclusion follows.
\end{proof}

\begin{lemma}[2/3-lemma]
    Consider the short exact sequence $0\rightarrow M'\rightarrow M\rightarrow M''\rightarrow 0$. Then $M$ is noetherian (resp. artinian) if and only if both $M'$ and $M''$ are noetherian (resp. artinian).
\end{lemma}
\begin{proof}
    
\end{proof}
\begin{corollary}
    Let $\{M_i\}_{i = 1}^n$ be $A$-modules. Then, $\displaystyle\bigoplus_{i = 1}^n M_i$ is noetherian (resp. artinian) if and only if each $M_i$ is noetherian (resp. artinian).
\end{corollary}
\begin{proof}
    The forward direction is obvious. For the converse, induct on $n$ using the short exact sequence: 
    \begin{equation*}
        0\longrightarrow M_n\longrightarrow\bigoplus_{i = 1}^n M_i\longrightarrow\bigoplus_{i = 1}^{n - 1}M_i\longrightarrow 0
    \end{equation*}
\end{proof}

\begin{proposition}
    If $A$ is a noethering (resp. artinian ring), then so is $A/\fraka$ for any ideal $\fraka$ in $A$.
\end{proposition}
\begin{proof}
    $A/\fraka$ is a noetherian (resp. artinian) $A$-module and thus a noetherian (resp. artinian) $A/\fraka$-module.
\end{proof}

\begin{proposition}
    Let $M$ be an $A$-module and $\phi\in\End_A(M)$. 
    \begin{enumerate}[label=(\alph*)]
        \item If $M$ is noetherian and $\phi$ is surjective, then $\phi$ is injective.
        \item If $M$ is artinian and $\phi$ is injective, then $\phi$ is surjective.
    \end{enumerate}
\end{proposition}
\begin{proof}
\begin{enumerate}[label=(\alph*)]
\item Consider the ascending chain of submodules
\begin{equation*}
    \ker\phi\subseteq\ker\phi^2\subseteq\cdots
\end{equation*}
Since $M$ is noetherian, there is an index $n$ such that $\ker\phi^n = \ker\phi^{n + 1}$. Let $x\in\ker\phi^n$. Due to the surjectivity of $\phi$, there is $y\in M$ such that $\phi(y) = x$, whence $\phi^{n + 1}(y) = 0$ and $y\in\ker\phi^{n + 1} = \ker\phi^n$. Therefore, $\ker\phi^n = 0$ and $\phi$ is injective.

\item Consider the descending chain of submodules
\begin{equation*}
    \im\phi\supseteq\im\phi^2\supseteq\cdots
\end{equation*}
Since $M$ is artinian, there is an index $n$ such that $\im\phi^n = \im\phi^{n + 1}$. Then, for every $x\in M$, there is $y\in M$ such that $\phi^n(x) = \phi^{n + 1}(y)$, whence $x = \phi(y)$, this establishes surjectivity.
\end{enumerate}
\end{proof}

\begin{lemma}
    Supose there is a sequence of maximal ideals $\frakm_1,\ldots,\frakm_n$ in $A$ such that $(0) = \frakm_1\cdots\frakm_n$. Then, $A$ is a noethering if and only if it is artinian.
\end{lemma}
\begin{proof}
    Suppose $A$ is a noethering. Consider the $A/\frakm_i$ module $\frakm_1\cdots\frakm_{i - 1}/\frakm_1\cdots\frakm_i$
\end{proof}

\section{Noetherian Rings}

Recall that $A$ is a noetherian ring if it is a noetherian $A$-module.

\begin{lemma}
    If $A$ is Noetherian and $\phi: A\to B$ is a surjective ring homomorphism, then $B$ is also Noetherian.
\end{lemma}
\begin{proof}
    Since $B\cong A/\ker\phi$, the conclusion follows.
\end{proof}

\begin{proposition}
    If $A$ is a noethering and $S\subseteq A$ is a multiplicative subset, then $S^{-1}A$ is a noethering.
\end{proposition}
\begin{proof}
    Recall that every ideal in $S^{-1}A$ is finitely generated. Let $I\subseteq S^{-1}A$ be an ideal then there is $\fraka\subseteq A$ an ideal such that $S^{-1}\fraka = I$. Since $A$ is noetherian, $\fraka$ is generated by a finite set $\{x_1,\ldots,x_n\}$, whereby $I$ is generated by the set $\{x_1/1,\ldots,x_n/1\}$. This completes the proof.
\end{proof}

But recall, as we have seen earlier, that being a noethering is not a local property, a counterexample to which is an infinite product of fields.

\begin{theorem}[Hilbert Basis Theorem]
    If $A$ is Noetherian, then so is $A[x]$.
\end{theorem}
Note that the converse is also true since $A\cong A[x]/(x)$. The following proof is due to Sarges.
\begin{proof}
    We shall show that every ideal in $A[x]$ is finitely generated. Suppose not and let $I\subseteq A[x]$ be an ideal that is not finitely generated. Choose $f_1\in I$ with minimum degree. Now, inductively, choose $f_{k + 1}\in I\backslash(f_1,\ldots,f_k)$ with the minimum degree. Obviously, this process goes on indefinitely, since we have assumed $I$ to not be finitely generated. We now have 
    \begin{align*}
        f_1 &= a_1x^{d_1} + \text{lower degree terms}\\
        f_2 &= a_2x^{d_2} + \text{lower degree terms}\\
        &\vdots\\
        f_n &= a_nx^{d_n} + \text{lower degree terms}\\
        &\vdots
    \end{align*}
    with $d_1\le d_2\le\cdots$. We also have the following ascending chain of ideals in $A$, 
    \begin{equation*}
        (a_1)\subseteq(a_1,a_2)\subseteq\cdots
    \end{equation*}
    Therefore, there is $n\in\N$ such that $(a_1,\ldots,a_n) = (a_1,\ldots,a_n,a_{n + 1})$. Consequently, we may write $a_{n + 1}$ as a linear combination of $a_1,\ldots,a_n$, say 
    \begin{equation*}
        a_{n + 1} = b_1a_1 + \cdots + b_na_n
    \end{equation*}
    for some $b_1,\ldots,b_n\in A$. Let 
    \begin{equation*}
        g = f_{n + 1} - (b_1x^{d_{n + 1} - d_1}f_1 + \cdots + b_nx^{d_{n + 1} - d_n}f_n)
    \end{equation*}
    It is not hard to argue that $g\in I\backslash(f_1,\ldots,f_n)$, but $\deg g\le\deg f_{n + 1}$, a contradiction. This completes the proof.
\end{proof}

An analogous theorem, with an analogous proof is true wherein $A[x]$ is replaced by $A\llbracket x\rrbracket$.

\begin{corollary}
    For a field $k$, the polynomial ring $k[x_1,\ldots,x_n]$ in finitely many indeterminates is noetherian.
\end{corollary}

\begin{corollary}
    If $A$ is a noethering, then every $A$-algebra of finite type is a noethering.
\end{corollary}

If $A\subseteq B$ is a ring extension with both $A$ and $B$ noetherian, it is not necessary that $B$ is an $A$-algebra of finite type. Indeed, consider $\overline\Q/\Q$ an extension of fields.

On the other hand, even if $B$ is an $A$-algebra of finite type and noetherian, it is not necessary for $A$ to be noetherian. Indeed, consider the ring inclusion 
\begin{equation*}
    k[xy,xy^2,\ldots]\subsetneq k[x,y]
\end{equation*}
The former is not noetherian owing to the chain of ideals 
\begin{equation*}
    (xy)\subsetneq(xy,xy^2)\subsetneq\cdots
\end{equation*}
while the latter obviously is noetherian.

\begin{proposition}
    Let $M$ be a noetherian $A$-module. Then, $A/\Ann_A(M)$ is a noethering.
\end{proposition}
\begin{proof}
    Since $M$ is noetherian, it is finitely generated. Let $\{m_1,\ldots,m_n\}$ be a set of generators. Then, $\displaystyle\Ann_A(M) = \bigcap_{i = 1}^n\Ann_M(m_i)$. Consider the map $\phi: A\to M^n$ given by $\phi(a) = (am_1,\ldots,am_n)$. Note that $\ker\phi = \Ann_A(M)$. Thus, we have a short exact sequence 
    \begin{equation*}
        0\longrightarrow A/\Ann_A(M)\longrightarrow A\longrightarrow\phi(A)\longrightarrow 0.
    \end{equation*}
    Consequently, $A/\Ann_A(M)$ is a noetherian $A$-module and thus a noetherian $A/\Ann_A(M)$-module, whence a noethering.
\end{proof}

\begin{lemma}[Artin-Tate Lemma]\thlabel{lem:artin-tate}
    Let $A\subseteq B\subseteq C$ be rings with $A$ noetherian, and $C$ an $A$-algebra of finite type. If either 
    \begin{enumerate}[label=(\alph*)]
        \item $C$ is a finite $B$-algebra\footnote{Recall that this is the same as being finitely generated as a $B$-module}, or 
        \item $C$ is integral over $B$,
    \end{enumerate}
    then $B$ is an $A$-algebra of finite type.
\end{lemma}
\begin{proof}
    Note that $(a)\iff(b)$ due to \thref{thm:equivalence-integral-extension}. We shall show that $(a)$ implies the desired conclusion. Since $C$ is an $A$-algebra of finite type, say it is generated by $\{x_1,\ldots,x_n\}$ as an $A$-algebra. Similarly, since it is a finite $B$-algebra, it is finitely generated as a $B$-module, say by $\{y_1,\ldots,y_m\}$. Therefore, there are coefficients $b_{ij}$ and $b_{ijk}$ in $B$ such that 
    \begin{align*}
        x_i &= \sum_{j = 1}^m b_{ij}y_j\\
        y_iy_j &= \sum_{k = 1}^m b_{ijk}y_k.
    \end{align*}
    Let $B_0 = A[\{b_{ij}\}\cup\{b_{ijk}\}]\subseteq B$. Since $A$ is noetherian, and $B_0$ is an $A$-algebra of finite type, it is a noethering. 
    
    Now, since $C$ is a finite type $A$-algebra, every element of $C$ is a polynomial in the $x_i$'s with coefficients in $A$. Using the first set of relations, it is a polynomial in the $y_i$'s with coefficients in $B_0$. Using the second set of relations, it is a linear combination of the $y_i$'s with coefficients in $B_0$, whereby $C$ is a finite $B_0$-algebra.

    Since $C$ is a finitely generated $B_0$-module it is noetherian and thus $B$, being a $B_0$-submodule, is a finitely generated $B_0$-module and consequently, a $B_0$-algebra of finite type. Thus, $B$ is an $A$-algebra of finite type.
\end{proof}

\begin{lemma}[Cohen]
    $A$ is a noethering if and only if every prime ideal in $A$ is finitely generated.
\end{lemma}
\begin{proof}
    We shall prove the converse. Let $\Sigma$ be the poset of proper ideals that are not finitely generated, which we suppose is nonempty. If $\mathscr C$ is a chain in $\Sigma$, then $I = \bigcup_{\fraka\in\mathscr C}\fraka$ may not be finitely generated for if it were, then there is a set of generators $\{r_1,\ldots,r_n\}$ and thus there would exist $\fraka\in\mathscr C$ containing $\{r_1,\ldots,r_n\}$ whereby equal to $I$, contradiction. Hence, $I$ is an upper bound for $\mathscr C$ and due to Zorn's Lemma, there is a maximal element $\frakp\in\Sigma$.

    Since $\frakp$ may not be prime, there are $x,y\notin\frakp$ such that $xy\in\frakp$. Consider $\frakp + (x)$. This strictly contains $\frakp$ and therefore, is finitely generated. The generators of $\frakp + (x)$ are of the form $p_i + a_ix$ for $1\le i\le n$ for some positive integer $n$. 

    Consider the ideal $(\frakp: x)$. This contains $\frakp + (y)$ which strictly contains $\frakp$ an thus, is finitely generated. Say $(\frakp: x) = (x_1,\ldots,x_m)$ for some positive integer $m$. Let $\fraka = (p_1,\ldots,p_n,xx_1,\ldots,xx_m)$. We contend that $\fraka = \frakp$.

    Obviously, $\fraka\subseteq\frakp$. On the other hand, for any $p\in\frakp$, there is a representation 
    \begin{equation*}
        p + x = b_1p_1 + \cdots + b_np_n + cx
    \end{equation*}
    for some $b_1,\ldots,b_n,c\in A$, consequently, $p\in\fraka$. Thus, $\fraka = \frakp$, which is a contradiction to the choice of $\frakp$. Hence, $\Sigma$ is empty and $A$ is a noethering.
\end{proof}

\subsection{Primary Decomposition}

\begin{definition}[Irreducible]
    An ideal $\fraka\subseteq A$ is said to be \textit{irreducible} if for all ideals $\frakb,\frakc\subseteq A$,
    \begin{equation*}
        \fraka = \frakb\cap\frakc\Longrightarrow \fraka = \frakb\text{ or }\fraka = \frakc
    \end{equation*}
\end{definition}

\begin{lemma}
    In a noethering, every ideal can be expressed as a finite intersection of irreducible ideals.
\end{lemma}
\begin{proof}
    Let $\Sigma$ be the poset of ideals that cannot be expressed as a finite intersection of irreducible ideals in $A$. Suppose $\Sigma$ is nonempty, then every chain in $\Sigma$ is finite (owing to noetherian-ness) whence has an upper bound, thus $\Sigma$ has a maximal element (Zorn's Lemma), say $\fraka$. Note that $\fraka$ cannot be irreducible, therefore, there are ideals $\frakb,\frakc$ properly containing $\fraka$ such that $\fraka = \frakb\cap\frakc$. Due to the maximality of $\fraka$, both $\frakb$ and $\frakc$ can be expressed as a finite intersection of irreducible ideals in $A$, as a result, so can $\fraka$, a contradiction. Thus $\Sigma$ must be empty and the proof is complete.
\end{proof}

\begin{lemma}
    Every irreducible ideal in a noethering is primary.
\end{lemma}
\begin{proof}
    Let $\frakq\subseteq A$ be an irreducible ideal. We shall show that $(0)$ is primary in $A/\frakq$, which is equivalent to $\frakq$ being primary. Let $x,y\in A/\frakq$ such that $xy = 0$. If $x\ne 0$, then consider the chain 
    \begin{equation*}
        \Ann(y)\subseteq\Ann(y^2)\subseteq\cdots
    \end{equation*}
    Since $A/\frakq$ is a noethering, there is a positive integer $n$ such that $\Ann(y^n) = \Ann(y^{n + 1})$. We contend that $(x)\cap(y^n) = 0$. Indeed, if $z\in(x)\cap(y^n)$, then there are $u,v\in A/\frakq$ such that $z = ux = vy^n$. Then,
    \begin{equation*}
        vy^{n + 1} = zy = uxy = 0
    \end{equation*}
    whence $v\in\Ann(y^{n + 1}) = \Ann(y^n)$, whereby $z = 0$. But since $(0)$ is irreducible and $x\ne 0$, we must have $y^n = 0$ and $(0)$ is primary. This completes the proof.
\end{proof}

\begin{corollary}
    A noethering has finitely many minimal prime ideals.
\end{corollary}
\begin{proof}
    Since $A$ is noetherian, the ideal $(0)$ has a primary decomposition and the minimal primes belonging to $(0)$ are precisely the minimal primes in $A$ and thus are finite.
\end{proof}


\section{Artinian Rings}

Recall that $A$ is artinian if it is an artinian module over itself.

\begin{proposition}
    Let $A$ be an artinian ring. Then $A$ has finitely many maximal ideals.
\end{proposition}
\begin{proof}
    Suppose not. Then, we have a sequence $\{\frakm_i\}_{i = 1}^\infty$ of pairwise distinct maximal ideals. Consider the sequence of ideals $\{\frakm_1\cdots\frakm_n\}_{n = 1}^\infty$. We contend that the inclusion $\frakm_1\cdots\frakm_{n - 1}\supseteq\frakm_1\cdots\frakm_n$ is strict. Indeed, for all $1\le i\le n - 1$, pick $x_i\in\frakm_i\backslash\frakm_n$. Then, $x_1\cdots x_{n - 1}\notin\frakm_n$, since $A\backslash\frakm_n$ is a multiplicatively closed subset. Thus, $x_1\cdots x_{n - 1}\in\frakm_1\cdots\frakm_{n - 1}\backslash\frakm_1\cdots\frakm_n$. This is a contradiction to $A$ being artinian.
\end{proof}

\begin{proposition}
    Let $A$ be an artinian ring. Then every prime ideal in $A$ is maximal.
\end{proposition}
\begin{proof}
    Let $\frakp$ be a prime ideal in $A$. Then $A' = A/\frakp$ is an Artinian integral domain. We shall show that this is a field, for which it suffices to show that every element is invertible. Choose $x'\in A'$ and let $\phi: A'\to A'$ be the $A'$-module homomorphism that maps $a\mapsto x'a$. Since $A'$ is an integral domain, this map is injective and since $A'$ is artinian, it is also an isomorphism. Consequently, there is some $y'\in A'$ such that $x'y' = 1$ and the conclusion follows.
\end{proof}

\begin{corollary}
    Let $A$ be an artinian ring. Then $\frakN(A) = \frakR(A)$.
\end{corollary}

\begin{lemma}
    Let $A$ be an artinian ring. Then $\frakN(A)$ is nilpotent.
\end{lemma}
\begin{proof}
    We shall denote $\frakN(A)$ by $\frakN$ for the sake of brevity. Consider the decreasing chain 
    \begin{equation*}
        \frakN\supseteq\frakN^2\supseteq\cdots
    \end{equation*}
    Then there is an index $n$ such that $\fraka = \frakN^n = \frakN^{n + 1} = \cdots$. Suppose for the sake of contradiction that $\fraka\ne 0$. Let $\Sigma$ be the set of ideals $\frakb$ such that $\fraka\frakb\ne 0$. Obviously $\Sigma$ is empty since it contains $\fraka$. Since $A$ is artinian, $\Sigma$ has a minimal element $\frakc$\footnote{We have not invoked Zorn to conclude this.}. 

    We contend that $\frakc$ is principal. Indeed, there is an element $x\in\frakc$ such that $x\fraka\ne 0$. Thus, $(x)\fraka\ne 0$. Owing to the minimality of $\frakc$, we must have $\frakc = (x)$. 

    Consider now the ideal $(x)\fraka$. This is a subset of $(x)$ and 
    \begin{equation*}
        ((x)\fraka)\fraka^k = (x)\fraka^{k + 1} = (x)\fraka\ne0
    \end{equation*}
    whence $(x)\fraka\in\Sigma$ and again, owing to the minimality of $(x) = \frakc$, we have $(x)\fraka = (x)$. Hence, there is some $y\in\fraka$ such that $xy = x$. We now have 
    \begin{equation*}
        x = xy = xy^2 = \cdots
    \end{equation*}
    Since $y\in\fraka\subseteq\frakN$, it is nilpotent, whence $x = 0$, a contradiction. Thus $\fraka = 0$ and this completes the proof.
\end{proof}

\begin{theorem}
    $A$ is artinian if and only if it is a noethering with krull dimension zero.
\end{theorem}
\begin{proof}
\end{proof}

\chapter{DVRs and Dedekind Domains}
\section{General Valuation Rings}

\begin{definition}[Valuation]
    A \emph{valuation} on a field $K$ is a map $v: K\to\Gamma\cup\{\infty\}$ where $\Gamma$ is an ordered abelian group such that for all $x,y\in K$,
    \begin{enumerate}
        \item $v(xy) = v(x) + v(y)$
        \item $v(x + y)\ge\min\{v(x), v(y)\}$
    \end{enumerate}
    The set 
    \begin{equation*}
        A = \{x\in K^\times\mid v(x)\ge 0\}
    \end{equation*}
    is called the \emph{valuation ring} of $K$ with respect to the valuation $v$.
\end{definition}

That the set $A$ forms a ring follows from the fact that it is closed under addition, multiplication and subtraction.

\begin{proposition}
    Let $B$ be an integral domain and $K = Q(B)$, its field of fractions. Then, $B$ is a \emph{valuation ring} of $K$ if for every $x\in K\backslash\{0\}$, we have $x\in B$ or $x^{-1}\in B$.
\end{proposition}
\begin{proof}
    Follows from the fact that $0 = v(1) = v(xx^{-1}) = v(x) + v(x^{-1})$.
\end{proof}

\begin{proposition}
    Let $B$ be a valuation ring. Then 
    \begin{enumerate}[label=(\alph*)]
        \item $B$ is a local ring. 
        \item $B$ is normal.
    \end{enumerate}
\end{proposition}
\begin{proof}
\begin{enumerate}[label=(\alph*)]
    \item We shall show that the nonunits in $B$ form an ideal. Let $\frakm$ be the set of nonunits in $B$ and choose $x\in\frakm\backslash\{0\}$, $b\in B$. Then, $bx\ne 0$ since $x$ is not a zero divisor. We contend that $bx$ is a nonunit. For if not, then $b(bx)^{-1}$ would be an inverse of $x$.

    Next, let $x,y\in\frakm\backslash\{0\}$. According to the given condition, either $x/y$ or $y/x$ are in $B$. Without loss of generality, suppose $x/y\in B$. Then $x + y = y(1 + x/y)\in\frakm$ from the conclusion of the previous paragraph. Thus $\frakm$ is an ideal and $B$ is local.

    \item Indeed, let $\alpha\in K$ be integral over $B$. If $\alpha\in B$, there is nothing to prove. If not, then it satisifes an equation of the form 
    \begin{equation*}
        \alpha^n + b_{n - 1}\alpha^{n - 1} + \cdots + b_1\alpha + b_0
    \end{equation*}
    Upon multiplying by $\alpha^{-(n - 1)}$, we can represent $\alpha$ as a sum of elements in $B$, consequently, is an element of $B$, a contradiction.
\end{enumerate}
\end{proof}


\section{Discrete Valuation Rings}

\begin{definition}[Discrete Valuation Ring]
    A valuation $v: K\to\Gamma\cup\{\infty\}$ is said to be a \emph{discrete valuation} when $\Gamma = \Z$ and $v$ is surjective. An integral domain $A$ is said to be a \emph{discrete valuation ring} if there is a discrete valuation $v$ on the field of fractions of $A$ such that $A$ is the corresponding valuation ring.
\end{definition}

First, since $A$ is a valuation ring of its field of fractions, say $K$, it is local and normal, i.e. integrally closed in $K$. Further, the maximal ideal $\frakm$ in $A$ is the set of all $x\in A$ with \underline{positive} valuations.

\begin{proposition}
    Let $A$ be a DVR. Then, $A$ is a noetherian local domain of Krull dimension $1$ such that every non-zero ideal is a power of the unique maximal ideal.
\end{proposition}
\begin{proof}
    Let $\frakm_k = \{x\in A\mid v(x)\ge k\}$. We first show that $\frakm_k$ is an ideal. Indeed, for all $x,y\in\frakm_k$, 
    \begin{equation*}
        v(x - y)\ge\min\{v(x), v(-y)\} = \min\{v(x),v(y)\}\ge k
    \end{equation*}
    and $v(xy) = v(x) + v(y)\ge k$.

    Next, we show that every non-zero ideal $\fraka$ in $A$ is one of the $\frakm_i$'s. Due to the well ordering of the naturals, there is an $x\in\fraka$ with $\displaystyle k = v(x) = \min_{a\in\fraka}v(a)$. Then, by the choice of $k$, $\fraka\subseteq\frakm_k$. Now, let $y\in\frakm_k$. Since $v$ is surjective, there is an element $z\in A$ with $v(z) = v(y) - v(x)$. Whence $xz\in\fraka$ and $v(xz) = v(y)$. Since $(xz) = (y)$, we must have $y\in\fraka$.

    Notice that these ideals form a descending chain 
    \begin{equation*}
        \frakm = \frakm_1\supseteq\frakm_2\supseteq\cdots
    \end{equation*}
    whereby any ascending chain must be finite and $A$ is noetherian.

    Choose some $a\in A$ with $v(a) = 1$, which exists due to the surjectivity of $v$. Then, $\frakm = (a)$ and consequently, $\frakm_k = (a^k) = \frakm^k$. From this, we may conclude that $\frakm$ is the unique non-zero prime ideal in $A$ and every other ideal is a power of $\frakm$. This also establishes the result about the Krull dimension.
\end{proof}

\begin{theorem}
    Let $A$ be a noetherian local domain of Krull dimension $1$, $\frakm$ its maximal ideal and $k = A/\frakm$ its residue field. Then the following are equivalent: 
    \begin{enumerate}[label=(\alph*)]
        \item $A$ is a discrete valuation ring.
        \item $A$ is normal.
        \item $\frakm$ is principal. 
        \item $\dim_k(\frakm/\frakm^2) = 1$.
        \item Every non-zero ideal is a power of $\frakm$. 
        \item There is $x\in A$ such that every nonzero ideal is of the form $(x^k)$ for $k\ge 0$. 
    \end{enumerate}
\end{theorem}
\begin{proof}
    $(a)\implies(b)$ is obvious. 

    $(b)\implies(c)$. Let $a\in\frakm$. Since the ring is noetherian, $(a)$ has a primary decomposition, but since the Krull dimension is $1$, the only non-zero prime ideal is $\frakm$, we see that $\sqrt{(a)} = \frakm$. Since we are in a noethering, there is a positive integer $n$ such that $\frakm^n\subseteq (a)$ but $\frakm^{n - 1}\subsetneq(a)$. Let $b\in\frakm^{n - 1}\backslash(a)$ and $x = a/b$, $y = x^{-1} = b/a$ in $K = Q(A)$, the field of fractions. 

    First, since $b\notin(a)$, $y\notin A$ and therefore, is not integral over $A$. Since $\frakm$ is a finitely generated $A$-module, it cannot be an $A[y]$-module lest $y$ be integral over $A$ due to \thref{thm:equivalence-integral-extension}. Hence, $y\frakm\subsetneq\frakm$.

    Now consider $y\frakm$. For any $z\in\frakm$, $yz = bz/a\in A$ since $bz\in\frakm^n\subseteq(a)$. Thus, $y\frakm\subseteq A$. Since this is an ideal and is not contained in $\frakm$, we must have $y\frakm = A$, whence $\frakm = Ax = (x)$ and is principal. 

    $(c)\implies(d)$. Let $\frakm = (a)$ for some $a\in A$. Then, $\frakm/\frakm^2 = (\overline a)$ where $\overline a$ is the image of $a$. Thus, $\dim_k(\frakm/\frakm^2)\le 1$. Now, note that $\frakm\ne\frakm^2$, lest due to \thref{lem:nakayama}, we have $\frakm = 0$. Thus, $\dim_k(\frakm/\frakm^2)\ge 1$ and the conclusion follows.

    $(d)\implies(e)$. Let $\fraka$ be a proper non-zero ideal in $A$. Then, $\sqrt{\fraka} = \frakm$ as we have argued earlier and thus, there is a least positive integer $n$ such that $\frakm^n\subseteq\fraka$. Now, $A/\frakm^n$ is an artinian local ring with maximal ideal $\overline\frakm = \frakm/\frakm^2$. Consequently, 
    \begin{equation*}
        \dim_k(\overline\frakm/\overline\frakm^2) = \dim_k(\frakm/\frakm^2) = 1
    \end{equation*}
    whence, due to \texttt{<insert reference>}, every ideal in $A/\frakm^n$ is principal, in particular, $\overline\fraka$ is principal. \textcolor{red}{TODO: complete this argument}

    $(e)\implies(f)$. Due to \thref{lem:nakayama}, $\frakm\supsetneq\frakm^2$, hence there is $x\in\frakm\backslash\frakm^2$. According to our hypothesis, $(x) = \frakm^n$ for some positive integer $n$. Due to our choice of $x$, we must have $n = 1$, whence $\frakm = (x)$. The conclusion now follows. 

    $(f)\implies(a)$. We shall explicitly create a valuation. First, note that we have $\frakm = (x)$ due to maximality and due to Nakayama's Lemma, $\frakm^k\ne\frakm^{k + 1}$ for if not, then $\frakm^k = 0$ whereby, $\frakm = 0$, upon taking radicals, a contradiction. 

    For each $a\in A$, $(a) = (x^k)$ for a unique $k$, since $(x^n)\supsetneq(x^{n + 1})$. Define $v(a) = k$ and extend it to $K = Q(A)$ by defining $v(a/b) = v(a) - v(b)$. This is obviously a well defined valuation and $v(a/b)\ge 0$ if and only if $(a) = (x^n)$ and $(b) = (x^m)$ for $n\ge m$, whence $a\in (b)$ and $a/b\in A$. Thus $A$ is the valuation ring of $K$ with respect to $v$. This completes the proof.
\end{proof}

\section{Dedekind Domains}

\begin{theorem}\thlabel{thm:dedekind-domain-equivalence}
    Let $A$ be a noetherian domain of Krull dimension $1$. Then, the following are equivalent 
    \begin{enumerate}[label=(\alph*)]
        \item $A$ is integrally closed.
        \item Every primary ideal in $A$ is a prime power in a unique way.
        \item Every local ring $A_\frakp$ is a discrete valuation ring.
    \end{enumerate}
\end{theorem}
\begin{proof}
\end{proof}

\begin{definition}
    A ring satisfying the equivalent conditions of \thref{thm:dedekind-domain-equivalence}, is said to be a \emph{Dedekind domain}.
\end{definition}

\begin{theorem}
    In a Dedekind domain, every non-zero ideal has a unique factorization as a product of prime\footnote{Which in this case, are maximal.} ideals.
\end{theorem}
\begin{proof}
    From \thref{lem:noether-dim1-ideal-pp}, every ideal in a noetherian domain of Krull dimension $1$ has a unique factorization as a product of prime ideals. Then, from \thref{thm:dedekind-domain-equivalence} and \thref{thm:chinese-remainder}, the conclusion follows.
\end{proof}

\begin{theorem}
    The ring of integers $\mathcal{O}_K$ in an \underline{algebraic number field}\footnote{An algebraic number field is a finite field extension of $\Q$} $K\supseteq\Q$ is a Dedekind domain.
\end{theorem}
\begin{proof}

\end{proof}

\section{Fractional Ideals}

\begin{definition}
    Let $A$ be an integral domain. A \emph{fractional ideal} of $A$ is a nonzero $A$-submodule of $K = Q(A)$, the field of fractions such that there is $d\in A$ with $d\fraka\subseteq A$.
\end{definition}

\chapter{Completions}
\section{Filtrations of Rings and Modules}

\begin{definition}[Filtered Ring]
    A \emph{filtered ring} $A$ is a ring $A$ together with a family $(A_n)_{n\ge 0}$ of additive subgroups of $A$ satisfying the conditions: 
    \begin{enumerate}[label=(\alph*)]
        \item $A_0 = A$,
        \item $A_{n + 1}\subseteq A_n$ for all $n\ge 0$,
        \item $A_mA_n\subseteq A_{m + n}$ for all $m,n\ge 0$.
    \end{enumerate}
\end{definition}

Substituting $m = 0$ In the last condition, we get $AA_n\subseteq A_n$ for all $n\ge 0$ whence each $A_n$ is in fact an ideal in $A$.

\begin{example}
\begin{enumerate}[label=(\alph*)]
    \item Let $\fraka\subseteq A$ be an ideal. Then, $A_n = \fraka^n$ for $n\ge 0$ gives the \emph{$\fraka$-adic filtration} on $A$.
    \item Let $B\subseteq A$ be a subring. Then, given any filtration $(A_n)_{n\ge 0}$ on $A$, the sequence $(B\cap A_n)_{n\ge 0}$ is a filtration on $B$, called the \emph{induced filtration on $B$}.
\end{enumerate}
\end{example}

\begin{definition}[Filtered Module]
    Let $A$ be a filtered ring with filtration $(A_n)_{n\ge 0}$. A \emph{filtered $A$-module} $M$ is an $A$-module $M$ together with a family $(M_n)_{n\ge 0}$ of additive subgroups of $M$ satisfying: 
    \begin{enumerate}[label=(\alph*)]
        \item $M_0 = M$,
        \item $M_{n + 1}\subseteq M_n$ for all $n\ge 0$,
        \item $A_mM_n\subseteq M_{m + n}$ for all $m,n\ge 0$.
    \end{enumerate}
\end{definition}

Substituting $m = 0$ in the last condition, we obtain $AM_n\subseteq M_n$ for all $n\ge 0$ whence each $M_n$ is an $A$-submodule of $M$.

\begin{example}
\begin{enumerate}[label=(\alph*)]
    \item A filtered ring is a filtered module over itself (with the filtration being the same).
    \item Let $\fraka\subseteq A$ be an ideal, then the sequence $(\fraka^nM)_{n\ge 0}$ of $A$-submodules of $M$ forms a filtration on $M$, called the \emph{$\fraka$-adic filtration}.
    \item More generally, given a filtration $(A_n)_{n\ge 0}$ on a ring $A$, define $M_n := A_nM$, which gives $M$ the structure of a filtered $A$-module.
    \item Let $M$ be a filtered $A$-module and $N$ an $A$-submodule of $M$. Then, we have an \emph{induced filtration} on $N$ and $M/N$ given by 
    \begin{equation*}
        (N\cap M_n)_{n\ge 0}\quad\text{and}\quad\left(\frac{N + M_n}{N}\right)_{n\ge 0}
    \end{equation*}
    respectively.
\end{enumerate}
\end{example}

\begin{definition}
    Let $M$ and $N$ be filtered $A$-modules (over a filtered ring). A \emph{homomorphism of filtered modules} is an $A$-module homomorphism $f: M\to N$ such that $f(M_n)\subseteq N_n$ for all $n\ge 0$.
\end{definition}

\begin{definition}
    A filtration $(M_n)_{n\ge 0}$ of an $A$-module $M$ is said to be an \emph{$\fraka$-filtration} if $\fraka M_n\subseteq M_{n + 1}$ for all $n\ge 0$. And a \emph{stable $\fraka$-filtration} if there is a positive integer $N$ such that $\fraka M_n = M_{n + 1}$ for $n\ge N$.
\end{definition}

\begin{definition}[Graded Ring]
    A \emph{graded ring} is a ring $A$ together with a family $(A_n)_{n\ge 0}$ of additive subgroups such that $A = \bigoplus_{n\ge 0} A_n$ and $A_mA_n\subseteq A_{m + n}$ for all $m,n\ge 0$. A nonzero element of $A_n$ is said to be a \emph{homogeneous element of degree $n$}.
\end{definition}

\begin{proposition}
    Let $A = (A_n)_{n\ge 0}$ be a graded ring with the specified grading. Then, 
    \begin{enumerate}[label=(\alph*)]
        \item $A_0$ is a subring, 
        \item $A$ is an $A_0$-module, 
        \item $A_n$ is an $A_0$-submodule for all $n\ge 0$.
    \end{enumerate}
\end{proposition}
\begin{proof}
\begin{enumerate}[label=(\alph*)]
    \item 
\end{enumerate}
\end{proof}

\begin{definition}
    Let $A$ be a graded ring. A \emph{graded $A$-module} is an $A$-module $M$ together with a family $(M_n)_{n\ge 0}$ of subgroups of $M$ such that $M = \bigoplus_{n\ge 0} M_n$ and $A_m M_n\subseteq M_{m + n}$. A nonzero element of $M_n$ is said to be a \emph{homogeneous element of degree $n$}.
\end{definition}

\begin{definition}
    If $M$ and $N$ are graded $A$-modules, then a \emph{homomorphism of graded $A$-modules} is an $A$-module homomorphism $f: M\to N$ such that $f(M_n)\subseteq N_n$ for all $n\ge 0$.
\end{definition}

\begin{proposition}
    Let $A = \bigoplus_{n\ge0} A_n$ be a graded ring. Then, the following are equivalent: 
    \begin{enumerate}[label=(\alph*)]
        \item $A$ is a Noetherian ring. 
        \item $A_0$ is noetherian and $A$ is an $A$-algebra of finite type.
    \end{enumerate}
\end{proposition}
\begin{proof}
    $\implies$ 
\end{proof}

\begin{definition}[Rees Algebra]
    Let $\fraka\subseteq A$ be an ideal. Define the \emph{Rees algebra} to be 
    \begin{equation*}
        A^\ast := \bigoplus_{n\ge 0}\fraka^n
    \end{equation*}
    where element multiplication is the analogue of polynomial multiplication. That is, represent every element of $A^\ast$ as a polynomial
    \begin{equation*}
        a_0 + a_1T + \dots + a_nT^n
    \end{equation*}
    in some indeterminate $T$, where $a_i\in\fraka^i$. It is now easy to see how multiplication is defined. The identity element is simply given by $(1,0,\dots)$ or in the polynomial notation, simply the monomial $1$. This gives $A^\ast$ the structure of a commutative ring.
\end{definition}

\begin{definition}
    Let $M$ be a filtered $A$-module with filtration $(M_n)_{n\ge 0}$ over $A$ with the $\fraka$-adic filtration for some ideal $\fraka\subseteq A$. Define 
    \begin{equation*}
        M^\ast := \bigoplus_{n\ge 0} M_n.
    \end{equation*}
    As in the definition of the Rees algebra, we view elements of $M^\ast$ as formal polynomials 
    \begin{equation*}
        m_0 + m_1T + \dots + m_nT^n
    \end{equation*}
    in some indeterminate $T$, where $m_i\in M_i$. This has a natural action of the Rees algebra, $A^\ast$, by polynomial multiplication, which is well defined, since $\fraka^iM_j\subseteq M_{i + j}$ due to the filtered structure of $M$. This structure also shows that $M^\ast$ is a graded $A^\ast$-module with the above grading.
\end{definition}

\begin{proposition}
    $A$ is a noethering if and only if $A^\ast$ is a noethering.
\end{proposition}
\begin{proof}
    The converse is obvious since $A$ can be realized as a quotient of $A^\ast$. Suppose $A$ is a noethering. Then, $\fraka$ is finitely generated, say $\fraka = (a_1,\dots,a_n)$. Consider the map $\varphi: A[x_1,\dots,x_n]\to A[T]$ mapping $x_i\mapsto x_iT$ (this map exists due to the universal property of the polynomial ring). It is not hard to see that $\im\varphi = A^\ast\subseteq A[T]$, whence we are done due to \thref{thm:hilbert-basis}.
\end{proof}

\begin{proposition}\thlabel{prop:before-artin-rees}
    Let $A$ be a noethering, $M$ a finitely generated $A$-module and $(M_n)_{n\ge 0}$ an $\fraka$-filtration of $M$. Then, the following are equivalent: 
    \begin{enumerate}[label=(\alph*)]
        \item $M^*$ is a finitely generated $A^*$-module. 
        \item The filtration $(M_n)_{n\ge 0}$ is $\fraka$-stable.
    \end{enumerate}
\end{proposition}
\begin{proof}
    Since $M$ is a finitely generated module over a noethering, it is a noetherian $A$-module whence each $M_n$ is finitely generated. Let 
    \begin{equation*}
        Q_n := \bigoplus_{k = 1}^n M_kT^k
    \end{equation*}
    be an $A$-module and $M_n^\ast$ be the $A^\ast$-module generated by it. Note that $M_n^\ast$ is finitely generated since each $M_k$ is finitely generated. Further, these form an ascending chain 
    \begin{equation}\label{eq:ascending-m-ast}
        \left(M_0^\ast\subseteq M_1^\ast\subseteq\cdots\right)\subseteq M^\ast\tag{$\dagger$}
    \end{equation}
    Recall that $A^\ast$ is noetherian. Thus, $M^\ast$ is finitely generated if and only if $M^\ast$ is noetherian if and only if \eqref{eq:ascending-m-ast} stabilizes if and only if $M^\ast = M_{n_0}^\ast$ for some $n_0\in\N$. Now, let $n\ge n_0$. Let $m_{n + 1}\in M_{n + 1}$. Then, $m_{n + 1}T^{n + 1}\in M_{n + 1}^\ast = M_n^\ast$ and thus 
    \begin{equation*}
        m_{n + 1}T^{n + 1} = \sum_{k = 1}^r P^A_k(T)P^M_k(T)
    \end{equation*}
    where each $P^A_k$ is a polynomial in $A^\ast$ while $P^M_k$ is a polynomial in $Q_n$. Looking at the coefficient of $T^{n + 1}$, we see that $m_{n + 1}\in\fraka M_n$ whence $M_{n + 1} = \fraka M_n$ whereby the filtration $(M_n)_{n\ge 0}$ is stable. The converse is obvious and thus this is an equivalence thereby completing the proof.
\end{proof}

\begin{lemma}[Artin-Rees Lemma]\thlabel{lem:artin-rees}
    Let $A$ be a noethering, $\fraka\subseteq A$ an ideal, $M$ a finitely-gennerated $A$-module, $(M_n)_{n\ge 0}$ a stable $\fraka$-filtration of $M$. If $M'$ is an $A$-submodule of $M$, then $(M'\cap M_n)_{n\ge 0}$, the induced filtration on $M'$ is a stable $\fraka$-filtration of $M'$.
\end{lemma}
\begin{proof}
    We have 
    \begin{equation*}
        \fraka(M'\cap M_n)\subseteq\fraka M'\cap\fraka M_n\subseteq M'\cap M_{n + 1}
    \end{equation*}
    whence the induced filtration $(M'\cap M_n)_{n\ge 0}$ is an $\fraka$-filtration. Consider $M'^\ast$ induced by this filtration. This is an $A^\ast$-submodule of $M^\ast$. Due to \thref{prop:before-artin-rees}, $M^\ast$ is a finitely generated $A^\ast$-module whence is noetherian and thus $M'^\ast$ is a finitely generated $A^\ast$-module. Again, \thref{prop:before-artin-rees}, the filtration $(M'\cap M_n)_{n\ge 0}$ is $\fraka$-stable. This completes the proof.
\end{proof}

\begin{corollary}[Krull's Intersection Theorem]
    Let $A$ be a noethering and $\fraka\subseteq\frakR(A)$ a proper ideal. Let $M$ be a finitely generated $A$-module. Then $\bigcap_{n\ge 0}\fraka^nM=0$.
\end{corollary}
\begin{proof}
    Let $N := \bigcap_{n\ge 0}\fraka^n M$. Then, $\fraka^nM\cap N = N$ for all $n\in\N$. The filtration $\fraka^nM$ is $\fraka$-stable and thus, so is the induced filtration on $N$. But this means $(N)_{n\ge 0}$ is a stable $\fraka$-filtration, implying that $\fraka N = N$ and thus $N = 0$ from \thref{lem:nakayama}.
\end{proof}


\section{Completion}

\begin{definition}
    An \emph{inverse system} of $A$-modules is a collection of $A$-modules $(M_n)_{n\ge 0}$ and homomorphisms $(\theta_n)_{n\ge 1}$ where $\theta_n: M_n\to M_{n - 1}$. If $\theta_n$ is surjective for all $n$, then the system is said to be a \emph{surjective system}.

    The \emph{inverse limit} of this system is the \emph{categorical limit} over the diagram 
    \begin{equation*}
        M_0\stackrel{\theta_1}{\longleftarrow} M_1\stackrel{\theta_2}{\longleftarrow} M_2\stackrel{\theta_3}{\longleftarrow}\cdots
    \end{equation*}
    in $A-\catMod$.
\end{definition}

\begin{example}
    Suppose we have a filtration $M = M_0\supseteq M_1\supseteq\cdots$, then we have an inverse system $(M/M_n)_{n\ge 0}$ with 
    \begin{equation*}
        \theta_{n + 1}: M/M_{n + 1}\onto M/M_n
    \end{equation*}
    being the natural map $x + M_{n + 1}\mapsto x + M_n$. Moreover, this is a \emph{surjective system}.
\end{example}

\begin{proposition}
    The inverse limit of an inverse system $((M_n)_{n\ge 0}, (\theta_n)_{n\ge1})$ exists and is unique upto unique isomorphism.
\end{proposition}
\begin{proof}
    It suffices to show existence since the ``unique upto unique isomorphism'' simply follows from the fact that the inverse limit is a ``universal object''.

    Let $N := \prod_{i\ge 0} M_i$ and $\pi_i: N\to M_i$ denote the projection. Let 
    \begin{equation*}
        M := \{(x_i)_{i\ge 0}\in N\mid \theta_{i + 1}(x_{i + 1}) = x_i\text{ for all }i\ge 0\}.
    \end{equation*}
    That this is a submodule is easy to verify. This is called the submodule of \emph{coherent sequences}. Next, define $f_i: M\to N_i$ by the restriction $f_i = \pi_i|_M$. We contend that 
    \begin{equation*}
        M = \limit_n M_n.
    \end{equation*}
    Indeed, let $P$ be another $A$-module with maps $g_i: P\to M_i$ such that $\theta_{i + 1}\circ g_{i + 1} = g_{i}$ for all $i\ge 0$. Define the map $h: P\to M$ by $h(p) = (g_0(p),g_1(p),\dots)$. Since this sequence is coherent, it is a valid map into $M$. Morover, for any $a\in A$, and $p,p'\in P$,
    \begin{equation*}
        h(p + ap') = (g_i(p) + ag_i(p')) = (g_i(p)) + a(g_i(p')) = h(p) + ah(p')
    \end{equation*}
    and thus, $h$ is an $A$-module homomorphism. Finally, 
    \begin{equation*}
        f_i\circ h(p) = f_i((g_j(p))_{i\ge 0}) = g_i(p)
    \end{equation*}
    as desired.
\end{proof}

\subsection*{Topological Interlude}

\begin{definition}
    Let $G$ be a topological abelian group. A \emph{fundamental system} of neighborhoods of $\{0\}$ is a descending chain of subgroups 
    \begin{equation*}
        G = G_0\supseteq G_1\supseteq\cdots.
    \end{equation*}
    such that $U\subseteq G$ is a neighborhood of $0$ if and only if it contains some $G_n$.
\end{definition}

\begin{proposition}\thlabel{prop:filtration-induces-topology}
    Let $G$ be an abelian group and $G=G_0\supseteq G_1\supseteq\cdots$ be a descending chain of subgroups of $G$. The collection 
    \begin{equation*}
        \scrB := \{g + G_i\mid g\in G\}
    \end{equation*}
    forms a basis for a topology on $G$. Under this topology, $G$ is a topological group.
\end{proposition}
\begin{proof}
    
\end{proof}

\begin{definition}
    A sequence $(x_n)$ in a topological abelian group $G$ is said to be \emph{Cauchy} if for every open neighborhood $U$ of $0$, there is a positive integer $N$ such that $x_n - x_m\in U$ for all $m,n\ge N$.
\end{definition}

We shall now construct the completion of a group using Cauchy sequences.
\begin{itemize}
    \item Define a relation on the set of all Cauchy sequences in $G$ by $(x_n)\sim (y_n)$ if and only if $x_n - y_n\to 0$ as $n\to\infty$.
    \item That this is an equivalence relation is easy to see, for if $(x_n)\sim(y_n)$ and $(y_n)\sim(z_n)$, then 
    \begin{equation*}
        \lim_{n\to\infty}(x_n - z_n) = \lim_{n\to\infty}\left((x_n - y_n) + (y_n - z_n)\right) = \lim_{n\to\infty}(x_n - y_n) + \lim_{n\to\infty}(y_n - z_n) = 0.
    \end{equation*}
    \item Let $\wh G$ denote the equivalence classes under the above relation. Define the operation $[(x_n)] + [(y_n)] = [(x_n + y_n)]$. It is not hard to verify that this is well defined and endows $\wh G$ with the structure of an abelian group.
\end{itemize}

\begin{proposition}
    Let $\varphi: G\to\wh G$ denote the map $g\mapsto[(g)]$, the equivalence class of the constant sequence. This is a homomorphism of groups and $\ker\varphi = \bigcap U$ where the intersection ranges over all neighborhoods of $0$.
\end{proposition}
\begin{proof}
\end{proof}

\subsection*{Back to Completions}

Let $M$ be a filtered module with filtration $(M_n)_{n\ge 0}$ over a filtered ring $A$ with filtration $(A_n)_{n\ge 0}$. In accordance with \thref{prop:filtration-induces-topology}, both $M$ and $A$ have the structure of abelian topological groups.

\begin{proposition}
    Under the aforementioned induced topology, $A$ is a topological ring and $M$ is a topological module. This topology is called the \textbf{topology induced by the filtration}.
\end{proposition}
\begin{proof}
\end{proof}

We now have a topology on the module $M$ whence, we can form its completion, $\wh M$, as outlined in the previous (sub)section.
\begin{itemize}
    \item Let $(x_n)$ be Cauchy in $M$ and $a\in A$, in particular, let $a\in A_{m_0}$. Let $U$ be a neighborhood of $0$, which contains $M_{n_0}$ for some positive integer $n_0$. Then, there is a positive integer $n_1$ such that for all $m,n\ge n_1$, $x_m - x_n\in M_{n_0}$, whereby $a(x_m - x_n)\in A_{m_0}M_{n_0}\subseteq M_{n_0}\subseteq U$.

    \item Further, if $(x_n)\sim (y_n)$ and $a\in A$, we must have $(ax_n)\sim (ay_n)$ using a similar argument as above.
\end{itemize}

Thus $\wh M$ is also an $A$-module. \textcolor{red}{Note that we haven't yet defined a topology on $\wh M$.}

\begin{proposition}
    $\displaystyle\wh M\cong\limit_n M/M_n$ as $A$-modules.
\end{proposition}
\begin{proof}
    We shall define a map $\displaystyle\alpha: \wt M:=\limit_{n}M/M_n\to\wh M$. Let $(y_n)\in\wt M$ be a coherent sequence. For each $n\ge 0$, pick any $x_n\in M_n$ such that $\pi_n(x_n) = y_n$ where $\pi_n: M\to M/M_n$ is the natural projection. First, note that \todo{complete this}
\end{proof}

Now, let $A$ be a filtered ring, which can be regarded as a filtered module over itself. Then, $\wh A$ is an $A$-module. There is a natural product on this module, which can be seen easily using coherent sequences. That is, 
\begin{equation*}
    [(x_n)_{n\ge 0}]\cdot[(y_n)_{n\ge 0}] = [(x_ny_n)_{n\ge 0}].
\end{equation*}
Thus, $\wh A$ is an $A$-algebra, in particular, a ring in its own right.

\begin{definition}
    Given inverse systems $(M_n,\theta_n)$ and $(M_n',\theta_n')$, a \emph{morphism of inverse systems} $f: (M_n')_n\to (M_n)_n$ is a family of maps $f_n: M_n'\to M_n$ for $n\ge 0$ such that the diagram 
    \begin{equation*}
        \xymatrix {
            M_n'\ar[d]_{f_n} & M_{n + 1}'\ar[d]^{f_{n + 1}}\ar[l]_{\theta_{n + 1}'}\\
            M_n & M_{n + 1}\ar[l]_{\theta_{n + 1}}
        }
    \end{equation*}
    commutes for all $n\ge 0$. Exactness of such a sequence has the obvious definition.
\end{definition}

A morphism as above induces a map $\displaystyle f_\ast: \limit_n M_n'\to\limit_n M_n$ given by $(x_n)\mapsto (f_n(x_n))$. The commutativity of the diagram ensures that the sequence on the right is coherent.

\begin{proposition}
    Let $0\to\{A_n\}\to\{B_n\}\to\{C_n\}\to0$ be an exact sequence of inverse systems. Then 
    \begin{equation*}
        0\to\limit A_n\to\limit B_n\to\limit C_n
    \end{equation*}
    is exact. Further, if $\{A_n\}$ is a surjective system, then 
    \begin{equation*}
        0\to\limit A_n\to\limit B_n\to\limit C_n\to 0
    \end{equation*}
    is exact.
\end{proposition}
\begin{proof}
\end{proof}

% \setcounter{chapter}{0}
% \part{Solutions to Atiyah-Macdonald}
% \chapter{Rings and Ideals}
% The origins of Fourier analysis lie in solving the heat equation:
\begin{equation*}
    \Delta u = \partial_t u
\end{equation*}
where $\Delta$ denotes the Laplacian.

In order to solve this, Fourier believed for a long time that one could expand a function as a series 
\begin{equation*}
    f\sim \sum_k a_k\sin kx + \sum b_k\cos kx.
\end{equation*}

This is not true. In 1876, Paul Du Bois-Reymond gave an example of a continuous function whose Fourier series does not converge. But in 1966, Carleson showed that given an $L^2$ function on $[0,1]$, the points at which the Fourier series does not converge has measure $0$.

There are many applications to PDEs, in solving the 
\begin{description}
    \item[Laplace Equation:] $\Delta u = 0$,
    \item[Heat Equation:] $\partial_t u = \Delta u$,
    \item[Wave Equation:] $\partial_{tt} u = \Delta u$.
\end{description}


\begin{definition}[Fourier Series]
    Given $f\in L^1[a, b]$, its $k$-th \emph{Fourier coefficient} is defined as 
    \begin{equation*}
        \wh f(k) := \frac{1}{L}\int_a^b f(x)\exp\left(-\frac{2\pi i k}{L}x\right)~dx.
    \end{equation*}
    where $L = b - a$.

    The \emph{Fourier series} of $f$ is given formally by 
    \begin{equation*}
        f\sim\sum_{k\in\Z}\wh f(k)\exp\left(\frac{2\pi i k}{L}x\right).
    \end{equation*}
\end{definition}

The question is whether 
\begin{equation*}
    \lim_{n\to\infty}\sum_{k= -n}^n \wh f(k)\exp\left(\frac{2\pi ik}{L}x\right) = f(x)
\end{equation*}
in the following cases: 
\begin{itemize}
    \item if $f\in L^1[a,b]$. Here we cannot expect pointwise convergence because one can just change the value of $f$ at a single point without affecting its Fourier series.
    \item if $f\in C[a,b]$. This is not true because of an example by Paul Du Bois-Reymond.
    \item if $f\in C^1[a,b]$ then this is true.
    \item if $f\in L^2[a,b]$, then there may not be pointwise convergence but there is convergence in the $L^2$-norm.
\end{itemize}

There are notions of convergence other than pointwise and uniform. For example Ces\`aro and Abel. Fej\'er had proved that for continuous functions, the Ces\`aro sums converge uniformly to the function, whatever that means.

\begin{example}
    Consider the function $f:[-\pi,\pi]\to\R$ given by $f(x) = x$. Then, 
    \begin{equation*}
        \wh f(k) = \frac{1}{2\pi}\int_{-\pi}^\pi x\exp(-ikx)~dx = 
        \begin{cases}
            0 & k = 0\\
            \frac{(-1)^ki}{k} & k\ne 0.
        \end{cases}
    \end{equation*}
    The Fourier series is then given by 
    \begin{equation*}
        \sum_{k\in\Z\backslash\{0\}}(-1)^{k + 1}\frac{\sin kx}{k}.
    \end{equation*}
\end{example}

% \bibliographystyle{plain}
% \bibliography{ref}
\end{document}